% \documentclass[11pt,twoside,notitlepage]{report}
% %\usepackage[margin=1.5cm, right=2.3cm, left=2.3cm, top=2.2cm, bottom=1.4cm,includefoot]{geometry}
% \usepackage{brufaceStyle}
% \usepackage[utf8]{inputenc}
% \usepackage[english,french,dutch]{babel}
% \usepackage[T1]{fontenc}
% \usepackage[hidelinks]{hyperref}
% \usepackage{eurosym}
% \usepackage{amsmath,amsfonts,amssymb}
% \usepackage{rotating}
% \usepackage{graphicx} %Import Images
% \usepackage[table,xcdraw]{xcolor}
% %\usepackage{tikz}
%
% \usepackage{pdfpages}
% % \documentclass[tikz,border=2mm]{standalone}
% \usepackage{geometry}
% \usepackage{pdflscape}
% \usepackage{afterpage}
%
% \usepackage{titlesec}
%
% \setcounter{secnumdepth}{4}
%
% \titleformat{\paragraph}
% {\normalfont\normalsize\bfseries}{\theparagraph}{1em}{}
% \titlespacing*{\paragraph}
% {0pt}{3.25ex plus 1ex minus .2ex}{1.5ex plus .2ex}

\documentclass[a4paper, 12pt]{report}

\usepackage[utf8]{inputenc}
\usepackage[french]{babel} %
\usepackage{xcolor}
\usepackage[top=2.5cm, bottom=1.5cm, left=2.5cm, right=1cm]{geometry}

\usepackage{tikz}
\usepackage{url}
\usepackage{appendix}
\usepackage{lscape}
\usepackage{siunitx}
% \usepackage{import}
\usepackage{pgf}
\usepackage{svg}
\usepackage{textcomp}
\usepackage[T1]{fontenc}
\usepackage{booktabs}
\usepackage{listings}
\usepackage{dirtytalk}
\usepackage{hyperref}
\usepackage{minted}
\usepackage{float}
\usepackage{eurosym}
%\usepackage[pdftex]{graphicx}



\newcommand\JSONnumbervaluestyle{\color{blue}}
\newcommand\JSONstringvaluestyle{\color{red}}


% Logos
\newcommand{\ulb}{\includegraphics[scale=1.1]{img/front_page/logo_ULB2.pdf}}
\newcommand{\polytech}{\includegraphics[scale=0.35]{img/front_page/logo_polytech_FR.pdf}}

% Polices
\definecolor{ULBblue}{rgb}{0,0.2196,0.5765}
\newcommand{\fontTitle}{\sffamily \Huge\selectfont \color{ULBblue}}
\newcommand{\fontSubtitle}{\sffamily \LARGE \selectfont \color{ULBblue}}
\newcommand{\fontText}{\sffamily \selectfont}
\newcommand{\fontColor}{\sffamily \selectfont \color{ULBblue}}

% Titre
\newcommand{\titleA}{\fontTitle{Système intégré de monitoring}} % Titre identique au titre remis au secrétariat
\newcommand{\titleB}{\fontTitle{pour centre hospitalier en Afrique}} % (dans la langue de rédaction a priori)
% Sous-titre
%\newcommand{\subtitle}{\fontSubtitle{Ligne du sous-titre du mémoire}}
% Titre du diplôme
\newcommand{\diplomaA}{\fontText{Mémoire présenté en vue de l'obtention du diplôme}} % A laisser en Français
\newcommand{\diplomaB}{\fontText{d'Ingénieur Civil en informatique à finalité spécialisée}}

% Etudiant
\newcommand{\student}{\textbf{\sffamily \large João Marques Correia}}

% Supervision
\newcommand{\promAa}{\fontColor{Directeur}}
\newcommand{\promAb}{\fontText{Professeur François Quitin}}
\newcommand{\promBa}{\fontColor{Co-Promoteur}}
\newcommand{\promBb}{\fontText{Professeur Antoine Nonclercq}}
\newcommand{\promCa}{\fontColor{Superviseur}}
\newcommand{\promCb}{\fontText{Quentin Delhaye}}
\newcommand{\deptA}{\fontColor{Service}}
\newcommand{\deptB}{\fontText{BEAMS}}

% Année académique
\newcommand{\yearA}{\fontColor{Année académique}}
\newcommand{\yearB}{\fontText{2019 - 2020}}


%JSON listing code
\colorlet{punct}{red!60!black}
\definecolor{background}{HTML}{EEEEEE}
\definecolor{delim}{RGB}{20,105,176}
\colorlet{numb}{magenta!60!black}

\lstdefinelanguage{json}{
    basicstyle=\normalfont\ttfamily,
    numbers=left,
    numberstyle=\scriptsize,
    stepnumber=1,
    numbersep=8pt,
    showstringspaces=false,
    breaklines=true,
    frame=lines,
    backgroundcolor=\color{background},
    literate=
     *{0}{{{\color{numb}0}}}{1}
      {1}{{{\color{numb}1}}}{1}
      {2}{{{\color{numb}2}}}{1}
      {3}{{{\color{numb}3}}}{1}
      {4}{{{\color{numb}4}}}{1}
      {5}{{{\color{numb}5}}}{1}
      {6}{{{\color{numb}6}}}{1}
      {7}{{{\color{numb}7}}}{1}
      {8}{{{\color{numb}8}}}{1}
      {9}{{{\color{numb}9}}}{1}
      {:}{{{\color{punct}{:}}}}{1}
      {,}{{{\color{punct}{,}}}}{1}
      {\{}{{{\color{delim}{\{}}}}{1}
      {\}}{{{\color{delim}{\}}}}}{1}
      {[}{{{\color{delim}{[}}}}{1}
      {]}{{{\color{delim}{]}}}}{1},
}

%YAML Listing



\begin{document}
\thispagestyle{empty}
\newgeometry{top=2.5cm, bottom=1.5cm, left=2.5cm, right=1cm}
\setlength{\unitlength}{1mm}
\noindent\begin{picture}(175,257)

	\put(0,245){\polytech}
	\put(153,139.5){\ulb}

	\put(8,155){\makebox(150,10)[l]{\titleA}}
	\put(8,145){\makebox(150,10)[l]{\titleB}}
	%\put(8,135){\makebox(150,10)[l]{\subtitle}}

	\put(0,75){
	\begin{tikzpicture}[scale=0.1]
	\fill [fill=ULBblue](0,0) rectangle (0.8,90);
	\fill [fill=ULBblue](0,57) rectangle (152,57.8);
	\end{tikzpicture}}

	\put(8,120){\makebox(150,5)[l]{\diplomaA}}
	\put(8,115){\makebox(150,5)[l]{\diplomaB}}

	\put(8,75){\makebox(150,10)[l]{\selectfont \student}}

	\put(8,44){\makebox(80,5)[l]{\promAa}}
	\put(8,39){\makebox(80,5)[l]{\promAb}}
	\put(8,31){\makebox(80,5)[l]{\promBa}} % Commenter la ligne si pas nécessaire
	\put(8,26){\makebox(80,5)[l]{\promBb}} % Commenter la ligne si pas nécessaire
	\put(8,18){\makebox(80,5)[l]{\promCa}} % Commenter la ligne si pas nécessaire
	\put(8,13){\makebox(80,5)[l]{\promCb}} % Commenter la ligne si pas nécessaire
	\put(8,5){\makebox(80,5)[l]{\deptA}}
	\put(8,0){\makebox(80,5)[l]{\deptB}}

	\put(145,5){\makebox(30,5)[r]{\yearA}}
	\put(145,0){\makebox(30,5)[r]{\yearB}}

\end{picture}
\restoregeometry


\chapter*{\huge Remerciements}

    \noindent
    En tout premier lieu, je voudrais remercier le directeur de ce mémoire, le Professeur François Quitin, et mon co-promoteur, le Professeur Antoine Nonclercq, pour leur disponibilité, leurs conseils et les connaissances techniques qu’ils ont apportées à ce mémoire.

    ~

    \noindent
    Je tiens également à remercier mon superviseur, Quentin Delhaye, de m’avoir tant aidé tout au long de ce mémoire et d’avoir suivi de près tous les progrès réalisés.


    ~

    \noindent
    Je voudrais aussi remercier Loïc Vaes, chef de projet chez AEDES, et Erica Berghman, ULB-Coopération, de m’avoir accordé l’opportunité de travailler sur ce mémoire.


    ~

    \noindent
    Je tiens à témoigner toute ma gratitude envers mon ami, Jacky Trinh, et mon frère, Tiago Correia, pour avoir pris le temps de relire ce mémoire et pour tous les conseils qu’ils m’ont donnés concernant son style.


    ~

    \noindent
    Enfin, un grand merci à mes parents qui m’ont toujours encouragé tout au long de mes études, surtout dans les moments plus difficiles.


\chapter*{\huge Résumé}


  \vspace{5cm}

  \noindent
   L’Organisation mondiale de la Santé a déclaré que les systèmes d’information hospitaliers sont un des composants principaux d’une structure de santé. CERHIS est un système d’information qui a été développé pour répondre aux besoins particuliers des centres hospitaliers en Afrique. L’infrastructure de CERHIS est constituée de plusieurs composants, dont des tablettes Android, un serveur local et un système d’alimentation sans interruption. Afin d’assurer l’opérabilité de CERHIS, les différents éléments doivent être surveillés en permanence et les techniciens doivent être alertés en cas d’une défaillance. C’est ce dernier problème que ce mémoire essaye de tacler en créant une solution de monitoring basée sur un Raspberry Pi. Cependant, les conditions particulières de l’environnement imposent certaines contraintes sur cette même solution. En effet, le dispositif doit pouvoir contacter des techniciens par SMS et envoyer des informations sur un serveur distant, mais sans avoir accès à une connexion Internet filaire. En outre, le système doit aussi posséder une batterie pour que le monitoring continue à se faire même en cas de coupure du réseau électrique. La solution présentée ici combine un SIM800L avec le Raspberry Pi de telle sorte que ce dernier puisse se connecter aux réseaux mobiles pour transmettre des données. Thingstream, une plateforme \textit{IoT Communication-as-a-Service}, a été exploitée afin de gérer la connectivité et l'échange d'informations entre SIM800L et le serveur distant. En ce qui concerne le logiciel de monitoring, une solution de pointe a été développée suivant une architecture de microservices et basée sur l'application de monitoring Prometheus couplé à l’outil de visualisation de données Grafana. Les tests effectués avec le prototype ont été très prometteurs et permettent d’établir la direction à prendre à l'avenir pour l'améliorer.

   ~

   \noindent
   \textbf{Mots-clés:} CERHIS, \hspace{0.2cm} Système de Monitoring, \hspace{0.2cm} Prometheus, \hspace{0.2cm} Microservices





\tableofcontents
\newpage


\chapter{Introduction}
\section{Contexte du mémoire}

\noindent
De nos jours, les solutions de stockage d'informations au format papier sont remplacées au fur et à mesure par des solutions électroniques. Ces dernières permettent de stocker une énorme quantité de documents sur des supports physiques de petite taille, mais cela est loin d'être leur seul avantage. En effet, les documents électroniques ouvrent la porte à tant d'autres possibilités. Ils facilitent la collaboration, l'échange et la sauvegarde de données et proposent énormément de différents formats pour présenter une très grande variété d'informations. De plus, lorsque les informations informatisées se retrouvent sur une base de données, les langages de requête aident à très rapidement récupérer les données cherchées.

~

\noindent
D'après l'Organisation mondiale de la santé (OMS), les systèmes d'information hospitaliers sont un des six piliers d'une structure de santé \cite{world2010monitoring}. Avoir accès à des informations fiables et solides est indispensable pour conduire toutes les activités liées aux autres composantes du système de santé \cite{Mutale2013}. Avec ce but est né CERHIS, \say{un système d'information adapté aux structures de santé situées dans des environnements à faibles ressources}. Cette solution est composée de tablettes Android et d'un serveur local qui centralise le stockage de toutes les données. Grâce à un réseau Wi-Fi local, les tablettes peuvent communiquer avec le serveur afin d'échanger les données médicales. En plus de cela, CERHIS possède un appareil d'alimentation mixte\footnote{réseau électrique combiné avec des batteries alimentées par des panneaux solaires} dans le but de faire face à d'éventuelles coupures d'électricité.


~


\noindent
Toutefois, comme tout système informatique et électronique, les différents composants peuvent tomber en panne, ce qui peut affecter les performances du système sur le long terme. De ce fait, un dispositif de monitoring de l'infrastructure est requis afin de détecter des problèmes et d'avertir les parties concernées le plus rapidement possible. Lors des années académiques antérieures, plusieurs projets ont eu lieu dans l'école polytechnique de Bruxelles dans la finalité de créer ce dispositif de monitoring, dont un projet développé par moi-même l'année dernière.

~

\noindent
Le but de ce mémoire est donc de réunir toutes les connaissances acquises tout au long des programmes précédents pour construire un nouveau prototype capable de surveiller l'installation de CERHIS.  En particulier, le prototype de l'année dernière a présenté plusieurs problèmes qui doivent être résolus en vue d'avoir un prototype plus proche du potentiel produit final.

~

\noindent
Il faut cependant noter que cette thèse s'encadre dans un projet de coopération et d'aide au développement. Le but n'est pas de proposer une boîte noire avec un guide d'instructions sur son utilisation. Au contraire, ce mémoire vise à partager les connaissances requises pour construire un tel produit, ou à pointer le lecteur dans la bonne direction. Les choix effectués seront décrits en détail, ainsi que le fonctionnement du prototype.

~

\noindent
Un dernier aspect très important à considérer est que ce mémoire ne se concentre pas sur un seul domaine de l'informatique. En effet, comme il sera présenté plus tard, le prototype encadre une partie de communication utilisant les réseaux sans fil, mais aussi une partie concernant les procédés de monitoring qui est souvent associé au domaine de DevOps. Finalement, il y a également une petite part liée au développement d'un boîtier qui logera tous les composants électroniques.


\section{Parties prenantes}

\noindent
Ce mémoire étant axé sur la coopération et l'aide au développement, plusieurs parties prenantes sont étroitement associées à sa mise en œuvre. Elles sont présentées ci-dessous :

~

\begin{itemize}
  \item \textbf{AEDES:} AEDES, ou Agence Européenne pour le Développement et la Santé, \say{est une société de consultance spécialisée en santé publique}\cite{aedes} fondée en 1985. Acteur majeur en santé publique, la mission d'AEDES est \say{l'amélioration de la qualité et de l'accès aux soins de santé partout dans le monde.}\cite{aedes} Cela est accompli grâce à la mise à disposition de moyens humains et le partage de connaissances. Les contacts avec AEDES ont été réalisés à travers de Loïc Vaes.

  ~

  \item \textbf{ULB-Coopération:} \say{ULB-Coopération est l'ONG de l'Université libre de Bruxelles.} \cite{ulb_coop} Elle est active dans plusieurs thématiques, dont la santé et les systèmes de santé. \say{ULB-Coopération s'engage à contribuer à la construction d'une société dans laquelle il fait bon vivre, une société juste, émancipatrice, solidaire, responsable où toutes les citoyennes et tous les citoyens, sans discrimination de genre, sont traité·e·s avec égalité.}\cite{ulb_coop} Erica Berghman était le point de contact avec ULB-Coopération.

  ~

  \item \textbf{Codepo:} Fondée en 2007, Codepo est la cellule de coopération au développement de l'École polytechnique de Bruxelles. Elle propose aux étudiants de Master de participer dans un projet de coopération au développement. Ses principaux objectifs sont la pédagogie, la recherche scientifique et technique et l'éducation au développement. \cite{codepo} Le Professeur Antoine Nonclercq a représenté la cellule Codepo.

\end{itemize}

\section{Projet précédent}

\noindent
Lors des années académiques 2016-2017 et 2017-2018, une solution de monitoring a été créée par des étudiants à l'École polytechnique de Bruxelles. Cette solution utilisait un smartphone Android comme composant principal pour tourner de logiciel de monitoring. Cependant, comme expliqué dans \cite{delobbe_2017}, le système d'exploitation Android peut mettre certaines applications en pause pour économiser la batterie du smartphone. Ceci a posé un énorme problème puisque le logiciel de monitoring doit être capable de surveiller l'état des différents éléments de façon continue.

~

\noindent
Un prototype conçu lors de l'année académique 2018-2019 a essayé de pallier ce problème en remplaçant le smartphone par un Raspberry Pi. Plus de détails sur ce projet seront fournis dans la section \ref{sec:old_boy}. Ce mémoire s'appuie sur les connaissances acquises tout au long des projets précédents afin de proposer un nouveau prototype.


\section{Structure du mémoire}

\noindent
Ce mémoire commence par une présentation de l’état de l’art des trois technologies principales de ce mémoire, c’est-à-dire les technologies de communication à distance, les solutions pour le monitoring d’un serveur Linux et les plateformes de visualisation de données. De plus, dans chaque section, il est aussi expliqué la façon dont chacune de ces solutions s'intègre dans cette thèse.

~

\noindent
Ensuite, le cahier des charges établi en début d’année avec AEDES est détaillé. Ce cahier des charges présente non seulement les fonctionnalités du logiciel de monitoring, mais couvre encore certaines propriétés requises afin que le prototype soit adapté à son environnement de fonctionnement. Cette section détaille également l’ordre dans lequel chaque tâche a été réalisée ainsi que la méthodologie adoptée pour accomplir ces dernières.

~

\noindent
L’intégralité du prototype construit lors de mémoire est couverte dans le chapitre 4. Tous les composants et solutions considérés sont exposés ainsi que les raisons derrière chaque choix.  De plus, le prototype réalisé l’année précédente est également présenté dans cette section puisqu’il a une grande influence sur le prototype conçu cette année. Finalement, le résultat des premiers tests réalisés ainsi que les conclusions à tirer sont détaillés.

~

\noindent
Le chapitre 5 présente le logiciel de monitoring implémenté. En particulier, ce chapitre décrit comment les différentes parties du logiciel de monitoring interagissent afin de surveiller l’infrastructure de CERHIS. L’application créée dans ce mémoire est composée des solutions Open Source qui y sont également détaillées. Une section de ce chapitre est aussi réservée à la présentation du serveur distant responsable du stockage de toutes les données envoyées par le logiciel de monitoring.

~

\noindent
Valider le bon fonctionnement de toutes les parties constituant le prototype est un élément essentiel de cette thèse. En effet, il faut comprendre si le résultat de ce mémoire respecte les attentes du cahier des charges ou si un des choix a guidé le prototype dans la mauvaise direction. Le chapitre 6 présente donc en détail les tests réalisés sur l’entière de la solution de monitoring produite lors de ce mémoire. De plus, les conclusions à tirer de ces tests sont également explicitées.

~

\noindent
Finalement, le dernier chapitre parcourt brièvement les accomplissements de ce mémoire et les prochaines étapes en vue de transformer le résultat en un produit prêt à être déployé dans un environnement de production.

\chapter{État de l'art}

\section{Technologies de communication à distance}


Lors des dernières années, le nombre de dispositifs Internet des Objets (IoT) connaît une croissance fulgurante qui se maintiendra au cours des années qui suivent (voir figure \ref{fig:iot_number}). Ces objets sont capables d’acquérir des données sur leur environnement et/ou de prendre des actions sur celui-ci. Ils communiquent avec d’autres machines pour transmettre les informations acquises et recevoir des commandes lorsque cela est nécessaire. Les dispositifs IoT ont de diverses applications telles que les thermostats intelligents, les voitures connectées, suivi et monitorage d’actifs, et bien d’autres. Selon les conditions d’utilisation et son objectif, ces dispositifs doivent être capables d’envoyer des données sur des courtes ou des longues distances. En outre, leur consommation énergétique doit généralement rester faible, notamment lorsqu’ils sont alimentés par une batterie.

~

\begin{figure}[ht!]
  \includegraphics[width=\textwidth]{img/state_of_the_art/iot_number_connected.png}
  \caption{Nombre de dispositifs connectés à Internet \cite{lueth_2018}}
  \label{fig:iot_number}
\end{figure}

~

\noindent
Un réseau de communication unique et employable indépendamment du contexte du projet n’existe pas \cite{vannieuwenborg_iot}. Toutefois, une multitude de réseaux avec des caractéristiques distinctes sont disponibles, tels que LoRaWAN et GSM. La figure 1 présente l’architecture simplifiée des réseaux utilisés par les dispositifs IoT pour échanger des informations avec des serveurs ou des utilisateurs qui sont connectés à Internet. Nous ne nous intéressons ici qu’aux technologies responsables de la transmission de données dans la phase 1 de la figure.

~

\begin{figure}[ht!]
  \includegraphics[width=\textwidth]{img/state_of_the_art/network_iot.png}
  \caption{Architecture simplifiée d'un réseau avec dispositifs IoT (Basé sur \cite{sanchez2016state, mekki2018overview})}
  \label{fig:network_archi}
\end{figure}

~

\noindent
Actuellement, les principaux réseaux sans fil disponibles pour les dispositifs IoT sont :

\begin{itemize}
  \item LoRaWAN
  \item Réseaux mobiles (2G/3G/4G/5G)
  \item Sigfox
  \item Satellite
  \item Wi-Fi
  \item Zigbee
  \item Bluetooth
  \item Réseaux mobiles IoT (NB-IoT et LTE-M)
\end{itemize}

~

\noindent
Ces technologies utilisent toutes des ondes électromagnétiques pour transmettre les données, mais les fréquences sur lesquelles elles opèrent sont parfois très différentes. Certaines technologies utilisent une implémentation propriétaire, d’autres se basent sur des standards open source. \cite{foubert_iot} De façon similaire, certaines technologies exploitent la bande ISM\footnote{Bande industrielle, scientifique et médicale}, alors que d’autres exploitent des fréquences non réglementées. Ces dernières permettent de déployer son propre réseau privé, à condition d’acheter et configurer tout le matériel requis ce qui peut demander un investissement initial assez important. Ces différences accordent des propriétés distinctes aux réseaux en ce qui concerne le coût d’utilisation et déploiement, la bande passante, et la portée du signal. À noter également, certains réseaux imposent une taille maximale sur chaque message, qu’il soit envoyé ou reçu. Le tableau comparatif de l’annexe \ref{ap:table_network} offre une vue d’ensemble sur les différentes technologies et leurs propriétés. Il faut cependant remarquer que la portée et le débit sont juste donnés à titre comparatif et ils sont à prendre avec des pincettes. En effet, ces valeurs varient beaucoup en fonction de certains paramètres tels que la géographie du milieu (urbain ou rural) et des éventuelles interférences. Néanmoins, le débit et la portée sont très importants pour classifier ces réseaux et, avec l’aide de la figure \ref{fig:range_iot}, ces technologies peuvent être classées de la façon suivante \cite{orange_iot} :


\begin{figure}
  \includegraphics[width=\textwidth]{img/state_of_the_art/range_iot.png}
  \caption{Classement des réseaux en fonction de leur portée et débit (Basé sur )}
  \label{fig:range_iot}
\end{figure}

\begin{itemize}
  \item IoT haut débit : Dans cette catégorie se trouvent principalement les technologies comme les réseaux mobiles, le Wi-Fi et le satellite. Les débits proposés sont généralement supérieurs au Mb/s. En contrepartie, l'utilisation de ces technologies engendre une consommation énergétique plus élevée.
  \item IoT bas débit : Ici se trouvent les réseaux mobiles IoT, LoRaWAN, Sigfox, et Zigbee. En fonction du réseau, le débit peut varier entre quelques bits par seconde jusqu'à un Mb/s. Ces technologies offrent une consommation énergétique faible.
  \item IoT critique : C'est-à-dire lorsqu'il s'avère nécessaire de transmettre des données sur un intervalle de temps donné \cite{orange_iot}. C'est une propriété des réseaux 5G.
\end{itemize}

~

\noindent
Un point également important est que ces technologies ont de différents niveaux de maturité. Au moins un réseau mobile est disponible dans chaque pays, mais les nouveaux réseaux comme Sigfox, LoRaWAN, et 5G sont en revanche toujours en cours de déploiement. De ce fait, ces dernières technologies ne sont disponibles que dans certaines régions possédant une infrastructure moderne. La couverture du réseau est un critère important à prendre en compte avant de faire un choix afin d'éviter les mauvaises surprises.
Finalement, toujours dues à ce différent degré de maturité, certaines technologies plus anciennes pourraient voir la fin de ses jours dans les années à venir. Par exemple, certains opérateurs vont décommissionner leur réseau 2G à partir de 2020 \cite{swisscom_2g}. Les opérateurs restants devraient suivre la même tendance dans les années suivantes. De plus, le réseau 3G devrait suivre la même tendance comme affichée sur la figure \ref{fig:evo_network}.

\begin{figure}[ht]
  \includegraphics[width=\textwidth]{img/state_of_the_art/network_evolution.png}
  \caption{Évolution du nombre de dispositifs connectés aux différents réseaux \cite{report_cisco}}
  \label{fig:evo_network}
\end{figure}



\section{Monitoring de l'état d'un serveur linux}

La nécessité de surveiller toute l’infrastructure IT, et en particulier des serveurs tournant sous Linux, existe depuis un bon nombre d’années. Cependant, surveiller un système n’est pas trivial. Cela peut être accompli de différentes manières en fonction des évènements ou statistiques qui doivent être observées dans la machine hôte. Par exemple, l’utilisation de logs pour débugger une application ou voir l’historique de transactions d’une base de données est une technique de monitoring utilisée depuis de nombreuses années. D’autres techniques de monitoring existantes sont le profiling, le tracing et l’acquisition de métriques \cite{brazil2018prometheus}. Dans cette section, seulement les technologies d’acquisition de métriques sont présentées puisqu’elles sont les plus utiles pour connaître l’état de la machine.

~

\noindent
Créé à la fin des années 80, le protocole SNMP \cite{RFC1098, RFC1157} permet de monitorer et configurer des dispositifs différents connectés au réseau. Tout au long des années, des solutions de monitorage de métriques de plus haut niveau, ou tout-en-un, sont apparues sur le marché. Ces solutions utilisent différents protocoles, dont SNMP, pour surveiller les dispositifs sur le réseau et stockent toutes les informations recueillies sur une base de données. De plus, elles permettent de définir un ensemble de conditions à surveiller de plus près et si une de ces conditions est violée, une alerte est transmise vers l’équipe responsable de l’infrastructure. Deux exemples de solutions créées à la fin des années 90 et qui demeurent disponibles sont Nagios et Zabbix.

~

\noindent
Actuellement, un grand nombre de solutions de monitoring sont disponibles sur le marché proposant différents modèles commerciaux. En effet, certaines solutions sont vendues comme un produit, mais d’autres préfèrent suivre un modèle open source. Dans le cas de ces dernières, les entreprises responsables de l’application proposent souvent leurs services en matière d’assistance technique afin d'installer et maintenir la solution. Un autre modèle aussi appliqué est de proposer une version gratuite avec un nombre limité de fonctionnalités, et une version payante débloquant toutes les capacités de l’application. Avant d’entamer toute recherche d’une solution, il faut comprendre quelle formule s’adapte le mieux aux besoins de l’utilisateur.
Comme il est affiché sur la figure \ref{fig:mon_archi}, deux architectures existent pour les solutions de monitoring : \textit{pull} et \textit{push} \cite{brazil2018prometheus, techhub_monitoring, blog_monitoring, techhub_monitoring}. Chaque application est entièrement basée sur une de ces deux approches. Dans certains cas, une solution peut parfois supporter les deux architectures (voir exemple figure \ref{fig:archi_prom_simple}). Cependant, ces solutions détiennent toujours un modèle préféré, et l’autre possibilité sert juste à combler certaines lacunes du premier modèle puisque, en effet, chaque architecture possède ses avantages et faiblesses.

~

\begin{figure}[ht!]
  \includegraphics[width=\textwidth]{img/state_of_the_art/monitoring_architecture.png}
  \caption{\textbf{Gauche :} Solution de monitoring avec architecture \textit{push}.  \textbf{Droite :} Solution de monitoring avec architecture \textit{pull}.}
  \label{fig:mon_archi}
\end{figure}



\begin{figure}[ht!]
  \centering
  \includegraphics[scale=0.4]{img/state_of_the_art/prometheus_archi_simple.png}
  \caption{Architecture \textit{pull} de Prometheus. Le modèle \textit{push} est supporté grâce au Pushgateway.}
  \label{fig:archi_prom_simple}
\end{figure}

~

\noindent
Dans le modèle de type \textit{push}, les agents dans la machine hôte envoient un ensemble de métriques ou des événements au serveur de monitoring. Cette technique permet notamment de définir des conditions d’alerte dans l’agent, et c’est celui-ci qui se charge de vérifier ces conditions. Lorsque le test des conditions passe, l’agent transmet un événement au serveur de monitoring. De plus, seul le système de push est capable de correctement acquérir des données à propos de jobs Batch de courte durée. \cite{blog_monitoring, prometheus_tuto} En effet, comme affiché sur la figure \ref{fig:batch_end}, le modèle \textit{pull} peut rater la fin du job ce qui se traduit par une perte d’informations.

~

\begin{figure}[ht!]
  \centering
  \includegraphics{img/state_of_the_art/pull_batch_miss.png}
  \caption{Faiblesse du modèle \textit{pull}. Le système de monitoring a raté les informations de la fin du job}
  \label{fig:batch_end}
\end{figure}

~

\noindent
Dans le modèle de type \textit{pull}, les agents dans la machine hôte collectent et exposent les métriques, mais c’est au serveur de monitoring d’activement aller retrouver ces données. Par conséquent, les événements et alertes au niveau des agents ne sont pas pris en charge, dans la mesure où ces derniers ne savent pas quand le système de monitoring récupérera les informations. En revanche, cette méthode permet de mieux identifier si un problème s'est produit sur la machine hôte puisqu’elle ne répondra plus aux requêtes lorsque cela a lieu.\cite{prometheus_doc_pull_push} Ceci rend l’approche \textit{pull} légèrement plus fiable que la technique \textit{push}. Les deux méthodes possèdent donc leurs avantages et inconvénients qu’il faut tenir en compte lors du choix de la technologie. Cependant, dans la majorité des cas, les deux architectures offrent ces capacités très similaires \cite{interview_push_pull}.

~

\noindent
Indépendamment de l’architecture utilisée, toutes les informations recueillies ou reçues par le serveur de monitoring doivent être stockées sur une base de données. Les solutions plus anciennes comme Nagios et Zabbix, utilisent principalement des bases de données relationnelles \cite{nagios_db, zabbix_db}. Ceci n’est plus le cas pour les solutions plus récentes telles que Prometheus et le TIG Stack\footnote{Le TIG stack est la combinaison de Telegraf, InfluxDB et Grafana pour créer un outil de monitoring} qui utilisent des bases de données orientées séries temporelles pour la persistance des données. Cette sorte de base de données, comme son nom indique, est mieux optimisée pour des valeurs horodatées \cite{time_series_fr}. En effet, ces bases de données possèdent les propriétés suivantes \cite{alibaba_timeseries}:

\begin{itemize}
  \item Elles supportent des écritures simultanées et avec de grands débits. Typiquement, dans ces bases de données se produisent énormément d’écritures, mais pas beaucoup d’accès.
  \item Ces bases de données doivent stocker d’énormes quantités de données temporelles. Elles utilisent des algorithmes de compression spécifiques pour stocker les données de manière efficace. \cite{di2007efficient}
  \item Toutes les requêtes sont effectuées sur base d’un intervalle dans le temps
  \item Les données ont une durée de rétention. À partir d’une certaine date, la base de données élimine automatiquement les informations car elles sont jugées non utiles.
\end{itemize}

~

\noindent
De ce fait, les bases de données orientées séries temporelles offrent un stockage plus efficace pour les métriques surveillées. Un dernier aspect très important d’une solution de monitoring est la visualisation de données. La majorité des solutions offrent soit un outil de visualisation propriétaire (Nagios ou Zabbix), soit elles exploitent des outils existants tels que Grafana (Prometheus et le TIG Stack).


\section{Plateformes de visualisation de données}

\textbf{Cette section n'est pas encore terminée.}

~

La visualisation de données consiste à représenter plusieurs formats de données sous un format visuel tel que les graphiques. Le système visuel humain est capable d’identifier des tendances ou des aberrations. Le but de la visualisation des données consiste alors dans l’aide de la compréhension des données en exploitant cette capacité. \cite{zoo_data}

~

\noindent
Les outils de visualisations de données sont alors des solutions software qui aident à accomplir cet objectif. \cite{bikakis2018big} Ces outils fournissent une plateforme très puissante pour communiquer de l’information. Cependant, le concept de données est trop abstrait. En effet, elles peuvent être des coordonnées spatiales, les mesures de température d’une pièce, le nombre d’occurrences d’un évènement, le classement d’une course automobile, etc. Pour chacun de ces différents types de données, la méthode de visualisation à utiliser est différente. Par exemple, une carte serait utilisée pour afficher les coordonnées, mais une courbe serait préférée pour afficher l’évolution de la température. L’article \cite{zoo_data} propose différentes méthodes et graphiques pour observer les différents types de données existants.

~

\noindent
Comme présenté dans [source], la visualisation de données doit respecter les trois principes suivants :
Thrustworthy, accessible, and elegant

~

Avec l’arrivée de l’ère du Big Data, les technologies de visualisation sont devenues d’autant plus importantes. En effet, la quantité de données est devenue tellement importante au point qu’un humain n’est plus capable d’analyser toutes les données. Ceci coïncide avec l’apparition de nouvelles technologies capables de mieux guider l’utilisateur à trouver des tendances dans les données. Actuellement, trois grandes catégories d’outils de visualisations de données sont disponibles :

\begin{itemize}
  \item Les librairies software, telles que Matplotlib ou D3.js, qui aident les utilisateurs à créer de différentes visualisations de données. Ces outils demandent des compétences en programmation.

  \item Ensuite, il y a les outils purement pour la visualisation de données tels que Grafana ou Google Data Studio. À partir d’une interface web, ces outils permettent de facilement créer des graphiques en utilisant seulement des langages de requête comme SQL. Certains de ces outils sont capables d’auxilier l’utilisateur à formuler des requêtes. (Query Builders)

  \item Finalement, il y a les outils de Business Intelligence. Ils ont les mêmes fonctionnalités qu’un outil de visualisation de données. Cependant, ils permettent en plus de cela de faire un traitement des données et des prédictions sur celles-ci. Un exemple d’un tel outil est Tableau ou Power BI.
\end{itemize}


\chapter{Cahier des charges}
\label{chap:3}

\noindent
Le but de ce mémoire consiste à développer un prototype capable de surveiller l'infrastructure de CERHIS et s'assurer du bon fonctionnement de celle-ci. De plus, ce système de monitoring doit pouvoir avertir un technicien lorsqu'un problème est détecté. Bien évidemment, pour accomplir cet objectif, le système devra acquérir des données sur le fonctionnement des différents dispositifs de CERHIS. Ces données devront être sauvegardées localement, mais aussi sur un serveur distant. Une interface web devra également être présente pour faciliter la visualisation de ces données, que cela soit dans le périmètre du centre hospitalier ou partout dans le monde. La figure \ref{fig:mon_archi_simple} reprend l'architecture simplifiée de ce système de monitoring.

~

\noindent
Pour simplifier la division du travail, la création du dispositif a été scindée en deux parties.
Une première partie portant principalement sur le matériel, y compris le choix et l'assemblage de tous les composants nécessaires au développement du prototype. Cela englobe l'ordinateur requis pour tourner le logiciel, l'alimentation, la solution de communication et le logiciel qui permet aux différents composants de s'interfacer. Ensuite, une deuxième partie comprenant juste le logiciel nécessaire pour effectuer les tâches de monitoring. Cela comprend les fonctionnalités requises pour l'acquisition, traitement, stockage et visualisation des données. La présentation des spécificités de chaque partie se fera dans les sections suivantes (sections \ref{sec:cahier_proto} et \ref{sec:cahier_soft}).



\begin{figure}[ht!]
  \centering
  \includegraphics[width=\textwidth]{img/cahier_des_charges/baseline_archi.png}
  \caption{Architecture simplifiée de la solution de monitoring}
  \label{fig:mon_archi_simple}
\end{figure}

\section{Caractéristiques matérielles du dispositif}
\label{sec:cahier_proto}

Étant donné que le dispositif de monitoring sera utilisé dans des centres hospitaliers en Afrique centrale, il doit être capable de résister à certaines caractéristiques propres du milieu telles que la température élevée. Voici la liste exhaustive de toutes les caractéristiques matérielles que le produit doit posséder :

~


\begin{itemize}
  \item Le prototype doit être robuste. Mes composants électroniques doivent notamment être capables de supporter des températures ambiantes supérieures à \SI{30}{\celsius} pendant de longues périodes. \cite{temperature_kinshasa}

  \item Le dispositif doit être capable de communiquer à distance, même lorsqu'une connexion à internet filaire n'est pas disponible. De ce fait, le prototype doit être capable de tirer parti d'une des technologies présentées dans la section 2 pour pouvoir envoyer des informations et des alertes.

  \item Le dispositif doit être d'apparence discrète, évitant d'attirer l'attention sur sa valeur.

  \item Le coût doit rester abordable et réaliste.

  \item Le dispositif doit posséder un système d'alimentation mixte capable d'alimenter le prototype pendant un ou deux jours en cas de passe du service électrique.

  \item Le boîtier doit être fermé hermétiquement pour empêcher des insectes d'y faire son nid.
\end{itemize}



\section{Fonctionnalités à implémenter}
\label{sec:cahier_soft}

Pour la section logiciel du dispositif, les fonctionnalités à implémenter peuvent encore être divisées en deux sous-sections : la solution de monitoring tournant sur le site, et le serveur distant capable de recevoir, stocker et afficher des données. Les fonctionnalités présentées ci-dessous seront alors divisées en suivant ces deux sous-sections.

~

\begin{itemize}
  \item \textbf{Solution de monitoring dans le centre hospitalier}
  \begin{itemize}
    \item Effectuer un mappage du réseau local de façon à trouver tous les appareils qui y sont connectés.

    \item Monitorer la présence des appareils connectés au réseau local. Si un appareil se déconnecte du réseau, pouvoir indiquer combien de temps est passé depuis sa dernière connexion.

    \item Monitorer l'état de fonctionnement du serveur de CERHIS.

    \item Envoi d'alertes par SMS.

    \item Envoi de données vers un serveur distant.

    \item Interface utilisateur permettant de facilement visualiser toutes les données.

    \item Vérifier le niveau de la batterie de CERHIS et l'état de charge des tablettes.

    \item Créer un historique de connexion des tablettes au serveur de CERHIS
  \end{itemize}

~

  \item \textbf{Serveur distant}
  \begin{itemize}
    \item Serveur capable de stocker les données envoyées depuis plusieurs centres hospitaliers.

    \item Interface utilisateur permettant de facilement visualiser toutes les données disponibles sur le serveur distant.
  \end{itemize}
\end{itemize}



\section{Priorité des tâches et méthodologie de travail}

\noindent
Les listes des fonctionnalités et caractéristiques présentes ci-dessus reprennent tout ce qui serait nécessaire pour avoir un produit fini. L'objectif de ce mémoire est bien évidemment d'essayer d'implémenter le maximum de fonctionnalités possibles. Cependant, le temps est limité et des imprévus peuvent survenir au long du chemin. De ce fait, les diverses fonctionnalités ont été classées par ordre de priorité afin de garantir que le prototype rendu à la fin de ce mémoire possède au moins un certain nombre de fonctionnalités.

~

\noindent
En premier lieu, c'était la sélection du matériel requis pour implémenter les fonctionnalités. Cela reprend, le choix de la plateforme, l'alimentation et la solution de communication. Cette tâche reprend aussi des tests sur la plateforme du projet de l'année précédente afin de vérifier si des changements étaient nécessaires.

~

\noindent
Deuxièmement, c'était l'implémentation de tout le code nécessaire pour lier les divers composants hardware. En particulier, l'implémentation du code essentiel pour mettre en marche la solution de communication. Ici s'ajoutent aussi les premiers essais afin de prouver le bon fonctionnement de la plateforme.

~

\noindent
Une fois la plateforme préliminaire choisie et testée, la possibilité de commencer à développer les différentes fonctionnalités du logiciel s'est ouverte. À nouveau, en raison du grand nombre de fonctionnalités requises, les plus cruciales ont été sélectionnées lors de réunions avec les parties prenantes. La liste ci-dessous présente les fonctionnalités triées par ordre décroissant de priorité :

~


\begin{itemize}
  \item Monitorer l'état du serveur, en particulier surveiller l'uptime de celui-ci et avertir un technicien en cas de modification anormale de la valeur.
  \item Mapper le réseau local.
  \item Surveiller la présence des dispositifs de CERHIS.
  \item Monitorer l'état de certains processus dans le serveur.
  \item Envoi des données vers un serveur distant.
  \item Interface web pour la visualisation des données
\end{itemize}

~

\noindent
Finalement, les dernières tâches reprennent la finalisation d'un boîtier permettant de loger les différents composants électroniques et des tests approfondis pour s'assurer du bon fonctionnement et la fiabilité de la solution de monitoring.

% ~
%
% \noindent
% Pour essayer d'être le plus efficace possible, la méthodologie de travail consistait à écrire toutes les tâches liées à ce projet sur des notes Post-it. Les Post-its orange représentaient les missions liées au software, et les jaunes correspondaient aux missions liées au hardware. Chaque note possédait aussi un numéro coïncidant avec la priorité de la tâche. Le critère de sélection était en premier lieu la couleur, orange prioritaire sur le jaune, et ensuite la priorité (lorsque les Post-its avaient tous la même couleur).


\chapter{Prototype}

\section{Le premier prototype (2018-2019)}

\noindent
Dans le cadre du projet d’année de la première année du Master ingénieur civil en informatique, un prototype capable de surveiller l’infrastructure de CERHIS a été réalisé. Ce dispositif s’inspirait sur d’autres prototypes plus anciens créés par des étudiants de l’école polytechnique de Bruxelles. L’architecture du dispositif produit l’année dernière est présentée sur la figure \ref{fig:archi_prev}.

~

\begin{figure}[ht!]
  \includegraphics[width=\textwidth]{img/el_prototype/archi_prev.png}
  \caption{Architecture de la solution de monitoring développée lors de l'année académique 2018-2019}
  \label{fig:archi_prev}
\end{figure}



\noindent
Un Raspberry Pi 2B était utilisé pour tourner le logiciel de monitoring ainsi que l’interface web permettant la configuration de celui-ci. Un circuit imprimé (PCB) avait été réalisé dans le but de loger des composants électroniques différents tels que le module de communication GSM SIM800L et le convertisseur analogique numérique MCP3008. Ce même circuit imprimé s’insérait sur les ports GPIO du Raspberry Pi afin que ce dernier puisse interfacer avec les différents composants présents sur le PCB. Grâce au SIM800L, le réseau GSM d’un opérateur mobile local pouvait être utilisé pour envoyer des SMS vers un technicien, ou des informations quelques conques vers un serveur distant.  Dans ce premier essai, ce serveur était juste capable de recevoir les données envoyées par le prototype et de les stocker sur un fichier \textit{.txt}. Aucun traitement ou visualisation des données n’était offert aux techniciens. De plus, le monitoring des tablettes n'a pas pu être implémenté par manque de temps.

~

\noindent
De façon à alimenter le dispositif, un système d’alimentation mixte avait été mis en place. Comme le montre la figure \ref{fig:power_source}, ce système était basé sur une batterie acide-plomb de 12V, un régulateur de charge permettant de recharger la batterie et un convertisseur step-down transformant la tension de 12 à 5 volts.

~

\begin{figure}[ht!]
  \includegraphics[width=\textwidth]{img/el_prototype/power_source.png}
  \caption{Système d'alimentation de la solution de monitoring développée lors de l'année académique 2018-2019}
  \label{fig:power_source}
\end{figure}

~

\noindent
Ce prototype a été testé pendant plusieurs semaines\footnote{entre le 14 mars et le 10 avril 2019} à Goma par un étudiant de l’ISIG\footnote{Institut supérieur d'informatique et de gestion de Goma}. Lors de cette période de tests, plusieurs problèmes ont été identifiés. Ils seront présentés dans la section suivante.

\subsection{Problèmes détectés}

Plusieurs protocoles de tests avaient été mis en place l’année dernière dans le but d'évaluer ce premier dispositif. Les soucis qui ont été trouvés lors de ces tests peuvent être classés selon trois grandes catégories : communication, alimentation et température. Tous les problèmes sont documentés ci-dessous.

\begin{itemize}
  \item \textbf{Communication :}
  \begin{itemize}
    \item Corruption de certaines données échangées entre le SIM800L et le Raspberry Pi. Des bits semblaient être modifiés lors de la transmission des messages entre les deux composants. Cela provoquait une perte d’informations. De plus, cela rendait plus difficile l'utilisation du SIM800L, car ce dernier n’était pas capable de traiter toutes les commandes qui lui étaient envoyées.

    \item Implémenter toutes les fonctionnalités de communication requises avec le SIM800L était très laborieux. En effet, ce module peut être vu comme un automate fini déterministe. Des commandes AT sont utilisées pour contrôler le module, c’est-à-dire qu'elles sont employées pour changer l’état de celui-ci. Une succession correcte de multiples commandes est nécessaire afin d'accomplir une simple tâche telle que l'envoi d'un SMS. Ceci rendait le procès d’implémentation assez long, en particulier l’implémentation de la gestion des erreurs. Le problème mentionné ci-dessus a encore plus aggravé la complexité de la manipulation de ce module.
  \end{itemize}

  \item \textbf{Alimentation :}
  \begin{itemize}
    \item Le régulateur de charge initialement utilisé est tombé en panne lors de la réalisation des tests. Un simple remplacement de cette pièce semble avoir résolu le problème.

    \item Le transformateur step-down est tombé en panne lors des tests. Malheureusement, aucun remplacement pour cette pièce a été trouvé à Goma. Ce souci est possiblement lié au problème de température qui sera présenté ci-dessous. Cependant, n’ayant pas assez d’information pour prouver l‘existence d'un lien de causalité, chaque problème doit être considéré séparément.
  \end{itemize}

  \item \textbf{Température :}
  \begin{itemize}
    \item D’après les personnes en charge de tester le prototype, celui-ci semblait atteindre de hautes températures. Des températures élevées peuvent avoir un impact considérable sur la durée de vie des différents composants électroniques.
  \end{itemize}
\end{itemize}

% \subsubsection{Influence sur le nouveau prototype}

\section{Nouveau Prototype}

\subsection{Solutions considérées}
\subsection{Thingstream}
\subsection{Résultat}
\subsection{Tests réalisés}

\chapter{L'application}
\label{sec:archi}
%\subsection{Introduction}


\section{Client}

\subsection{Architecture}
\label{sec:c_arch}
\subsection{Base de données}

\subsection{Implémentation des fonctionnalités}

\subsubsection{Mappage du réseau local}


\subsubsection{Surveiller la présence des divers composants}

\subsubsection{Monitoring du serveur}

\subsubsection{Envoie des données}



\subsection{Sécurité}
\newpage
\section{Serveur}

\subsection{Architecture}
\subsection{Base de données}
\subsection{Implémentation des fonctionnalités}
\subsection{Sécurité}

\chapter{Tests et résultats}


\chapter{Conclusion}





\appendix

\chapter{Tableau comparatif des réseaux}
\label{ap:table_network}

\begin{landscape}


  \begin{table}[]
\resizebox{25cm}{!}{\begin{tabular}{l|l|l|l|l|l|l|l|l|}
\cline{2-9}
                                                                                                   & \multicolumn{1}{c|}{\textbf{LoRaWAN}} & \multicolumn{1}{c|}{\textbf{\begin{tabular}[c]{@{}c@{}}Réseaux mobiles \\ (2G/3G/4G/5G)\end{tabular}}}                                                                                         & \multicolumn{1}{c|}{\textbf{Sigfox}}                                                                        & \multicolumn{1}{c|}{\textbf{Satellite}}                                                                                             & \multicolumn{1}{c|}{\textbf{Wi-Fi}}                                       & \multicolumn{1}{c|}{\textbf{Zigbee}}                        & \multicolumn{1}{c|}{\textbf{Bluetooth}}                       & \multicolumn{1}{c|}{\textbf{\begin{tabular}[c]{@{}c@{}}Réseaux mobiles IoT\\ (NB-IoT, LTE-M)\end{tabular}}} \\ \hline
\multicolumn{1}{|l|}{\textbf{Débit}}                                                               & 0.3 - 50kbps                          & \begin{tabular}[c]{@{}l@{}}EDGE: 384 Kbps\\ HSPA+: 42 Mbps\\ LTE-A: 1 Gbps\\ 5G: 10 Gbps\end{tabular}                                                                                          & 100 ou 600 bps                                                                                              & \begin{tabular}[c]{@{}l@{}}Quelques kbps jusqu'à \\ plusieurs Gbps. Cela \\ dépend de la fréquence \\ de transmission.\end{tabular} & Wi-Fi 6: 9.6 Gbps                                                         & \multicolumn{1}{c|}{250 Kbps}                               & \multicolumn{1}{c|}{5.0 : 5 Mbps}                             & \begin{tabular}[c]{@{}l@{}}NB-IoT: 200 Kbps\\ LTE-M: 1 Mbps\end{tabular}                                    \\ \hline
\multicolumn{1}{|l|}{\textbf{\begin{tabular}[c]{@{}l@{}}Bande de \\ fréquences\end{tabular}}}      & 868 - 915 MHz                         & \begin{tabular}[c]{@{}l@{}}400 - 3000 Mhz + \\ 25 - 39 Ghz (5G)\end{tabular}                                                                                                                   & 862 - 928 Mhz                                                                                               & 1 - 40 GHz                                                                                                                          & 2.4 GHz / 5 GHz                                                           & \begin{tabular}[c]{@{}l@{}}868/915/\\ 2400 MHz\end{tabular} & 2400 Mhz                                                      & 800 - 2200 MHz                                                                                              \\ \hline
\multicolumn{1}{|l|}{\textbf{\begin{tabular}[c]{@{}l@{}}Canal de \\ communication\end{tabular}}}   & Half-duplex                           & Full-duplex                                                                                                                                                                                    & Half-duplex                                                                                                 & Full-duplex                                                                                                                         & Half-duplex                                                               & Full-duplex                                                 & Full-duplex                                                   & Half-duplex                                                                                                 \\ \hline
\multicolumn{1}{|l|}{\textbf{\begin{tabular}[c]{@{}l@{}}Consommation \\ énergétique\end{tabular}}} & Très Basse                            & Haute                                                                                                                                                                                          & Très Basse                                                                                                  & Haute                                                                                                                               & Moyenne                                                                   & Très Basse                                                  & Très Basse                                                    & Basse                                                                                                       \\ \hline
\multicolumn{1}{|l|}{\textbf{Portée}}                                                              & 5 - 20 km                             & 2 km - 35 km                                                                                                                                                                                   & 10 - 40 km                                                                                                  & Mondiale                                                                                                                            & 15 - 300 m                                                                & 30 - 100 m                                                  & 3 - 30 m                                                      & \begin{tabular}[c]{@{}l@{}}NB-IoT : 1 - 15 km\\ LTE-M : 1 - 11 km\end{tabular}                              \\ \hline
\multicolumn{1}{|l|}{\textbf{Couverture}}                                                          &                                       & Mondiale                                                                                                                                                                                       & \begin{tabular}[c]{@{}l@{}}Europe de l'ouest \\ principalement\end{tabular}                                 & Mondiale                                                                                                                            & Réseau privé                                                              & Réseau privé                                                & Réseau privé                                                  &                                                                                                             \\ \hline
\multicolumn{1}{|l|}{\textbf{Sécurisé}}                                                            & \multicolumn{1}{c|}{Oui (AES 128)}    & \begin{tabular}[c]{@{}l@{}}Exploits possibles\\ dans le réseau GSM.\\ \\ Nouvelles versions\\ sont plus sécurisées.\\ (Possible aussi \\ d'ajouter dans la \\ couche application)\end{tabular} & \multicolumn{1}{c|}{\begin{tabular}[c]{@{}c@{}}Non (Possible\\ dans la couche \\ application)\end{tabular}} & \begin{tabular}[c]{@{}l@{}}Dépend du standard\\ utilisé\end{tabular}                                                                & \begin{tabular}[c]{@{}l@{}}Oui, mais des\\ exploits existent\end{tabular} & Oui (AES 128)                                               & \begin{tabular}[c]{@{}l@{}}Oui \\ (propriétaire)\end{tabular} & Oui (propriétaire)                                                                                          \\ \hline
\end{tabular}}
\end{table}

\end{landscape}



\bibliographystyle{plain}
\bibliography{src/bibliography}
%\chapter{Test}

\end{document}
