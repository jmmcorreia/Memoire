\chapter{Prototype}
\label{chap:4}

\section{Le premier prototype (2018-2019)}
\label{sec:old_boy}

\noindent
Dans le cadre du projet d'année de la première année du Master ingénieur civil en informatique, un prototype capable de surveiller l'infrastructure de CERHIS a été réalisé. Ce dispositif s'inspirait sur d'autres prototypes plus anciens créés par des étudiants de l'École polytechnique de Bruxelles. L'architecture du dispositif produit l'année dernière est présentée sur la figure \ref{fig:archi_prev}.

~

\begin{figure}[ht!]
  \includegraphics[width=\textwidth]{img/el_prototype/archi_prev.png}
  \caption{Architecture de la solution de monitoring développée lors de l'année académique 2018-2019}
  \label{fig:archi_prev}
\end{figure}



\noindent
Un Raspberry Pi 2B était utilisé pour tourner le logiciel de monitoring ainsi que l'interface web permettant la configuration de celui-ci. Un circuit imprimé (PCB) avait été réalisé dans le but de loger des composants électroniques différents tels que le module de communication GSM SIM800L et le convertisseur analogique numérique MCP3008. Ce même circuit imprimé s'insérait sur les ports GPIO du Raspberry Pi afin que ce dernier puisse interfacer avec les différents composants présents sur le PCB. Grâce au SIM800L, le réseau GSM d'un opérateur mobile local pouvait être utilisé pour envoyer des SMS vers un technicien, ou des informations quelques conques vers un serveur distant. Le MCP3008 était connecté à la batterie de CERHIS dans le but de monitorer la tension électrique de celle-ci. Dans ce premier essai, le serveur distant était juste capable de recevoir les données envoyées par le prototype et de les stocker sur un fichier \textit{.txt}. Aucun traitement ou visualisation des données n'était offert aux techniciens. De plus, le monitoring des tablettes n'a pas pu être implémenté par manque de temps.

~

\noindent
De façon à alimenter le dispositif, un système d'alimentation mixte avait été mis en place. Comme le montre la figure \ref{fig:power_source}, ce système était basé sur une batterie acide-plomb de 12V, un régulateur de charge permettant de recharger la batterie et un convertisseur step-down transformant la tension de 12 à 5 volts.

~

\begin{figure}[ht!]
  \includegraphics[width=\textwidth]{img/el_prototype/power_source.png}
  \caption{Système d'alimentation de la solution de monitoring développée lors de l'année académique 2018-2019}
  \label{fig:power_source}
\end{figure}

~

\noindent
Ce prototype a été testé pendant plusieurs semaines\footnote{entre le 14 mars et le 10 avril 2019} à Goma par un étudiant de l'ISIG\footnote{Institut supérieur d'informatique et de gestion de Goma}. Lors de cette période de tests, plusieurs problèmes ont été identifiés. Ils seront présentés dans la section suivante.

\subsection{Problèmes détectés}

\noindent
Plusieurs protocoles de tests avaient été mis en place l'année dernière dans le but d'évaluer ce premier dispositif. Les soucis qui ont été trouvés lors de ces tests peuvent être classés selon trois grandes catégories : communication, alimentation et température. Tous les problèmes sont documentés ci-dessous.

~

\begin{itemize}
  \item \textbf{Communication :}
  \begin{itemize}
    \item Corruption de certaines données échangées entre le SIM800L et le Raspberry Pi. Des bits semblaient être modifiés lors de la transmission des messages entre les deux composants. Cela provoquait une perte d'informations. De plus, cela rendait plus difficile l'utilisation du SIM800L, car ce dernier n'était pas capable de traiter toutes les commandes qui lui étaient envoyées.

    \item Implémenter toutes les fonctionnalités de communication requises avec le SIM800L était très laborieux. En effet, ce module peut être vu comme un automate fini déterministe. Des commandes AT sont utilisées pour contrôler le module, c'est-à-dire qu'elles sont employées pour changer l'état de celui-ci. Une succession correcte de multiples commandes est nécessaire afin d'accomplir une simple tâche telle que l'envoi d'un SMS. Ceci rendait le procès d'implémentation assez long, en particulier l'implémentation de la gestion des erreurs. Le problème mentionné ci-dessus a encore plus aggravé la complexité de la manipulation de ce module.
  \end{itemize}

~

  \item \textbf{Alimentation :}
  \begin{itemize}
    \item Le régulateur de charge initialement utilisé est tombé en panne lors de la réalisation des tests. Un simple remplacement de cette pièce semble avoir résolu le problème.

    \item Le step-down est tombé en panne lors des tests. Malheureusement, aucun remplacement pour cette pièce a été trouvé à Goma. Ce souci est possiblement lié au problème de température qui sera présenté ci-dessous. Cependant, n'ayant pas assez d'information pour prouver l'existence d'un lien de causalité, chaque problème doit être considéré séparément.
  \end{itemize}

~

  \item \textbf{Température :}
  \begin{itemize}
    \item D'après les personnes en charge de tester le prototype, celui-ci semblait atteindre de hautes températures. Des températures élevées peuvent avoir un impact considérable sur la durée de vie des différents composants électroniques.
  \end{itemize}
\end{itemize}

% \subsubsection{Influence sur le nouveau prototype}

\section{Nouveau Prototype}

\noindent
Compte tenu des problèmes rencontrés l'année précédente, un nouveau prototype a dû être conçu. Toutefois, énormément d'informations ont été acquises au cours de l'année dernière. Il ne serait pas judicieux de partir de zéro, car cela demanderait un effort assez important pour atteindre le niveau précédent. Ce nouveau prototype correspond alors à une évolution du résultat de l'année antérieure, ayant comme but de combler toutes les lacunes de ce dernier. Le choix de remplacer chaque composant est abordé ci-dessous.

\subsection{Solutions considérées}
\label{sec:sol_hard}

\subsubsection{Choix de la plateforme}

\noindent
Lors de l'année précédente, un Raspberry Pi 2B a été utilisé en tant que plateforme principale du projet. Ce modèle en particulier avait été choisi en raison de sa consommation énergétique plus faible que celle des modèles suivants, tout en offrant des performances amplement suffisantes. Depuis le projet antérieur, une nouvelle version est disponible sur le marché, le Raspberry Pi 4. Cependant, ce dernier présente une consommation énergétique encore plus élevée sans offrir de nouveaux avantages au projet. De ce fait, puisque le Raspberry Pi 2B s'est montré très fiable et qu'il restera en production jusqu'en 2026 \cite{rasp_prod}, il sera à nouveau utilisé dans ce nouveau prototype. Dans l'annexe \ref{ap:choix}, le tableau comparatif de toutes les solutions considérées l'année dernière est présent à titre de référence.


\subsubsection{Solution de communication}

\noindent
La communication à distance est probablement un des éléments les plus complexes de ce projet. En effet, l'infrastructure des différents réseaux en Afrique Centrale est moins développée comparée à l'infrastructure en Europe. De plus, la partie de la communication est celle qui a apporté le plus de problèmes l'année précédente, et celle dont les problèmes sont les plus difficiles à résoudre. De ce fait, cette partie doit être abordée avec caution et la décision doit être très pondérée.

~

\noindent
Actuellement, le choix de réseaux accessibles en Afrique centrale est très restreint. En particulier, les réseaux orientés IoT (LTE-M, NB-IoT, Sigfox, LoRaWAN) ne sont actuellement pas disponibles dans la majorité des pays en Afrique Centrale ou leur couverture est extrêmement limitée.
De plus, les réseaux Sigfox et LoRaWAN\footnote{Les contraintes sur la quantité de données pour les réseaux LoRaWAN ne sont applicables que sur les réseaux publics.} sont très contraignants sur la quantité de données qui peut être envoyée sur une journée. De ce fait, ils ne s'encadrent pas dans le cahier des charges de ce projet. Le satellite aurait été le choix incontestable si le coût n'était pas un facteur limitant. Les options restantes telles que le Wi-Fi et le Bluetooth ont une portée très courte.  Les réseaux mobiles se présentent donc comme étant le meilleur compromis entre disponibilité, coût et consommation énergétique. L'accès à ces derniers est relativement facile, et la couverture des réseaux 2G et 3G semble être relativement bonne dans les régions plus densement peuplées. \cite{coverage_airtel, coverage_orange}


~

\noindent
Toutefois, le développement des réseaux mobiles orientés IoT (NB-IoT et LTE-M) dans ces pays doit être suivi de près au cours des prochaines années. Ils s'encadrent dans l'esprit de ce projet et offrent potentiellement un meilleur compromis que les réseaux mobiles classiques en raison de leur faible consommation énergétique.  Un autre aspect important à surveiller est la suppression des réseaux 2G au cours des années qui suivent. Cependant, vu le retard de l'infrastructure dans certains pays d'Afrique Centrale, il est très peu probable que cela ait lieu prochainement.

~

\noindent
Les réseaux mobiles proposent plusieurs protocoles pour transmettre des données. Afin de choisir le protocole le plus approprié pour ce projet, il est d'abord nécessaire de comprendre exactement quels types de communication sont requis. En premier lieu, il faut considérer la communication de machine à personne. Le prototype doit être capable d'envoyer un message d'alerte à un technicien. Le cahier des charges impose que ce type de communication soit effectué par SMS. Les techniciens ayant tous accès à un téléphone, cette méthode semble être la plus efficace pour communiquer avec eux.

~

\noindent
Ensuite, il y a la communication de machine à machine (M2M). Le prototype doit être capable d'envoyer les données recueillies vers un serveur distant qui les rendra accessibles aux techniciens. Cet échange de données peut être effectué également par SMS. Cependant, comme démontré dans le rapport de l'année dernière, le SMS est une méthode très coûteuse pour l'envoi de données M2M. D'autres protocoles permettent de transporter des données à un moindre coût comme l'USSD et le GPRS/HSPA/LTE. De ce fait, une comparaison a été effectuée afin de comprendre les avantages de chaque protocole, ainsi que celui qui est le plus approprié pour ce prototype.


~

\noindent
L'USSD, ou Unstructured Supplementary Service Data, est un protocole de communication présent dans les réseaux mobiles (GSM, 3G, 4G) qui permet à un client de communiquer avec les serveurs de l'opérateur du réseau mobile. L'USSD est un stateful protocol et les échanges de données entre le client et le serveur se font en temps réel.  Les opérateurs utilisent très souvent ce protocole pour donner accès à certains services spéciaux, tels que vérifier le solde restant dans une carte prépayée ($\*XXX\#$). L'USSD présente la caractéristique intéressante que toutes les connexions initiées par le client sont toujours routées vers les serveurs de l'opérateur de la carte SIM, même lorsque le client se trouve en itinérance \cite{lakshmi2017ussd}. Cela garantit que, indépendamment du lieu où se trouve le client dans le monde, l'échange d'informations se déroule toujours de manière similaire. De plus, cela permet aussi de diminuer les coûts élevés souvent associés à l'itinérance. Mais le protocole USSD possède bien d'autres avantages tels que :

~

\begin{itemize}
  \item Aucune configuration supplémentaire n'est requise pour utiliser l'USSD dans le réseau d'un opérateur. Au contraire, l'Access Point Name doit être configuré avant d'avoir accès au service GPRS.

  \item Le protocole USSD utilise moins de ressources d'un modem ou du réseau d'un opérateur que le GPRS. \cite{global_ussd}

  \item Le protocole USSD reste disponible même lorsque la connectivité à Internet du réseau est coupée. En contrepartie, cela rendrait le service GPRS inexploitable.\cite{transport_ussd}

  \item La majorité des modems sur le marché, dont le SIM800L, sont compatibles avec USSD. \cite{lakshmi2017ussd}
\end{itemize}

~

\noindent
Cependant, le protocole USSD possède aussi les désavantages suivants :

~

\begin{itemize}
  \item Déployer une application utilisant USSD est très difficile puisque cela demande de travailler de très près avec un opérateur de télécommunications. En effet, toutes les connexions sont dirigées vers les serveurs de ces derniers. \cite{perrier2015ussd}

  \item Lié au problème précédent, le protocole USSD ne donne pas directement accès à Internet. De ce fait, il n'est par exemple pas trivial d'utiliser un protocole application tel que HTTP pour envoyer des données vers un serveur.
\end{itemize}

~

\noindent
Pour pallier ce dernier problème, une solution assez particulière est exploitée actuellement dans certains pays en Afrique \cite{africa_ussd}. Elle permet aux utilisateurs d'accéder à des services qui ne sont disponibles que sur Internet, mais à travers le protocole USSD. La figure \ref{fig:ussdex} illustre comment fonctionne cette solution. Le serveur USSD joue le rôle de proxy et convertit les messages USSD initialement envoyés par le client en requêtes HTTP à envoyer à un serveur distant. Malheureusement, cette solution n'est disponible que dans un nombre restreint de pays, mais une solution similaire pourrait être éventuellement utilisée pour envoyer les données au serveur distant.

\begin{figure}[ht!]
  \centering
  \includegraphics[width=0.90\textwidth]{img/el_prototype/ussd_examples.png}
  \caption{Architecture simplifiée d'une solution USSD permettant d'accéder à des services sur Internet}
  \label{fig:ussdex}
\end{figure}

~

\noindent
L'alternative à l'USSD est de recourir aux normes General Packet Radio Service (GPRS), High Speed Packet Access (HSPA) ou Long Term Evolution (LTE) pour transmettre des données. Ces normes permettent de communiquer des données à travers le réseau de l'opérateur et fournissent au client un accès à Internet. Les protocoles de la couche application tels que HTTP sont très souvent utilisés en conjonction avec ces normes. Un exemple est l'emploi d'un smartphone connecté à Internet par le biais de données mobiles pour accéder à un site web. Ce niveau d'abstraction supplémentaire facilite l'utilisation de cette solution.

~

\noindent
Trancher entre l'USSD et le GPRS n'est pas simple. D'une part, l'USSD semble offrir une solution très fiable qui est couramment employée dans plusieurs pays africains. De l'autre côté, le GPRS permettrait d'exploiter les avantages d'une connexion standard à Internet et tous les protocoles d'application qui viennent avec. De ce fait, une comparaison entre les possibles de chaque solution a été effectuée.


\newpage

\textbf{Solution 1: Routeur LTE (GPRS/HSPA/LTE)}

\vspace{-0.2cm}
~

\noindent
Dans cette première implémentation, l'idée était de remplacer le module SIM800L par un routeur LTE. Ce type de routeur est capable de se connecter aux réseaux mobiles afin de permettre aux clients d'accéder à Internet. Toute la partie bas niveau des réseaux serait alors traitée par le Raspberry Pi et le routeur. De ce fait, le logiciel de monitoring n'aurait besoin que d'utiliser des protocoles de la couche application, simplifiant donc l'implémentation du logiciel. Certains de ces routeurs peuvent aussi être raccordés à un deuxième réseau WAN\footnote{Wide Area Network ou réseau étendu en français}, comme une potentielle connexion Internet du centre hospitalier. Cette solution est la seule à proposer ce type de redondance. Cependant, les routeurs 4G sont assez coûteux, principalement ceux destinés à des environnements industriels. Un investissement d'au moins 70 ou 80 euros serait requis pour acquérir tel dispositif. De plus, leur consommation énergétique est assez élevée comparée au SIM800L puisque les routeurs proposent un grand nombre de fonctionnalités telles que le Wi-Fi et plusieurs ports Ethernet. La figure \ref{fig:so1} illustre l'architecture simplifiée de cette solution.

\begin{figure}[ht!]
  \centering
  \includegraphics[width=0.93\textwidth]{img/el_prototype/solution1_com.png}
  \caption{Architecture simplifiée de la solution de communication avec un router LTE}
  \label{fig:so1}
\end{figure}

~


\textbf{Solution 2: Service USSD avec SIM800L}

\vspace{-0.2cm}
~

\noindent
Jusqu'à il y a quelques années, déployer une application USSD était assez complexe puisque l'aide d'un opérateur était nécessaire \cite{perrier2015ussd}. Toutefois, sont apparues récemment sur le marché des entreprises qui offrent des services de communication exploitant la technologie USSD. En fonction de l'entreprise en question, la couverture du service ainsi que ses fonctionnalités diffèrent. Utiliser une telle solution permet de très aisément exploiter les avantages du protocole USSD tout en mitigeant ses défauts. Thingstream est un service de communication qui exploite la technologie USSD pour permettre des dispositifs de communiquer entre eux presque partout dans le monde. Cette plateforme se distingue par le fait qu'elle propose le protocole MQTT sur le protocole USSD. De ce fait, comme pour la solution précédente, le logiciel de monitoring a accès à un protocole de la couche application ce qui facilite l'échange de données, mais son choix est imposé. La solution de Thinsgtream sera présentée plus en détail dans la section \ref{sec:thingstream}.

~

\textbf{Solution 3: SIM800L et protocole point à point}

\vspace{-0.2cm}
~

\noindent
Dans cette implémentation, le SIM800L est à nouveau exploité. En effet, l'architecture est très similaire à celle du prototype de l'année dernière, la différence réside dans l'emploi d'un protocole de plus haut niveau pour communiquer avec le modem. L'année précédente, les commandes AT étaient envoyées par le logiciel de monitoring au SIM800L à travers le protocole SERIAL. Pour simplifier ce processus, il est possible d'utiliser le protocole point à point (PPP). Ce protocole de la couche liaison de données gère toute la complexité des commandes AT et permet au Raspberry Pi de se connecter à Internet. Le logiciel de monitoring peut donc de recourir aux protocoles habituels de la couche application pour échanger les données avec le serveur distant. Veuillez noter que le protocole SERIAL est toujours utilisé dans la couche physique. La figure \ref{fig:so2} illustre le fonctionnement de cette solution.


\begin{figure}[ht!]
  \includegraphics[width=\textwidth]{img/el_prototype/solution3_com.png}
  \caption{Architecture simplifiée de la solution de communication avec le SIM800L et le protocole PPP}
  \label{fig:so2}
\end{figure}

\noindent
Finalement, la solution 2 (USSD) est celle qui a été implémentée lors de ce mémoire. La solution 1 demandait un investissement assez élevé. De plus, il n'y a pas énormément d'informations sur la qualité des réseaux en République Démocratique du Congo \cite{congo_tower}. De ce fait, utiliser une solution basée sur le protocole USSD est le choix le plus sûr puisque, théoriquement, ce protocole fonctionne même lorsque la connectivité Internet du réseau est indisponible.

~

\noindent
Veuillez noter que la troisième solution ne possède pas le même problème de coût que la première. Cependant, cette solution a seulement été identifiée quelques semaines plus tard que les deux autres et, à ce stade, la deuxième méthode avait déjà commencé à être implémentée. En outre, la troisième proposition reste théoriquement moins fiable qu'une solution basée sur USSD. De ce fait, il a été décidé de continuer l'implémentation de la deuxième méthode afin d'augmenter la probabilité d'avoir une solution complètement fonctionnelle à la fin de ce mémoire. Utiliser le temps disponible pour implémenter les deux solutions parallèlement pourrait conduire à des solutions sous-optimales ou non fonctionnelles.

\subsubsection{Alimentation}

\noindent
Le système d'alimentation créé l'année dernière a aussi présenté des problèmes. En effet, le régulateur de charge ainsi que le step-down sont tombés en panne. Cependant, l'idée semblait prometteuse et, en particulier, l'autonomie du système était très bonne. Par conséquent, le plan pour ce mémoire fut de poursuivre avec la même idée, mais de changer deux choses. Premièrement, le step-down précédent a été remplacé par un nouveau modèle mieux documenté. Ce nouveau step-down est le XL4015.

~

\noindent
Le XL4015 est un convertisseur Buck capable de supporter une tension d'entrée comprise entre 8V et 36V et des charges allant jusqu'à 5A. Le rendement de ce convertisseur est très élevé pouvant atteindre les $96\%$. Des applications typiques pour ce step-down sont des moniteurs LCD et des appareils télécom ou réseaux. Cette dernière application s'encadre dans ce projet puisqu'un modem est inclus dans le prototype.


\begin{figure}[ht!]
  \centering
  \includegraphics[scale=0.4]{img/el_prototype/rendement.png}
  \caption{Courbe de rendement du XL4015 lorsque $V_{out} = 5V$ \cite{xl4015_datasheet}}
  \label{fig:xlrend}
\end{figure}
~

\noindent
Ensuite, l'approche pour tester le système d'alimentation a été modifiée. Dans un premier temps, le prototype a été seulement alimenté à partir d'un chargeur de 12V. La batterie et le régulateur de charge n'étaient alors pas utilisés. Si le temps le permettait, le prototype serait testé avec la solution d'alimentation complète lors d'une deuxième phase. Ces tests déphasés ont comme but de mieux aider à identifier le composant provoquant les problèmes


\subsubsection{Boîtier}

\noindent
Une fois les composants définis, il ne restait plus qu'à choisir le boîtier. L'importance de celui-ci ne doit pas être sous-estimée. En effet, le problème de température du prototype précédent était probablement lié à la mauvaise ventilation de la boîte. De plus, le boîtier utilisé précédemment n'était pas adapté pour loger les composants électroniques différents. Les modifications requises pour l'ajuster aux composants n'étaient pas simples à effectuer et l'accès aux divers ports I/O du Raspberry Pi était très limité. Cependant, avant de partir sur la conception de la nouvelle boîte, plusieurs tests ont été réalisés sur l'ancien boîtier afin d'examiner les problèmes de température et la nécessité d'ajouter un ventilateur. Les résultats de ces tests seront présentés dans la section \ref{sec:prototests} et le nouveau design sera abordé dans la section \ref{sec:protores}.

\subsection{Thingstream}
\label{sec:thingstream}

\noindent
Thingstream est une entreprise qui propose \textit{IoT-Communication-as-a-Service}. En d'autres termes, ils fournissent une plateforme qui facilite la communication entre des dispositifs différents. Leur service est disponible dans plus de 190 pays grâce à l'utilisation des réseaux mobiles GSM des opérateurs de télécommunication locaux. En particulier, le service est capable de basculer entre les réseaux des différents opérateurs disponibles à un endroit donné. Cette redondance accorde une grande fiabilité au service.

~

\noindent
La grande particularité de Thingstream est que l'ensemble de la plateforme est basé sur le protocole MQTT\footnote{Message Queuing Telemetry Transport} pour la transmission des messages. Ce protocole a été spécifiquement conçu pour des "réseaux à faible bande passante, à haute latence et non fiables"\cite{mqtt_faq}. Un SDK\footnote{Software Development Kit ou kit de développement logiciel} est fourni afin de simplifier l'accès à la plateforme de Thinsgtream et d'exploiter le protocole MQTT. Actuellement, Thingstream ne possède pas de concurrents directs offrant les mêmes services. Des services similaires se concentrent souvent uniquement sur la proposition d'une connectivité Internet à travers les réseaux mobiles. Un exemple d'un tel service est EMnify. Cependant, aucun software est proposé pour aider à plus facilement exploiter la plateforme. Le niveau d'abstraction supplémentaire fourni par Thingstream est une plus-value très importante, particulièrement dans le cadre de ce mémoire où le temps est limité.

~

\noindent
L'architecture de la plateforme de Thingstream est présentée sur la figure \ref{fig:thing_archi}. Les SN-Things correspondent aux appareils qui se connectent au réseau de Thingstream par le biais de réseaux mobiles. Comme indiqué sur la figure, ces dispositifs utilisent le protocole USSD pour transporter les messages de la couche application. La couche application est gérée par le protocole MQTT-SN. Ce protocole est une variante du protocole MQTT qui a été spécifiquement conçue pour les Sensor Networks. Le MQTT-SN permet notamment aux applications qui ne peuvent pas utiliser les réseaux TCP/IP de communiquer avec des applications MQTT. La spécification complète de MQTT-SN est disponible sur \cite{stanford2013mqtt}.


~

\noindent
Ensuite, il y a les IP-Things. Ils correspondent aux différents clients qui se connectent au broker\footnote{Le broker est un serveur qui reçoit tous les messages publiés et les distribue vers les clients appropriés.} en utilisant le protocole MQTT et qui ont donc accès à un réseau TCP/IP. Grâce à une fonctionnalité nommée Data Flow, Thingstream permet aussi au broker de communiquer avec d'autres applications en utilisant d'autres protocoles que MQTT. Actuellement, les protocoles disponibles sont :

~

\begin{itemize}
  \item SMS
  \item HTTP
  \item FTP
  \item SMTP (email)
\end{itemize}

~

\begin{figure}[ht!]
  \includegraphics[width=\textwidth]{img/el_prototype/thingstream_archi.png}
  \caption{Architecture simplifiée de la plateforme Thingstream \cite{thing_archi}}
  \label{fig:thing_archi}
\end{figure}
~

\noindent
Plus de détails sur cette fonctionnalité sont disponible sur \cite{thing_dataflow}. Cependant, il est très important de noter que Thingstream offre la possibilité d'envoyer des SMS. De ce fait, les deux types de communication requis pour ce projet sont supportés par Thingstream.

~

\noindent
En résumé, Thingstream offre un service exploitant tous les avantages du protocole USSD, mais sans devoir passer directement par un opérateur. Cependant, l'utilisation de ce service n'est pas sans risque. Toute la partie de communication du projet sera dépendante de cette tierce partie. Par conséquent, si Thingstream, pour une raison quelconque, cesse d'offrir ce service, la partie communication du prototype sera complètement inopérante. De façon à limiter le possible impact de ce risque, des mesures ont été prises lors de la conception de l'architecture de l'application. Ceci sera présenté plus tard dans la section \ref{sec:archi}.

\subsection{Tests réalisés}
\label{sec:prototests}

\noindent
Une fois les principaux composants définis, le prototype a été testé pour s'assurer que tous les composants fonctionnaient correctement ensemble. En outre, il fallait vérifier si le SIM800L fonctionnait correctement avec le SDK de Thingstream, ou s'il causerait les mêmes problèmes que ceux rencontrés l'année précédente. De ce fait, un protocole de test a été mis en place.

~

\noindent
Ce protocole consiste à lancer un stress test sur deux cœurs du CPU du Raspberry Pi et, en même temps, de publier un message toutes les 4 minutes sur un topic MQTT en utilisant le SIM800L et la plateforme Thingstream. Pendant le test, un serveur est aussi connecté à la plateforme de Thingstream (IP-Thing) et son but est de répondre à toute demande envoyée par le Raspberry Pi. Après l'envoi du message, le Raspberry Pi attend pendant maximum 2 minutes la réponse du serveur. Si aucune réponse n'est reçue, l'occurrence est enregistrée sur un fichier .txt. Le diagramme de séquence \ref{fig:dia_test} montre comment fonctionne l'envoi d'un message pendant une expérimentation. Le test a une durée de 4 heures et 45 minutes. Après cette période, des mesures de température supplémentaires sont prises pendant 15 minutes. Ce protocole a été automatisé à l'aide de scripts python afin de garantir sa reproductibilité. Dans l'annexe \ref{ap:teeeest} se trouvent plus de détails concernant le matériel nécessaire, le protocole et la démarche à suivre.

\begin{figure}[ht!]
  \includegraphics[width=\textwidth]{img/el_prototype/diagram_test.png}
  \caption{Diagramme de séquence de la publication d'un message pendant un test}
  \label{fig:dia_test}
\end{figure}


~

\vspace{-0.5cm}

\noindent
Pendant le test, plusieurs métriques sont surveillées afin d'évaluer le fonctionnement du prototype. En premier lieu, il y a la température du CPU qui est mesurée toutes les 30 secondes. Ensuite, la température de deux zones distinctes est mesurée à l'aide de capteurs externes de température. Ces deux zones sont indiquées sur la figure \ref{fig:dia_test} et correspondent aux deux zones où la température est la plus élevée à l'intérieur du boîtier. Ces trois mesures de température servent principalement à analyser les capacités de refroidissement du boîtier, mais aussi pour vérifier si la température a un impact sur les performances du SIM800L.

\begin{figure}[ht!]
  \centering
  \includegraphics[scale=0.25]{img/el_prototype/zones_temperature.png}
  \caption{Zones où la température a été mesurée lors des tests}
  \label{fig:temp_zones}
\end{figure}

~

\noindent
Ensuite, il y a les métriques liées au fonctionnement du SIM800L.
Les logs produits par le SDK de Thingstream permettent de ressortir deux informations assez importantes : le nombre de messages impossibles à décoder (Receive:parseMessage Warning) et le nombre d'erreurs lors de l'envoi d'un paquet (Protocol:SendPackets Error). La quantité de messages envoyés et reçus par le Raspberry Pi a aussi été enregistrée pour tous les tests, mais cette métrique ne sera pas discutée ici puisque cette valeur était la même pour tous les tests. Toutefois, nous pouvons déjà conclure que le SDK de Thingstream est résilient aux erreurs du modem une fois que le nombre d'erreurs ne semble pas influencer les messages envoyés.

~


\noindent
Les tests présentés ci-dessous ont été réalisés avec le boîtier du prototype de l'année précédente. Le boîtier se trouvait dans une pièce à température contrôlée entre 19,5°C et 20,5°C. Lors de ces tests, le système d'alimentation avec batterie n'a pas utilisé. Au lieu de cela, un chargeur 12V 1.5A DC a été utilisé pour alimenter le XL4015. Chaque test a été répété trois fois et les résultats correspondent à la moyenne de ces trois essais.

~


\noindent
Ce protocole de test a été suivi pour effectuer quatre tests avec des conditions de fonctionnement différentes. Le premier test a été réalisé sans aucun refroidissement actif et avec le step-down à l'intérieur de la boîte. Ce test correspond aux conditions normales de fonctionnement du prototype de l'année dernière.



\begin{figure}[h!]
  \centering
  %% Creator: Matplotlib, PGF backend
%%
%% To include the figure in your LaTeX document, write
%%   \input{<filename>.pgf}
%%
%% Make sure the required packages are loaded in your preamble
%%   \usepackage{pgf}
%%
%% Figures using additional raster images can only be included by \input if
%% they are in the same directory as the main LaTeX file. For loading figures
%% from other directories you can use the `import` package
%%   \usepackage{import}
%% and then include the figures with
%%   \import{<path to file>}{<filename>.pgf}
%%
%% Matplotlib used the following preamble
%%
\begingroup%
\makeatletter%
\begin{pgfpicture}%
\pgfpathrectangle{\pgfpointorigin}{\pgfqpoint{6.400000in}{4.800000in}}%
\pgfusepath{use as bounding box, clip}%
\begin{pgfscope}%
\pgfsetbuttcap%
\pgfsetmiterjoin%
\definecolor{currentfill}{rgb}{1.000000,1.000000,1.000000}%
\pgfsetfillcolor{currentfill}%
\pgfsetlinewidth{0.000000pt}%
\definecolor{currentstroke}{rgb}{1.000000,1.000000,1.000000}%
\pgfsetstrokecolor{currentstroke}%
\pgfsetdash{}{0pt}%
\pgfpathmoveto{\pgfqpoint{0.000000in}{0.000000in}}%
\pgfpathlineto{\pgfqpoint{6.400000in}{0.000000in}}%
\pgfpathlineto{\pgfqpoint{6.400000in}{4.800000in}}%
\pgfpathlineto{\pgfqpoint{0.000000in}{4.800000in}}%
\pgfpathclose%
\pgfusepath{fill}%
\end{pgfscope}%
\begin{pgfscope}%
\pgfsetbuttcap%
\pgfsetmiterjoin%
\definecolor{currentfill}{rgb}{1.000000,1.000000,1.000000}%
\pgfsetfillcolor{currentfill}%
\pgfsetlinewidth{0.000000pt}%
\definecolor{currentstroke}{rgb}{0.000000,0.000000,0.000000}%
\pgfsetstrokecolor{currentstroke}%
\pgfsetstrokeopacity{0.000000}%
\pgfsetdash{}{0pt}%
\pgfpathmoveto{\pgfqpoint{0.800000in}{0.960000in}}%
\pgfpathlineto{\pgfqpoint{5.760000in}{0.960000in}}%
\pgfpathlineto{\pgfqpoint{5.760000in}{4.224000in}}%
\pgfpathlineto{\pgfqpoint{0.800000in}{4.224000in}}%
\pgfpathclose%
\pgfusepath{fill}%
\end{pgfscope}%
\begin{pgfscope}%
\pgfsetbuttcap%
\pgfsetroundjoin%
\definecolor{currentfill}{rgb}{0.000000,0.000000,0.000000}%
\pgfsetfillcolor{currentfill}%
\pgfsetlinewidth{0.803000pt}%
\definecolor{currentstroke}{rgb}{0.000000,0.000000,0.000000}%
\pgfsetstrokecolor{currentstroke}%
\pgfsetdash{}{0pt}%
\pgfsys@defobject{currentmarker}{\pgfqpoint{0.000000in}{-0.048611in}}{\pgfqpoint{0.000000in}{0.000000in}}{%
\pgfpathmoveto{\pgfqpoint{0.000000in}{0.000000in}}%
\pgfpathlineto{\pgfqpoint{0.000000in}{-0.048611in}}%
\pgfusepath{stroke,fill}%
}%
\begin{pgfscope}%
\pgfsys@transformshift{0.800000in}{0.960000in}%
\pgfsys@useobject{currentmarker}{}%
\end{pgfscope}%
\end{pgfscope}%
\begin{pgfscope}%
\definecolor{textcolor}{rgb}{0.000000,0.000000,0.000000}%
\pgfsetstrokecolor{textcolor}%
\pgfsetfillcolor{textcolor}%
\pgftext[x=0.512522in,y=0.621070in,left,base,rotate=30.000000]{\color{textcolor}\rmfamily\fontsize{10.000000}{12.000000}\selectfont 00:00}%
\end{pgfscope}%
\begin{pgfscope}%
\pgfsetbuttcap%
\pgfsetroundjoin%
\definecolor{currentfill}{rgb}{0.000000,0.000000,0.000000}%
\pgfsetfillcolor{currentfill}%
\pgfsetlinewidth{0.803000pt}%
\definecolor{currentstroke}{rgb}{0.000000,0.000000,0.000000}%
\pgfsetstrokecolor{currentstroke}%
\pgfsetdash{}{0pt}%
\pgfsys@defobject{currentmarker}{\pgfqpoint{0.000000in}{-0.048611in}}{\pgfqpoint{0.000000in}{0.000000in}}{%
\pgfpathmoveto{\pgfqpoint{0.000000in}{0.000000in}}%
\pgfpathlineto{\pgfqpoint{0.000000in}{-0.048611in}}%
\pgfusepath{stroke,fill}%
}%
\begin{pgfscope}%
\pgfsys@transformshift{1.791957in}{0.960000in}%
\pgfsys@useobject{currentmarker}{}%
\end{pgfscope}%
\end{pgfscope}%
\begin{pgfscope}%
\definecolor{textcolor}{rgb}{0.000000,0.000000,0.000000}%
\pgfsetstrokecolor{textcolor}%
\pgfsetfillcolor{textcolor}%
\pgftext[x=1.504479in,y=0.621070in,left,base,rotate=30.000000]{\color{textcolor}\rmfamily\fontsize{10.000000}{12.000000}\selectfont 01:00}%
\end{pgfscope}%
\begin{pgfscope}%
\pgfsetbuttcap%
\pgfsetroundjoin%
\definecolor{currentfill}{rgb}{0.000000,0.000000,0.000000}%
\pgfsetfillcolor{currentfill}%
\pgfsetlinewidth{0.803000pt}%
\definecolor{currentstroke}{rgb}{0.000000,0.000000,0.000000}%
\pgfsetstrokecolor{currentstroke}%
\pgfsetdash{}{0pt}%
\pgfsys@defobject{currentmarker}{\pgfqpoint{0.000000in}{-0.048611in}}{\pgfqpoint{0.000000in}{0.000000in}}{%
\pgfpathmoveto{\pgfqpoint{0.000000in}{0.000000in}}%
\pgfpathlineto{\pgfqpoint{0.000000in}{-0.048611in}}%
\pgfusepath{stroke,fill}%
}%
\begin{pgfscope}%
\pgfsys@transformshift{2.783914in}{0.960000in}%
\pgfsys@useobject{currentmarker}{}%
\end{pgfscope}%
\end{pgfscope}%
\begin{pgfscope}%
\definecolor{textcolor}{rgb}{0.000000,0.000000,0.000000}%
\pgfsetstrokecolor{textcolor}%
\pgfsetfillcolor{textcolor}%
\pgftext[x=2.496436in,y=0.621070in,left,base,rotate=30.000000]{\color{textcolor}\rmfamily\fontsize{10.000000}{12.000000}\selectfont 02:00}%
\end{pgfscope}%
\begin{pgfscope}%
\pgfsetbuttcap%
\pgfsetroundjoin%
\definecolor{currentfill}{rgb}{0.000000,0.000000,0.000000}%
\pgfsetfillcolor{currentfill}%
\pgfsetlinewidth{0.803000pt}%
\definecolor{currentstroke}{rgb}{0.000000,0.000000,0.000000}%
\pgfsetstrokecolor{currentstroke}%
\pgfsetdash{}{0pt}%
\pgfsys@defobject{currentmarker}{\pgfqpoint{0.000000in}{-0.048611in}}{\pgfqpoint{0.000000in}{0.000000in}}{%
\pgfpathmoveto{\pgfqpoint{0.000000in}{0.000000in}}%
\pgfpathlineto{\pgfqpoint{0.000000in}{-0.048611in}}%
\pgfusepath{stroke,fill}%
}%
\begin{pgfscope}%
\pgfsys@transformshift{3.775871in}{0.960000in}%
\pgfsys@useobject{currentmarker}{}%
\end{pgfscope}%
\end{pgfscope}%
\begin{pgfscope}%
\definecolor{textcolor}{rgb}{0.000000,0.000000,0.000000}%
\pgfsetstrokecolor{textcolor}%
\pgfsetfillcolor{textcolor}%
\pgftext[x=3.488393in,y=0.621070in,left,base,rotate=30.000000]{\color{textcolor}\rmfamily\fontsize{10.000000}{12.000000}\selectfont 03:00}%
\end{pgfscope}%
\begin{pgfscope}%
\pgfsetbuttcap%
\pgfsetroundjoin%
\definecolor{currentfill}{rgb}{0.000000,0.000000,0.000000}%
\pgfsetfillcolor{currentfill}%
\pgfsetlinewidth{0.803000pt}%
\definecolor{currentstroke}{rgb}{0.000000,0.000000,0.000000}%
\pgfsetstrokecolor{currentstroke}%
\pgfsetdash{}{0pt}%
\pgfsys@defobject{currentmarker}{\pgfqpoint{0.000000in}{-0.048611in}}{\pgfqpoint{0.000000in}{0.000000in}}{%
\pgfpathmoveto{\pgfqpoint{0.000000in}{0.000000in}}%
\pgfpathlineto{\pgfqpoint{0.000000in}{-0.048611in}}%
\pgfusepath{stroke,fill}%
}%
\begin{pgfscope}%
\pgfsys@transformshift{4.767828in}{0.960000in}%
\pgfsys@useobject{currentmarker}{}%
\end{pgfscope}%
\end{pgfscope}%
\begin{pgfscope}%
\definecolor{textcolor}{rgb}{0.000000,0.000000,0.000000}%
\pgfsetstrokecolor{textcolor}%
\pgfsetfillcolor{textcolor}%
\pgftext[x=4.480350in,y=0.621070in,left,base,rotate=30.000000]{\color{textcolor}\rmfamily\fontsize{10.000000}{12.000000}\selectfont 04:00}%
\end{pgfscope}%
\begin{pgfscope}%
\pgfsetbuttcap%
\pgfsetroundjoin%
\definecolor{currentfill}{rgb}{0.000000,0.000000,0.000000}%
\pgfsetfillcolor{currentfill}%
\pgfsetlinewidth{0.803000pt}%
\definecolor{currentstroke}{rgb}{0.000000,0.000000,0.000000}%
\pgfsetstrokecolor{currentstroke}%
\pgfsetdash{}{0pt}%
\pgfsys@defobject{currentmarker}{\pgfqpoint{0.000000in}{-0.048611in}}{\pgfqpoint{0.000000in}{0.000000in}}{%
\pgfpathmoveto{\pgfqpoint{0.000000in}{0.000000in}}%
\pgfpathlineto{\pgfqpoint{0.000000in}{-0.048611in}}%
\pgfusepath{stroke,fill}%
}%
\begin{pgfscope}%
\pgfsys@transformshift{5.759786in}{0.960000in}%
\pgfsys@useobject{currentmarker}{}%
\end{pgfscope}%
\end{pgfscope}%
\begin{pgfscope}%
\definecolor{textcolor}{rgb}{0.000000,0.000000,0.000000}%
\pgfsetstrokecolor{textcolor}%
\pgfsetfillcolor{textcolor}%
\pgftext[x=5.472308in,y=0.621070in,left,base,rotate=30.000000]{\color{textcolor}\rmfamily\fontsize{10.000000}{12.000000}\selectfont 05:00}%
\end{pgfscope}%
\begin{pgfscope}%
\definecolor{textcolor}{rgb}{0.000000,0.000000,0.000000}%
\pgfsetstrokecolor{textcolor}%
\pgfsetfillcolor{textcolor}%
\pgftext[x=3.280000in,y=0.542126in,,top]{\color{textcolor}\rmfamily\fontsize{10.000000}{12.000000}\selectfont Temps (hh:mm)}%
\end{pgfscope}%
\begin{pgfscope}%
\pgfsetbuttcap%
\pgfsetroundjoin%
\definecolor{currentfill}{rgb}{0.000000,0.000000,0.000000}%
\pgfsetfillcolor{currentfill}%
\pgfsetlinewidth{0.803000pt}%
\definecolor{currentstroke}{rgb}{0.000000,0.000000,0.000000}%
\pgfsetstrokecolor{currentstroke}%
\pgfsetdash{}{0pt}%
\pgfsys@defobject{currentmarker}{\pgfqpoint{-0.048611in}{0.000000in}}{\pgfqpoint{0.000000in}{0.000000in}}{%
\pgfpathmoveto{\pgfqpoint{0.000000in}{0.000000in}}%
\pgfpathlineto{\pgfqpoint{-0.048611in}{0.000000in}}%
\pgfusepath{stroke,fill}%
}%
\begin{pgfscope}%
\pgfsys@transformshift{0.800000in}{0.960000in}%
\pgfsys@useobject{currentmarker}{}%
\end{pgfscope}%
\end{pgfscope}%
\begin{pgfscope}%
\definecolor{textcolor}{rgb}{0.000000,0.000000,0.000000}%
\pgfsetstrokecolor{textcolor}%
\pgfsetfillcolor{textcolor}%
\pgftext[x=0.563888in,y=0.911775in,left,base]{\color{textcolor}\rmfamily\fontsize{10.000000}{12.000000}\selectfont \(\displaystyle 20\)}%
\end{pgfscope}%
\begin{pgfscope}%
\pgfsetbuttcap%
\pgfsetroundjoin%
\definecolor{currentfill}{rgb}{0.000000,0.000000,0.000000}%
\pgfsetfillcolor{currentfill}%
\pgfsetlinewidth{0.803000pt}%
\definecolor{currentstroke}{rgb}{0.000000,0.000000,0.000000}%
\pgfsetstrokecolor{currentstroke}%
\pgfsetdash{}{0pt}%
\pgfsys@defobject{currentmarker}{\pgfqpoint{-0.048611in}{0.000000in}}{\pgfqpoint{0.000000in}{0.000000in}}{%
\pgfpathmoveto{\pgfqpoint{0.000000in}{0.000000in}}%
\pgfpathlineto{\pgfqpoint{-0.048611in}{0.000000in}}%
\pgfusepath{stroke,fill}%
}%
\begin{pgfscope}%
\pgfsys@transformshift{0.800000in}{1.426286in}%
\pgfsys@useobject{currentmarker}{}%
\end{pgfscope}%
\end{pgfscope}%
\begin{pgfscope}%
\definecolor{textcolor}{rgb}{0.000000,0.000000,0.000000}%
\pgfsetstrokecolor{textcolor}%
\pgfsetfillcolor{textcolor}%
\pgftext[x=0.563888in,y=1.378060in,left,base]{\color{textcolor}\rmfamily\fontsize{10.000000}{12.000000}\selectfont \(\displaystyle 25\)}%
\end{pgfscope}%
\begin{pgfscope}%
\pgfsetbuttcap%
\pgfsetroundjoin%
\definecolor{currentfill}{rgb}{0.000000,0.000000,0.000000}%
\pgfsetfillcolor{currentfill}%
\pgfsetlinewidth{0.803000pt}%
\definecolor{currentstroke}{rgb}{0.000000,0.000000,0.000000}%
\pgfsetstrokecolor{currentstroke}%
\pgfsetdash{}{0pt}%
\pgfsys@defobject{currentmarker}{\pgfqpoint{-0.048611in}{0.000000in}}{\pgfqpoint{0.000000in}{0.000000in}}{%
\pgfpathmoveto{\pgfqpoint{0.000000in}{0.000000in}}%
\pgfpathlineto{\pgfqpoint{-0.048611in}{0.000000in}}%
\pgfusepath{stroke,fill}%
}%
\begin{pgfscope}%
\pgfsys@transformshift{0.800000in}{1.892571in}%
\pgfsys@useobject{currentmarker}{}%
\end{pgfscope}%
\end{pgfscope}%
\begin{pgfscope}%
\definecolor{textcolor}{rgb}{0.000000,0.000000,0.000000}%
\pgfsetstrokecolor{textcolor}%
\pgfsetfillcolor{textcolor}%
\pgftext[x=0.563888in,y=1.844346in,left,base]{\color{textcolor}\rmfamily\fontsize{10.000000}{12.000000}\selectfont \(\displaystyle 30\)}%
\end{pgfscope}%
\begin{pgfscope}%
\pgfsetbuttcap%
\pgfsetroundjoin%
\definecolor{currentfill}{rgb}{0.000000,0.000000,0.000000}%
\pgfsetfillcolor{currentfill}%
\pgfsetlinewidth{0.803000pt}%
\definecolor{currentstroke}{rgb}{0.000000,0.000000,0.000000}%
\pgfsetstrokecolor{currentstroke}%
\pgfsetdash{}{0pt}%
\pgfsys@defobject{currentmarker}{\pgfqpoint{-0.048611in}{0.000000in}}{\pgfqpoint{0.000000in}{0.000000in}}{%
\pgfpathmoveto{\pgfqpoint{0.000000in}{0.000000in}}%
\pgfpathlineto{\pgfqpoint{-0.048611in}{0.000000in}}%
\pgfusepath{stroke,fill}%
}%
\begin{pgfscope}%
\pgfsys@transformshift{0.800000in}{2.358857in}%
\pgfsys@useobject{currentmarker}{}%
\end{pgfscope}%
\end{pgfscope}%
\begin{pgfscope}%
\definecolor{textcolor}{rgb}{0.000000,0.000000,0.000000}%
\pgfsetstrokecolor{textcolor}%
\pgfsetfillcolor{textcolor}%
\pgftext[x=0.563888in,y=2.310632in,left,base]{\color{textcolor}\rmfamily\fontsize{10.000000}{12.000000}\selectfont \(\displaystyle 35\)}%
\end{pgfscope}%
\begin{pgfscope}%
\pgfsetbuttcap%
\pgfsetroundjoin%
\definecolor{currentfill}{rgb}{0.000000,0.000000,0.000000}%
\pgfsetfillcolor{currentfill}%
\pgfsetlinewidth{0.803000pt}%
\definecolor{currentstroke}{rgb}{0.000000,0.000000,0.000000}%
\pgfsetstrokecolor{currentstroke}%
\pgfsetdash{}{0pt}%
\pgfsys@defobject{currentmarker}{\pgfqpoint{-0.048611in}{0.000000in}}{\pgfqpoint{0.000000in}{0.000000in}}{%
\pgfpathmoveto{\pgfqpoint{0.000000in}{0.000000in}}%
\pgfpathlineto{\pgfqpoint{-0.048611in}{0.000000in}}%
\pgfusepath{stroke,fill}%
}%
\begin{pgfscope}%
\pgfsys@transformshift{0.800000in}{2.825143in}%
\pgfsys@useobject{currentmarker}{}%
\end{pgfscope}%
\end{pgfscope}%
\begin{pgfscope}%
\definecolor{textcolor}{rgb}{0.000000,0.000000,0.000000}%
\pgfsetstrokecolor{textcolor}%
\pgfsetfillcolor{textcolor}%
\pgftext[x=0.563888in,y=2.776918in,left,base]{\color{textcolor}\rmfamily\fontsize{10.000000}{12.000000}\selectfont \(\displaystyle 40\)}%
\end{pgfscope}%
\begin{pgfscope}%
\pgfsetbuttcap%
\pgfsetroundjoin%
\definecolor{currentfill}{rgb}{0.000000,0.000000,0.000000}%
\pgfsetfillcolor{currentfill}%
\pgfsetlinewidth{0.803000pt}%
\definecolor{currentstroke}{rgb}{0.000000,0.000000,0.000000}%
\pgfsetstrokecolor{currentstroke}%
\pgfsetdash{}{0pt}%
\pgfsys@defobject{currentmarker}{\pgfqpoint{-0.048611in}{0.000000in}}{\pgfqpoint{0.000000in}{0.000000in}}{%
\pgfpathmoveto{\pgfqpoint{0.000000in}{0.000000in}}%
\pgfpathlineto{\pgfqpoint{-0.048611in}{0.000000in}}%
\pgfusepath{stroke,fill}%
}%
\begin{pgfscope}%
\pgfsys@transformshift{0.800000in}{3.291429in}%
\pgfsys@useobject{currentmarker}{}%
\end{pgfscope}%
\end{pgfscope}%
\begin{pgfscope}%
\definecolor{textcolor}{rgb}{0.000000,0.000000,0.000000}%
\pgfsetstrokecolor{textcolor}%
\pgfsetfillcolor{textcolor}%
\pgftext[x=0.563888in,y=3.243203in,left,base]{\color{textcolor}\rmfamily\fontsize{10.000000}{12.000000}\selectfont \(\displaystyle 45\)}%
\end{pgfscope}%
\begin{pgfscope}%
\pgfsetbuttcap%
\pgfsetroundjoin%
\definecolor{currentfill}{rgb}{0.000000,0.000000,0.000000}%
\pgfsetfillcolor{currentfill}%
\pgfsetlinewidth{0.803000pt}%
\definecolor{currentstroke}{rgb}{0.000000,0.000000,0.000000}%
\pgfsetstrokecolor{currentstroke}%
\pgfsetdash{}{0pt}%
\pgfsys@defobject{currentmarker}{\pgfqpoint{-0.048611in}{0.000000in}}{\pgfqpoint{0.000000in}{0.000000in}}{%
\pgfpathmoveto{\pgfqpoint{0.000000in}{0.000000in}}%
\pgfpathlineto{\pgfqpoint{-0.048611in}{0.000000in}}%
\pgfusepath{stroke,fill}%
}%
\begin{pgfscope}%
\pgfsys@transformshift{0.800000in}{3.757714in}%
\pgfsys@useobject{currentmarker}{}%
\end{pgfscope}%
\end{pgfscope}%
\begin{pgfscope}%
\definecolor{textcolor}{rgb}{0.000000,0.000000,0.000000}%
\pgfsetstrokecolor{textcolor}%
\pgfsetfillcolor{textcolor}%
\pgftext[x=0.563888in,y=3.709489in,left,base]{\color{textcolor}\rmfamily\fontsize{10.000000}{12.000000}\selectfont \(\displaystyle 50\)}%
\end{pgfscope}%
\begin{pgfscope}%
\pgfsetbuttcap%
\pgfsetroundjoin%
\definecolor{currentfill}{rgb}{0.000000,0.000000,0.000000}%
\pgfsetfillcolor{currentfill}%
\pgfsetlinewidth{0.803000pt}%
\definecolor{currentstroke}{rgb}{0.000000,0.000000,0.000000}%
\pgfsetstrokecolor{currentstroke}%
\pgfsetdash{}{0pt}%
\pgfsys@defobject{currentmarker}{\pgfqpoint{-0.048611in}{0.000000in}}{\pgfqpoint{0.000000in}{0.000000in}}{%
\pgfpathmoveto{\pgfqpoint{0.000000in}{0.000000in}}%
\pgfpathlineto{\pgfqpoint{-0.048611in}{0.000000in}}%
\pgfusepath{stroke,fill}%
}%
\begin{pgfscope}%
\pgfsys@transformshift{0.800000in}{4.224000in}%
\pgfsys@useobject{currentmarker}{}%
\end{pgfscope}%
\end{pgfscope}%
\begin{pgfscope}%
\definecolor{textcolor}{rgb}{0.000000,0.000000,0.000000}%
\pgfsetstrokecolor{textcolor}%
\pgfsetfillcolor{textcolor}%
\pgftext[x=0.563888in,y=4.175775in,left,base]{\color{textcolor}\rmfamily\fontsize{10.000000}{12.000000}\selectfont \(\displaystyle 55\)}%
\end{pgfscope}%
\begin{pgfscope}%
\definecolor{textcolor}{rgb}{0.000000,0.000000,0.000000}%
\pgfsetstrokecolor{textcolor}%
\pgfsetfillcolor{textcolor}%
\pgftext[x=0.508333in,y=2.592000in,,bottom,rotate=90.000000]{\color{textcolor}\rmfamily\fontsize{10.000000}{12.000000}\selectfont Température (\textdegree{}C)}%
\end{pgfscope}%
\begin{pgfscope}%
\pgfpathrectangle{\pgfqpoint{0.800000in}{0.960000in}}{\pgfqpoint{4.960000in}{3.264000in}}%
\pgfusepath{clip}%
\pgfsetrectcap%
\pgfsetroundjoin%
\pgfsetlinewidth{1.505625pt}%
\definecolor{currentstroke}{rgb}{0.121569,0.466667,0.705882}%
\pgfsetstrokecolor{currentstroke}%
\pgfsetdash{}{0pt}%
\pgfpathmoveto{\pgfqpoint{0.800000in}{2.731886in}}%
\pgfpathlineto{\pgfqpoint{0.808310in}{3.184183in}}%
\pgfpathlineto{\pgfqpoint{0.816588in}{3.356709in}}%
\pgfpathlineto{\pgfqpoint{0.824865in}{3.384686in}}%
\pgfpathlineto{\pgfqpoint{0.833142in}{3.435977in}}%
\pgfpathlineto{\pgfqpoint{0.841419in}{3.435977in}}%
\pgfpathlineto{\pgfqpoint{0.849697in}{3.510583in}}%
\pgfpathlineto{\pgfqpoint{0.857974in}{3.510583in}}%
\pgfpathlineto{\pgfqpoint{0.866251in}{3.533897in}}%
\pgfpathlineto{\pgfqpoint{0.874529in}{3.585189in}}%
\pgfpathlineto{\pgfqpoint{0.891084in}{3.585189in}}%
\pgfpathlineto{\pgfqpoint{0.899362in}{3.636480in}}%
\pgfpathlineto{\pgfqpoint{0.907640in}{3.636480in}}%
\pgfpathlineto{\pgfqpoint{0.915918in}{3.659794in}}%
\pgfpathlineto{\pgfqpoint{0.924193in}{3.687771in}}%
\pgfpathlineto{\pgfqpoint{0.932467in}{3.711086in}}%
\pgfpathlineto{\pgfqpoint{0.940742in}{3.711086in}}%
\pgfpathlineto{\pgfqpoint{0.949019in}{3.687771in}}%
\pgfpathlineto{\pgfqpoint{0.957296in}{3.739063in}}%
\pgfpathlineto{\pgfqpoint{0.965574in}{3.687771in}}%
\pgfpathlineto{\pgfqpoint{0.973852in}{3.711086in}}%
\pgfpathlineto{\pgfqpoint{0.982130in}{3.739063in}}%
\pgfpathlineto{\pgfqpoint{0.990408in}{3.687771in}}%
\pgfpathlineto{\pgfqpoint{0.998683in}{3.757714in}}%
\pgfpathlineto{\pgfqpoint{1.006958in}{3.739063in}}%
\pgfpathlineto{\pgfqpoint{1.015230in}{3.785691in}}%
\pgfpathlineto{\pgfqpoint{1.023507in}{3.739063in}}%
\pgfpathlineto{\pgfqpoint{1.031784in}{3.785691in}}%
\pgfpathlineto{\pgfqpoint{1.040062in}{3.762377in}}%
\pgfpathlineto{\pgfqpoint{1.056618in}{3.809006in}}%
\pgfpathlineto{\pgfqpoint{1.073173in}{3.809006in}}%
\pgfpathlineto{\pgfqpoint{1.081451in}{3.832320in}}%
\pgfpathlineto{\pgfqpoint{1.089729in}{3.832320in}}%
\pgfpathlineto{\pgfqpoint{1.098005in}{3.785691in}}%
\pgfpathlineto{\pgfqpoint{1.106283in}{3.832320in}}%
\pgfpathlineto{\pgfqpoint{1.139394in}{3.832320in}}%
\pgfpathlineto{\pgfqpoint{1.147672in}{3.860297in}}%
\pgfpathlineto{\pgfqpoint{1.155944in}{3.860297in}}%
\pgfpathlineto{\pgfqpoint{1.164222in}{3.785691in}}%
\pgfpathlineto{\pgfqpoint{1.172500in}{3.832320in}}%
\pgfpathlineto{\pgfqpoint{1.189052in}{3.888274in}}%
\pgfpathlineto{\pgfqpoint{1.205597in}{3.888274in}}%
\pgfpathlineto{\pgfqpoint{1.213869in}{3.911589in}}%
\pgfpathlineto{\pgfqpoint{1.222142in}{3.832320in}}%
\pgfpathlineto{\pgfqpoint{1.230417in}{3.911589in}}%
\pgfpathlineto{\pgfqpoint{1.238690in}{3.860297in}}%
\pgfpathlineto{\pgfqpoint{1.246967in}{3.888274in}}%
\pgfpathlineto{\pgfqpoint{1.263515in}{3.888274in}}%
\pgfpathlineto{\pgfqpoint{1.271791in}{3.883611in}}%
\pgfpathlineto{\pgfqpoint{1.280068in}{3.939566in}}%
\pgfpathlineto{\pgfqpoint{1.288343in}{3.911589in}}%
\pgfpathlineto{\pgfqpoint{1.296616in}{3.911589in}}%
\pgfpathlineto{\pgfqpoint{1.304889in}{3.934903in}}%
\pgfpathlineto{\pgfqpoint{1.313166in}{3.860297in}}%
\pgfpathlineto{\pgfqpoint{1.321444in}{3.888274in}}%
\pgfpathlineto{\pgfqpoint{1.329715in}{3.911589in}}%
\pgfpathlineto{\pgfqpoint{1.337988in}{3.911589in}}%
\pgfpathlineto{\pgfqpoint{1.346263in}{3.939566in}}%
\pgfpathlineto{\pgfqpoint{1.354535in}{3.911589in}}%
\pgfpathlineto{\pgfqpoint{1.362807in}{3.934903in}}%
\pgfpathlineto{\pgfqpoint{1.379358in}{3.934903in}}%
\pgfpathlineto{\pgfqpoint{1.387635in}{3.911589in}}%
\pgfpathlineto{\pgfqpoint{1.395912in}{3.911589in}}%
\pgfpathlineto{\pgfqpoint{1.404190in}{3.934903in}}%
\pgfpathlineto{\pgfqpoint{1.420744in}{3.934903in}}%
\pgfpathlineto{\pgfqpoint{1.429022in}{3.962880in}}%
\pgfpathlineto{\pgfqpoint{1.437299in}{3.962880in}}%
\pgfpathlineto{\pgfqpoint{1.445569in}{3.934903in}}%
\pgfpathlineto{\pgfqpoint{1.453845in}{3.911589in}}%
\pgfpathlineto{\pgfqpoint{1.462116in}{3.939566in}}%
\pgfpathlineto{\pgfqpoint{1.470390in}{3.934903in}}%
\pgfpathlineto{\pgfqpoint{1.478663in}{3.934903in}}%
\pgfpathlineto{\pgfqpoint{1.486940in}{3.911589in}}%
\pgfpathlineto{\pgfqpoint{1.495218in}{3.934903in}}%
\pgfpathlineto{\pgfqpoint{1.503490in}{3.962880in}}%
\pgfpathlineto{\pgfqpoint{1.511762in}{3.934903in}}%
\pgfpathlineto{\pgfqpoint{1.520036in}{3.934903in}}%
\pgfpathlineto{\pgfqpoint{1.528310in}{3.888274in}}%
\pgfpathlineto{\pgfqpoint{1.536584in}{3.934903in}}%
\pgfpathlineto{\pgfqpoint{1.544861in}{3.911589in}}%
\pgfpathlineto{\pgfqpoint{1.553139in}{3.934903in}}%
\pgfpathlineto{\pgfqpoint{1.561411in}{3.911589in}}%
\pgfpathlineto{\pgfqpoint{1.569684in}{3.934903in}}%
\pgfpathlineto{\pgfqpoint{1.577958in}{3.911589in}}%
\pgfpathlineto{\pgfqpoint{1.586235in}{3.986194in}}%
\pgfpathlineto{\pgfqpoint{1.594505in}{3.962880in}}%
\pgfpathlineto{\pgfqpoint{1.611060in}{3.962880in}}%
\pgfpathlineto{\pgfqpoint{1.619338in}{3.986194in}}%
\pgfpathlineto{\pgfqpoint{1.635894in}{3.939566in}}%
\pgfpathlineto{\pgfqpoint{1.644168in}{3.934903in}}%
\pgfpathlineto{\pgfqpoint{1.652441in}{3.962880in}}%
\pgfpathlineto{\pgfqpoint{1.660718in}{3.986194in}}%
\pgfpathlineto{\pgfqpoint{1.668995in}{3.934903in}}%
\pgfpathlineto{\pgfqpoint{1.677270in}{3.962880in}}%
\pgfpathlineto{\pgfqpoint{1.685550in}{3.934903in}}%
\pgfpathlineto{\pgfqpoint{1.693825in}{3.986194in}}%
\pgfpathlineto{\pgfqpoint{1.702103in}{3.986194in}}%
\pgfpathlineto{\pgfqpoint{1.710375in}{3.962880in}}%
\pgfpathlineto{\pgfqpoint{1.718653in}{3.962880in}}%
\pgfpathlineto{\pgfqpoint{1.726929in}{4.014171in}}%
\pgfpathlineto{\pgfqpoint{1.743485in}{3.864960in}}%
\pgfpathlineto{\pgfqpoint{1.751758in}{3.986194in}}%
\pgfpathlineto{\pgfqpoint{1.760035in}{3.962880in}}%
\pgfpathlineto{\pgfqpoint{1.768312in}{3.986194in}}%
\pgfpathlineto{\pgfqpoint{1.776590in}{3.934903in}}%
\pgfpathlineto{\pgfqpoint{1.784867in}{3.939566in}}%
\pgfpathlineto{\pgfqpoint{1.793144in}{3.934903in}}%
\pgfpathlineto{\pgfqpoint{1.801422in}{3.986194in}}%
\pgfpathlineto{\pgfqpoint{1.809699in}{3.986194in}}%
\pgfpathlineto{\pgfqpoint{1.817977in}{3.934903in}}%
\pgfpathlineto{\pgfqpoint{1.826255in}{3.934903in}}%
\pgfpathlineto{\pgfqpoint{1.834533in}{3.962880in}}%
\pgfpathlineto{\pgfqpoint{1.842808in}{3.934903in}}%
\pgfpathlineto{\pgfqpoint{1.859363in}{3.934903in}}%
\pgfpathlineto{\pgfqpoint{1.867640in}{3.962880in}}%
\pgfpathlineto{\pgfqpoint{1.875917in}{3.962880in}}%
\pgfpathlineto{\pgfqpoint{1.884195in}{3.986194in}}%
\pgfpathlineto{\pgfqpoint{1.892474in}{3.934903in}}%
\pgfpathlineto{\pgfqpoint{1.900749in}{4.014171in}}%
\pgfpathlineto{\pgfqpoint{1.909027in}{4.014171in}}%
\pgfpathlineto{\pgfqpoint{1.917304in}{3.986194in}}%
\pgfpathlineto{\pgfqpoint{1.925581in}{3.986194in}}%
\pgfpathlineto{\pgfqpoint{1.933859in}{4.037486in}}%
\pgfpathlineto{\pgfqpoint{1.942136in}{4.014171in}}%
\pgfpathlineto{\pgfqpoint{1.950412in}{4.037486in}}%
\pgfpathlineto{\pgfqpoint{1.958690in}{3.911589in}}%
\pgfpathlineto{\pgfqpoint{1.966967in}{3.939566in}}%
\pgfpathlineto{\pgfqpoint{1.975244in}{3.990857in}}%
\pgfpathlineto{\pgfqpoint{1.983522in}{4.014171in}}%
\pgfpathlineto{\pgfqpoint{1.991797in}{3.986194in}}%
\pgfpathlineto{\pgfqpoint{2.000074in}{4.014171in}}%
\pgfpathlineto{\pgfqpoint{2.024901in}{4.014171in}}%
\pgfpathlineto{\pgfqpoint{2.033176in}{3.934903in}}%
\pgfpathlineto{\pgfqpoint{2.041449in}{3.962880in}}%
\pgfpathlineto{\pgfqpoint{2.049722in}{3.934903in}}%
\pgfpathlineto{\pgfqpoint{2.057995in}{3.962880in}}%
\pgfpathlineto{\pgfqpoint{2.066270in}{4.037486in}}%
\pgfpathlineto{\pgfqpoint{2.074543in}{4.037486in}}%
\pgfpathlineto{\pgfqpoint{2.082818in}{3.962880in}}%
\pgfpathlineto{\pgfqpoint{2.091095in}{4.037486in}}%
\pgfpathlineto{\pgfqpoint{2.099372in}{3.934903in}}%
\pgfpathlineto{\pgfqpoint{2.107646in}{3.939566in}}%
\pgfpathlineto{\pgfqpoint{2.115924in}{4.014171in}}%
\pgfpathlineto{\pgfqpoint{2.124201in}{4.014171in}}%
\pgfpathlineto{\pgfqpoint{2.132479in}{4.037486in}}%
\pgfpathlineto{\pgfqpoint{2.157309in}{4.037486in}}%
\pgfpathlineto{\pgfqpoint{2.165581in}{4.014171in}}%
\pgfpathlineto{\pgfqpoint{2.173855in}{3.986194in}}%
\pgfpathlineto{\pgfqpoint{2.182128in}{3.986194in}}%
\pgfpathlineto{\pgfqpoint{2.190404in}{4.037486in}}%
\pgfpathlineto{\pgfqpoint{2.215225in}{4.037486in}}%
\pgfpathlineto{\pgfqpoint{2.223497in}{4.014171in}}%
\pgfpathlineto{\pgfqpoint{2.240051in}{4.014171in}}%
\pgfpathlineto{\pgfqpoint{2.248329in}{3.962880in}}%
\pgfpathlineto{\pgfqpoint{2.256606in}{4.014171in}}%
\pgfpathlineto{\pgfqpoint{2.281438in}{4.014171in}}%
\pgfpathlineto{\pgfqpoint{2.289715in}{4.037486in}}%
\pgfpathlineto{\pgfqpoint{2.297993in}{4.014171in}}%
\pgfpathlineto{\pgfqpoint{2.306270in}{4.037486in}}%
\pgfpathlineto{\pgfqpoint{2.314547in}{4.037486in}}%
\pgfpathlineto{\pgfqpoint{2.322825in}{3.962880in}}%
\pgfpathlineto{\pgfqpoint{2.331101in}{3.986194in}}%
\pgfpathlineto{\pgfqpoint{2.339376in}{4.037486in}}%
\pgfpathlineto{\pgfqpoint{2.364208in}{4.037486in}}%
\pgfpathlineto{\pgfqpoint{2.372486in}{4.014171in}}%
\pgfpathlineto{\pgfqpoint{2.380763in}{4.014171in}}%
\pgfpathlineto{\pgfqpoint{2.389037in}{4.037486in}}%
\pgfpathlineto{\pgfqpoint{2.397313in}{3.986194in}}%
\pgfpathlineto{\pgfqpoint{2.405589in}{4.037486in}}%
\pgfpathlineto{\pgfqpoint{2.430422in}{4.037486in}}%
\pgfpathlineto{\pgfqpoint{2.438699in}{4.014171in}}%
\pgfpathlineto{\pgfqpoint{2.446976in}{4.037486in}}%
\pgfpathlineto{\pgfqpoint{2.480071in}{4.037486in}}%
\pgfpathlineto{\pgfqpoint{2.488346in}{4.014171in}}%
\pgfpathlineto{\pgfqpoint{2.496623in}{4.037486in}}%
\pgfpathlineto{\pgfqpoint{2.529712in}{4.037486in}}%
\pgfpathlineto{\pgfqpoint{2.537985in}{3.986194in}}%
\pgfpathlineto{\pgfqpoint{2.546260in}{3.962880in}}%
\pgfpathlineto{\pgfqpoint{2.554532in}{4.037486in}}%
\pgfpathlineto{\pgfqpoint{2.595916in}{4.037486in}}%
\pgfpathlineto{\pgfqpoint{2.604193in}{4.060800in}}%
\pgfpathlineto{\pgfqpoint{2.612467in}{3.986194in}}%
\pgfpathlineto{\pgfqpoint{2.620739in}{3.962880in}}%
\pgfpathlineto{\pgfqpoint{2.629015in}{4.037486in}}%
\pgfpathlineto{\pgfqpoint{2.662119in}{4.037486in}}%
\pgfpathlineto{\pgfqpoint{2.670390in}{4.014171in}}%
\pgfpathlineto{\pgfqpoint{2.678667in}{4.037486in}}%
\pgfpathlineto{\pgfqpoint{2.686945in}{3.911589in}}%
\pgfpathlineto{\pgfqpoint{2.695218in}{4.014171in}}%
\pgfpathlineto{\pgfqpoint{2.703494in}{4.037486in}}%
\pgfpathlineto{\pgfqpoint{2.711771in}{4.014171in}}%
\pgfpathlineto{\pgfqpoint{2.720049in}{4.037486in}}%
\pgfpathlineto{\pgfqpoint{2.728326in}{4.037486in}}%
\pgfpathlineto{\pgfqpoint{2.736603in}{4.060800in}}%
\pgfpathlineto{\pgfqpoint{2.744880in}{4.037486in}}%
\pgfpathlineto{\pgfqpoint{2.753158in}{4.037486in}}%
\pgfpathlineto{\pgfqpoint{2.761435in}{3.962880in}}%
\pgfpathlineto{\pgfqpoint{2.769709in}{4.037486in}}%
\pgfpathlineto{\pgfqpoint{2.786261in}{4.037486in}}%
\pgfpathlineto{\pgfqpoint{2.794533in}{4.014171in}}%
\pgfpathlineto{\pgfqpoint{2.802809in}{4.037486in}}%
\pgfpathlineto{\pgfqpoint{2.811080in}{4.037486in}}%
\pgfpathlineto{\pgfqpoint{2.819358in}{4.014171in}}%
\pgfpathlineto{\pgfqpoint{2.827635in}{4.037486in}}%
\pgfpathlineto{\pgfqpoint{2.835913in}{3.990857in}}%
\pgfpathlineto{\pgfqpoint{2.844189in}{4.037486in}}%
\pgfpathlineto{\pgfqpoint{2.893845in}{4.037486in}}%
\pgfpathlineto{\pgfqpoint{2.902117in}{3.939566in}}%
\pgfpathlineto{\pgfqpoint{2.918671in}{4.037486in}}%
\pgfpathlineto{\pgfqpoint{2.926942in}{4.014171in}}%
\pgfpathlineto{\pgfqpoint{2.935213in}{4.037486in}}%
\pgfpathlineto{\pgfqpoint{2.943489in}{4.037486in}}%
\pgfpathlineto{\pgfqpoint{2.951769in}{4.014171in}}%
\pgfpathlineto{\pgfqpoint{2.960046in}{4.037486in}}%
\pgfpathlineto{\pgfqpoint{2.968324in}{4.037486in}}%
\pgfpathlineto{\pgfqpoint{2.976601in}{3.986194in}}%
\pgfpathlineto{\pgfqpoint{2.984879in}{3.962880in}}%
\pgfpathlineto{\pgfqpoint{2.993152in}{4.037486in}}%
\pgfpathlineto{\pgfqpoint{3.017969in}{4.037486in}}%
\pgfpathlineto{\pgfqpoint{3.026241in}{4.014171in}}%
\pgfpathlineto{\pgfqpoint{3.034513in}{4.037486in}}%
\pgfpathlineto{\pgfqpoint{3.042786in}{4.014171in}}%
\pgfpathlineto{\pgfqpoint{3.051062in}{3.934903in}}%
\pgfpathlineto{\pgfqpoint{3.059334in}{4.037486in}}%
\pgfpathlineto{\pgfqpoint{3.092437in}{4.037486in}}%
\pgfpathlineto{\pgfqpoint{3.100709in}{4.014171in}}%
\pgfpathlineto{\pgfqpoint{3.108982in}{4.037486in}}%
\pgfpathlineto{\pgfqpoint{3.117254in}{4.014171in}}%
\pgfpathlineto{\pgfqpoint{3.125525in}{3.934903in}}%
\pgfpathlineto{\pgfqpoint{3.133804in}{4.037486in}}%
\pgfpathlineto{\pgfqpoint{3.158626in}{4.037486in}}%
\pgfpathlineto{\pgfqpoint{3.166898in}{3.986194in}}%
\pgfpathlineto{\pgfqpoint{3.175171in}{4.037486in}}%
\pgfpathlineto{\pgfqpoint{3.183444in}{4.014171in}}%
\pgfpathlineto{\pgfqpoint{3.191721in}{3.962880in}}%
\pgfpathlineto{\pgfqpoint{3.199998in}{3.934903in}}%
\pgfpathlineto{\pgfqpoint{3.208276in}{3.986194in}}%
\pgfpathlineto{\pgfqpoint{3.216547in}{4.014171in}}%
\pgfpathlineto{\pgfqpoint{3.224825in}{4.037486in}}%
\pgfpathlineto{\pgfqpoint{3.233103in}{4.037486in}}%
\pgfpathlineto{\pgfqpoint{3.241374in}{4.014171in}}%
\pgfpathlineto{\pgfqpoint{3.249652in}{4.037486in}}%
\pgfpathlineto{\pgfqpoint{3.266206in}{3.934903in}}%
\pgfpathlineto{\pgfqpoint{3.274481in}{3.986194in}}%
\pgfpathlineto{\pgfqpoint{3.282757in}{3.934903in}}%
\pgfpathlineto{\pgfqpoint{3.291035in}{4.014171in}}%
\pgfpathlineto{\pgfqpoint{3.299306in}{4.037486in}}%
\pgfpathlineto{\pgfqpoint{3.332409in}{4.037486in}}%
\pgfpathlineto{\pgfqpoint{3.340686in}{3.883611in}}%
\pgfpathlineto{\pgfqpoint{3.348963in}{3.990857in}}%
\pgfpathlineto{\pgfqpoint{3.357240in}{4.037486in}}%
\pgfpathlineto{\pgfqpoint{3.365511in}{4.014171in}}%
\pgfpathlineto{\pgfqpoint{3.373784in}{4.037486in}}%
\pgfpathlineto{\pgfqpoint{3.382060in}{3.962880in}}%
\pgfpathlineto{\pgfqpoint{3.390337in}{4.037486in}}%
\pgfpathlineto{\pgfqpoint{3.398613in}{4.014171in}}%
\pgfpathlineto{\pgfqpoint{3.406888in}{4.014171in}}%
\pgfpathlineto{\pgfqpoint{3.415167in}{3.934903in}}%
\pgfpathlineto{\pgfqpoint{3.423438in}{3.986194in}}%
\pgfpathlineto{\pgfqpoint{3.431714in}{3.986194in}}%
\pgfpathlineto{\pgfqpoint{3.439985in}{4.014171in}}%
\pgfpathlineto{\pgfqpoint{3.448257in}{4.037486in}}%
\pgfpathlineto{\pgfqpoint{3.456536in}{4.037486in}}%
\pgfpathlineto{\pgfqpoint{3.464811in}{3.962880in}}%
\pgfpathlineto{\pgfqpoint{3.473082in}{3.986194in}}%
\pgfpathlineto{\pgfqpoint{3.481361in}{3.934903in}}%
\pgfpathlineto{\pgfqpoint{3.489638in}{3.934903in}}%
\pgfpathlineto{\pgfqpoint{3.497913in}{4.037486in}}%
\pgfpathlineto{\pgfqpoint{3.506186in}{3.962880in}}%
\pgfpathlineto{\pgfqpoint{3.514464in}{3.986194in}}%
\pgfpathlineto{\pgfqpoint{3.522738in}{4.037486in}}%
\pgfpathlineto{\pgfqpoint{3.531009in}{3.986194in}}%
\pgfpathlineto{\pgfqpoint{3.539287in}{3.986194in}}%
\pgfpathlineto{\pgfqpoint{3.547564in}{4.014171in}}%
\pgfpathlineto{\pgfqpoint{3.555842in}{3.962880in}}%
\pgfpathlineto{\pgfqpoint{3.564116in}{3.962880in}}%
\pgfpathlineto{\pgfqpoint{3.572394in}{3.990857in}}%
\pgfpathlineto{\pgfqpoint{3.588948in}{4.037486in}}%
\pgfpathlineto{\pgfqpoint{3.597219in}{4.037486in}}%
\pgfpathlineto{\pgfqpoint{3.605497in}{4.014171in}}%
\pgfpathlineto{\pgfqpoint{3.613775in}{4.037486in}}%
\pgfpathlineto{\pgfqpoint{3.622052in}{3.986194in}}%
\pgfpathlineto{\pgfqpoint{3.630329in}{3.911589in}}%
\pgfpathlineto{\pgfqpoint{3.638607in}{3.986194in}}%
\pgfpathlineto{\pgfqpoint{3.646885in}{3.990857in}}%
\pgfpathlineto{\pgfqpoint{3.655163in}{4.014171in}}%
\pgfpathlineto{\pgfqpoint{3.663433in}{4.014171in}}%
\pgfpathlineto{\pgfqpoint{3.671711in}{3.990857in}}%
\pgfpathlineto{\pgfqpoint{3.679983in}{4.014171in}}%
\pgfpathlineto{\pgfqpoint{3.688259in}{3.962880in}}%
\pgfpathlineto{\pgfqpoint{3.696532in}{3.986194in}}%
\pgfpathlineto{\pgfqpoint{3.704807in}{3.934903in}}%
\pgfpathlineto{\pgfqpoint{3.721361in}{3.990857in}}%
\pgfpathlineto{\pgfqpoint{3.729639in}{4.037486in}}%
\pgfpathlineto{\pgfqpoint{3.746194in}{4.037486in}}%
\pgfpathlineto{\pgfqpoint{3.754471in}{3.986194in}}%
\pgfpathlineto{\pgfqpoint{3.762749in}{4.014171in}}%
\pgfpathlineto{\pgfqpoint{3.771027in}{3.911589in}}%
\pgfpathlineto{\pgfqpoint{3.779303in}{3.939566in}}%
\pgfpathlineto{\pgfqpoint{3.787574in}{4.014171in}}%
\pgfpathlineto{\pgfqpoint{3.795852in}{3.986194in}}%
\pgfpathlineto{\pgfqpoint{3.804130in}{3.986194in}}%
\pgfpathlineto{\pgfqpoint{3.812407in}{4.014171in}}%
\pgfpathlineto{\pgfqpoint{3.820684in}{4.014171in}}%
\pgfpathlineto{\pgfqpoint{3.828963in}{3.986194in}}%
\pgfpathlineto{\pgfqpoint{3.837241in}{3.986194in}}%
\pgfpathlineto{\pgfqpoint{3.845513in}{3.911589in}}%
\pgfpathlineto{\pgfqpoint{3.853792in}{3.962880in}}%
\pgfpathlineto{\pgfqpoint{3.862069in}{3.962880in}}%
\pgfpathlineto{\pgfqpoint{3.870346in}{4.014171in}}%
\pgfpathlineto{\pgfqpoint{3.878622in}{3.986194in}}%
\pgfpathlineto{\pgfqpoint{3.886900in}{4.014171in}}%
\pgfpathlineto{\pgfqpoint{3.895177in}{3.962880in}}%
\pgfpathlineto{\pgfqpoint{3.903450in}{3.986194in}}%
\pgfpathlineto{\pgfqpoint{3.911723in}{3.990857in}}%
\pgfpathlineto{\pgfqpoint{3.919998in}{3.911589in}}%
\pgfpathlineto{\pgfqpoint{3.928274in}{3.962880in}}%
\pgfpathlineto{\pgfqpoint{3.936551in}{3.986194in}}%
\pgfpathlineto{\pgfqpoint{3.944825in}{4.037486in}}%
\pgfpathlineto{\pgfqpoint{3.953098in}{3.986194in}}%
\pgfpathlineto{\pgfqpoint{3.961375in}{3.962880in}}%
\pgfpathlineto{\pgfqpoint{3.969650in}{4.014171in}}%
\pgfpathlineto{\pgfqpoint{3.977925in}{4.014171in}}%
\pgfpathlineto{\pgfqpoint{3.986197in}{3.939566in}}%
\pgfpathlineto{\pgfqpoint{3.994472in}{3.962880in}}%
\pgfpathlineto{\pgfqpoint{4.002748in}{3.990857in}}%
\pgfpathlineto{\pgfqpoint{4.011024in}{4.037486in}}%
\pgfpathlineto{\pgfqpoint{4.019302in}{3.962880in}}%
\pgfpathlineto{\pgfqpoint{4.027580in}{4.014171in}}%
\pgfpathlineto{\pgfqpoint{4.035851in}{3.962880in}}%
\pgfpathlineto{\pgfqpoint{4.044126in}{4.014171in}}%
\pgfpathlineto{\pgfqpoint{4.052401in}{4.014171in}}%
\pgfpathlineto{\pgfqpoint{4.060678in}{4.037486in}}%
\pgfpathlineto{\pgfqpoint{4.068952in}{3.962880in}}%
\pgfpathlineto{\pgfqpoint{4.077229in}{4.014171in}}%
\pgfpathlineto{\pgfqpoint{4.085505in}{4.014171in}}%
\pgfpathlineto{\pgfqpoint{4.093781in}{4.037486in}}%
\pgfpathlineto{\pgfqpoint{4.102056in}{4.014171in}}%
\pgfpathlineto{\pgfqpoint{4.110334in}{4.014171in}}%
\pgfpathlineto{\pgfqpoint{4.118609in}{3.962880in}}%
\pgfpathlineto{\pgfqpoint{4.126887in}{3.986194in}}%
\pgfpathlineto{\pgfqpoint{4.135165in}{3.939566in}}%
\pgfpathlineto{\pgfqpoint{4.143437in}{3.986194in}}%
\pgfpathlineto{\pgfqpoint{4.151715in}{3.986194in}}%
\pgfpathlineto{\pgfqpoint{4.159992in}{4.014171in}}%
\pgfpathlineto{\pgfqpoint{4.168268in}{4.037486in}}%
\pgfpathlineto{\pgfqpoint{4.176539in}{4.014171in}}%
\pgfpathlineto{\pgfqpoint{4.184816in}{4.037486in}}%
\pgfpathlineto{\pgfqpoint{4.193093in}{4.014171in}}%
\pgfpathlineto{\pgfqpoint{4.201366in}{3.986194in}}%
\pgfpathlineto{\pgfqpoint{4.209644in}{3.962880in}}%
\pgfpathlineto{\pgfqpoint{4.217921in}{3.990857in}}%
\pgfpathlineto{\pgfqpoint{4.234471in}{4.037486in}}%
\pgfpathlineto{\pgfqpoint{4.242748in}{4.014171in}}%
\pgfpathlineto{\pgfqpoint{4.251023in}{4.037486in}}%
\pgfpathlineto{\pgfqpoint{4.259298in}{3.990857in}}%
\pgfpathlineto{\pgfqpoint{4.267576in}{4.037486in}}%
\pgfpathlineto{\pgfqpoint{4.275850in}{3.934903in}}%
\pgfpathlineto{\pgfqpoint{4.284126in}{3.990857in}}%
\pgfpathlineto{\pgfqpoint{4.292402in}{4.037486in}}%
\pgfpathlineto{\pgfqpoint{4.300679in}{4.014171in}}%
\pgfpathlineto{\pgfqpoint{4.308956in}{3.986194in}}%
\pgfpathlineto{\pgfqpoint{4.317234in}{4.014171in}}%
\pgfpathlineto{\pgfqpoint{4.325511in}{4.037486in}}%
\pgfpathlineto{\pgfqpoint{4.333789in}{4.014171in}}%
\pgfpathlineto{\pgfqpoint{4.342066in}{4.037486in}}%
\pgfpathlineto{\pgfqpoint{4.350344in}{3.962880in}}%
\pgfpathlineto{\pgfqpoint{4.358622in}{3.934903in}}%
\pgfpathlineto{\pgfqpoint{4.366896in}{4.014171in}}%
\pgfpathlineto{\pgfqpoint{4.375168in}{4.014171in}}%
\pgfpathlineto{\pgfqpoint{4.383440in}{4.037486in}}%
\pgfpathlineto{\pgfqpoint{4.391717in}{4.014171in}}%
\pgfpathlineto{\pgfqpoint{4.399994in}{4.037486in}}%
\pgfpathlineto{\pgfqpoint{4.416549in}{4.037486in}}%
\pgfpathlineto{\pgfqpoint{4.424823in}{3.911589in}}%
\pgfpathlineto{\pgfqpoint{4.433101in}{4.037486in}}%
\pgfpathlineto{\pgfqpoint{4.466206in}{4.037486in}}%
\pgfpathlineto{\pgfqpoint{4.474476in}{4.060800in}}%
\pgfpathlineto{\pgfqpoint{4.482753in}{4.037486in}}%
\pgfpathlineto{\pgfqpoint{4.491024in}{4.037486in}}%
\pgfpathlineto{\pgfqpoint{4.499300in}{4.014171in}}%
\pgfpathlineto{\pgfqpoint{4.507574in}{4.014171in}}%
\pgfpathlineto{\pgfqpoint{4.515845in}{4.037486in}}%
\pgfpathlineto{\pgfqpoint{4.540674in}{4.037486in}}%
\pgfpathlineto{\pgfqpoint{4.548947in}{4.060800in}}%
\pgfpathlineto{\pgfqpoint{4.565489in}{4.014171in}}%
\pgfpathlineto{\pgfqpoint{4.573767in}{3.911589in}}%
\pgfpathlineto{\pgfqpoint{4.582045in}{3.986194in}}%
\pgfpathlineto{\pgfqpoint{4.590323in}{4.037486in}}%
\pgfpathlineto{\pgfqpoint{4.639976in}{4.037486in}}%
\pgfpathlineto{\pgfqpoint{4.648254in}{3.986194in}}%
\pgfpathlineto{\pgfqpoint{4.656525in}{4.037486in}}%
\pgfpathlineto{\pgfqpoint{4.714458in}{4.037486in}}%
\pgfpathlineto{\pgfqpoint{4.722730in}{3.962880in}}%
\pgfpathlineto{\pgfqpoint{4.731008in}{4.014171in}}%
\pgfpathlineto{\pgfqpoint{4.747555in}{4.060800in}}%
\pgfpathlineto{\pgfqpoint{4.755833in}{4.037486in}}%
\pgfpathlineto{\pgfqpoint{4.788937in}{4.037486in}}%
\pgfpathlineto{\pgfqpoint{4.797216in}{4.014171in}}%
\pgfpathlineto{\pgfqpoint{4.805492in}{4.014171in}}%
\pgfpathlineto{\pgfqpoint{4.813769in}{4.037486in}}%
\pgfpathlineto{\pgfqpoint{4.838600in}{4.037486in}}%
\pgfpathlineto{\pgfqpoint{4.846877in}{4.060800in}}%
\pgfpathlineto{\pgfqpoint{4.855153in}{4.037486in}}%
\pgfpathlineto{\pgfqpoint{4.863428in}{3.986194in}}%
\pgfpathlineto{\pgfqpoint{4.871703in}{3.986194in}}%
\pgfpathlineto{\pgfqpoint{4.879980in}{4.037486in}}%
\pgfpathlineto{\pgfqpoint{4.921367in}{4.037486in}}%
\pgfpathlineto{\pgfqpoint{4.929645in}{4.060800in}}%
\pgfpathlineto{\pgfqpoint{4.937922in}{4.014171in}}%
\pgfpathlineto{\pgfqpoint{4.946196in}{4.014171in}}%
\pgfpathlineto{\pgfqpoint{4.954469in}{4.037486in}}%
\pgfpathlineto{\pgfqpoint{4.971022in}{4.037486in}}%
\pgfpathlineto{\pgfqpoint{4.979295in}{4.060800in}}%
\pgfpathlineto{\pgfqpoint{4.995849in}{4.060800in}}%
\pgfpathlineto{\pgfqpoint{5.004127in}{4.037486in}}%
\pgfpathlineto{\pgfqpoint{5.012398in}{3.986194in}}%
\pgfpathlineto{\pgfqpoint{5.020675in}{4.037486in}}%
\pgfpathlineto{\pgfqpoint{5.070325in}{4.037486in}}%
\pgfpathlineto{\pgfqpoint{5.078599in}{3.962880in}}%
\pgfpathlineto{\pgfqpoint{5.086873in}{3.962880in}}%
\pgfpathlineto{\pgfqpoint{5.095147in}{4.037486in}}%
\pgfpathlineto{\pgfqpoint{5.111701in}{4.037486in}}%
\pgfpathlineto{\pgfqpoint{5.119978in}{4.014171in}}%
\pgfpathlineto{\pgfqpoint{5.128253in}{4.037486in}}%
\pgfpathlineto{\pgfqpoint{5.144802in}{4.037486in}}%
\pgfpathlineto{\pgfqpoint{5.153080in}{3.962880in}}%
\pgfpathlineto{\pgfqpoint{5.161357in}{4.014171in}}%
\pgfpathlineto{\pgfqpoint{5.169634in}{4.014171in}}%
\pgfpathlineto{\pgfqpoint{5.177911in}{4.037486in}}%
\pgfpathlineto{\pgfqpoint{5.202738in}{4.037486in}}%
\pgfpathlineto{\pgfqpoint{5.211016in}{4.014171in}}%
\pgfpathlineto{\pgfqpoint{5.219287in}{4.037486in}}%
\pgfpathlineto{\pgfqpoint{5.227565in}{3.986194in}}%
\pgfpathlineto{\pgfqpoint{5.235843in}{4.037486in}}%
\pgfpathlineto{\pgfqpoint{5.244121in}{4.014171in}}%
\pgfpathlineto{\pgfqpoint{5.252399in}{4.060800in}}%
\pgfpathlineto{\pgfqpoint{5.260670in}{4.014171in}}%
\pgfpathlineto{\pgfqpoint{5.268946in}{4.037486in}}%
\pgfpathlineto{\pgfqpoint{5.277223in}{4.037486in}}%
\pgfpathlineto{\pgfqpoint{5.285500in}{4.014171in}}%
\pgfpathlineto{\pgfqpoint{5.293778in}{4.060800in}}%
\pgfpathlineto{\pgfqpoint{5.302050in}{4.009509in}}%
\pgfpathlineto{\pgfqpoint{5.310328in}{4.037486in}}%
\pgfpathlineto{\pgfqpoint{5.318606in}{4.014171in}}%
\pgfpathlineto{\pgfqpoint{5.326878in}{4.014171in}}%
\pgfpathlineto{\pgfqpoint{5.343434in}{4.060800in}}%
\pgfpathlineto{\pgfqpoint{5.351705in}{4.037486in}}%
\pgfpathlineto{\pgfqpoint{5.359981in}{4.060800in}}%
\pgfpathlineto{\pgfqpoint{5.368254in}{3.986194in}}%
\pgfpathlineto{\pgfqpoint{5.376528in}{3.934903in}}%
\pgfpathlineto{\pgfqpoint{5.384803in}{4.037486in}}%
\pgfpathlineto{\pgfqpoint{5.409635in}{4.037486in}}%
\pgfpathlineto{\pgfqpoint{5.417913in}{3.986194in}}%
\pgfpathlineto{\pgfqpoint{5.426191in}{4.037486in}}%
\pgfpathlineto{\pgfqpoint{5.434469in}{4.014171in}}%
\pgfpathlineto{\pgfqpoint{5.442745in}{3.986194in}}%
\pgfpathlineto{\pgfqpoint{5.451023in}{3.986194in}}%
\pgfpathlineto{\pgfqpoint{5.459297in}{4.037486in}}%
\pgfpathlineto{\pgfqpoint{5.467574in}{4.037486in}}%
\pgfpathlineto{\pgfqpoint{5.475851in}{4.014171in}}%
\pgfpathlineto{\pgfqpoint{5.484128in}{4.060800in}}%
\pgfpathlineto{\pgfqpoint{5.492406in}{4.037486in}}%
\pgfpathlineto{\pgfqpoint{5.508960in}{4.037486in}}%
\pgfpathlineto{\pgfqpoint{5.517238in}{3.510583in}}%
\pgfpathlineto{\pgfqpoint{5.533787in}{3.310080in}}%
\pgfpathlineto{\pgfqpoint{5.542066in}{3.282103in}}%
\pgfpathlineto{\pgfqpoint{5.550339in}{3.282103in}}%
\pgfpathlineto{\pgfqpoint{5.566884in}{3.188846in}}%
\pgfpathlineto{\pgfqpoint{5.575156in}{3.184183in}}%
\pgfpathlineto{\pgfqpoint{5.583428in}{3.132891in}}%
\pgfpathlineto{\pgfqpoint{5.591700in}{3.132891in}}%
\pgfpathlineto{\pgfqpoint{5.599972in}{3.109577in}}%
\pgfpathlineto{\pgfqpoint{5.608251in}{3.132891in}}%
\pgfpathlineto{\pgfqpoint{5.616530in}{3.086263in}}%
\pgfpathlineto{\pgfqpoint{5.633088in}{3.086263in}}%
\pgfpathlineto{\pgfqpoint{5.641368in}{3.081600in}}%
\pgfpathlineto{\pgfqpoint{5.649647in}{3.058286in}}%
\pgfpathlineto{\pgfqpoint{5.657926in}{3.058286in}}%
\pgfpathlineto{\pgfqpoint{5.666205in}{3.030309in}}%
\pgfpathlineto{\pgfqpoint{5.674485in}{3.058286in}}%
\pgfpathlineto{\pgfqpoint{5.682764in}{3.058286in}}%
\pgfpathlineto{\pgfqpoint{5.691040in}{3.030309in}}%
\pgfpathlineto{\pgfqpoint{5.715872in}{3.030309in}}%
\pgfpathlineto{\pgfqpoint{5.732421in}{2.983680in}}%
\pgfpathlineto{\pgfqpoint{5.740695in}{3.030309in}}%
\pgfpathlineto{\pgfqpoint{5.748970in}{2.983680in}}%
\pgfpathlineto{\pgfqpoint{5.757245in}{3.030309in}}%
\pgfpathlineto{\pgfqpoint{5.757245in}{3.030309in}}%
\pgfusepath{stroke}%
\end{pgfscope}%
\begin{pgfscope}%
\pgfpathrectangle{\pgfqpoint{0.800000in}{0.960000in}}{\pgfqpoint{4.960000in}{3.264000in}}%
\pgfusepath{clip}%
\pgfsetrectcap%
\pgfsetroundjoin%
\pgfsetlinewidth{1.505625pt}%
\definecolor{currentstroke}{rgb}{1.000000,0.498039,0.054902}%
\pgfsetstrokecolor{currentstroke}%
\pgfsetdash{}{0pt}%
\pgfpathmoveto{\pgfqpoint{0.790000in}{1.624639in}}%
\pgfpathlineto{\pgfqpoint{0.794114in}{1.627371in}}%
\pgfpathlineto{\pgfqpoint{0.802887in}{1.630286in}}%
\pgfpathlineto{\pgfqpoint{0.811661in}{1.639029in}}%
\pgfpathlineto{\pgfqpoint{0.820434in}{1.641943in}}%
\pgfpathlineto{\pgfqpoint{0.829207in}{1.650686in}}%
\pgfpathlineto{\pgfqpoint{0.837980in}{1.653600in}}%
\pgfpathlineto{\pgfqpoint{0.846754in}{1.662343in}}%
\pgfpathlineto{\pgfqpoint{0.864300in}{1.674000in}}%
\pgfpathlineto{\pgfqpoint{0.873074in}{1.682743in}}%
\pgfpathlineto{\pgfqpoint{0.881846in}{1.685657in}}%
\pgfpathlineto{\pgfqpoint{0.890620in}{1.694400in}}%
\pgfpathlineto{\pgfqpoint{0.899394in}{1.700229in}}%
\pgfpathlineto{\pgfqpoint{0.908161in}{1.708971in}}%
\pgfpathlineto{\pgfqpoint{0.916918in}{1.714800in}}%
\pgfpathlineto{\pgfqpoint{0.925692in}{1.723543in}}%
\pgfpathlineto{\pgfqpoint{0.934465in}{1.729371in}}%
\pgfpathlineto{\pgfqpoint{0.943238in}{1.738114in}}%
\pgfpathlineto{\pgfqpoint{0.960785in}{1.749771in}}%
\pgfpathlineto{\pgfqpoint{0.969558in}{1.758514in}}%
\pgfpathlineto{\pgfqpoint{0.978331in}{1.761429in}}%
\pgfpathlineto{\pgfqpoint{1.004651in}{1.778914in}}%
\pgfpathlineto{\pgfqpoint{1.013419in}{1.787657in}}%
\pgfpathlineto{\pgfqpoint{1.022176in}{1.793486in}}%
\pgfpathlineto{\pgfqpoint{1.030943in}{1.796400in}}%
\pgfpathlineto{\pgfqpoint{1.048474in}{1.808057in}}%
\pgfpathlineto{\pgfqpoint{1.057247in}{1.810971in}}%
\pgfpathlineto{\pgfqpoint{1.074794in}{1.822629in}}%
\pgfpathlineto{\pgfqpoint{1.083567in}{1.822629in}}%
\pgfpathlineto{\pgfqpoint{1.101110in}{1.834286in}}%
\pgfpathlineto{\pgfqpoint{1.153754in}{1.851771in}}%
\pgfpathlineto{\pgfqpoint{1.162527in}{1.857600in}}%
\pgfpathlineto{\pgfqpoint{1.171300in}{1.857600in}}%
\pgfpathlineto{\pgfqpoint{1.180073in}{1.863429in}}%
\pgfpathlineto{\pgfqpoint{1.188847in}{1.863429in}}%
\pgfpathlineto{\pgfqpoint{1.215167in}{1.872171in}}%
\pgfpathlineto{\pgfqpoint{1.223934in}{1.872171in}}%
\pgfpathlineto{\pgfqpoint{1.232691in}{1.878000in}}%
\pgfpathlineto{\pgfqpoint{1.250216in}{1.878000in}}%
\pgfpathlineto{\pgfqpoint{1.258989in}{1.883829in}}%
\pgfpathlineto{\pgfqpoint{1.267762in}{1.883829in}}%
\pgfpathlineto{\pgfqpoint{1.294082in}{1.892571in}}%
\pgfpathlineto{\pgfqpoint{1.311626in}{1.892571in}}%
\pgfpathlineto{\pgfqpoint{1.320402in}{1.895486in}}%
\pgfpathlineto{\pgfqpoint{1.329176in}{1.895486in}}%
\pgfpathlineto{\pgfqpoint{1.346722in}{1.901314in}}%
\pgfpathlineto{\pgfqpoint{1.364269in}{1.901314in}}%
\pgfpathlineto{\pgfqpoint{1.381815in}{1.907143in}}%
\pgfpathlineto{\pgfqpoint{1.390589in}{1.907143in}}%
\pgfpathlineto{\pgfqpoint{1.399362in}{1.910057in}}%
\pgfpathlineto{\pgfqpoint{1.416909in}{1.910057in}}%
\pgfpathlineto{\pgfqpoint{1.434449in}{1.915886in}}%
\pgfpathlineto{\pgfqpoint{1.460753in}{1.915886in}}%
\pgfpathlineto{\pgfqpoint{1.469527in}{1.918800in}}%
\pgfpathlineto{\pgfqpoint{1.487073in}{1.918800in}}%
\pgfpathlineto{\pgfqpoint{1.504620in}{1.924629in}}%
\pgfpathlineto{\pgfqpoint{1.530940in}{1.924629in}}%
\pgfpathlineto{\pgfqpoint{1.539707in}{1.927543in}}%
\pgfpathlineto{\pgfqpoint{1.557238in}{1.927543in}}%
\pgfpathlineto{\pgfqpoint{1.566011in}{1.930457in}}%
\pgfpathlineto{\pgfqpoint{1.574784in}{1.930457in}}%
\pgfpathlineto{\pgfqpoint{1.583558in}{1.933371in}}%
\pgfpathlineto{\pgfqpoint{1.618629in}{1.933371in}}%
\pgfpathlineto{\pgfqpoint{1.627399in}{1.936286in}}%
\pgfpathlineto{\pgfqpoint{1.653722in}{1.936286in}}%
\pgfpathlineto{\pgfqpoint{1.662495in}{1.939200in}}%
\pgfpathlineto{\pgfqpoint{1.671269in}{1.936286in}}%
\pgfpathlineto{\pgfqpoint{1.680042in}{1.942114in}}%
\pgfpathlineto{\pgfqpoint{1.723908in}{1.942114in}}%
\pgfpathlineto{\pgfqpoint{1.741455in}{1.947943in}}%
\pgfpathlineto{\pgfqpoint{1.767753in}{1.947943in}}%
\pgfpathlineto{\pgfqpoint{1.776526in}{1.950857in}}%
\pgfpathlineto{\pgfqpoint{1.794073in}{1.950857in}}%
\pgfpathlineto{\pgfqpoint{1.802846in}{1.953771in}}%
\pgfpathlineto{\pgfqpoint{1.811620in}{1.953771in}}%
\pgfpathlineto{\pgfqpoint{1.820393in}{1.956686in}}%
\pgfpathlineto{\pgfqpoint{1.829166in}{1.956686in}}%
\pgfpathlineto{\pgfqpoint{1.837939in}{1.959600in}}%
\pgfpathlineto{\pgfqpoint{1.855480in}{1.959600in}}%
\pgfpathlineto{\pgfqpoint{1.864237in}{1.962514in}}%
\pgfpathlineto{\pgfqpoint{1.881784in}{1.962514in}}%
\pgfpathlineto{\pgfqpoint{1.890557in}{1.965429in}}%
\pgfpathlineto{\pgfqpoint{1.908104in}{1.965429in}}%
\pgfpathlineto{\pgfqpoint{1.925647in}{1.971257in}}%
\pgfpathlineto{\pgfqpoint{1.951970in}{1.971257in}}%
\pgfpathlineto{\pgfqpoint{1.960738in}{1.974171in}}%
\pgfpathlineto{\pgfqpoint{1.978268in}{1.974171in}}%
\pgfpathlineto{\pgfqpoint{1.987042in}{1.980000in}}%
\pgfpathlineto{\pgfqpoint{2.022135in}{1.980000in}}%
\pgfpathlineto{\pgfqpoint{2.030908in}{1.982914in}}%
\pgfpathlineto{\pgfqpoint{2.039681in}{1.982914in}}%
\pgfpathlineto{\pgfqpoint{2.048455in}{1.985829in}}%
\pgfpathlineto{\pgfqpoint{2.057228in}{1.982914in}}%
\pgfpathlineto{\pgfqpoint{2.074753in}{1.988743in}}%
\pgfpathlineto{\pgfqpoint{2.083526in}{1.985829in}}%
\pgfpathlineto{\pgfqpoint{2.092299in}{1.991657in}}%
\pgfpathlineto{\pgfqpoint{2.118619in}{1.991657in}}%
\pgfpathlineto{\pgfqpoint{2.127393in}{1.994571in}}%
\pgfpathlineto{\pgfqpoint{2.144939in}{1.994571in}}%
\pgfpathlineto{\pgfqpoint{2.153712in}{1.997486in}}%
\pgfpathlineto{\pgfqpoint{2.162486in}{1.997486in}}%
\pgfpathlineto{\pgfqpoint{2.171253in}{2.000400in}}%
\pgfpathlineto{\pgfqpoint{2.197533in}{2.000400in}}%
\pgfpathlineto{\pgfqpoint{2.206308in}{2.003314in}}%
\pgfpathlineto{\pgfqpoint{2.232628in}{2.003314in}}%
\pgfpathlineto{\pgfqpoint{2.241401in}{2.006229in}}%
\pgfpathlineto{\pgfqpoint{2.250175in}{2.006229in}}%
\pgfpathlineto{\pgfqpoint{2.258945in}{2.009143in}}%
\pgfpathlineto{\pgfqpoint{2.302815in}{2.009143in}}%
\pgfpathlineto{\pgfqpoint{2.311588in}{2.012057in}}%
\pgfpathlineto{\pgfqpoint{2.337908in}{2.012057in}}%
\pgfpathlineto{\pgfqpoint{2.346681in}{2.014971in}}%
\pgfpathlineto{\pgfqpoint{2.417107in}{2.014971in}}%
\pgfpathlineto{\pgfqpoint{2.425883in}{2.017886in}}%
\pgfpathlineto{\pgfqpoint{2.434657in}{2.017886in}}%
\pgfpathlineto{\pgfqpoint{2.443430in}{2.020800in}}%
\pgfpathlineto{\pgfqpoint{2.452203in}{2.017886in}}%
\pgfpathlineto{\pgfqpoint{2.460977in}{2.020800in}}%
\pgfpathlineto{\pgfqpoint{2.539930in}{2.020800in}}%
\pgfpathlineto{\pgfqpoint{2.548688in}{2.023714in}}%
\pgfpathlineto{\pgfqpoint{2.557461in}{2.020800in}}%
\pgfpathlineto{\pgfqpoint{2.566234in}{2.020800in}}%
\pgfpathlineto{\pgfqpoint{2.575008in}{2.023714in}}%
\pgfpathlineto{\pgfqpoint{2.583781in}{2.020800in}}%
\pgfpathlineto{\pgfqpoint{2.592554in}{2.023714in}}%
\pgfpathlineto{\pgfqpoint{2.601328in}{2.020800in}}%
\pgfpathlineto{\pgfqpoint{2.618874in}{2.020800in}}%
\pgfpathlineto{\pgfqpoint{2.627648in}{2.023714in}}%
\pgfpathlineto{\pgfqpoint{2.636421in}{2.023714in}}%
\pgfpathlineto{\pgfqpoint{2.645188in}{2.026629in}}%
\pgfpathlineto{\pgfqpoint{2.662719in}{2.026629in}}%
\pgfpathlineto{\pgfqpoint{2.671492in}{2.023714in}}%
\pgfpathlineto{\pgfqpoint{2.680265in}{2.026629in}}%
\pgfpathlineto{\pgfqpoint{2.750446in}{2.026629in}}%
\pgfpathlineto{\pgfqpoint{2.759203in}{2.023714in}}%
\pgfpathlineto{\pgfqpoint{2.776750in}{2.029543in}}%
\pgfpathlineto{\pgfqpoint{2.811843in}{2.029543in}}%
\pgfpathlineto{\pgfqpoint{2.820616in}{2.026629in}}%
\pgfpathlineto{\pgfqpoint{2.829389in}{2.029543in}}%
\pgfpathlineto{\pgfqpoint{2.838163in}{2.029543in}}%
\pgfpathlineto{\pgfqpoint{2.846936in}{2.026629in}}%
\pgfpathlineto{\pgfqpoint{2.855703in}{2.029543in}}%
\pgfpathlineto{\pgfqpoint{2.864461in}{2.026629in}}%
\pgfpathlineto{\pgfqpoint{3.013585in}{2.026629in}}%
\pgfpathlineto{\pgfqpoint{3.022358in}{2.023714in}}%
\pgfpathlineto{\pgfqpoint{3.101296in}{2.023714in}}%
\pgfpathlineto{\pgfqpoint{3.110069in}{2.020800in}}%
\pgfpathlineto{\pgfqpoint{3.171476in}{2.020800in}}%
\pgfpathlineto{\pgfqpoint{3.180234in}{2.014971in}}%
\pgfpathlineto{\pgfqpoint{3.189007in}{2.017886in}}%
\pgfpathlineto{\pgfqpoint{3.197780in}{2.014971in}}%
\pgfpathlineto{\pgfqpoint{3.206554in}{2.017886in}}%
\pgfpathlineto{\pgfqpoint{3.215326in}{2.014971in}}%
\pgfpathlineto{\pgfqpoint{3.259194in}{2.014971in}}%
\pgfpathlineto{\pgfqpoint{3.267967in}{2.012057in}}%
\pgfpathlineto{\pgfqpoint{3.285491in}{2.012057in}}%
\pgfpathlineto{\pgfqpoint{3.294265in}{2.009143in}}%
\pgfpathlineto{\pgfqpoint{3.355677in}{2.009143in}}%
\pgfpathlineto{\pgfqpoint{3.364451in}{2.006229in}}%
\pgfpathlineto{\pgfqpoint{3.417069in}{2.006229in}}%
\pgfpathlineto{\pgfqpoint{3.425842in}{2.003314in}}%
\pgfpathlineto{\pgfqpoint{3.460935in}{2.003314in}}%
\pgfpathlineto{\pgfqpoint{3.469709in}{2.000400in}}%
\pgfpathlineto{\pgfqpoint{3.513553in}{2.000400in}}%
\pgfpathlineto{\pgfqpoint{3.522327in}{1.997486in}}%
\pgfpathlineto{\pgfqpoint{3.601287in}{1.997486in}}%
\pgfpathlineto{\pgfqpoint{3.610060in}{1.994571in}}%
\pgfpathlineto{\pgfqpoint{3.618833in}{1.997486in}}%
\pgfpathlineto{\pgfqpoint{3.636380in}{1.991657in}}%
\pgfpathlineto{\pgfqpoint{3.662700in}{1.991657in}}%
\pgfpathlineto{\pgfqpoint{3.671473in}{1.994571in}}%
\pgfpathlineto{\pgfqpoint{3.706566in}{1.994571in}}%
\pgfpathlineto{\pgfqpoint{3.715340in}{1.991657in}}%
\pgfpathlineto{\pgfqpoint{3.724113in}{1.994571in}}%
\pgfpathlineto{\pgfqpoint{3.732886in}{1.991657in}}%
\pgfpathlineto{\pgfqpoint{3.776752in}{1.991657in}}%
\pgfpathlineto{\pgfqpoint{3.785525in}{1.994571in}}%
\pgfpathlineto{\pgfqpoint{3.794299in}{1.991657in}}%
\pgfpathlineto{\pgfqpoint{3.803073in}{1.994571in}}%
\pgfpathlineto{\pgfqpoint{3.811840in}{1.991657in}}%
\pgfpathlineto{\pgfqpoint{3.820597in}{1.994571in}}%
\pgfpathlineto{\pgfqpoint{3.829607in}{1.991657in}}%
\pgfpathlineto{\pgfqpoint{3.838361in}{1.994571in}}%
\pgfpathlineto{\pgfqpoint{3.847138in}{1.991657in}}%
\pgfpathlineto{\pgfqpoint{3.855911in}{1.991657in}}%
\pgfpathlineto{\pgfqpoint{3.864684in}{1.994571in}}%
\pgfpathlineto{\pgfqpoint{3.917324in}{1.994571in}}%
\pgfpathlineto{\pgfqpoint{3.926098in}{1.997486in}}%
\pgfpathlineto{\pgfqpoint{3.969942in}{1.997486in}}%
\pgfpathlineto{\pgfqpoint{3.978715in}{2.000400in}}%
\pgfpathlineto{\pgfqpoint{3.996262in}{2.000400in}}%
\pgfpathlineto{\pgfqpoint{4.005029in}{2.003314in}}%
\pgfpathlineto{\pgfqpoint{4.040106in}{2.003314in}}%
\pgfpathlineto{\pgfqpoint{4.048876in}{2.006229in}}%
\pgfpathlineto{\pgfqpoint{4.083951in}{2.006229in}}%
\pgfpathlineto{\pgfqpoint{4.092724in}{2.009143in}}%
\pgfpathlineto{\pgfqpoint{4.101497in}{2.009143in}}%
\pgfpathlineto{\pgfqpoint{4.110271in}{2.012057in}}%
\pgfpathlineto{\pgfqpoint{4.119044in}{2.012057in}}%
\pgfpathlineto{\pgfqpoint{4.127817in}{2.014971in}}%
\pgfpathlineto{\pgfqpoint{4.162911in}{2.014971in}}%
\pgfpathlineto{\pgfqpoint{4.180457in}{2.020800in}}%
\pgfpathlineto{\pgfqpoint{4.206777in}{2.020800in}}%
\pgfpathlineto{\pgfqpoint{4.215573in}{2.023714in}}%
\pgfpathlineto{\pgfqpoint{4.241892in}{2.023714in}}%
\pgfpathlineto{\pgfqpoint{4.259439in}{2.029543in}}%
\pgfpathlineto{\pgfqpoint{4.294532in}{2.029543in}}%
\pgfpathlineto{\pgfqpoint{4.303306in}{2.032457in}}%
\pgfpathlineto{\pgfqpoint{4.312079in}{2.032457in}}%
\pgfpathlineto{\pgfqpoint{4.320852in}{2.035371in}}%
\pgfpathlineto{\pgfqpoint{4.355942in}{2.035371in}}%
\pgfpathlineto{\pgfqpoint{4.364718in}{2.038286in}}%
\pgfpathlineto{\pgfqpoint{4.382265in}{2.038286in}}%
\pgfpathlineto{\pgfqpoint{4.391033in}{2.041200in}}%
\pgfpathlineto{\pgfqpoint{4.408563in}{2.041200in}}%
\pgfpathlineto{\pgfqpoint{4.417337in}{2.044114in}}%
\pgfpathlineto{\pgfqpoint{4.469976in}{2.044114in}}%
\pgfpathlineto{\pgfqpoint{4.478750in}{2.047029in}}%
\pgfpathlineto{\pgfqpoint{4.505048in}{2.047029in}}%
\pgfpathlineto{\pgfqpoint{4.513821in}{2.049943in}}%
\pgfpathlineto{\pgfqpoint{4.575234in}{2.049943in}}%
\pgfpathlineto{\pgfqpoint{4.584007in}{2.052857in}}%
\pgfpathlineto{\pgfqpoint{4.636647in}{2.052857in}}%
\pgfpathlineto{\pgfqpoint{4.645421in}{2.055771in}}%
\pgfpathlineto{\pgfqpoint{4.654194in}{2.052857in}}%
\pgfpathlineto{\pgfqpoint{4.662967in}{2.052857in}}%
\pgfpathlineto{\pgfqpoint{4.671740in}{2.055771in}}%
\pgfpathlineto{\pgfqpoint{4.750943in}{2.055771in}}%
\pgfpathlineto{\pgfqpoint{4.759716in}{2.058686in}}%
\pgfpathlineto{\pgfqpoint{4.856200in}{2.058686in}}%
\pgfpathlineto{\pgfqpoint{4.864974in}{2.055771in}}%
\pgfpathlineto{\pgfqpoint{4.873747in}{2.058686in}}%
\pgfpathlineto{\pgfqpoint{4.952707in}{2.058686in}}%
\pgfpathlineto{\pgfqpoint{4.961480in}{2.055771in}}%
\pgfpathlineto{\pgfqpoint{4.970253in}{2.058686in}}%
\pgfpathlineto{\pgfqpoint{4.987800in}{2.058686in}}%
\pgfpathlineto{\pgfqpoint{4.996573in}{2.055771in}}%
\pgfpathlineto{\pgfqpoint{5.040462in}{2.055771in}}%
\pgfpathlineto{\pgfqpoint{5.049235in}{2.052857in}}%
\pgfpathlineto{\pgfqpoint{5.058009in}{2.055771in}}%
\pgfpathlineto{\pgfqpoint{5.145742in}{2.055771in}}%
\pgfpathlineto{\pgfqpoint{5.154515in}{2.052857in}}%
\pgfpathlineto{\pgfqpoint{5.163288in}{2.052857in}}%
\pgfpathlineto{\pgfqpoint{5.172084in}{2.055771in}}%
\pgfpathlineto{\pgfqpoint{5.180857in}{2.049943in}}%
\pgfpathlineto{\pgfqpoint{5.189630in}{2.055771in}}%
\pgfpathlineto{\pgfqpoint{5.207177in}{2.049943in}}%
\pgfpathlineto{\pgfqpoint{5.365339in}{2.049943in}}%
\pgfpathlineto{\pgfqpoint{5.374112in}{2.047029in}}%
\pgfpathlineto{\pgfqpoint{5.382885in}{2.049943in}}%
\pgfpathlineto{\pgfqpoint{5.391659in}{2.049943in}}%
\pgfpathlineto{\pgfqpoint{5.400432in}{2.044114in}}%
\pgfpathlineto{\pgfqpoint{5.496917in}{2.044114in}}%
\pgfpathlineto{\pgfqpoint{5.505690in}{2.041200in}}%
\pgfpathlineto{\pgfqpoint{5.523236in}{2.041200in}}%
\pgfpathlineto{\pgfqpoint{5.532010in}{2.038286in}}%
\pgfpathlineto{\pgfqpoint{5.540783in}{2.038286in}}%
\pgfpathlineto{\pgfqpoint{5.584628in}{2.023714in}}%
\pgfpathlineto{\pgfqpoint{5.602174in}{2.012057in}}%
\pgfpathlineto{\pgfqpoint{5.619721in}{2.006229in}}%
\pgfpathlineto{\pgfqpoint{5.646040in}{1.988743in}}%
\pgfpathlineto{\pgfqpoint{5.654813in}{1.985829in}}%
\pgfpathlineto{\pgfqpoint{5.663587in}{1.977086in}}%
\pgfpathlineto{\pgfqpoint{5.672361in}{1.977086in}}%
\pgfpathlineto{\pgfqpoint{5.707432in}{1.953771in}}%
\pgfpathlineto{\pgfqpoint{5.724978in}{1.947943in}}%
\pgfpathlineto{\pgfqpoint{5.742525in}{1.936286in}}%
\pgfpathlineto{\pgfqpoint{5.751298in}{1.936286in}}%
\pgfpathlineto{\pgfqpoint{5.760071in}{1.930457in}}%
\pgfpathlineto{\pgfqpoint{5.770000in}{1.927159in}}%
\pgfpathlineto{\pgfqpoint{5.770000in}{1.927159in}}%
\pgfusepath{stroke}%
\end{pgfscope}%
\begin{pgfscope}%
\pgfpathrectangle{\pgfqpoint{0.800000in}{0.960000in}}{\pgfqpoint{4.960000in}{3.264000in}}%
\pgfusepath{clip}%
\pgfsetrectcap%
\pgfsetroundjoin%
\pgfsetlinewidth{1.505625pt}%
\definecolor{currentstroke}{rgb}{0.172549,0.627451,0.172549}%
\pgfsetstrokecolor{currentstroke}%
\pgfsetdash{}{0pt}%
\pgfpathmoveto{\pgfqpoint{0.790000in}{1.261720in}}%
\pgfpathlineto{\pgfqpoint{0.794114in}{1.263086in}}%
\pgfpathlineto{\pgfqpoint{0.802887in}{1.260171in}}%
\pgfpathlineto{\pgfqpoint{0.811661in}{1.263086in}}%
\pgfpathlineto{\pgfqpoint{0.837980in}{1.263086in}}%
\pgfpathlineto{\pgfqpoint{0.846754in}{1.266000in}}%
\pgfpathlineto{\pgfqpoint{0.864300in}{1.266000in}}%
\pgfpathlineto{\pgfqpoint{0.873074in}{1.271829in}}%
\pgfpathlineto{\pgfqpoint{0.881846in}{1.271829in}}%
\pgfpathlineto{\pgfqpoint{0.890620in}{1.274743in}}%
\pgfpathlineto{\pgfqpoint{0.908161in}{1.274743in}}%
\pgfpathlineto{\pgfqpoint{0.925692in}{1.280571in}}%
\pgfpathlineto{\pgfqpoint{0.943238in}{1.280571in}}%
\pgfpathlineto{\pgfqpoint{0.952011in}{1.286400in}}%
\pgfpathlineto{\pgfqpoint{0.960785in}{1.289314in}}%
\pgfpathlineto{\pgfqpoint{0.978331in}{1.289314in}}%
\pgfpathlineto{\pgfqpoint{0.987104in}{1.295143in}}%
\pgfpathlineto{\pgfqpoint{1.013419in}{1.295143in}}%
\pgfpathlineto{\pgfqpoint{1.022176in}{1.300971in}}%
\pgfpathlineto{\pgfqpoint{1.048474in}{1.300971in}}%
\pgfpathlineto{\pgfqpoint{1.057247in}{1.306800in}}%
\pgfpathlineto{\pgfqpoint{1.083567in}{1.306800in}}%
\pgfpathlineto{\pgfqpoint{1.101110in}{1.312629in}}%
\pgfpathlineto{\pgfqpoint{1.109887in}{1.312629in}}%
\pgfpathlineto{\pgfqpoint{1.118660in}{1.315543in}}%
\pgfpathlineto{\pgfqpoint{1.144980in}{1.315543in}}%
\pgfpathlineto{\pgfqpoint{1.153754in}{1.321371in}}%
\pgfpathlineto{\pgfqpoint{1.171300in}{1.321371in}}%
\pgfpathlineto{\pgfqpoint{1.180073in}{1.324286in}}%
\pgfpathlineto{\pgfqpoint{1.188847in}{1.324286in}}%
\pgfpathlineto{\pgfqpoint{1.197620in}{1.327200in}}%
\pgfpathlineto{\pgfqpoint{1.206393in}{1.327200in}}%
\pgfpathlineto{\pgfqpoint{1.215167in}{1.330114in}}%
\pgfpathlineto{\pgfqpoint{1.258989in}{1.330114in}}%
\pgfpathlineto{\pgfqpoint{1.267762in}{1.333029in}}%
\pgfpathlineto{\pgfqpoint{1.276536in}{1.333029in}}%
\pgfpathlineto{\pgfqpoint{1.285309in}{1.335943in}}%
\pgfpathlineto{\pgfqpoint{1.311626in}{1.335943in}}%
\pgfpathlineto{\pgfqpoint{1.320402in}{1.341771in}}%
\pgfpathlineto{\pgfqpoint{1.329176in}{1.341771in}}%
\pgfpathlineto{\pgfqpoint{1.337949in}{1.338857in}}%
\pgfpathlineto{\pgfqpoint{1.346722in}{1.341771in}}%
\pgfpathlineto{\pgfqpoint{1.399362in}{1.341771in}}%
\pgfpathlineto{\pgfqpoint{1.408135in}{1.344686in}}%
\pgfpathlineto{\pgfqpoint{1.416909in}{1.341771in}}%
\pgfpathlineto{\pgfqpoint{1.425682in}{1.341771in}}%
\pgfpathlineto{\pgfqpoint{1.434449in}{1.344686in}}%
\pgfpathlineto{\pgfqpoint{1.504620in}{1.344686in}}%
\pgfpathlineto{\pgfqpoint{1.513393in}{1.347600in}}%
\pgfpathlineto{\pgfqpoint{1.601098in}{1.347600in}}%
\pgfpathlineto{\pgfqpoint{1.609855in}{1.350514in}}%
\pgfpathlineto{\pgfqpoint{1.618629in}{1.350514in}}%
\pgfpathlineto{\pgfqpoint{1.627399in}{1.353429in}}%
\pgfpathlineto{\pgfqpoint{1.671269in}{1.353429in}}%
\pgfpathlineto{\pgfqpoint{1.680042in}{1.356343in}}%
\pgfpathlineto{\pgfqpoint{1.715135in}{1.356343in}}%
\pgfpathlineto{\pgfqpoint{1.723908in}{1.359257in}}%
\pgfpathlineto{\pgfqpoint{1.732682in}{1.359257in}}%
\pgfpathlineto{\pgfqpoint{1.741455in}{1.362171in}}%
\pgfpathlineto{\pgfqpoint{1.758980in}{1.362171in}}%
\pgfpathlineto{\pgfqpoint{1.767753in}{1.365086in}}%
\pgfpathlineto{\pgfqpoint{1.785300in}{1.365086in}}%
\pgfpathlineto{\pgfqpoint{1.794073in}{1.368000in}}%
\pgfpathlineto{\pgfqpoint{1.820393in}{1.368000in}}%
\pgfpathlineto{\pgfqpoint{1.829166in}{1.373829in}}%
\pgfpathlineto{\pgfqpoint{1.837939in}{1.373829in}}%
\pgfpathlineto{\pgfqpoint{1.846713in}{1.376743in}}%
\pgfpathlineto{\pgfqpoint{1.855480in}{1.373829in}}%
\pgfpathlineto{\pgfqpoint{1.864237in}{1.373829in}}%
\pgfpathlineto{\pgfqpoint{1.873011in}{1.376743in}}%
\pgfpathlineto{\pgfqpoint{1.899331in}{1.376743in}}%
\pgfpathlineto{\pgfqpoint{1.908104in}{1.379657in}}%
\pgfpathlineto{\pgfqpoint{1.925647in}{1.379657in}}%
\pgfpathlineto{\pgfqpoint{1.934423in}{1.382571in}}%
\pgfpathlineto{\pgfqpoint{1.951970in}{1.382571in}}%
\pgfpathlineto{\pgfqpoint{1.960738in}{1.385486in}}%
\pgfpathlineto{\pgfqpoint{1.978268in}{1.385486in}}%
\pgfpathlineto{\pgfqpoint{1.987042in}{1.388400in}}%
\pgfpathlineto{\pgfqpoint{1.995815in}{1.388400in}}%
\pgfpathlineto{\pgfqpoint{2.004588in}{1.385486in}}%
\pgfpathlineto{\pgfqpoint{2.013362in}{1.391314in}}%
\pgfpathlineto{\pgfqpoint{2.022135in}{1.388400in}}%
\pgfpathlineto{\pgfqpoint{2.030908in}{1.388400in}}%
\pgfpathlineto{\pgfqpoint{2.039681in}{1.394229in}}%
\pgfpathlineto{\pgfqpoint{2.048455in}{1.391314in}}%
\pgfpathlineto{\pgfqpoint{2.057228in}{1.391314in}}%
\pgfpathlineto{\pgfqpoint{2.065995in}{1.394229in}}%
\pgfpathlineto{\pgfqpoint{2.083526in}{1.394229in}}%
\pgfpathlineto{\pgfqpoint{2.092299in}{1.397143in}}%
\pgfpathlineto{\pgfqpoint{2.118619in}{1.397143in}}%
\pgfpathlineto{\pgfqpoint{2.127393in}{1.400057in}}%
\pgfpathlineto{\pgfqpoint{2.162486in}{1.400057in}}%
\pgfpathlineto{\pgfqpoint{2.171253in}{1.402971in}}%
\pgfpathlineto{\pgfqpoint{2.197533in}{1.402971in}}%
\pgfpathlineto{\pgfqpoint{2.206308in}{1.400057in}}%
\pgfpathlineto{\pgfqpoint{2.276495in}{1.400057in}}%
\pgfpathlineto{\pgfqpoint{2.285268in}{1.402971in}}%
\pgfpathlineto{\pgfqpoint{2.294041in}{1.400057in}}%
\pgfpathlineto{\pgfqpoint{2.478523in}{1.400057in}}%
\pgfpathlineto{\pgfqpoint{2.487297in}{1.405886in}}%
\pgfpathlineto{\pgfqpoint{2.496070in}{1.405886in}}%
\pgfpathlineto{\pgfqpoint{2.504843in}{1.402971in}}%
\pgfpathlineto{\pgfqpoint{2.513616in}{1.405886in}}%
\pgfpathlineto{\pgfqpoint{2.531163in}{1.400057in}}%
\pgfpathlineto{\pgfqpoint{2.539930in}{1.402971in}}%
\pgfpathlineto{\pgfqpoint{2.548688in}{1.402971in}}%
\pgfpathlineto{\pgfqpoint{2.557461in}{1.400057in}}%
\pgfpathlineto{\pgfqpoint{2.627648in}{1.400057in}}%
\pgfpathlineto{\pgfqpoint{2.636421in}{1.397143in}}%
\pgfpathlineto{\pgfqpoint{2.653945in}{1.397143in}}%
\pgfpathlineto{\pgfqpoint{2.662719in}{1.394229in}}%
\pgfpathlineto{\pgfqpoint{2.689039in}{1.394229in}}%
\pgfpathlineto{\pgfqpoint{2.697812in}{1.397143in}}%
\pgfpathlineto{\pgfqpoint{2.706585in}{1.394229in}}%
\pgfpathlineto{\pgfqpoint{2.741679in}{1.394229in}}%
\pgfpathlineto{\pgfqpoint{2.750446in}{1.391314in}}%
\pgfpathlineto{\pgfqpoint{2.829389in}{1.391314in}}%
\pgfpathlineto{\pgfqpoint{2.838163in}{1.388400in}}%
\pgfpathlineto{\pgfqpoint{2.846936in}{1.391314in}}%
\pgfpathlineto{\pgfqpoint{2.855703in}{1.388400in}}%
\pgfpathlineto{\pgfqpoint{2.960961in}{1.388400in}}%
\pgfpathlineto{\pgfqpoint{2.969718in}{1.385486in}}%
\pgfpathlineto{\pgfqpoint{2.996038in}{1.385486in}}%
\pgfpathlineto{\pgfqpoint{3.004812in}{1.382571in}}%
\pgfpathlineto{\pgfqpoint{3.048678in}{1.382571in}}%
\pgfpathlineto{\pgfqpoint{3.057452in}{1.379657in}}%
\pgfpathlineto{\pgfqpoint{3.066219in}{1.379657in}}%
\pgfpathlineto{\pgfqpoint{3.074975in}{1.376743in}}%
\pgfpathlineto{\pgfqpoint{3.110069in}{1.376743in}}%
\pgfpathlineto{\pgfqpoint{3.118843in}{1.373829in}}%
\pgfpathlineto{\pgfqpoint{3.145162in}{1.373829in}}%
\pgfpathlineto{\pgfqpoint{3.153936in}{1.370914in}}%
\pgfpathlineto{\pgfqpoint{3.162709in}{1.370914in}}%
\pgfpathlineto{\pgfqpoint{3.171476in}{1.368000in}}%
\pgfpathlineto{\pgfqpoint{3.206554in}{1.368000in}}%
\pgfpathlineto{\pgfqpoint{3.215326in}{1.365086in}}%
\pgfpathlineto{\pgfqpoint{3.224100in}{1.365086in}}%
\pgfpathlineto{\pgfqpoint{3.232874in}{1.362171in}}%
\pgfpathlineto{\pgfqpoint{3.241647in}{1.365086in}}%
\pgfpathlineto{\pgfqpoint{3.250420in}{1.362171in}}%
\pgfpathlineto{\pgfqpoint{3.267967in}{1.362171in}}%
\pgfpathlineto{\pgfqpoint{3.276734in}{1.359257in}}%
\pgfpathlineto{\pgfqpoint{3.320585in}{1.359257in}}%
\pgfpathlineto{\pgfqpoint{3.338131in}{1.353429in}}%
\pgfpathlineto{\pgfqpoint{3.355677in}{1.353429in}}%
\pgfpathlineto{\pgfqpoint{3.364451in}{1.350514in}}%
\pgfpathlineto{\pgfqpoint{3.399522in}{1.350514in}}%
\pgfpathlineto{\pgfqpoint{3.408296in}{1.347600in}}%
\pgfpathlineto{\pgfqpoint{3.460935in}{1.347600in}}%
\pgfpathlineto{\pgfqpoint{3.469709in}{1.344686in}}%
\pgfpathlineto{\pgfqpoint{3.478482in}{1.344686in}}%
\pgfpathlineto{\pgfqpoint{3.487249in}{1.341771in}}%
\pgfpathlineto{\pgfqpoint{3.513553in}{1.341771in}}%
\pgfpathlineto{\pgfqpoint{3.522327in}{1.338857in}}%
\pgfpathlineto{\pgfqpoint{3.548647in}{1.338857in}}%
\pgfpathlineto{\pgfqpoint{3.557420in}{1.335943in}}%
\pgfpathlineto{\pgfqpoint{3.583740in}{1.335943in}}%
\pgfpathlineto{\pgfqpoint{3.592510in}{1.338857in}}%
\pgfpathlineto{\pgfqpoint{3.610060in}{1.338857in}}%
\pgfpathlineto{\pgfqpoint{3.618833in}{1.335943in}}%
\pgfpathlineto{\pgfqpoint{3.636380in}{1.335943in}}%
\pgfpathlineto{\pgfqpoint{3.645153in}{1.338857in}}%
\pgfpathlineto{\pgfqpoint{3.653926in}{1.338857in}}%
\pgfpathlineto{\pgfqpoint{3.662700in}{1.335943in}}%
\pgfpathlineto{\pgfqpoint{3.671473in}{1.338857in}}%
\pgfpathlineto{\pgfqpoint{3.680246in}{1.335943in}}%
\pgfpathlineto{\pgfqpoint{3.697793in}{1.335943in}}%
\pgfpathlineto{\pgfqpoint{3.706566in}{1.338857in}}%
\pgfpathlineto{\pgfqpoint{3.794299in}{1.338857in}}%
\pgfpathlineto{\pgfqpoint{3.803073in}{1.341771in}}%
\pgfpathlineto{\pgfqpoint{3.811840in}{1.338857in}}%
\pgfpathlineto{\pgfqpoint{3.829607in}{1.338857in}}%
\pgfpathlineto{\pgfqpoint{3.838361in}{1.341771in}}%
\pgfpathlineto{\pgfqpoint{3.847138in}{1.341771in}}%
\pgfpathlineto{\pgfqpoint{3.855911in}{1.344686in}}%
\pgfpathlineto{\pgfqpoint{3.864684in}{1.341771in}}%
\pgfpathlineto{\pgfqpoint{3.873458in}{1.344686in}}%
\pgfpathlineto{\pgfqpoint{3.899777in}{1.344686in}}%
\pgfpathlineto{\pgfqpoint{3.908551in}{1.347600in}}%
\pgfpathlineto{\pgfqpoint{3.934870in}{1.347600in}}%
\pgfpathlineto{\pgfqpoint{3.943644in}{1.350514in}}%
\pgfpathlineto{\pgfqpoint{3.961184in}{1.350514in}}%
\pgfpathlineto{\pgfqpoint{3.969942in}{1.353429in}}%
\pgfpathlineto{\pgfqpoint{3.978715in}{1.353429in}}%
\pgfpathlineto{\pgfqpoint{3.987488in}{1.356343in}}%
\pgfpathlineto{\pgfqpoint{4.005029in}{1.356343in}}%
\pgfpathlineto{\pgfqpoint{4.013786in}{1.359257in}}%
\pgfpathlineto{\pgfqpoint{4.031333in}{1.359257in}}%
\pgfpathlineto{\pgfqpoint{4.048876in}{1.365086in}}%
\pgfpathlineto{\pgfqpoint{4.066426in}{1.365086in}}%
\pgfpathlineto{\pgfqpoint{4.075193in}{1.368000in}}%
\pgfpathlineto{\pgfqpoint{4.083951in}{1.368000in}}%
\pgfpathlineto{\pgfqpoint{4.092724in}{1.370914in}}%
\pgfpathlineto{\pgfqpoint{4.101497in}{1.370914in}}%
\pgfpathlineto{\pgfqpoint{4.110271in}{1.373829in}}%
\pgfpathlineto{\pgfqpoint{4.119044in}{1.373829in}}%
\pgfpathlineto{\pgfqpoint{4.127817in}{1.376743in}}%
\pgfpathlineto{\pgfqpoint{4.145364in}{1.376743in}}%
\pgfpathlineto{\pgfqpoint{4.162911in}{1.382571in}}%
\pgfpathlineto{\pgfqpoint{4.189231in}{1.382571in}}%
\pgfpathlineto{\pgfqpoint{4.198004in}{1.385486in}}%
\pgfpathlineto{\pgfqpoint{4.206777in}{1.385486in}}%
\pgfpathlineto{\pgfqpoint{4.215573in}{1.388400in}}%
\pgfpathlineto{\pgfqpoint{4.250665in}{1.388400in}}%
\pgfpathlineto{\pgfqpoint{4.268209in}{1.394229in}}%
\pgfpathlineto{\pgfqpoint{4.276986in}{1.394229in}}%
\pgfpathlineto{\pgfqpoint{4.285759in}{1.397143in}}%
\pgfpathlineto{\pgfqpoint{4.320852in}{1.397143in}}%
\pgfpathlineto{\pgfqpoint{4.329626in}{1.400057in}}%
\pgfpathlineto{\pgfqpoint{4.347172in}{1.400057in}}%
\pgfpathlineto{\pgfqpoint{4.355942in}{1.402971in}}%
\pgfpathlineto{\pgfqpoint{4.391033in}{1.402971in}}%
\pgfpathlineto{\pgfqpoint{4.399790in}{1.405886in}}%
\pgfpathlineto{\pgfqpoint{4.443656in}{1.405886in}}%
\pgfpathlineto{\pgfqpoint{4.452430in}{1.408800in}}%
\pgfpathlineto{\pgfqpoint{4.487523in}{1.408800in}}%
\pgfpathlineto{\pgfqpoint{4.496290in}{1.405886in}}%
\pgfpathlineto{\pgfqpoint{4.505048in}{1.408800in}}%
\pgfpathlineto{\pgfqpoint{4.707076in}{1.408800in}}%
\pgfpathlineto{\pgfqpoint{4.715849in}{1.405886in}}%
\pgfpathlineto{\pgfqpoint{4.724623in}{1.408800in}}%
\pgfpathlineto{\pgfqpoint{4.733396in}{1.405886in}}%
\pgfpathlineto{\pgfqpoint{4.838654in}{1.405886in}}%
\pgfpathlineto{\pgfqpoint{4.847427in}{1.400057in}}%
\pgfpathlineto{\pgfqpoint{4.856200in}{1.402971in}}%
\pgfpathlineto{\pgfqpoint{4.864974in}{1.402971in}}%
\pgfpathlineto{\pgfqpoint{4.873747in}{1.400057in}}%
\pgfpathlineto{\pgfqpoint{4.926387in}{1.400057in}}%
\pgfpathlineto{\pgfqpoint{4.935160in}{1.397143in}}%
\pgfpathlineto{\pgfqpoint{4.943933in}{1.397143in}}%
\pgfpathlineto{\pgfqpoint{4.952707in}{1.400057in}}%
\pgfpathlineto{\pgfqpoint{4.961480in}{1.397143in}}%
\pgfpathlineto{\pgfqpoint{5.040462in}{1.397143in}}%
\pgfpathlineto{\pgfqpoint{5.049235in}{1.394229in}}%
\pgfpathlineto{\pgfqpoint{5.066782in}{1.394229in}}%
\pgfpathlineto{\pgfqpoint{5.075555in}{1.391314in}}%
\pgfpathlineto{\pgfqpoint{5.172084in}{1.391314in}}%
\pgfpathlineto{\pgfqpoint{5.180857in}{1.385486in}}%
\pgfpathlineto{\pgfqpoint{5.189630in}{1.391314in}}%
\pgfpathlineto{\pgfqpoint{5.207177in}{1.385486in}}%
\pgfpathlineto{\pgfqpoint{5.251043in}{1.385486in}}%
\pgfpathlineto{\pgfqpoint{5.259817in}{1.382571in}}%
\pgfpathlineto{\pgfqpoint{5.295152in}{1.382571in}}%
\pgfpathlineto{\pgfqpoint{5.303926in}{1.379657in}}%
\pgfpathlineto{\pgfqpoint{5.312699in}{1.379657in}}%
\pgfpathlineto{\pgfqpoint{5.321472in}{1.382571in}}%
\pgfpathlineto{\pgfqpoint{5.330246in}{1.376743in}}%
\pgfpathlineto{\pgfqpoint{5.382885in}{1.376743in}}%
\pgfpathlineto{\pgfqpoint{5.391659in}{1.373829in}}%
\pgfpathlineto{\pgfqpoint{5.400432in}{1.376743in}}%
\pgfpathlineto{\pgfqpoint{5.409205in}{1.376743in}}%
\pgfpathlineto{\pgfqpoint{5.417979in}{1.370914in}}%
\pgfpathlineto{\pgfqpoint{5.426746in}{1.376743in}}%
\pgfpathlineto{\pgfqpoint{5.435503in}{1.370914in}}%
\pgfpathlineto{\pgfqpoint{5.496917in}{1.370914in}}%
\pgfpathlineto{\pgfqpoint{5.505690in}{1.368000in}}%
\pgfpathlineto{\pgfqpoint{5.514463in}{1.368000in}}%
\pgfpathlineto{\pgfqpoint{5.523236in}{1.365086in}}%
\pgfpathlineto{\pgfqpoint{5.540783in}{1.365086in}}%
\pgfpathlineto{\pgfqpoint{5.558330in}{1.359257in}}%
\pgfpathlineto{\pgfqpoint{5.575870in}{1.359257in}}%
\pgfpathlineto{\pgfqpoint{5.584628in}{1.353429in}}%
\pgfpathlineto{\pgfqpoint{5.602174in}{1.353429in}}%
\pgfpathlineto{\pgfqpoint{5.610947in}{1.347600in}}%
\pgfpathlineto{\pgfqpoint{5.619721in}{1.347600in}}%
\pgfpathlineto{\pgfqpoint{5.628494in}{1.341771in}}%
\pgfpathlineto{\pgfqpoint{5.646040in}{1.341771in}}%
\pgfpathlineto{\pgfqpoint{5.663587in}{1.335943in}}%
\pgfpathlineto{\pgfqpoint{5.672361in}{1.335943in}}%
\pgfpathlineto{\pgfqpoint{5.681128in}{1.330114in}}%
\pgfpathlineto{\pgfqpoint{5.689885in}{1.330114in}}%
\pgfpathlineto{\pgfqpoint{5.707432in}{1.324286in}}%
\pgfpathlineto{\pgfqpoint{5.724978in}{1.324286in}}%
\pgfpathlineto{\pgfqpoint{5.742525in}{1.318457in}}%
\pgfpathlineto{\pgfqpoint{5.751298in}{1.318457in}}%
\pgfpathlineto{\pgfqpoint{5.770000in}{1.312245in}}%
\pgfpathlineto{\pgfqpoint{5.770000in}{1.312245in}}%
\pgfusepath{stroke}%
\end{pgfscope}%
\begin{pgfscope}%
\pgfsetrectcap%
\pgfsetmiterjoin%
\pgfsetlinewidth{0.803000pt}%
\definecolor{currentstroke}{rgb}{0.000000,0.000000,0.000000}%
\pgfsetstrokecolor{currentstroke}%
\pgfsetdash{}{0pt}%
\pgfpathmoveto{\pgfqpoint{0.800000in}{0.960000in}}%
\pgfpathlineto{\pgfqpoint{0.800000in}{4.224000in}}%
\pgfusepath{stroke}%
\end{pgfscope}%
\begin{pgfscope}%
\pgfsetrectcap%
\pgfsetmiterjoin%
\pgfsetlinewidth{0.803000pt}%
\definecolor{currentstroke}{rgb}{0.000000,0.000000,0.000000}%
\pgfsetstrokecolor{currentstroke}%
\pgfsetdash{}{0pt}%
\pgfpathmoveto{\pgfqpoint{5.760000in}{0.960000in}}%
\pgfpathlineto{\pgfqpoint{5.760000in}{4.224000in}}%
\pgfusepath{stroke}%
\end{pgfscope}%
\begin{pgfscope}%
\pgfsetrectcap%
\pgfsetmiterjoin%
\pgfsetlinewidth{0.803000pt}%
\definecolor{currentstroke}{rgb}{0.000000,0.000000,0.000000}%
\pgfsetstrokecolor{currentstroke}%
\pgfsetdash{}{0pt}%
\pgfpathmoveto{\pgfqpoint{0.800000in}{0.960000in}}%
\pgfpathlineto{\pgfqpoint{5.760000in}{0.960000in}}%
\pgfusepath{stroke}%
\end{pgfscope}%
\begin{pgfscope}%
\pgfsetrectcap%
\pgfsetmiterjoin%
\pgfsetlinewidth{0.803000pt}%
\definecolor{currentstroke}{rgb}{0.000000,0.000000,0.000000}%
\pgfsetstrokecolor{currentstroke}%
\pgfsetdash{}{0pt}%
\pgfpathmoveto{\pgfqpoint{0.800000in}{4.224000in}}%
\pgfpathlineto{\pgfqpoint{5.760000in}{4.224000in}}%
\pgfusepath{stroke}%
\end{pgfscope}%
\begin{pgfscope}%
\definecolor{textcolor}{rgb}{0.000000,0.000000,0.000000}%
\pgfsetstrokecolor{textcolor}%
\pgfsetfillcolor{textcolor}%
%\pgftext[x=3.280000in,y=4.307333in,,base]{\color{textcolor}\rmfamily\fontsize{12.000000}{14.400000}\selectfont Évolution de la température dans les tests avec step-down et sans ventilateur}%
\end{pgfscope}%
\begin{pgfscope}%
\pgfsetbuttcap%
\pgfsetmiterjoin%
\definecolor{currentfill}{rgb}{1.000000,1.000000,1.000000}%
\pgfsetfillcolor{currentfill}%
\pgfsetfillopacity{0.800000}%
\pgfsetlinewidth{1.003750pt}%
\definecolor{currentstroke}{rgb}{0.800000,0.800000,0.800000}%
\pgfsetstrokecolor{currentstroke}%
\pgfsetstrokeopacity{0.800000}%
\pgfsetdash{}{0pt}%
\pgfpathmoveto{\pgfqpoint{4.070955in}{2.280657in}}%
\pgfpathlineto{\pgfqpoint{5.662778in}{2.280657in}}%
\pgfpathquadraticcurveto{\pgfqpoint{5.690556in}{2.280657in}}{\pgfqpoint{5.690556in}{2.308435in}}%
\pgfpathlineto{\pgfqpoint{5.690556in}{2.875565in}}%
\pgfpathquadraticcurveto{\pgfqpoint{5.690556in}{2.903342in}}{\pgfqpoint{5.662778in}{2.903342in}}%
\pgfpathlineto{\pgfqpoint{4.070955in}{2.903342in}}%
\pgfpathquadraticcurveto{\pgfqpoint{4.043177in}{2.903342in}}{\pgfqpoint{4.043177in}{2.875565in}}%
\pgfpathlineto{\pgfqpoint{4.043177in}{2.308435in}}%
\pgfpathquadraticcurveto{\pgfqpoint{4.043177in}{2.280657in}}{\pgfqpoint{4.070955in}{2.280657in}}%
\pgfpathclose%
\pgfusepath{stroke,fill}%
\end{pgfscope}%
\begin{pgfscope}%
\pgfsetrectcap%
\pgfsetroundjoin%
\pgfsetlinewidth{1.505625pt}%
\definecolor{currentstroke}{rgb}{0.121569,0.466667,0.705882}%
\pgfsetstrokecolor{currentstroke}%
\pgfsetdash{}{0pt}%
\pgfpathmoveto{\pgfqpoint{4.098733in}{2.799176in}}%
\pgfpathlineto{\pgfqpoint{4.376510in}{2.799176in}}%
\pgfusepath{stroke}%
\end{pgfscope}%
\begin{pgfscope}%
\definecolor{textcolor}{rgb}{0.000000,0.000000,0.000000}%
\pgfsetstrokecolor{textcolor}%
\pgfsetfillcolor{textcolor}%
\pgftext[x=4.487622in,y=2.750565in,left,base]{\color{textcolor}\rmfamily\fontsize{10.000000}{12.000000}\selectfont CPU}%
\end{pgfscope}%
\begin{pgfscope}%
\pgfsetrectcap%
\pgfsetroundjoin%
\pgfsetlinewidth{1.505625pt}%
\definecolor{currentstroke}{rgb}{1.000000,0.498039,0.054902}%
\pgfsetstrokecolor{currentstroke}%
\pgfsetdash{}{0pt}%
\pgfpathmoveto{\pgfqpoint{4.098733in}{2.605503in}}%
\pgfpathlineto{\pgfqpoint{4.376510in}{2.605503in}}%
\pgfusepath{stroke}%
\end{pgfscope}%
\begin{pgfscope}%
\definecolor{textcolor}{rgb}{0.000000,0.000000,0.000000}%
\pgfsetstrokecolor{textcolor}%
\pgfsetfillcolor{textcolor}%
\pgftext[x=4.487622in,y=2.556892in,left,base]{\color{textcolor}\rmfamily\fontsize{10.000000}{12.000000}\selectfont Zone Raspberry Pi}%
\end{pgfscope}%
\begin{pgfscope}%
\pgfsetrectcap%
\pgfsetroundjoin%
\pgfsetlinewidth{1.505625pt}%
\definecolor{currentstroke}{rgb}{0.172549,0.627451,0.172549}%
\pgfsetstrokecolor{currentstroke}%
\pgfsetdash{}{0pt}%
\pgfpathmoveto{\pgfqpoint{4.098733in}{2.411830in}}%
\pgfpathlineto{\pgfqpoint{4.376510in}{2.411830in}}%
\pgfusepath{stroke}%
\end{pgfscope}%
\begin{pgfscope}%
\definecolor{textcolor}{rgb}{0.000000,0.000000,0.000000}%
\pgfsetstrokecolor{textcolor}%
\pgfsetfillcolor{textcolor}%
\pgftext[x=4.487622in,y=2.363219in,left,base]{\color{textcolor}\rmfamily\fontsize{10.000000}{12.000000}\selectfont Zone step-down}%
\end{pgfscope}%
\end{pgfpicture}%
\makeatother%
\endgroup%

  \label{fig:test_1}
  \vspace{-1cm}
  \caption{\textbf{Test 1 :} Évolution de la température dans les tests avec step-down et sans ventilateur}
\end{figure}

~

\noindent
Ensuite, dans le test 2, le XL4015 a été retiré de la boîte et placé à une distance de 50 cm de celle-ci. Des fils d'une longueur de 60 cm étaient alors utilisés pour relier le XL4015 aux différents composants de la boîte. Ce test avait comme but de vérifier l'impact du XL4015, et en particulier du champ magnétique produit par celui-ci, sur les performances du SIM800L.
%\vspace{-0.4cm}

\begin{figure}[ht!]
  \centering
  %% Creator: Matplotlib, PGF backend
%%
%% To include the figure in your LaTeX document, write
%%   \input{<filename>.pgf}
%%
%% Make sure the required packages are loaded in your preamble
%%   \usepackage{pgf}
%%
%% Figures using additional raster images can only be included by \input if
%% they are in the same directory as the main LaTeX file. For loading figures
%% from other directories you can use the `import` package
%%   \usepackage{import}
%% and then include the figures with
%%   \import{<path to file>}{<filename>.pgf}
%%
%% Matplotlib used the following preamble
%%
\begingroup%
\makeatletter%
\begin{pgfpicture}%
\pgfpathrectangle{\pgfpointorigin}{\pgfqpoint{6.400000in}{4.800000in}}%
\pgfusepath{use as bounding box, clip}%
\begin{pgfscope}%
\pgfsetbuttcap%
\pgfsetmiterjoin%
\definecolor{currentfill}{rgb}{1.000000,1.000000,1.000000}%
\pgfsetfillcolor{currentfill}%
\pgfsetlinewidth{0.000000pt}%
\definecolor{currentstroke}{rgb}{1.000000,1.000000,1.000000}%
\pgfsetstrokecolor{currentstroke}%
\pgfsetdash{}{0pt}%
\pgfpathmoveto{\pgfqpoint{0.000000in}{0.000000in}}%
\pgfpathlineto{\pgfqpoint{6.400000in}{0.000000in}}%
\pgfpathlineto{\pgfqpoint{6.400000in}{4.800000in}}%
\pgfpathlineto{\pgfqpoint{0.000000in}{4.800000in}}%
\pgfpathclose%
\pgfusepath{fill}%
\end{pgfscope}%
\begin{pgfscope}%
\pgfsetbuttcap%
\pgfsetmiterjoin%
\definecolor{currentfill}{rgb}{1.000000,1.000000,1.000000}%
\pgfsetfillcolor{currentfill}%
\pgfsetlinewidth{0.000000pt}%
\definecolor{currentstroke}{rgb}{0.000000,0.000000,0.000000}%
\pgfsetstrokecolor{currentstroke}%
\pgfsetstrokeopacity{0.000000}%
\pgfsetdash{}{0pt}%
\pgfpathmoveto{\pgfqpoint{0.800000in}{0.960000in}}%
\pgfpathlineto{\pgfqpoint{5.760000in}{0.960000in}}%
\pgfpathlineto{\pgfqpoint{5.760000in}{4.224000in}}%
\pgfpathlineto{\pgfqpoint{0.800000in}{4.224000in}}%
\pgfpathclose%
\pgfusepath{fill}%
\end{pgfscope}%
\begin{pgfscope}%
\pgfsetbuttcap%
\pgfsetroundjoin%
\definecolor{currentfill}{rgb}{0.000000,0.000000,0.000000}%
\pgfsetfillcolor{currentfill}%
\pgfsetlinewidth{0.803000pt}%
\definecolor{currentstroke}{rgb}{0.000000,0.000000,0.000000}%
\pgfsetstrokecolor{currentstroke}%
\pgfsetdash{}{0pt}%
\pgfsys@defobject{currentmarker}{\pgfqpoint{0.000000in}{-0.048611in}}{\pgfqpoint{0.000000in}{0.000000in}}{%
\pgfpathmoveto{\pgfqpoint{0.000000in}{0.000000in}}%
\pgfpathlineto{\pgfqpoint{0.000000in}{-0.048611in}}%
\pgfusepath{stroke,fill}%
}%
\begin{pgfscope}%
\pgfsys@transformshift{0.800000in}{0.960000in}%
\pgfsys@useobject{currentmarker}{}%
\end{pgfscope}%
\end{pgfscope}%
\begin{pgfscope}%
\definecolor{textcolor}{rgb}{0.000000,0.000000,0.000000}%
\pgfsetstrokecolor{textcolor}%
\pgfsetfillcolor{textcolor}%
\pgftext[x=0.512522in,y=0.621070in,left,base,rotate=30.000000]{\color{textcolor}\rmfamily\fontsize{10.000000}{12.000000}\selectfont 00:00}%
\end{pgfscope}%
\begin{pgfscope}%
\pgfsetbuttcap%
\pgfsetroundjoin%
\definecolor{currentfill}{rgb}{0.000000,0.000000,0.000000}%
\pgfsetfillcolor{currentfill}%
\pgfsetlinewidth{0.803000pt}%
\definecolor{currentstroke}{rgb}{0.000000,0.000000,0.000000}%
\pgfsetstrokecolor{currentstroke}%
\pgfsetdash{}{0pt}%
\pgfsys@defobject{currentmarker}{\pgfqpoint{0.000000in}{-0.048611in}}{\pgfqpoint{0.000000in}{0.000000in}}{%
\pgfpathmoveto{\pgfqpoint{0.000000in}{0.000000in}}%
\pgfpathlineto{\pgfqpoint{0.000000in}{-0.048611in}}%
\pgfusepath{stroke,fill}%
}%
\begin{pgfscope}%
\pgfsys@transformshift{1.791869in}{0.960000in}%
\pgfsys@useobject{currentmarker}{}%
\end{pgfscope}%
\end{pgfscope}%
\begin{pgfscope}%
\definecolor{textcolor}{rgb}{0.000000,0.000000,0.000000}%
\pgfsetstrokecolor{textcolor}%
\pgfsetfillcolor{textcolor}%
\pgftext[x=1.504391in,y=0.621070in,left,base,rotate=30.000000]{\color{textcolor}\rmfamily\fontsize{10.000000}{12.000000}\selectfont 01:00}%
\end{pgfscope}%
\begin{pgfscope}%
\pgfsetbuttcap%
\pgfsetroundjoin%
\definecolor{currentfill}{rgb}{0.000000,0.000000,0.000000}%
\pgfsetfillcolor{currentfill}%
\pgfsetlinewidth{0.803000pt}%
\definecolor{currentstroke}{rgb}{0.000000,0.000000,0.000000}%
\pgfsetstrokecolor{currentstroke}%
\pgfsetdash{}{0pt}%
\pgfsys@defobject{currentmarker}{\pgfqpoint{0.000000in}{-0.048611in}}{\pgfqpoint{0.000000in}{0.000000in}}{%
\pgfpathmoveto{\pgfqpoint{0.000000in}{0.000000in}}%
\pgfpathlineto{\pgfqpoint{0.000000in}{-0.048611in}}%
\pgfusepath{stroke,fill}%
}%
\begin{pgfscope}%
\pgfsys@transformshift{2.783738in}{0.960000in}%
\pgfsys@useobject{currentmarker}{}%
\end{pgfscope}%
\end{pgfscope}%
\begin{pgfscope}%
\definecolor{textcolor}{rgb}{0.000000,0.000000,0.000000}%
\pgfsetstrokecolor{textcolor}%
\pgfsetfillcolor{textcolor}%
\pgftext[x=2.496260in,y=0.621070in,left,base,rotate=30.000000]{\color{textcolor}\rmfamily\fontsize{10.000000}{12.000000}\selectfont 02:00}%
\end{pgfscope}%
\begin{pgfscope}%
\pgfsetbuttcap%
\pgfsetroundjoin%
\definecolor{currentfill}{rgb}{0.000000,0.000000,0.000000}%
\pgfsetfillcolor{currentfill}%
\pgfsetlinewidth{0.803000pt}%
\definecolor{currentstroke}{rgb}{0.000000,0.000000,0.000000}%
\pgfsetstrokecolor{currentstroke}%
\pgfsetdash{}{0pt}%
\pgfsys@defobject{currentmarker}{\pgfqpoint{0.000000in}{-0.048611in}}{\pgfqpoint{0.000000in}{0.000000in}}{%
\pgfpathmoveto{\pgfqpoint{0.000000in}{0.000000in}}%
\pgfpathlineto{\pgfqpoint{0.000000in}{-0.048611in}}%
\pgfusepath{stroke,fill}%
}%
\begin{pgfscope}%
\pgfsys@transformshift{3.775607in}{0.960000in}%
\pgfsys@useobject{currentmarker}{}%
\end{pgfscope}%
\end{pgfscope}%
\begin{pgfscope}%
\definecolor{textcolor}{rgb}{0.000000,0.000000,0.000000}%
\pgfsetstrokecolor{textcolor}%
\pgfsetfillcolor{textcolor}%
\pgftext[x=3.488129in,y=0.621070in,left,base,rotate=30.000000]{\color{textcolor}\rmfamily\fontsize{10.000000}{12.000000}\selectfont 03:00}%
\end{pgfscope}%
\begin{pgfscope}%
\pgfsetbuttcap%
\pgfsetroundjoin%
\definecolor{currentfill}{rgb}{0.000000,0.000000,0.000000}%
\pgfsetfillcolor{currentfill}%
\pgfsetlinewidth{0.803000pt}%
\definecolor{currentstroke}{rgb}{0.000000,0.000000,0.000000}%
\pgfsetstrokecolor{currentstroke}%
\pgfsetdash{}{0pt}%
\pgfsys@defobject{currentmarker}{\pgfqpoint{0.000000in}{-0.048611in}}{\pgfqpoint{0.000000in}{0.000000in}}{%
\pgfpathmoveto{\pgfqpoint{0.000000in}{0.000000in}}%
\pgfpathlineto{\pgfqpoint{0.000000in}{-0.048611in}}%
\pgfusepath{stroke,fill}%
}%
\begin{pgfscope}%
\pgfsys@transformshift{4.767476in}{0.960000in}%
\pgfsys@useobject{currentmarker}{}%
\end{pgfscope}%
\end{pgfscope}%
\begin{pgfscope}%
\definecolor{textcolor}{rgb}{0.000000,0.000000,0.000000}%
\pgfsetstrokecolor{textcolor}%
\pgfsetfillcolor{textcolor}%
\pgftext[x=4.479998in,y=0.621070in,left,base,rotate=30.000000]{\color{textcolor}\rmfamily\fontsize{10.000000}{12.000000}\selectfont 04:00}%
\end{pgfscope}%
\begin{pgfscope}%
\pgfsetbuttcap%
\pgfsetroundjoin%
\definecolor{currentfill}{rgb}{0.000000,0.000000,0.000000}%
\pgfsetfillcolor{currentfill}%
\pgfsetlinewidth{0.803000pt}%
\definecolor{currentstroke}{rgb}{0.000000,0.000000,0.000000}%
\pgfsetstrokecolor{currentstroke}%
\pgfsetdash{}{0pt}%
\pgfsys@defobject{currentmarker}{\pgfqpoint{0.000000in}{-0.048611in}}{\pgfqpoint{0.000000in}{0.000000in}}{%
\pgfpathmoveto{\pgfqpoint{0.000000in}{0.000000in}}%
\pgfpathlineto{\pgfqpoint{0.000000in}{-0.048611in}}%
\pgfusepath{stroke,fill}%
}%
\begin{pgfscope}%
\pgfsys@transformshift{5.759345in}{0.960000in}%
\pgfsys@useobject{currentmarker}{}%
\end{pgfscope}%
\end{pgfscope}%
\begin{pgfscope}%
\definecolor{textcolor}{rgb}{0.000000,0.000000,0.000000}%
\pgfsetstrokecolor{textcolor}%
\pgfsetfillcolor{textcolor}%
\pgftext[x=5.471867in,y=0.621070in,left,base,rotate=30.000000]{\color{textcolor}\rmfamily\fontsize{10.000000}{12.000000}\selectfont 05:00}%
\end{pgfscope}%
\begin{pgfscope}%
\definecolor{textcolor}{rgb}{0.000000,0.000000,0.000000}%
\pgfsetstrokecolor{textcolor}%
\pgfsetfillcolor{textcolor}%
\pgftext[x=3.280000in,y=0.542126in,,top]{\color{textcolor}\rmfamily\fontsize{10.000000}{12.000000}\selectfont Temps (hh:mm)}%
\end{pgfscope}%
\begin{pgfscope}%
\pgfsetbuttcap%
\pgfsetroundjoin%
\definecolor{currentfill}{rgb}{0.000000,0.000000,0.000000}%
\pgfsetfillcolor{currentfill}%
\pgfsetlinewidth{0.803000pt}%
\definecolor{currentstroke}{rgb}{0.000000,0.000000,0.000000}%
\pgfsetstrokecolor{currentstroke}%
\pgfsetdash{}{0pt}%
\pgfsys@defobject{currentmarker}{\pgfqpoint{-0.048611in}{0.000000in}}{\pgfqpoint{0.000000in}{0.000000in}}{%
\pgfpathmoveto{\pgfqpoint{0.000000in}{0.000000in}}%
\pgfpathlineto{\pgfqpoint{-0.048611in}{0.000000in}}%
\pgfusepath{stroke,fill}%
}%
\begin{pgfscope}%
\pgfsys@transformshift{0.800000in}{0.960000in}%
\pgfsys@useobject{currentmarker}{}%
\end{pgfscope}%
\end{pgfscope}%
\begin{pgfscope}%
\definecolor{textcolor}{rgb}{0.000000,0.000000,0.000000}%
\pgfsetstrokecolor{textcolor}%
\pgfsetfillcolor{textcolor}%
\pgftext[x=0.563888in,y=0.911775in,left,base]{\color{textcolor}\rmfamily\fontsize{10.000000}{12.000000}\selectfont \(\displaystyle 20\)}%
\end{pgfscope}%
\begin{pgfscope}%
\pgfsetbuttcap%
\pgfsetroundjoin%
\definecolor{currentfill}{rgb}{0.000000,0.000000,0.000000}%
\pgfsetfillcolor{currentfill}%
\pgfsetlinewidth{0.803000pt}%
\definecolor{currentstroke}{rgb}{0.000000,0.000000,0.000000}%
\pgfsetstrokecolor{currentstroke}%
\pgfsetdash{}{0pt}%
\pgfsys@defobject{currentmarker}{\pgfqpoint{-0.048611in}{0.000000in}}{\pgfqpoint{0.000000in}{0.000000in}}{%
\pgfpathmoveto{\pgfqpoint{0.000000in}{0.000000in}}%
\pgfpathlineto{\pgfqpoint{-0.048611in}{0.000000in}}%
\pgfusepath{stroke,fill}%
}%
\begin{pgfscope}%
\pgfsys@transformshift{0.800000in}{1.426286in}%
\pgfsys@useobject{currentmarker}{}%
\end{pgfscope}%
\end{pgfscope}%
\begin{pgfscope}%
\definecolor{textcolor}{rgb}{0.000000,0.000000,0.000000}%
\pgfsetstrokecolor{textcolor}%
\pgfsetfillcolor{textcolor}%
\pgftext[x=0.563888in,y=1.378060in,left,base]{\color{textcolor}\rmfamily\fontsize{10.000000}{12.000000}\selectfont \(\displaystyle 25\)}%
\end{pgfscope}%
\begin{pgfscope}%
\pgfsetbuttcap%
\pgfsetroundjoin%
\definecolor{currentfill}{rgb}{0.000000,0.000000,0.000000}%
\pgfsetfillcolor{currentfill}%
\pgfsetlinewidth{0.803000pt}%
\definecolor{currentstroke}{rgb}{0.000000,0.000000,0.000000}%
\pgfsetstrokecolor{currentstroke}%
\pgfsetdash{}{0pt}%
\pgfsys@defobject{currentmarker}{\pgfqpoint{-0.048611in}{0.000000in}}{\pgfqpoint{0.000000in}{0.000000in}}{%
\pgfpathmoveto{\pgfqpoint{0.000000in}{0.000000in}}%
\pgfpathlineto{\pgfqpoint{-0.048611in}{0.000000in}}%
\pgfusepath{stroke,fill}%
}%
\begin{pgfscope}%
\pgfsys@transformshift{0.800000in}{1.892571in}%
\pgfsys@useobject{currentmarker}{}%
\end{pgfscope}%
\end{pgfscope}%
\begin{pgfscope}%
\definecolor{textcolor}{rgb}{0.000000,0.000000,0.000000}%
\pgfsetstrokecolor{textcolor}%
\pgfsetfillcolor{textcolor}%
\pgftext[x=0.563888in,y=1.844346in,left,base]{\color{textcolor}\rmfamily\fontsize{10.000000}{12.000000}\selectfont \(\displaystyle 30\)}%
\end{pgfscope}%
\begin{pgfscope}%
\pgfsetbuttcap%
\pgfsetroundjoin%
\definecolor{currentfill}{rgb}{0.000000,0.000000,0.000000}%
\pgfsetfillcolor{currentfill}%
\pgfsetlinewidth{0.803000pt}%
\definecolor{currentstroke}{rgb}{0.000000,0.000000,0.000000}%
\pgfsetstrokecolor{currentstroke}%
\pgfsetdash{}{0pt}%
\pgfsys@defobject{currentmarker}{\pgfqpoint{-0.048611in}{0.000000in}}{\pgfqpoint{0.000000in}{0.000000in}}{%
\pgfpathmoveto{\pgfqpoint{0.000000in}{0.000000in}}%
\pgfpathlineto{\pgfqpoint{-0.048611in}{0.000000in}}%
\pgfusepath{stroke,fill}%
}%
\begin{pgfscope}%
\pgfsys@transformshift{0.800000in}{2.358857in}%
\pgfsys@useobject{currentmarker}{}%
\end{pgfscope}%
\end{pgfscope}%
\begin{pgfscope}%
\definecolor{textcolor}{rgb}{0.000000,0.000000,0.000000}%
\pgfsetstrokecolor{textcolor}%
\pgfsetfillcolor{textcolor}%
\pgftext[x=0.563888in,y=2.310632in,left,base]{\color{textcolor}\rmfamily\fontsize{10.000000}{12.000000}\selectfont \(\displaystyle 35\)}%
\end{pgfscope}%
\begin{pgfscope}%
\pgfsetbuttcap%
\pgfsetroundjoin%
\definecolor{currentfill}{rgb}{0.000000,0.000000,0.000000}%
\pgfsetfillcolor{currentfill}%
\pgfsetlinewidth{0.803000pt}%
\definecolor{currentstroke}{rgb}{0.000000,0.000000,0.000000}%
\pgfsetstrokecolor{currentstroke}%
\pgfsetdash{}{0pt}%
\pgfsys@defobject{currentmarker}{\pgfqpoint{-0.048611in}{0.000000in}}{\pgfqpoint{0.000000in}{0.000000in}}{%
\pgfpathmoveto{\pgfqpoint{0.000000in}{0.000000in}}%
\pgfpathlineto{\pgfqpoint{-0.048611in}{0.000000in}}%
\pgfusepath{stroke,fill}%
}%
\begin{pgfscope}%
\pgfsys@transformshift{0.800000in}{2.825143in}%
\pgfsys@useobject{currentmarker}{}%
\end{pgfscope}%
\end{pgfscope}%
\begin{pgfscope}%
\definecolor{textcolor}{rgb}{0.000000,0.000000,0.000000}%
\pgfsetstrokecolor{textcolor}%
\pgfsetfillcolor{textcolor}%
\pgftext[x=0.563888in,y=2.776918in,left,base]{\color{textcolor}\rmfamily\fontsize{10.000000}{12.000000}\selectfont \(\displaystyle 40\)}%
\end{pgfscope}%
\begin{pgfscope}%
\pgfsetbuttcap%
\pgfsetroundjoin%
\definecolor{currentfill}{rgb}{0.000000,0.000000,0.000000}%
\pgfsetfillcolor{currentfill}%
\pgfsetlinewidth{0.803000pt}%
\definecolor{currentstroke}{rgb}{0.000000,0.000000,0.000000}%
\pgfsetstrokecolor{currentstroke}%
\pgfsetdash{}{0pt}%
\pgfsys@defobject{currentmarker}{\pgfqpoint{-0.048611in}{0.000000in}}{\pgfqpoint{0.000000in}{0.000000in}}{%
\pgfpathmoveto{\pgfqpoint{0.000000in}{0.000000in}}%
\pgfpathlineto{\pgfqpoint{-0.048611in}{0.000000in}}%
\pgfusepath{stroke,fill}%
}%
\begin{pgfscope}%
\pgfsys@transformshift{0.800000in}{3.291429in}%
\pgfsys@useobject{currentmarker}{}%
\end{pgfscope}%
\end{pgfscope}%
\begin{pgfscope}%
\definecolor{textcolor}{rgb}{0.000000,0.000000,0.000000}%
\pgfsetstrokecolor{textcolor}%
\pgfsetfillcolor{textcolor}%
\pgftext[x=0.563888in,y=3.243203in,left,base]{\color{textcolor}\rmfamily\fontsize{10.000000}{12.000000}\selectfont \(\displaystyle 45\)}%
\end{pgfscope}%
\begin{pgfscope}%
\pgfsetbuttcap%
\pgfsetroundjoin%
\definecolor{currentfill}{rgb}{0.000000,0.000000,0.000000}%
\pgfsetfillcolor{currentfill}%
\pgfsetlinewidth{0.803000pt}%
\definecolor{currentstroke}{rgb}{0.000000,0.000000,0.000000}%
\pgfsetstrokecolor{currentstroke}%
\pgfsetdash{}{0pt}%
\pgfsys@defobject{currentmarker}{\pgfqpoint{-0.048611in}{0.000000in}}{\pgfqpoint{0.000000in}{0.000000in}}{%
\pgfpathmoveto{\pgfqpoint{0.000000in}{0.000000in}}%
\pgfpathlineto{\pgfqpoint{-0.048611in}{0.000000in}}%
\pgfusepath{stroke,fill}%
}%
\begin{pgfscope}%
\pgfsys@transformshift{0.800000in}{3.757714in}%
\pgfsys@useobject{currentmarker}{}%
\end{pgfscope}%
\end{pgfscope}%
\begin{pgfscope}%
\definecolor{textcolor}{rgb}{0.000000,0.000000,0.000000}%
\pgfsetstrokecolor{textcolor}%
\pgfsetfillcolor{textcolor}%
\pgftext[x=0.563888in,y=3.709489in,left,base]{\color{textcolor}\rmfamily\fontsize{10.000000}{12.000000}\selectfont \(\displaystyle 50\)}%
\end{pgfscope}%
\begin{pgfscope}%
\pgfsetbuttcap%
\pgfsetroundjoin%
\definecolor{currentfill}{rgb}{0.000000,0.000000,0.000000}%
\pgfsetfillcolor{currentfill}%
\pgfsetlinewidth{0.803000pt}%
\definecolor{currentstroke}{rgb}{0.000000,0.000000,0.000000}%
\pgfsetstrokecolor{currentstroke}%
\pgfsetdash{}{0pt}%
\pgfsys@defobject{currentmarker}{\pgfqpoint{-0.048611in}{0.000000in}}{\pgfqpoint{0.000000in}{0.000000in}}{%
\pgfpathmoveto{\pgfqpoint{0.000000in}{0.000000in}}%
\pgfpathlineto{\pgfqpoint{-0.048611in}{0.000000in}}%
\pgfusepath{stroke,fill}%
}%
\begin{pgfscope}%
\pgfsys@transformshift{0.800000in}{4.224000in}%
\pgfsys@useobject{currentmarker}{}%
\end{pgfscope}%
\end{pgfscope}%
\begin{pgfscope}%
\definecolor{textcolor}{rgb}{0.000000,0.000000,0.000000}%
\pgfsetstrokecolor{textcolor}%
\pgfsetfillcolor{textcolor}%
\pgftext[x=0.563888in,y=4.175775in,left,base]{\color{textcolor}\rmfamily\fontsize{10.000000}{12.000000}\selectfont \(\displaystyle 55\)}%
\end{pgfscope}%
\begin{pgfscope}%
\definecolor{textcolor}{rgb}{0.000000,0.000000,0.000000}%
\pgfsetstrokecolor{textcolor}%
\pgfsetfillcolor{textcolor}%
\pgftext[x=0.508333in,y=2.592000in,,bottom,rotate=90.000000]{\color{textcolor}\rmfamily\fontsize{10.000000}{12.000000}\selectfont Température (\textdegree{}C)}%
\end{pgfscope}%
\begin{pgfscope}%
\pgfpathrectangle{\pgfqpoint{0.800000in}{0.960000in}}{\pgfqpoint{4.960000in}{3.264000in}}%
\pgfusepath{clip}%
\pgfsetrectcap%
\pgfsetroundjoin%
\pgfsetlinewidth{1.505625pt}%
\definecolor{currentstroke}{rgb}{0.121569,0.466667,0.705882}%
\pgfsetstrokecolor{currentstroke}%
\pgfsetdash{}{0pt}%
\pgfpathmoveto{\pgfqpoint{0.800000in}{2.605989in}}%
\pgfpathlineto{\pgfqpoint{0.808277in}{3.132891in}}%
\pgfpathlineto{\pgfqpoint{0.816555in}{3.235474in}}%
\pgfpathlineto{\pgfqpoint{0.824832in}{3.310080in}}%
\pgfpathlineto{\pgfqpoint{0.833107in}{3.356709in}}%
\pgfpathlineto{\pgfqpoint{0.841384in}{3.356709in}}%
\pgfpathlineto{\pgfqpoint{0.849660in}{3.435977in}}%
\pgfpathlineto{\pgfqpoint{0.857937in}{3.408000in}}%
\pgfpathlineto{\pgfqpoint{0.866214in}{3.435977in}}%
\pgfpathlineto{\pgfqpoint{0.874490in}{3.459291in}}%
\pgfpathlineto{\pgfqpoint{0.882767in}{3.384686in}}%
\pgfpathlineto{\pgfqpoint{0.891043in}{3.459291in}}%
\pgfpathlineto{\pgfqpoint{0.899320in}{3.482606in}}%
\pgfpathlineto{\pgfqpoint{0.907599in}{3.463954in}}%
\pgfpathlineto{\pgfqpoint{0.915876in}{3.510583in}}%
\pgfpathlineto{\pgfqpoint{0.924152in}{3.538560in}}%
\pgfpathlineto{\pgfqpoint{0.932429in}{3.585189in}}%
\pgfpathlineto{\pgfqpoint{0.940706in}{3.533897in}}%
\pgfpathlineto{\pgfqpoint{0.948982in}{3.561874in}}%
\pgfpathlineto{\pgfqpoint{0.957259in}{3.538560in}}%
\pgfpathlineto{\pgfqpoint{0.965536in}{3.538560in}}%
\pgfpathlineto{\pgfqpoint{0.973813in}{3.533897in}}%
\pgfpathlineto{\pgfqpoint{0.982090in}{3.561874in}}%
\pgfpathlineto{\pgfqpoint{0.990368in}{3.561874in}}%
\pgfpathlineto{\pgfqpoint{0.998645in}{3.585189in}}%
\pgfpathlineto{\pgfqpoint{1.006921in}{3.585189in}}%
\pgfpathlineto{\pgfqpoint{1.015198in}{3.608503in}}%
\pgfpathlineto{\pgfqpoint{1.023474in}{3.585189in}}%
\pgfpathlineto{\pgfqpoint{1.040028in}{3.585189in}}%
\pgfpathlineto{\pgfqpoint{1.048305in}{3.608503in}}%
\pgfpathlineto{\pgfqpoint{1.056582in}{3.613166in}}%
\pgfpathlineto{\pgfqpoint{1.064857in}{3.585189in}}%
\pgfpathlineto{\pgfqpoint{1.073134in}{3.608503in}}%
\pgfpathlineto{\pgfqpoint{1.081411in}{3.608503in}}%
\pgfpathlineto{\pgfqpoint{1.089688in}{3.585189in}}%
\pgfpathlineto{\pgfqpoint{1.097964in}{3.585189in}}%
\pgfpathlineto{\pgfqpoint{1.106240in}{3.608503in}}%
\pgfpathlineto{\pgfqpoint{1.114517in}{3.636480in}}%
\pgfpathlineto{\pgfqpoint{1.131070in}{3.636480in}}%
\pgfpathlineto{\pgfqpoint{1.139347in}{3.664457in}}%
\pgfpathlineto{\pgfqpoint{1.147623in}{3.687771in}}%
\pgfpathlineto{\pgfqpoint{1.155900in}{3.659794in}}%
\pgfpathlineto{\pgfqpoint{1.164177in}{3.687771in}}%
\pgfpathlineto{\pgfqpoint{1.172453in}{3.636480in}}%
\pgfpathlineto{\pgfqpoint{1.180730in}{3.636480in}}%
\pgfpathlineto{\pgfqpoint{1.189007in}{3.687771in}}%
\pgfpathlineto{\pgfqpoint{1.197284in}{3.659794in}}%
\pgfpathlineto{\pgfqpoint{1.205561in}{3.659794in}}%
\pgfpathlineto{\pgfqpoint{1.213839in}{3.636480in}}%
\pgfpathlineto{\pgfqpoint{1.222115in}{3.659794in}}%
\pgfpathlineto{\pgfqpoint{1.238669in}{3.659794in}}%
\pgfpathlineto{\pgfqpoint{1.246945in}{3.687771in}}%
\pgfpathlineto{\pgfqpoint{1.255223in}{3.687771in}}%
\pgfpathlineto{\pgfqpoint{1.263500in}{3.659794in}}%
\pgfpathlineto{\pgfqpoint{1.271777in}{3.687771in}}%
\pgfpathlineto{\pgfqpoint{1.280055in}{3.659794in}}%
\pgfpathlineto{\pgfqpoint{1.288332in}{3.687771in}}%
\pgfpathlineto{\pgfqpoint{1.296609in}{3.711086in}}%
\pgfpathlineto{\pgfqpoint{1.304886in}{3.687771in}}%
\pgfpathlineto{\pgfqpoint{1.346270in}{3.687771in}}%
\pgfpathlineto{\pgfqpoint{1.354544in}{3.659794in}}%
\pgfpathlineto{\pgfqpoint{1.362820in}{3.687771in}}%
\pgfpathlineto{\pgfqpoint{1.371097in}{3.687771in}}%
\pgfpathlineto{\pgfqpoint{1.379367in}{3.683109in}}%
\pgfpathlineto{\pgfqpoint{1.387639in}{3.687771in}}%
\pgfpathlineto{\pgfqpoint{1.395916in}{3.659794in}}%
\pgfpathlineto{\pgfqpoint{1.404190in}{3.687771in}}%
\pgfpathlineto{\pgfqpoint{1.412466in}{3.659794in}}%
\pgfpathlineto{\pgfqpoint{1.420744in}{3.711086in}}%
\pgfpathlineto{\pgfqpoint{1.429014in}{3.683109in}}%
\pgfpathlineto{\pgfqpoint{1.437291in}{3.687771in}}%
\pgfpathlineto{\pgfqpoint{1.445566in}{3.734400in}}%
\pgfpathlineto{\pgfqpoint{1.462121in}{3.734400in}}%
\pgfpathlineto{\pgfqpoint{1.470398in}{3.687771in}}%
\pgfpathlineto{\pgfqpoint{1.478674in}{3.711086in}}%
\pgfpathlineto{\pgfqpoint{1.486943in}{3.711086in}}%
\pgfpathlineto{\pgfqpoint{1.495218in}{3.687771in}}%
\pgfpathlineto{\pgfqpoint{1.503495in}{3.687771in}}%
\pgfpathlineto{\pgfqpoint{1.511774in}{3.734400in}}%
\pgfpathlineto{\pgfqpoint{1.520052in}{3.687771in}}%
\pgfpathlineto{\pgfqpoint{1.528331in}{3.734400in}}%
\pgfpathlineto{\pgfqpoint{1.536610in}{3.711086in}}%
\pgfpathlineto{\pgfqpoint{1.544887in}{3.711086in}}%
\pgfpathlineto{\pgfqpoint{1.553163in}{3.734400in}}%
\pgfpathlineto{\pgfqpoint{1.561435in}{3.687771in}}%
\pgfpathlineto{\pgfqpoint{1.577987in}{3.734400in}}%
\pgfpathlineto{\pgfqpoint{1.586264in}{3.711086in}}%
\pgfpathlineto{\pgfqpoint{1.611088in}{3.711086in}}%
\pgfpathlineto{\pgfqpoint{1.619365in}{3.739063in}}%
\pgfpathlineto{\pgfqpoint{1.627638in}{3.739063in}}%
\pgfpathlineto{\pgfqpoint{1.635909in}{3.687771in}}%
\pgfpathlineto{\pgfqpoint{1.644185in}{3.687771in}}%
\pgfpathlineto{\pgfqpoint{1.652461in}{3.711086in}}%
\pgfpathlineto{\pgfqpoint{1.660733in}{3.711086in}}%
\pgfpathlineto{\pgfqpoint{1.669005in}{3.734400in}}%
\pgfpathlineto{\pgfqpoint{1.677277in}{3.739063in}}%
\pgfpathlineto{\pgfqpoint{1.685548in}{3.711086in}}%
\pgfpathlineto{\pgfqpoint{1.693825in}{3.711086in}}%
\pgfpathlineto{\pgfqpoint{1.702099in}{3.734400in}}%
\pgfpathlineto{\pgfqpoint{1.710375in}{3.734400in}}%
\pgfpathlineto{\pgfqpoint{1.726919in}{3.687771in}}%
\pgfpathlineto{\pgfqpoint{1.735196in}{3.762377in}}%
\pgfpathlineto{\pgfqpoint{1.743467in}{3.711086in}}%
\pgfpathlineto{\pgfqpoint{1.751744in}{3.734400in}}%
\pgfpathlineto{\pgfqpoint{1.760020in}{3.711086in}}%
\pgfpathlineto{\pgfqpoint{1.768291in}{3.711086in}}%
\pgfpathlineto{\pgfqpoint{1.776563in}{3.757714in}}%
\pgfpathlineto{\pgfqpoint{1.784836in}{3.734400in}}%
\pgfpathlineto{\pgfqpoint{1.793112in}{3.687771in}}%
\pgfpathlineto{\pgfqpoint{1.801390in}{3.762377in}}%
\pgfpathlineto{\pgfqpoint{1.809665in}{3.687771in}}%
\pgfpathlineto{\pgfqpoint{1.817942in}{3.659794in}}%
\pgfpathlineto{\pgfqpoint{1.826217in}{3.711086in}}%
\pgfpathlineto{\pgfqpoint{1.842767in}{3.757714in}}%
\pgfpathlineto{\pgfqpoint{1.851043in}{3.762377in}}%
\pgfpathlineto{\pgfqpoint{1.859320in}{3.734400in}}%
\pgfpathlineto{\pgfqpoint{1.867598in}{3.711086in}}%
\pgfpathlineto{\pgfqpoint{1.875875in}{3.711086in}}%
\pgfpathlineto{\pgfqpoint{1.884149in}{3.762377in}}%
\pgfpathlineto{\pgfqpoint{1.892427in}{3.734400in}}%
\pgfpathlineto{\pgfqpoint{1.900702in}{3.711086in}}%
\pgfpathlineto{\pgfqpoint{1.908979in}{3.739063in}}%
\pgfpathlineto{\pgfqpoint{1.917256in}{3.711086in}}%
\pgfpathlineto{\pgfqpoint{1.925533in}{3.734400in}}%
\pgfpathlineto{\pgfqpoint{1.933811in}{3.734400in}}%
\pgfpathlineto{\pgfqpoint{1.942088in}{3.762377in}}%
\pgfpathlineto{\pgfqpoint{1.950365in}{3.734400in}}%
\pgfpathlineto{\pgfqpoint{1.958643in}{3.762377in}}%
\pgfpathlineto{\pgfqpoint{1.966920in}{3.734400in}}%
\pgfpathlineto{\pgfqpoint{1.975189in}{3.711086in}}%
\pgfpathlineto{\pgfqpoint{1.983462in}{3.785691in}}%
\pgfpathlineto{\pgfqpoint{1.991738in}{3.739063in}}%
\pgfpathlineto{\pgfqpoint{2.000013in}{3.785691in}}%
\pgfpathlineto{\pgfqpoint{2.008285in}{3.734400in}}%
\pgfpathlineto{\pgfqpoint{2.016561in}{3.734400in}}%
\pgfpathlineto{\pgfqpoint{2.024838in}{3.739063in}}%
\pgfpathlineto{\pgfqpoint{2.033115in}{3.711086in}}%
\pgfpathlineto{\pgfqpoint{2.041386in}{3.757714in}}%
\pgfpathlineto{\pgfqpoint{2.049663in}{3.734400in}}%
\pgfpathlineto{\pgfqpoint{2.057933in}{3.734400in}}%
\pgfpathlineto{\pgfqpoint{2.066210in}{3.762377in}}%
\pgfpathlineto{\pgfqpoint{2.074487in}{3.785691in}}%
\pgfpathlineto{\pgfqpoint{2.082757in}{3.711086in}}%
\pgfpathlineto{\pgfqpoint{2.091034in}{3.757714in}}%
\pgfpathlineto{\pgfqpoint{2.099312in}{3.762377in}}%
\pgfpathlineto{\pgfqpoint{2.107583in}{3.711086in}}%
\pgfpathlineto{\pgfqpoint{2.115860in}{3.739063in}}%
\pgfpathlineto{\pgfqpoint{2.124130in}{3.785691in}}%
\pgfpathlineto{\pgfqpoint{2.140673in}{3.785691in}}%
\pgfpathlineto{\pgfqpoint{2.148951in}{3.711086in}}%
\pgfpathlineto{\pgfqpoint{2.157228in}{3.785691in}}%
\pgfpathlineto{\pgfqpoint{2.165505in}{3.785691in}}%
\pgfpathlineto{\pgfqpoint{2.182057in}{3.739063in}}%
\pgfpathlineto{\pgfqpoint{2.190336in}{3.711086in}}%
\pgfpathlineto{\pgfqpoint{2.198613in}{3.785691in}}%
\pgfpathlineto{\pgfqpoint{2.206890in}{3.762377in}}%
\pgfpathlineto{\pgfqpoint{2.215161in}{3.762377in}}%
\pgfpathlineto{\pgfqpoint{2.223435in}{3.809006in}}%
\pgfpathlineto{\pgfqpoint{2.239982in}{3.762377in}}%
\pgfpathlineto{\pgfqpoint{2.248259in}{3.757714in}}%
\pgfpathlineto{\pgfqpoint{2.256536in}{3.762377in}}%
\pgfpathlineto{\pgfqpoint{2.264812in}{3.785691in}}%
\pgfpathlineto{\pgfqpoint{2.297919in}{3.785691in}}%
\pgfpathlineto{\pgfqpoint{2.306195in}{3.739063in}}%
\pgfpathlineto{\pgfqpoint{2.314472in}{3.785691in}}%
\pgfpathlineto{\pgfqpoint{2.322748in}{3.785691in}}%
\pgfpathlineto{\pgfqpoint{2.331025in}{3.711086in}}%
\pgfpathlineto{\pgfqpoint{2.339302in}{3.739063in}}%
\pgfpathlineto{\pgfqpoint{2.347578in}{3.785691in}}%
\pgfpathlineto{\pgfqpoint{2.355855in}{3.762377in}}%
\pgfpathlineto{\pgfqpoint{2.364131in}{3.809006in}}%
\pgfpathlineto{\pgfqpoint{2.372408in}{3.762377in}}%
\pgfpathlineto{\pgfqpoint{2.380685in}{3.785691in}}%
\pgfpathlineto{\pgfqpoint{2.388961in}{3.734400in}}%
\pgfpathlineto{\pgfqpoint{2.397238in}{3.762377in}}%
\pgfpathlineto{\pgfqpoint{2.405514in}{3.762377in}}%
\pgfpathlineto{\pgfqpoint{2.413791in}{3.785691in}}%
\pgfpathlineto{\pgfqpoint{2.422068in}{3.785691in}}%
\pgfpathlineto{\pgfqpoint{2.430344in}{3.809006in}}%
\pgfpathlineto{\pgfqpoint{2.438618in}{3.785691in}}%
\pgfpathlineto{\pgfqpoint{2.455172in}{3.785691in}}%
\pgfpathlineto{\pgfqpoint{2.463442in}{3.762377in}}%
\pgfpathlineto{\pgfqpoint{2.471719in}{3.785691in}}%
\pgfpathlineto{\pgfqpoint{2.479997in}{3.762377in}}%
\pgfpathlineto{\pgfqpoint{2.488272in}{3.809006in}}%
\pgfpathlineto{\pgfqpoint{2.496549in}{3.762377in}}%
\pgfpathlineto{\pgfqpoint{2.504826in}{3.739063in}}%
\pgfpathlineto{\pgfqpoint{2.513104in}{3.809006in}}%
\pgfpathlineto{\pgfqpoint{2.529658in}{3.809006in}}%
\pgfpathlineto{\pgfqpoint{2.537934in}{3.785691in}}%
\pgfpathlineto{\pgfqpoint{2.546211in}{3.809006in}}%
\pgfpathlineto{\pgfqpoint{2.554488in}{3.785691in}}%
\pgfpathlineto{\pgfqpoint{2.579317in}{3.785691in}}%
\pgfpathlineto{\pgfqpoint{2.587594in}{3.762377in}}%
\pgfpathlineto{\pgfqpoint{2.595871in}{3.785691in}}%
\pgfpathlineto{\pgfqpoint{2.604147in}{3.762377in}}%
\pgfpathlineto{\pgfqpoint{2.612423in}{3.785691in}}%
\pgfpathlineto{\pgfqpoint{2.620700in}{3.739063in}}%
\pgfpathlineto{\pgfqpoint{2.628977in}{3.762377in}}%
\pgfpathlineto{\pgfqpoint{2.637254in}{3.739063in}}%
\pgfpathlineto{\pgfqpoint{2.645530in}{3.785691in}}%
\pgfpathlineto{\pgfqpoint{2.653807in}{3.809006in}}%
\pgfpathlineto{\pgfqpoint{2.662084in}{3.762377in}}%
\pgfpathlineto{\pgfqpoint{2.670361in}{3.762377in}}%
\pgfpathlineto{\pgfqpoint{2.678638in}{3.785691in}}%
\pgfpathlineto{\pgfqpoint{2.686912in}{3.762377in}}%
\pgfpathlineto{\pgfqpoint{2.695188in}{3.762377in}}%
\pgfpathlineto{\pgfqpoint{2.711730in}{3.809006in}}%
\pgfpathlineto{\pgfqpoint{2.720002in}{3.739063in}}%
\pgfpathlineto{\pgfqpoint{2.736551in}{3.785691in}}%
\pgfpathlineto{\pgfqpoint{2.753104in}{3.785691in}}%
\pgfpathlineto{\pgfqpoint{2.761376in}{3.711086in}}%
\pgfpathlineto{\pgfqpoint{2.769653in}{3.785691in}}%
\pgfpathlineto{\pgfqpoint{2.786207in}{3.785691in}}%
\pgfpathlineto{\pgfqpoint{2.794484in}{3.739063in}}%
\pgfpathlineto{\pgfqpoint{2.802761in}{3.809006in}}%
\pgfpathlineto{\pgfqpoint{2.819315in}{3.809006in}}%
\pgfpathlineto{\pgfqpoint{2.827592in}{3.762377in}}%
\pgfpathlineto{\pgfqpoint{2.835869in}{3.832320in}}%
\pgfpathlineto{\pgfqpoint{2.844147in}{3.762377in}}%
\pgfpathlineto{\pgfqpoint{2.852423in}{3.785691in}}%
\pgfpathlineto{\pgfqpoint{2.860700in}{3.762377in}}%
\pgfpathlineto{\pgfqpoint{2.868976in}{3.785691in}}%
\pgfpathlineto{\pgfqpoint{2.885529in}{3.785691in}}%
\pgfpathlineto{\pgfqpoint{2.893806in}{3.762377in}}%
\pgfpathlineto{\pgfqpoint{2.902083in}{3.785691in}}%
\pgfpathlineto{\pgfqpoint{2.910359in}{3.785691in}}%
\pgfpathlineto{\pgfqpoint{2.918636in}{3.809006in}}%
\pgfpathlineto{\pgfqpoint{2.926912in}{3.762377in}}%
\pgfpathlineto{\pgfqpoint{2.935189in}{3.762377in}}%
\pgfpathlineto{\pgfqpoint{2.943466in}{3.809006in}}%
\pgfpathlineto{\pgfqpoint{2.951742in}{3.785691in}}%
\pgfpathlineto{\pgfqpoint{2.960019in}{3.809006in}}%
\pgfpathlineto{\pgfqpoint{2.968295in}{3.739063in}}%
\pgfpathlineto{\pgfqpoint{2.976572in}{3.739063in}}%
\pgfpathlineto{\pgfqpoint{2.993125in}{3.785691in}}%
\pgfpathlineto{\pgfqpoint{3.001402in}{3.785691in}}%
\pgfpathlineto{\pgfqpoint{3.009678in}{3.762377in}}%
\pgfpathlineto{\pgfqpoint{3.017955in}{3.809006in}}%
\pgfpathlineto{\pgfqpoint{3.026234in}{3.785691in}}%
\pgfpathlineto{\pgfqpoint{3.042787in}{3.785691in}}%
\pgfpathlineto{\pgfqpoint{3.051064in}{3.711086in}}%
\pgfpathlineto{\pgfqpoint{3.059341in}{3.739063in}}%
\pgfpathlineto{\pgfqpoint{3.075895in}{3.785691in}}%
\pgfpathlineto{\pgfqpoint{3.092440in}{3.785691in}}%
\pgfpathlineto{\pgfqpoint{3.100717in}{3.762377in}}%
\pgfpathlineto{\pgfqpoint{3.108995in}{3.785691in}}%
\pgfpathlineto{\pgfqpoint{3.117270in}{3.785691in}}%
\pgfpathlineto{\pgfqpoint{3.125545in}{3.762377in}}%
\pgfpathlineto{\pgfqpoint{3.158652in}{3.762377in}}%
\pgfpathlineto{\pgfqpoint{3.166928in}{3.785691in}}%
\pgfpathlineto{\pgfqpoint{3.175199in}{3.739063in}}%
\pgfpathlineto{\pgfqpoint{3.183476in}{3.785691in}}%
\pgfpathlineto{\pgfqpoint{3.191750in}{3.711086in}}%
\pgfpathlineto{\pgfqpoint{3.200025in}{3.762377in}}%
\pgfpathlineto{\pgfqpoint{3.208302in}{3.762377in}}%
\pgfpathlineto{\pgfqpoint{3.216579in}{3.739063in}}%
\pgfpathlineto{\pgfqpoint{3.224857in}{3.739063in}}%
\pgfpathlineto{\pgfqpoint{3.249683in}{3.809006in}}%
\pgfpathlineto{\pgfqpoint{3.257960in}{3.785691in}}%
\pgfpathlineto{\pgfqpoint{3.266236in}{3.739063in}}%
\pgfpathlineto{\pgfqpoint{3.274513in}{3.762377in}}%
\pgfpathlineto{\pgfqpoint{3.282790in}{3.739063in}}%
\pgfpathlineto{\pgfqpoint{3.291066in}{3.739063in}}%
\pgfpathlineto{\pgfqpoint{3.299343in}{3.762377in}}%
\pgfpathlineto{\pgfqpoint{3.307616in}{3.762377in}}%
\pgfpathlineto{\pgfqpoint{3.315893in}{3.785691in}}%
\pgfpathlineto{\pgfqpoint{3.324170in}{3.762377in}}%
\pgfpathlineto{\pgfqpoint{3.332446in}{3.762377in}}%
\pgfpathlineto{\pgfqpoint{3.340722in}{3.785691in}}%
\pgfpathlineto{\pgfqpoint{3.349000in}{3.762377in}}%
\pgfpathlineto{\pgfqpoint{3.357270in}{3.785691in}}%
\pgfpathlineto{\pgfqpoint{3.365547in}{3.762377in}}%
\pgfpathlineto{\pgfqpoint{3.373822in}{3.785691in}}%
\pgfpathlineto{\pgfqpoint{3.382094in}{3.785691in}}%
\pgfpathlineto{\pgfqpoint{3.390365in}{3.809006in}}%
\pgfpathlineto{\pgfqpoint{3.398643in}{3.762377in}}%
\pgfpathlineto{\pgfqpoint{3.406917in}{3.739063in}}%
\pgfpathlineto{\pgfqpoint{3.415193in}{3.785691in}}%
\pgfpathlineto{\pgfqpoint{3.423470in}{3.762377in}}%
\pgfpathlineto{\pgfqpoint{3.431746in}{3.785691in}}%
\pgfpathlineto{\pgfqpoint{3.440020in}{3.739063in}}%
\pgfpathlineto{\pgfqpoint{3.448297in}{3.762377in}}%
\pgfpathlineto{\pgfqpoint{3.456574in}{3.809006in}}%
\pgfpathlineto{\pgfqpoint{3.473123in}{3.809006in}}%
\pgfpathlineto{\pgfqpoint{3.481401in}{3.711086in}}%
\pgfpathlineto{\pgfqpoint{3.489677in}{3.762377in}}%
\pgfpathlineto{\pgfqpoint{3.497948in}{3.711086in}}%
\pgfpathlineto{\pgfqpoint{3.506225in}{3.739063in}}%
\pgfpathlineto{\pgfqpoint{3.514503in}{3.739063in}}%
\pgfpathlineto{\pgfqpoint{3.522780in}{3.762377in}}%
\pgfpathlineto{\pgfqpoint{3.531058in}{3.762377in}}%
\pgfpathlineto{\pgfqpoint{3.539335in}{3.785691in}}%
\pgfpathlineto{\pgfqpoint{3.547610in}{3.739063in}}%
\pgfpathlineto{\pgfqpoint{3.564163in}{3.739063in}}%
\pgfpathlineto{\pgfqpoint{3.572440in}{3.762377in}}%
\pgfpathlineto{\pgfqpoint{3.580711in}{3.739063in}}%
\pgfpathlineto{\pgfqpoint{3.588988in}{3.785691in}}%
\pgfpathlineto{\pgfqpoint{3.597265in}{3.762377in}}%
\pgfpathlineto{\pgfqpoint{3.613815in}{3.762377in}}%
\pgfpathlineto{\pgfqpoint{3.622092in}{3.734400in}}%
\pgfpathlineto{\pgfqpoint{3.630362in}{3.739063in}}%
\pgfpathlineto{\pgfqpoint{3.638640in}{3.757714in}}%
\pgfpathlineto{\pgfqpoint{3.646916in}{3.762377in}}%
\pgfpathlineto{\pgfqpoint{3.655193in}{3.739063in}}%
\pgfpathlineto{\pgfqpoint{3.663469in}{3.785691in}}%
\pgfpathlineto{\pgfqpoint{3.688298in}{3.785691in}}%
\pgfpathlineto{\pgfqpoint{3.696575in}{3.762377in}}%
\pgfpathlineto{\pgfqpoint{3.704852in}{3.711086in}}%
\pgfpathlineto{\pgfqpoint{3.713128in}{3.739063in}}%
\pgfpathlineto{\pgfqpoint{3.721402in}{3.762377in}}%
\pgfpathlineto{\pgfqpoint{3.729678in}{3.762377in}}%
\pgfpathlineto{\pgfqpoint{3.737949in}{3.739063in}}%
\pgfpathlineto{\pgfqpoint{3.746224in}{3.762377in}}%
\pgfpathlineto{\pgfqpoint{3.754500in}{3.739063in}}%
\pgfpathlineto{\pgfqpoint{3.762771in}{3.762377in}}%
\pgfpathlineto{\pgfqpoint{3.771048in}{3.757714in}}%
\pgfpathlineto{\pgfqpoint{3.779325in}{3.734400in}}%
\pgfpathlineto{\pgfqpoint{3.787601in}{3.739063in}}%
\pgfpathlineto{\pgfqpoint{3.804154in}{3.739063in}}%
\pgfpathlineto{\pgfqpoint{3.812432in}{3.762377in}}%
\pgfpathlineto{\pgfqpoint{3.820709in}{3.739063in}}%
\pgfpathlineto{\pgfqpoint{3.828987in}{3.785691in}}%
\pgfpathlineto{\pgfqpoint{3.837264in}{3.739063in}}%
\pgfpathlineto{\pgfqpoint{3.845541in}{3.739063in}}%
\pgfpathlineto{\pgfqpoint{3.853816in}{3.711086in}}%
\pgfpathlineto{\pgfqpoint{3.862090in}{3.762377in}}%
\pgfpathlineto{\pgfqpoint{3.870362in}{3.785691in}}%
\pgfpathlineto{\pgfqpoint{3.878638in}{3.739063in}}%
\pgfpathlineto{\pgfqpoint{3.886911in}{3.762377in}}%
\pgfpathlineto{\pgfqpoint{3.911735in}{3.762377in}}%
\pgfpathlineto{\pgfqpoint{3.920011in}{3.785691in}}%
\pgfpathlineto{\pgfqpoint{3.928284in}{3.739063in}}%
\pgfpathlineto{\pgfqpoint{3.936556in}{3.711086in}}%
\pgfpathlineto{\pgfqpoint{3.944832in}{3.739063in}}%
\pgfpathlineto{\pgfqpoint{3.953108in}{3.739063in}}%
\pgfpathlineto{\pgfqpoint{3.961385in}{3.711086in}}%
\pgfpathlineto{\pgfqpoint{3.969662in}{3.762377in}}%
\pgfpathlineto{\pgfqpoint{3.977938in}{3.762377in}}%
\pgfpathlineto{\pgfqpoint{3.986215in}{3.785691in}}%
\pgfpathlineto{\pgfqpoint{4.002766in}{3.739063in}}%
\pgfpathlineto{\pgfqpoint{4.011043in}{3.739063in}}%
\pgfpathlineto{\pgfqpoint{4.019320in}{3.734400in}}%
\pgfpathlineto{\pgfqpoint{4.027597in}{3.739063in}}%
\pgfpathlineto{\pgfqpoint{4.035875in}{3.739063in}}%
\pgfpathlineto{\pgfqpoint{4.044152in}{3.762377in}}%
\pgfpathlineto{\pgfqpoint{4.052429in}{3.687771in}}%
\pgfpathlineto{\pgfqpoint{4.060702in}{3.785691in}}%
\pgfpathlineto{\pgfqpoint{4.068976in}{3.687771in}}%
\pgfpathlineto{\pgfqpoint{4.077253in}{3.711086in}}%
\pgfpathlineto{\pgfqpoint{4.085530in}{3.762377in}}%
\pgfpathlineto{\pgfqpoint{4.093804in}{3.785691in}}%
\pgfpathlineto{\pgfqpoint{4.102076in}{3.785691in}}%
\pgfpathlineto{\pgfqpoint{4.110353in}{3.739063in}}%
\pgfpathlineto{\pgfqpoint{4.118629in}{3.785691in}}%
\pgfpathlineto{\pgfqpoint{4.126900in}{3.711086in}}%
\pgfpathlineto{\pgfqpoint{4.135177in}{3.739063in}}%
\pgfpathlineto{\pgfqpoint{4.151730in}{3.739063in}}%
\pgfpathlineto{\pgfqpoint{4.160007in}{3.711086in}}%
\pgfpathlineto{\pgfqpoint{4.168284in}{3.762377in}}%
\pgfpathlineto{\pgfqpoint{4.176561in}{3.762377in}}%
\pgfpathlineto{\pgfqpoint{4.184836in}{3.711086in}}%
\pgfpathlineto{\pgfqpoint{4.193113in}{3.739063in}}%
\pgfpathlineto{\pgfqpoint{4.201391in}{3.687771in}}%
\pgfpathlineto{\pgfqpoint{4.217939in}{3.785691in}}%
\pgfpathlineto{\pgfqpoint{4.226216in}{3.739063in}}%
\pgfpathlineto{\pgfqpoint{4.234487in}{3.785691in}}%
\pgfpathlineto{\pgfqpoint{4.242764in}{3.734400in}}%
\pgfpathlineto{\pgfqpoint{4.251041in}{3.762377in}}%
\pgfpathlineto{\pgfqpoint{4.259313in}{3.762377in}}%
\pgfpathlineto{\pgfqpoint{4.267589in}{3.739063in}}%
\pgfpathlineto{\pgfqpoint{4.275866in}{3.711086in}}%
\pgfpathlineto{\pgfqpoint{4.284142in}{3.739063in}}%
\pgfpathlineto{\pgfqpoint{4.292418in}{3.762377in}}%
\pgfpathlineto{\pgfqpoint{4.300693in}{3.762377in}}%
\pgfpathlineto{\pgfqpoint{4.308969in}{3.785691in}}%
\pgfpathlineto{\pgfqpoint{4.317246in}{3.762377in}}%
\pgfpathlineto{\pgfqpoint{4.325522in}{3.762377in}}%
\pgfpathlineto{\pgfqpoint{4.333799in}{3.785691in}}%
\pgfpathlineto{\pgfqpoint{4.350347in}{3.739063in}}%
\pgfpathlineto{\pgfqpoint{4.375166in}{3.739063in}}%
\pgfpathlineto{\pgfqpoint{4.383442in}{3.762377in}}%
\pgfpathlineto{\pgfqpoint{4.391719in}{3.739063in}}%
\pgfpathlineto{\pgfqpoint{4.399996in}{3.785691in}}%
\pgfpathlineto{\pgfqpoint{4.408272in}{3.739063in}}%
\pgfpathlineto{\pgfqpoint{4.416541in}{3.762377in}}%
\pgfpathlineto{\pgfqpoint{4.424817in}{3.711086in}}%
\pgfpathlineto{\pgfqpoint{4.433094in}{3.762377in}}%
\pgfpathlineto{\pgfqpoint{4.441365in}{3.739063in}}%
\pgfpathlineto{\pgfqpoint{4.449637in}{3.762377in}}%
\pgfpathlineto{\pgfqpoint{4.457910in}{3.762377in}}%
\pgfpathlineto{\pgfqpoint{4.466186in}{3.739063in}}%
\pgfpathlineto{\pgfqpoint{4.482739in}{3.739063in}}%
\pgfpathlineto{\pgfqpoint{4.491016in}{3.762377in}}%
\pgfpathlineto{\pgfqpoint{4.499292in}{3.739063in}}%
\pgfpathlineto{\pgfqpoint{4.507567in}{3.739063in}}%
\pgfpathlineto{\pgfqpoint{4.515842in}{3.762377in}}%
\pgfpathlineto{\pgfqpoint{4.524119in}{3.739063in}}%
\pgfpathlineto{\pgfqpoint{4.532396in}{3.739063in}}%
\pgfpathlineto{\pgfqpoint{4.540672in}{3.762377in}}%
\pgfpathlineto{\pgfqpoint{4.557225in}{3.762377in}}%
\pgfpathlineto{\pgfqpoint{4.565502in}{3.785691in}}%
\pgfpathlineto{\pgfqpoint{4.573779in}{3.762377in}}%
\pgfpathlineto{\pgfqpoint{4.582055in}{3.809006in}}%
\pgfpathlineto{\pgfqpoint{4.590332in}{3.762377in}}%
\pgfpathlineto{\pgfqpoint{4.598608in}{3.739063in}}%
\pgfpathlineto{\pgfqpoint{4.606885in}{3.809006in}}%
\pgfpathlineto{\pgfqpoint{4.615162in}{3.809006in}}%
\pgfpathlineto{\pgfqpoint{4.623439in}{3.762377in}}%
\pgfpathlineto{\pgfqpoint{4.631710in}{3.785691in}}%
\pgfpathlineto{\pgfqpoint{4.639987in}{3.739063in}}%
\pgfpathlineto{\pgfqpoint{4.648264in}{3.739063in}}%
\pgfpathlineto{\pgfqpoint{4.656542in}{3.832320in}}%
\pgfpathlineto{\pgfqpoint{4.664819in}{3.762377in}}%
\pgfpathlineto{\pgfqpoint{4.673096in}{3.762377in}}%
\pgfpathlineto{\pgfqpoint{4.681374in}{3.785691in}}%
\pgfpathlineto{\pgfqpoint{4.689648in}{3.762377in}}%
\pgfpathlineto{\pgfqpoint{4.697925in}{3.762377in}}%
\pgfpathlineto{\pgfqpoint{4.706202in}{3.809006in}}%
\pgfpathlineto{\pgfqpoint{4.714479in}{3.762377in}}%
\pgfpathlineto{\pgfqpoint{4.722756in}{3.785691in}}%
\pgfpathlineto{\pgfqpoint{4.747590in}{3.785691in}}%
\pgfpathlineto{\pgfqpoint{4.755867in}{3.762377in}}%
\pgfpathlineto{\pgfqpoint{4.764143in}{3.809006in}}%
\pgfpathlineto{\pgfqpoint{4.772420in}{3.762377in}}%
\pgfpathlineto{\pgfqpoint{4.780696in}{3.785691in}}%
\pgfpathlineto{\pgfqpoint{4.788973in}{3.711086in}}%
\pgfpathlineto{\pgfqpoint{4.797249in}{3.785691in}}%
\pgfpathlineto{\pgfqpoint{4.805518in}{3.762377in}}%
\pgfpathlineto{\pgfqpoint{4.813796in}{3.785691in}}%
\pgfpathlineto{\pgfqpoint{4.822073in}{3.762377in}}%
\pgfpathlineto{\pgfqpoint{4.830349in}{3.762377in}}%
\pgfpathlineto{\pgfqpoint{4.838625in}{3.785691in}}%
\pgfpathlineto{\pgfqpoint{4.846901in}{3.762377in}}%
\pgfpathlineto{\pgfqpoint{4.855177in}{3.813669in}}%
\pgfpathlineto{\pgfqpoint{4.863455in}{3.762377in}}%
\pgfpathlineto{\pgfqpoint{4.871732in}{3.739063in}}%
\pgfpathlineto{\pgfqpoint{4.880002in}{3.785691in}}%
\pgfpathlineto{\pgfqpoint{4.888280in}{3.785691in}}%
\pgfpathlineto{\pgfqpoint{4.896557in}{3.739063in}}%
\pgfpathlineto{\pgfqpoint{4.904834in}{3.762377in}}%
\pgfpathlineto{\pgfqpoint{4.913107in}{3.809006in}}%
\pgfpathlineto{\pgfqpoint{4.921384in}{3.739063in}}%
\pgfpathlineto{\pgfqpoint{4.929661in}{3.785691in}}%
\pgfpathlineto{\pgfqpoint{4.954487in}{3.785691in}}%
\pgfpathlineto{\pgfqpoint{4.962764in}{3.739063in}}%
\pgfpathlineto{\pgfqpoint{4.971042in}{3.809006in}}%
\pgfpathlineto{\pgfqpoint{4.979313in}{3.762377in}}%
\pgfpathlineto{\pgfqpoint{4.995868in}{3.809006in}}%
\pgfpathlineto{\pgfqpoint{5.004143in}{3.739063in}}%
\pgfpathlineto{\pgfqpoint{5.012421in}{3.762377in}}%
\pgfpathlineto{\pgfqpoint{5.020696in}{3.762377in}}%
\pgfpathlineto{\pgfqpoint{5.028974in}{3.785691in}}%
\pgfpathlineto{\pgfqpoint{5.037245in}{3.762377in}}%
\pgfpathlineto{\pgfqpoint{5.045520in}{3.762377in}}%
\pgfpathlineto{\pgfqpoint{5.062069in}{3.809006in}}%
\pgfpathlineto{\pgfqpoint{5.070344in}{3.762377in}}%
\pgfpathlineto{\pgfqpoint{5.086897in}{3.762377in}}%
\pgfpathlineto{\pgfqpoint{5.095175in}{3.785691in}}%
\pgfpathlineto{\pgfqpoint{5.103452in}{3.785691in}}%
\pgfpathlineto{\pgfqpoint{5.111729in}{3.762377in}}%
\pgfpathlineto{\pgfqpoint{5.128281in}{3.762377in}}%
\pgfpathlineto{\pgfqpoint{5.136559in}{3.785691in}}%
\pgfpathlineto{\pgfqpoint{5.153113in}{3.785691in}}%
\pgfpathlineto{\pgfqpoint{5.161389in}{3.762377in}}%
\pgfpathlineto{\pgfqpoint{5.177941in}{3.762377in}}%
\pgfpathlineto{\pgfqpoint{5.186218in}{3.785691in}}%
\pgfpathlineto{\pgfqpoint{5.202765in}{3.739063in}}%
\pgfpathlineto{\pgfqpoint{5.211041in}{3.711086in}}%
\pgfpathlineto{\pgfqpoint{5.219319in}{3.687771in}}%
\pgfpathlineto{\pgfqpoint{5.227600in}{3.739063in}}%
\pgfpathlineto{\pgfqpoint{5.244153in}{3.785691in}}%
\pgfpathlineto{\pgfqpoint{5.252429in}{3.757714in}}%
\pgfpathlineto{\pgfqpoint{5.260701in}{3.785691in}}%
\pgfpathlineto{\pgfqpoint{5.268974in}{3.734400in}}%
\pgfpathlineto{\pgfqpoint{5.277251in}{3.762377in}}%
\pgfpathlineto{\pgfqpoint{5.285525in}{3.785691in}}%
\pgfpathlineto{\pgfqpoint{5.293802in}{3.711086in}}%
\pgfpathlineto{\pgfqpoint{5.302078in}{3.762377in}}%
\pgfpathlineto{\pgfqpoint{5.310354in}{3.739063in}}%
\pgfpathlineto{\pgfqpoint{5.318628in}{3.785691in}}%
\pgfpathlineto{\pgfqpoint{5.326905in}{3.809006in}}%
\pgfpathlineto{\pgfqpoint{5.335179in}{3.739063in}}%
\pgfpathlineto{\pgfqpoint{5.343453in}{3.762377in}}%
\pgfpathlineto{\pgfqpoint{5.351730in}{3.762377in}}%
\pgfpathlineto{\pgfqpoint{5.360002in}{3.739063in}}%
\pgfpathlineto{\pgfqpoint{5.368278in}{3.711086in}}%
\pgfpathlineto{\pgfqpoint{5.376555in}{3.762377in}}%
\pgfpathlineto{\pgfqpoint{5.384830in}{3.739063in}}%
\pgfpathlineto{\pgfqpoint{5.393107in}{3.785691in}}%
\pgfpathlineto{\pgfqpoint{5.401383in}{3.734400in}}%
\pgfpathlineto{\pgfqpoint{5.409660in}{3.785691in}}%
\pgfpathlineto{\pgfqpoint{5.417933in}{3.762377in}}%
\pgfpathlineto{\pgfqpoint{5.426205in}{3.762377in}}%
\pgfpathlineto{\pgfqpoint{5.434476in}{3.711086in}}%
\pgfpathlineto{\pgfqpoint{5.442748in}{3.762377in}}%
\pgfpathlineto{\pgfqpoint{5.451018in}{3.734400in}}%
\pgfpathlineto{\pgfqpoint{5.459295in}{3.762377in}}%
\pgfpathlineto{\pgfqpoint{5.467571in}{3.739063in}}%
\pgfpathlineto{\pgfqpoint{5.475848in}{3.762377in}}%
\pgfpathlineto{\pgfqpoint{5.492401in}{3.762377in}}%
\pgfpathlineto{\pgfqpoint{5.500677in}{3.739063in}}%
\pgfpathlineto{\pgfqpoint{5.508954in}{3.739063in}}%
\pgfpathlineto{\pgfqpoint{5.517228in}{3.235474in}}%
\pgfpathlineto{\pgfqpoint{5.525498in}{3.137554in}}%
\pgfpathlineto{\pgfqpoint{5.533777in}{3.109577in}}%
\pgfpathlineto{\pgfqpoint{5.542055in}{3.030309in}}%
\pgfpathlineto{\pgfqpoint{5.550333in}{3.030309in}}%
\pgfpathlineto{\pgfqpoint{5.558612in}{2.932389in}}%
\pgfpathlineto{\pgfqpoint{5.566891in}{2.937051in}}%
\pgfpathlineto{\pgfqpoint{5.575169in}{2.937051in}}%
\pgfpathlineto{\pgfqpoint{5.583447in}{2.909074in}}%
\pgfpathlineto{\pgfqpoint{5.591726in}{2.909074in}}%
\pgfpathlineto{\pgfqpoint{5.600004in}{2.885760in}}%
\pgfpathlineto{\pgfqpoint{5.608275in}{2.834469in}}%
\pgfpathlineto{\pgfqpoint{5.616553in}{2.857783in}}%
\pgfpathlineto{\pgfqpoint{5.624829in}{2.834469in}}%
\pgfpathlineto{\pgfqpoint{5.633102in}{2.857783in}}%
\pgfpathlineto{\pgfqpoint{5.649654in}{2.857783in}}%
\pgfpathlineto{\pgfqpoint{5.657930in}{2.834469in}}%
\pgfpathlineto{\pgfqpoint{5.666206in}{2.834469in}}%
\pgfpathlineto{\pgfqpoint{5.674482in}{2.783177in}}%
\pgfpathlineto{\pgfqpoint{5.682755in}{2.806491in}}%
\pgfpathlineto{\pgfqpoint{5.715869in}{2.806491in}}%
\pgfpathlineto{\pgfqpoint{5.724140in}{2.834469in}}%
\pgfpathlineto{\pgfqpoint{5.732416in}{2.806491in}}%
\pgfpathlineto{\pgfqpoint{5.740688in}{2.806491in}}%
\pgfpathlineto{\pgfqpoint{5.748966in}{2.755200in}}%
\pgfpathlineto{\pgfqpoint{5.757245in}{2.755200in}}%
\pgfpathlineto{\pgfqpoint{5.757245in}{2.755200in}}%
\pgfusepath{stroke}%
\end{pgfscope}%
\begin{pgfscope}%
\pgfpathrectangle{\pgfqpoint{0.800000in}{0.960000in}}{\pgfqpoint{4.960000in}{3.264000in}}%
\pgfusepath{clip}%
\pgfsetrectcap%
\pgfsetroundjoin%
\pgfsetlinewidth{1.505625pt}%
\definecolor{currentstroke}{rgb}{1.000000,0.498039,0.054902}%
\pgfsetstrokecolor{currentstroke}%
\pgfsetdash{}{0pt}%
\pgfpathmoveto{\pgfqpoint{0.790000in}{1.516809in}}%
\pgfpathlineto{\pgfqpoint{0.801515in}{1.528286in}}%
\pgfpathlineto{\pgfqpoint{0.810288in}{1.534114in}}%
\pgfpathlineto{\pgfqpoint{0.819060in}{1.542857in}}%
\pgfpathlineto{\pgfqpoint{0.827833in}{1.545771in}}%
\pgfpathlineto{\pgfqpoint{0.854151in}{1.563257in}}%
\pgfpathlineto{\pgfqpoint{0.862923in}{1.572000in}}%
\pgfpathlineto{\pgfqpoint{0.871696in}{1.577829in}}%
\pgfpathlineto{\pgfqpoint{0.880468in}{1.589486in}}%
\pgfpathlineto{\pgfqpoint{0.898013in}{1.601143in}}%
\pgfpathlineto{\pgfqpoint{0.906786in}{1.609886in}}%
\pgfpathlineto{\pgfqpoint{0.915562in}{1.615714in}}%
\pgfpathlineto{\pgfqpoint{0.924331in}{1.624457in}}%
\pgfpathlineto{\pgfqpoint{0.933103in}{1.630286in}}%
\pgfpathlineto{\pgfqpoint{0.941876in}{1.639029in}}%
\pgfpathlineto{\pgfqpoint{0.968193in}{1.656514in}}%
\pgfpathlineto{\pgfqpoint{0.976966in}{1.665257in}}%
\pgfpathlineto{\pgfqpoint{0.994511in}{1.676914in}}%
\pgfpathlineto{\pgfqpoint{1.003284in}{1.679829in}}%
\pgfpathlineto{\pgfqpoint{1.029601in}{1.697314in}}%
\pgfpathlineto{\pgfqpoint{1.038374in}{1.700229in}}%
\pgfpathlineto{\pgfqpoint{1.055919in}{1.711886in}}%
\pgfpathlineto{\pgfqpoint{1.073464in}{1.717714in}}%
\pgfpathlineto{\pgfqpoint{1.082236in}{1.723543in}}%
\pgfpathlineto{\pgfqpoint{1.091009in}{1.723543in}}%
\pgfpathlineto{\pgfqpoint{1.099781in}{1.729371in}}%
\pgfpathlineto{\pgfqpoint{1.108576in}{1.732286in}}%
\pgfpathlineto{\pgfqpoint{1.117348in}{1.738114in}}%
\pgfpathlineto{\pgfqpoint{1.126121in}{1.738114in}}%
\pgfpathlineto{\pgfqpoint{1.134894in}{1.743943in}}%
\pgfpathlineto{\pgfqpoint{1.152439in}{1.749771in}}%
\pgfpathlineto{\pgfqpoint{1.161454in}{1.749771in}}%
\pgfpathlineto{\pgfqpoint{1.170226in}{1.755600in}}%
\pgfpathlineto{\pgfqpoint{1.187771in}{1.761429in}}%
\pgfpathlineto{\pgfqpoint{1.196538in}{1.761429in}}%
\pgfpathlineto{\pgfqpoint{1.205294in}{1.767257in}}%
\pgfpathlineto{\pgfqpoint{1.214067in}{1.767257in}}%
\pgfpathlineto{\pgfqpoint{1.222839in}{1.773086in}}%
\pgfpathlineto{\pgfqpoint{1.240384in}{1.773086in}}%
\pgfpathlineto{\pgfqpoint{1.249157in}{1.778914in}}%
\pgfpathlineto{\pgfqpoint{1.257929in}{1.778914in}}%
\pgfpathlineto{\pgfqpoint{1.275474in}{1.784743in}}%
\pgfpathlineto{\pgfqpoint{1.293019in}{1.784743in}}%
\pgfpathlineto{\pgfqpoint{1.319337in}{1.793486in}}%
\pgfpathlineto{\pgfqpoint{1.336882in}{1.793486in}}%
\pgfpathlineto{\pgfqpoint{1.354427in}{1.799314in}}%
\pgfpathlineto{\pgfqpoint{1.371972in}{1.799314in}}%
\pgfpathlineto{\pgfqpoint{1.380745in}{1.805143in}}%
\pgfpathlineto{\pgfqpoint{1.398532in}{1.805143in}}%
\pgfpathlineto{\pgfqpoint{1.407305in}{1.808057in}}%
\pgfpathlineto{\pgfqpoint{1.424850in}{1.808057in}}%
\pgfpathlineto{\pgfqpoint{1.433622in}{1.813886in}}%
\pgfpathlineto{\pgfqpoint{1.477970in}{1.813886in}}%
\pgfpathlineto{\pgfqpoint{1.486743in}{1.816800in}}%
\pgfpathlineto{\pgfqpoint{1.495515in}{1.816800in}}%
\pgfpathlineto{\pgfqpoint{1.504284in}{1.819714in}}%
\pgfpathlineto{\pgfqpoint{1.521833in}{1.819714in}}%
\pgfpathlineto{\pgfqpoint{1.530605in}{1.822629in}}%
\pgfpathlineto{\pgfqpoint{1.539378in}{1.822629in}}%
\pgfpathlineto{\pgfqpoint{1.548150in}{1.825543in}}%
\pgfpathlineto{\pgfqpoint{1.574710in}{1.825543in}}%
\pgfpathlineto{\pgfqpoint{1.583482in}{1.828457in}}%
\pgfpathlineto{\pgfqpoint{1.618566in}{1.828457in}}%
\pgfpathlineto{\pgfqpoint{1.627323in}{1.831371in}}%
\pgfpathlineto{\pgfqpoint{1.662414in}{1.831371in}}%
\pgfpathlineto{\pgfqpoint{1.671186in}{1.834286in}}%
\pgfpathlineto{\pgfqpoint{1.697504in}{1.834286in}}%
\pgfpathlineto{\pgfqpoint{1.706276in}{1.837200in}}%
\pgfpathlineto{\pgfqpoint{1.750139in}{1.837200in}}%
\pgfpathlineto{\pgfqpoint{1.758911in}{1.840114in}}%
\pgfpathlineto{\pgfqpoint{1.776456in}{1.840114in}}%
\pgfpathlineto{\pgfqpoint{1.785229in}{1.843029in}}%
\pgfpathlineto{\pgfqpoint{1.829092in}{1.843029in}}%
\pgfpathlineto{\pgfqpoint{1.837864in}{1.845943in}}%
\pgfpathlineto{\pgfqpoint{1.890765in}{1.845943in}}%
\pgfpathlineto{\pgfqpoint{1.899536in}{1.848857in}}%
\pgfpathlineto{\pgfqpoint{1.908309in}{1.845943in}}%
\pgfpathlineto{\pgfqpoint{1.917081in}{1.848857in}}%
\pgfpathlineto{\pgfqpoint{1.952172in}{1.848857in}}%
\pgfpathlineto{\pgfqpoint{1.960944in}{1.851771in}}%
\pgfpathlineto{\pgfqpoint{1.969717in}{1.848857in}}%
\pgfpathlineto{\pgfqpoint{1.978489in}{1.848857in}}%
\pgfpathlineto{\pgfqpoint{1.987262in}{1.854686in}}%
\pgfpathlineto{\pgfqpoint{2.039919in}{1.854686in}}%
\pgfpathlineto{\pgfqpoint{2.048691in}{1.857600in}}%
\pgfpathlineto{\pgfqpoint{2.083782in}{1.857600in}}%
\pgfpathlineto{\pgfqpoint{2.092554in}{1.860514in}}%
\pgfpathlineto{\pgfqpoint{2.136417in}{1.860514in}}%
\pgfpathlineto{\pgfqpoint{2.145189in}{1.863429in}}%
\pgfpathlineto{\pgfqpoint{2.171507in}{1.863429in}}%
\pgfpathlineto{\pgfqpoint{2.180279in}{1.866343in}}%
\pgfpathlineto{\pgfqpoint{2.215369in}{1.866343in}}%
\pgfpathlineto{\pgfqpoint{2.224142in}{1.869257in}}%
\pgfpathlineto{\pgfqpoint{2.259226in}{1.869257in}}%
\pgfpathlineto{\pgfqpoint{2.267983in}{1.872171in}}%
\pgfpathlineto{\pgfqpoint{2.303073in}{1.872171in}}%
\pgfpathlineto{\pgfqpoint{2.311845in}{1.875086in}}%
\pgfpathlineto{\pgfqpoint{2.346913in}{1.875086in}}%
\pgfpathlineto{\pgfqpoint{2.355686in}{1.878000in}}%
\pgfpathlineto{\pgfqpoint{2.417094in}{1.878000in}}%
\pgfpathlineto{\pgfqpoint{2.425863in}{1.880914in}}%
\pgfpathlineto{\pgfqpoint{2.504819in}{1.880914in}}%
\pgfpathlineto{\pgfqpoint{2.513591in}{1.883829in}}%
\pgfpathlineto{\pgfqpoint{2.636649in}{1.883829in}}%
\pgfpathlineto{\pgfqpoint{2.645422in}{1.886743in}}%
\pgfpathlineto{\pgfqpoint{2.654194in}{1.886743in}}%
\pgfpathlineto{\pgfqpoint{2.662967in}{1.883829in}}%
\pgfpathlineto{\pgfqpoint{2.864735in}{1.883829in}}%
\pgfpathlineto{\pgfqpoint{2.873508in}{1.880914in}}%
\pgfpathlineto{\pgfqpoint{2.987573in}{1.880914in}}%
\pgfpathlineto{\pgfqpoint{2.996345in}{1.878000in}}%
\pgfpathlineto{\pgfqpoint{3.084070in}{1.878000in}}%
\pgfpathlineto{\pgfqpoint{3.092843in}{1.875086in}}%
\pgfpathlineto{\pgfqpoint{3.101615in}{1.878000in}}%
\pgfpathlineto{\pgfqpoint{3.119161in}{1.878000in}}%
\pgfpathlineto{\pgfqpoint{3.136706in}{1.872171in}}%
\pgfpathlineto{\pgfqpoint{3.259521in}{1.872171in}}%
\pgfpathlineto{\pgfqpoint{3.268294in}{1.869257in}}%
\pgfpathlineto{\pgfqpoint{3.277066in}{1.869257in}}%
\pgfpathlineto{\pgfqpoint{3.285839in}{1.872171in}}%
\pgfpathlineto{\pgfqpoint{3.294611in}{1.869257in}}%
\pgfpathlineto{\pgfqpoint{3.399882in}{1.869257in}}%
\pgfpathlineto{\pgfqpoint{3.408654in}{1.866343in}}%
\pgfpathlineto{\pgfqpoint{3.426199in}{1.866343in}}%
\pgfpathlineto{\pgfqpoint{3.434972in}{1.863429in}}%
\pgfpathlineto{\pgfqpoint{3.443744in}{1.866343in}}%
\pgfpathlineto{\pgfqpoint{3.452759in}{1.863429in}}%
\pgfpathlineto{\pgfqpoint{3.496644in}{1.863429in}}%
\pgfpathlineto{\pgfqpoint{3.505416in}{1.860514in}}%
\pgfpathlineto{\pgfqpoint{3.558052in}{1.860514in}}%
\pgfpathlineto{\pgfqpoint{3.566824in}{1.857600in}}%
\pgfpathlineto{\pgfqpoint{3.584369in}{1.857600in}}%
\pgfpathlineto{\pgfqpoint{3.593142in}{1.854686in}}%
\pgfpathlineto{\pgfqpoint{3.707185in}{1.854686in}}%
\pgfpathlineto{\pgfqpoint{3.715957in}{1.851771in}}%
\pgfpathlineto{\pgfqpoint{3.856318in}{1.851771in}}%
\pgfpathlineto{\pgfqpoint{3.865090in}{1.848857in}}%
\pgfpathlineto{\pgfqpoint{3.970360in}{1.848857in}}%
\pgfpathlineto{\pgfqpoint{3.979133in}{1.845943in}}%
\pgfpathlineto{\pgfqpoint{3.996678in}{1.845943in}}%
\pgfpathlineto{\pgfqpoint{4.005451in}{1.848857in}}%
\pgfpathlineto{\pgfqpoint{4.014223in}{1.845943in}}%
\pgfpathlineto{\pgfqpoint{4.022996in}{1.848857in}}%
\pgfpathlineto{\pgfqpoint{4.031768in}{1.848857in}}%
\pgfpathlineto{\pgfqpoint{4.040541in}{1.845943in}}%
\pgfpathlineto{\pgfqpoint{4.154584in}{1.845943in}}%
\pgfpathlineto{\pgfqpoint{4.163356in}{1.848857in}}%
\pgfpathlineto{\pgfqpoint{4.172128in}{1.845943in}}%
\pgfpathlineto{\pgfqpoint{4.180901in}{1.848857in}}%
\pgfpathlineto{\pgfqpoint{4.189674in}{1.845943in}}%
\pgfpathlineto{\pgfqpoint{4.233779in}{1.845943in}}%
\pgfpathlineto{\pgfqpoint{4.242550in}{1.848857in}}%
\pgfpathlineto{\pgfqpoint{4.286414in}{1.848857in}}%
\pgfpathlineto{\pgfqpoint{4.295187in}{1.851771in}}%
\pgfpathlineto{\pgfqpoint{4.330277in}{1.851771in}}%
\pgfpathlineto{\pgfqpoint{4.339049in}{1.854686in}}%
\pgfpathlineto{\pgfqpoint{4.374139in}{1.854686in}}%
\pgfpathlineto{\pgfqpoint{4.382912in}{1.857600in}}%
\pgfpathlineto{\pgfqpoint{4.409229in}{1.857600in}}%
\pgfpathlineto{\pgfqpoint{4.418002in}{1.854686in}}%
\pgfpathlineto{\pgfqpoint{4.426775in}{1.857600in}}%
\pgfpathlineto{\pgfqpoint{4.444320in}{1.857600in}}%
\pgfpathlineto{\pgfqpoint{4.461865in}{1.863429in}}%
\pgfpathlineto{\pgfqpoint{4.496977in}{1.863429in}}%
\pgfpathlineto{\pgfqpoint{4.514522in}{1.869257in}}%
\pgfpathlineto{\pgfqpoint{4.549612in}{1.869257in}}%
\pgfpathlineto{\pgfqpoint{4.558385in}{1.872171in}}%
\pgfpathlineto{\pgfqpoint{4.611020in}{1.872171in}}%
\pgfpathlineto{\pgfqpoint{4.619792in}{1.875086in}}%
\pgfpathlineto{\pgfqpoint{4.681200in}{1.875086in}}%
\pgfpathlineto{\pgfqpoint{4.689972in}{1.878000in}}%
\pgfpathlineto{\pgfqpoint{4.795485in}{1.878000in}}%
\pgfpathlineto{\pgfqpoint{4.804258in}{1.875086in}}%
\pgfpathlineto{\pgfqpoint{4.813027in}{1.875086in}}%
\pgfpathlineto{\pgfqpoint{4.821803in}{1.878000in}}%
\pgfpathlineto{\pgfqpoint{4.830575in}{1.875086in}}%
\pgfpathlineto{\pgfqpoint{4.891983in}{1.875086in}}%
\pgfpathlineto{\pgfqpoint{4.900756in}{1.872171in}}%
\pgfpathlineto{\pgfqpoint{4.909528in}{1.875086in}}%
\pgfpathlineto{\pgfqpoint{4.918301in}{1.872171in}}%
\pgfpathlineto{\pgfqpoint{5.032586in}{1.872171in}}%
\pgfpathlineto{\pgfqpoint{5.041359in}{1.869257in}}%
\pgfpathlineto{\pgfqpoint{5.120308in}{1.869257in}}%
\pgfpathlineto{\pgfqpoint{5.129081in}{1.866343in}}%
\pgfpathlineto{\pgfqpoint{5.199507in}{1.866343in}}%
\pgfpathlineto{\pgfqpoint{5.208279in}{1.863429in}}%
\pgfpathlineto{\pgfqpoint{5.269687in}{1.863429in}}%
\pgfpathlineto{\pgfqpoint{5.278481in}{1.860514in}}%
\pgfpathlineto{\pgfqpoint{5.287254in}{1.863429in}}%
\pgfpathlineto{\pgfqpoint{5.296027in}{1.860514in}}%
\pgfpathlineto{\pgfqpoint{5.331117in}{1.860514in}}%
\pgfpathlineto{\pgfqpoint{5.339889in}{1.857600in}}%
\pgfpathlineto{\pgfqpoint{5.366207in}{1.857600in}}%
\pgfpathlineto{\pgfqpoint{5.374979in}{1.860514in}}%
\pgfpathlineto{\pgfqpoint{5.383752in}{1.857600in}}%
\pgfpathlineto{\pgfqpoint{5.427615in}{1.857600in}}%
\pgfpathlineto{\pgfqpoint{5.436387in}{1.854686in}}%
\pgfpathlineto{\pgfqpoint{5.532885in}{1.854686in}}%
\pgfpathlineto{\pgfqpoint{5.550430in}{1.848857in}}%
\pgfpathlineto{\pgfqpoint{5.559202in}{1.848857in}}%
\pgfpathlineto{\pgfqpoint{5.567969in}{1.843029in}}%
\pgfpathlineto{\pgfqpoint{5.576722in}{1.840114in}}%
\pgfpathlineto{\pgfqpoint{5.673223in}{1.776000in}}%
\pgfpathlineto{\pgfqpoint{5.681996in}{1.773086in}}%
\pgfpathlineto{\pgfqpoint{5.690768in}{1.767257in}}%
\pgfpathlineto{\pgfqpoint{5.699535in}{1.764343in}}%
\pgfpathlineto{\pgfqpoint{5.734609in}{1.741029in}}%
\pgfpathlineto{\pgfqpoint{5.752154in}{1.735200in}}%
\pgfpathlineto{\pgfqpoint{5.760927in}{1.729371in}}%
\pgfpathlineto{\pgfqpoint{5.770000in}{1.726357in}}%
\pgfpathlineto{\pgfqpoint{5.770000in}{1.726357in}}%
\pgfusepath{stroke}%
\end{pgfscope}%
\begin{pgfscope}%
\pgfpathrectangle{\pgfqpoint{0.800000in}{0.960000in}}{\pgfqpoint{4.960000in}{3.264000in}}%
\pgfusepath{clip}%
\pgfsetrectcap%
\pgfsetroundjoin%
\pgfsetlinewidth{1.505625pt}%
\definecolor{currentstroke}{rgb}{0.172549,0.627451,0.172549}%
\pgfsetstrokecolor{currentstroke}%
\pgfsetdash{}{0pt}%
\pgfpathmoveto{\pgfqpoint{0.790000in}{1.129029in}}%
\pgfpathlineto{\pgfqpoint{0.792743in}{1.129029in}}%
\pgfpathlineto{\pgfqpoint{0.810288in}{1.134857in}}%
\pgfpathlineto{\pgfqpoint{0.819060in}{1.134857in}}%
\pgfpathlineto{\pgfqpoint{0.845378in}{1.143600in}}%
\pgfpathlineto{\pgfqpoint{0.854151in}{1.143600in}}%
\pgfpathlineto{\pgfqpoint{0.862923in}{1.149429in}}%
\pgfpathlineto{\pgfqpoint{0.871696in}{1.149429in}}%
\pgfpathlineto{\pgfqpoint{0.880468in}{1.152343in}}%
\pgfpathlineto{\pgfqpoint{0.889241in}{1.158171in}}%
\pgfpathlineto{\pgfqpoint{0.898013in}{1.158171in}}%
\pgfpathlineto{\pgfqpoint{0.906786in}{1.164000in}}%
\pgfpathlineto{\pgfqpoint{0.915562in}{1.164000in}}%
\pgfpathlineto{\pgfqpoint{0.976966in}{1.184400in}}%
\pgfpathlineto{\pgfqpoint{0.985738in}{1.184400in}}%
\pgfpathlineto{\pgfqpoint{0.994511in}{1.190229in}}%
\pgfpathlineto{\pgfqpoint{1.003284in}{1.190229in}}%
\pgfpathlineto{\pgfqpoint{1.012056in}{1.193143in}}%
\pgfpathlineto{\pgfqpoint{1.020829in}{1.193143in}}%
\pgfpathlineto{\pgfqpoint{1.029601in}{1.198971in}}%
\pgfpathlineto{\pgfqpoint{1.038374in}{1.198971in}}%
\pgfpathlineto{\pgfqpoint{1.073464in}{1.210629in}}%
\pgfpathlineto{\pgfqpoint{1.082236in}{1.210629in}}%
\pgfpathlineto{\pgfqpoint{1.108576in}{1.219371in}}%
\pgfpathlineto{\pgfqpoint{1.117348in}{1.219371in}}%
\pgfpathlineto{\pgfqpoint{1.126121in}{1.225200in}}%
\pgfpathlineto{\pgfqpoint{1.134894in}{1.225200in}}%
\pgfpathlineto{\pgfqpoint{1.143666in}{1.228114in}}%
\pgfpathlineto{\pgfqpoint{1.152439in}{1.228114in}}%
\pgfpathlineto{\pgfqpoint{1.161454in}{1.233943in}}%
\pgfpathlineto{\pgfqpoint{1.170226in}{1.233943in}}%
\pgfpathlineto{\pgfqpoint{1.178999in}{1.236857in}}%
\pgfpathlineto{\pgfqpoint{1.187771in}{1.236857in}}%
\pgfpathlineto{\pgfqpoint{1.214067in}{1.245600in}}%
\pgfpathlineto{\pgfqpoint{1.240384in}{1.245600in}}%
\pgfpathlineto{\pgfqpoint{1.257929in}{1.251429in}}%
\pgfpathlineto{\pgfqpoint{1.293019in}{1.251429in}}%
\pgfpathlineto{\pgfqpoint{1.301792in}{1.254343in}}%
\pgfpathlineto{\pgfqpoint{1.336882in}{1.254343in}}%
\pgfpathlineto{\pgfqpoint{1.345655in}{1.260171in}}%
\pgfpathlineto{\pgfqpoint{1.371972in}{1.260171in}}%
\pgfpathlineto{\pgfqpoint{1.380745in}{1.263086in}}%
\pgfpathlineto{\pgfqpoint{1.407305in}{1.263086in}}%
\pgfpathlineto{\pgfqpoint{1.416093in}{1.266000in}}%
\pgfpathlineto{\pgfqpoint{1.442637in}{1.266000in}}%
\pgfpathlineto{\pgfqpoint{1.451410in}{1.268914in}}%
\pgfpathlineto{\pgfqpoint{1.477970in}{1.268914in}}%
\pgfpathlineto{\pgfqpoint{1.486743in}{1.271829in}}%
\pgfpathlineto{\pgfqpoint{1.513060in}{1.271829in}}%
\pgfpathlineto{\pgfqpoint{1.521833in}{1.274743in}}%
\pgfpathlineto{\pgfqpoint{1.548150in}{1.274743in}}%
\pgfpathlineto{\pgfqpoint{1.556923in}{1.277657in}}%
\pgfpathlineto{\pgfqpoint{1.583482in}{1.277657in}}%
\pgfpathlineto{\pgfqpoint{1.592255in}{1.280571in}}%
\pgfpathlineto{\pgfqpoint{1.636096in}{1.280571in}}%
\pgfpathlineto{\pgfqpoint{1.644868in}{1.283486in}}%
\pgfpathlineto{\pgfqpoint{1.688731in}{1.283486in}}%
\pgfpathlineto{\pgfqpoint{1.697504in}{1.286400in}}%
\pgfpathlineto{\pgfqpoint{1.750139in}{1.286400in}}%
\pgfpathlineto{\pgfqpoint{1.758911in}{1.283486in}}%
\pgfpathlineto{\pgfqpoint{1.767684in}{1.286400in}}%
\pgfpathlineto{\pgfqpoint{1.785229in}{1.286400in}}%
\pgfpathlineto{\pgfqpoint{1.794002in}{1.289314in}}%
\pgfpathlineto{\pgfqpoint{1.820319in}{1.289314in}}%
\pgfpathlineto{\pgfqpoint{1.829092in}{1.286400in}}%
\pgfpathlineto{\pgfqpoint{1.837864in}{1.289314in}}%
\pgfpathlineto{\pgfqpoint{1.846637in}{1.289314in}}%
\pgfpathlineto{\pgfqpoint{1.855652in}{1.292229in}}%
\pgfpathlineto{\pgfqpoint{1.899536in}{1.292229in}}%
\pgfpathlineto{\pgfqpoint{1.908309in}{1.295143in}}%
\pgfpathlineto{\pgfqpoint{1.917081in}{1.292229in}}%
\pgfpathlineto{\pgfqpoint{1.925854in}{1.295143in}}%
\pgfpathlineto{\pgfqpoint{1.960944in}{1.295143in}}%
\pgfpathlineto{\pgfqpoint{1.969717in}{1.298057in}}%
\pgfpathlineto{\pgfqpoint{2.004807in}{1.298057in}}%
\pgfpathlineto{\pgfqpoint{2.013579in}{1.300971in}}%
\pgfpathlineto{\pgfqpoint{2.048691in}{1.300971in}}%
\pgfpathlineto{\pgfqpoint{2.057464in}{1.303886in}}%
\pgfpathlineto{\pgfqpoint{2.075009in}{1.303886in}}%
\pgfpathlineto{\pgfqpoint{2.083782in}{1.306800in}}%
\pgfpathlineto{\pgfqpoint{2.110099in}{1.306800in}}%
\pgfpathlineto{\pgfqpoint{2.118872in}{1.309714in}}%
\pgfpathlineto{\pgfqpoint{2.145189in}{1.309714in}}%
\pgfpathlineto{\pgfqpoint{2.153965in}{1.312629in}}%
\pgfpathlineto{\pgfqpoint{2.189052in}{1.312629in}}%
\pgfpathlineto{\pgfqpoint{2.197824in}{1.315543in}}%
\pgfpathlineto{\pgfqpoint{2.206597in}{1.312629in}}%
\pgfpathlineto{\pgfqpoint{2.215369in}{1.315543in}}%
\pgfpathlineto{\pgfqpoint{2.241687in}{1.315543in}}%
\pgfpathlineto{\pgfqpoint{2.250460in}{1.318457in}}%
\pgfpathlineto{\pgfqpoint{2.267983in}{1.318457in}}%
\pgfpathlineto{\pgfqpoint{2.276755in}{1.321371in}}%
\pgfpathlineto{\pgfqpoint{2.338141in}{1.321371in}}%
\pgfpathlineto{\pgfqpoint{2.346913in}{1.324286in}}%
\pgfpathlineto{\pgfqpoint{2.417094in}{1.324286in}}%
\pgfpathlineto{\pgfqpoint{2.425863in}{1.327200in}}%
\pgfpathlineto{\pgfqpoint{2.434639in}{1.324286in}}%
\pgfpathlineto{\pgfqpoint{2.443411in}{1.327200in}}%
\pgfpathlineto{\pgfqpoint{2.601560in}{1.327200in}}%
\pgfpathlineto{\pgfqpoint{2.610332in}{1.324286in}}%
\pgfpathlineto{\pgfqpoint{2.689285in}{1.324286in}}%
\pgfpathlineto{\pgfqpoint{2.698057in}{1.321371in}}%
\pgfpathlineto{\pgfqpoint{2.777010in}{1.321371in}}%
\pgfpathlineto{\pgfqpoint{2.785782in}{1.324286in}}%
\pgfpathlineto{\pgfqpoint{2.794555in}{1.321371in}}%
\pgfpathlineto{\pgfqpoint{2.803327in}{1.321371in}}%
\pgfpathlineto{\pgfqpoint{2.812100in}{1.318457in}}%
\pgfpathlineto{\pgfqpoint{2.820872in}{1.321371in}}%
\pgfpathlineto{\pgfqpoint{2.829645in}{1.321371in}}%
\pgfpathlineto{\pgfqpoint{2.838418in}{1.318457in}}%
\pgfpathlineto{\pgfqpoint{2.961233in}{1.318457in}}%
\pgfpathlineto{\pgfqpoint{2.970005in}{1.315543in}}%
\pgfpathlineto{\pgfqpoint{2.978778in}{1.318457in}}%
\pgfpathlineto{\pgfqpoint{2.987573in}{1.315543in}}%
\pgfpathlineto{\pgfqpoint{3.066525in}{1.315543in}}%
\pgfpathlineto{\pgfqpoint{3.075298in}{1.312629in}}%
\pgfpathlineto{\pgfqpoint{3.084070in}{1.312629in}}%
\pgfpathlineto{\pgfqpoint{3.092843in}{1.309714in}}%
\pgfpathlineto{\pgfqpoint{3.127933in}{1.309714in}}%
\pgfpathlineto{\pgfqpoint{3.136706in}{1.306800in}}%
\pgfpathlineto{\pgfqpoint{3.154251in}{1.306800in}}%
\pgfpathlineto{\pgfqpoint{3.163023in}{1.309714in}}%
\pgfpathlineto{\pgfqpoint{3.180568in}{1.303886in}}%
\pgfpathlineto{\pgfqpoint{3.189341in}{1.306800in}}%
\pgfpathlineto{\pgfqpoint{3.198113in}{1.303886in}}%
\pgfpathlineto{\pgfqpoint{3.206886in}{1.306800in}}%
\pgfpathlineto{\pgfqpoint{3.215658in}{1.306800in}}%
\pgfpathlineto{\pgfqpoint{3.224431in}{1.303886in}}%
\pgfpathlineto{\pgfqpoint{3.303384in}{1.303886in}}%
\pgfpathlineto{\pgfqpoint{3.312156in}{1.300971in}}%
\pgfpathlineto{\pgfqpoint{3.320929in}{1.303886in}}%
\pgfpathlineto{\pgfqpoint{3.329701in}{1.300971in}}%
\pgfpathlineto{\pgfqpoint{3.443744in}{1.300971in}}%
\pgfpathlineto{\pgfqpoint{3.461554in}{1.295143in}}%
\pgfpathlineto{\pgfqpoint{3.470326in}{1.295143in}}%
\pgfpathlineto{\pgfqpoint{3.479099in}{1.292229in}}%
\pgfpathlineto{\pgfqpoint{3.496644in}{1.292229in}}%
\pgfpathlineto{\pgfqpoint{3.505416in}{1.289314in}}%
\pgfpathlineto{\pgfqpoint{3.531734in}{1.289314in}}%
\pgfpathlineto{\pgfqpoint{3.540506in}{1.286400in}}%
\pgfpathlineto{\pgfqpoint{3.584369in}{1.286400in}}%
\pgfpathlineto{\pgfqpoint{3.593142in}{1.283486in}}%
\pgfpathlineto{\pgfqpoint{3.663322in}{1.283486in}}%
\pgfpathlineto{\pgfqpoint{3.672094in}{1.280571in}}%
\pgfpathlineto{\pgfqpoint{3.689640in}{1.280571in}}%
\pgfpathlineto{\pgfqpoint{3.698412in}{1.283486in}}%
\pgfpathlineto{\pgfqpoint{3.786137in}{1.283486in}}%
\pgfpathlineto{\pgfqpoint{3.794910in}{1.286400in}}%
\pgfpathlineto{\pgfqpoint{3.803682in}{1.283486in}}%
\pgfpathlineto{\pgfqpoint{3.812455in}{1.286400in}}%
\pgfpathlineto{\pgfqpoint{3.821227in}{1.283486in}}%
\pgfpathlineto{\pgfqpoint{3.847545in}{1.283486in}}%
\pgfpathlineto{\pgfqpoint{3.856318in}{1.280571in}}%
\pgfpathlineto{\pgfqpoint{3.865090in}{1.283486in}}%
\pgfpathlineto{\pgfqpoint{3.873863in}{1.283486in}}%
\pgfpathlineto{\pgfqpoint{3.882635in}{1.280571in}}%
\pgfpathlineto{\pgfqpoint{3.891408in}{1.283486in}}%
\pgfpathlineto{\pgfqpoint{3.900180in}{1.283486in}}%
\pgfpathlineto{\pgfqpoint{3.908953in}{1.280571in}}%
\pgfpathlineto{\pgfqpoint{4.022996in}{1.280571in}}%
\pgfpathlineto{\pgfqpoint{4.031768in}{1.277657in}}%
\pgfpathlineto{\pgfqpoint{4.154584in}{1.277657in}}%
\pgfpathlineto{\pgfqpoint{4.163356in}{1.280571in}}%
\pgfpathlineto{\pgfqpoint{4.189674in}{1.280571in}}%
\pgfpathlineto{\pgfqpoint{4.198446in}{1.283486in}}%
\pgfpathlineto{\pgfqpoint{4.207219in}{1.280571in}}%
\pgfpathlineto{\pgfqpoint{4.215991in}{1.283486in}}%
\pgfpathlineto{\pgfqpoint{4.225006in}{1.283486in}}%
\pgfpathlineto{\pgfqpoint{4.233779in}{1.286400in}}%
\pgfpathlineto{\pgfqpoint{4.242550in}{1.286400in}}%
\pgfpathlineto{\pgfqpoint{4.251324in}{1.289314in}}%
\pgfpathlineto{\pgfqpoint{4.277642in}{1.289314in}}%
\pgfpathlineto{\pgfqpoint{4.286414in}{1.292229in}}%
\pgfpathlineto{\pgfqpoint{4.303959in}{1.292229in}}%
\pgfpathlineto{\pgfqpoint{4.312732in}{1.295143in}}%
\pgfpathlineto{\pgfqpoint{4.339049in}{1.295143in}}%
\pgfpathlineto{\pgfqpoint{4.347822in}{1.298057in}}%
\pgfpathlineto{\pgfqpoint{4.365367in}{1.298057in}}%
\pgfpathlineto{\pgfqpoint{4.374139in}{1.300971in}}%
\pgfpathlineto{\pgfqpoint{4.382912in}{1.300971in}}%
\pgfpathlineto{\pgfqpoint{4.391684in}{1.298057in}}%
\pgfpathlineto{\pgfqpoint{4.400457in}{1.300971in}}%
\pgfpathlineto{\pgfqpoint{4.418002in}{1.300971in}}%
\pgfpathlineto{\pgfqpoint{4.426775in}{1.303886in}}%
\pgfpathlineto{\pgfqpoint{4.435547in}{1.300971in}}%
\pgfpathlineto{\pgfqpoint{4.461865in}{1.300971in}}%
\pgfpathlineto{\pgfqpoint{4.470637in}{1.303886in}}%
\pgfpathlineto{\pgfqpoint{4.514522in}{1.303886in}}%
\pgfpathlineto{\pgfqpoint{4.523294in}{1.306800in}}%
\pgfpathlineto{\pgfqpoint{4.575930in}{1.306800in}}%
\pgfpathlineto{\pgfqpoint{4.584702in}{1.309714in}}%
\pgfpathlineto{\pgfqpoint{4.689972in}{1.309714in}}%
\pgfpathlineto{\pgfqpoint{4.698987in}{1.312629in}}%
\pgfpathlineto{\pgfqpoint{4.707760in}{1.309714in}}%
\pgfpathlineto{\pgfqpoint{4.742850in}{1.309714in}}%
\pgfpathlineto{\pgfqpoint{4.751623in}{1.312629in}}%
\pgfpathlineto{\pgfqpoint{4.760395in}{1.312629in}}%
\pgfpathlineto{\pgfqpoint{4.769168in}{1.309714in}}%
\pgfpathlineto{\pgfqpoint{4.786713in}{1.309714in}}%
\pgfpathlineto{\pgfqpoint{4.795485in}{1.306800in}}%
\pgfpathlineto{\pgfqpoint{4.830575in}{1.306800in}}%
\pgfpathlineto{\pgfqpoint{4.839348in}{1.303886in}}%
\pgfpathlineto{\pgfqpoint{4.900756in}{1.303886in}}%
\pgfpathlineto{\pgfqpoint{4.909528in}{1.300971in}}%
\pgfpathlineto{\pgfqpoint{4.988496in}{1.300971in}}%
\pgfpathlineto{\pgfqpoint{4.997496in}{1.298057in}}%
\pgfpathlineto{\pgfqpoint{5.006268in}{1.300971in}}%
\pgfpathlineto{\pgfqpoint{5.015041in}{1.298057in}}%
\pgfpathlineto{\pgfqpoint{5.050131in}{1.298057in}}%
\pgfpathlineto{\pgfqpoint{5.058904in}{1.295143in}}%
\pgfpathlineto{\pgfqpoint{5.111539in}{1.295143in}}%
\pgfpathlineto{\pgfqpoint{5.120308in}{1.292229in}}%
\pgfpathlineto{\pgfqpoint{5.155402in}{1.292229in}}%
\pgfpathlineto{\pgfqpoint{5.164174in}{1.289314in}}%
\pgfpathlineto{\pgfqpoint{5.173189in}{1.292229in}}%
\pgfpathlineto{\pgfqpoint{5.181962in}{1.286400in}}%
\pgfpathlineto{\pgfqpoint{5.217052in}{1.286400in}}%
\pgfpathlineto{\pgfqpoint{5.225824in}{1.289314in}}%
\pgfpathlineto{\pgfqpoint{5.234593in}{1.286400in}}%
\pgfpathlineto{\pgfqpoint{5.331117in}{1.286400in}}%
\pgfpathlineto{\pgfqpoint{5.339889in}{1.283486in}}%
\pgfpathlineto{\pgfqpoint{5.348661in}{1.283486in}}%
\pgfpathlineto{\pgfqpoint{5.357434in}{1.286400in}}%
\pgfpathlineto{\pgfqpoint{5.366207in}{1.283486in}}%
\pgfpathlineto{\pgfqpoint{5.462705in}{1.283486in}}%
\pgfpathlineto{\pgfqpoint{5.471474in}{1.280571in}}%
\pgfpathlineto{\pgfqpoint{5.480250in}{1.283486in}}%
\pgfpathlineto{\pgfqpoint{5.524112in}{1.283486in}}%
\pgfpathlineto{\pgfqpoint{5.532885in}{1.280571in}}%
\pgfpathlineto{\pgfqpoint{5.541657in}{1.280571in}}%
\pgfpathlineto{\pgfqpoint{5.550430in}{1.277657in}}%
\pgfpathlineto{\pgfqpoint{5.559202in}{1.277657in}}%
\pgfpathlineto{\pgfqpoint{5.603043in}{1.263086in}}%
\pgfpathlineto{\pgfqpoint{5.611816in}{1.263086in}}%
\pgfpathlineto{\pgfqpoint{5.620588in}{1.260171in}}%
\pgfpathlineto{\pgfqpoint{5.629361in}{1.254343in}}%
\pgfpathlineto{\pgfqpoint{5.638133in}{1.254343in}}%
\pgfpathlineto{\pgfqpoint{5.690768in}{1.236857in}}%
\pgfpathlineto{\pgfqpoint{5.699535in}{1.236857in}}%
\pgfpathlineto{\pgfqpoint{5.708291in}{1.231029in}}%
\pgfpathlineto{\pgfqpoint{5.717064in}{1.231029in}}%
\pgfpathlineto{\pgfqpoint{5.734609in}{1.225200in}}%
\pgfpathlineto{\pgfqpoint{5.743382in}{1.225200in}}%
\pgfpathlineto{\pgfqpoint{5.752154in}{1.219371in}}%
\pgfpathlineto{\pgfqpoint{5.760927in}{1.219371in}}%
\pgfpathlineto{\pgfqpoint{5.770000in}{1.216357in}}%
\pgfpathlineto{\pgfqpoint{5.770000in}{1.216357in}}%
\pgfusepath{stroke}%
\end{pgfscope}%
\begin{pgfscope}%
\pgfsetrectcap%
\pgfsetmiterjoin%
\pgfsetlinewidth{0.803000pt}%
\definecolor{currentstroke}{rgb}{0.000000,0.000000,0.000000}%
\pgfsetstrokecolor{currentstroke}%
\pgfsetdash{}{0pt}%
\pgfpathmoveto{\pgfqpoint{0.800000in}{0.960000in}}%
\pgfpathlineto{\pgfqpoint{0.800000in}{4.224000in}}%
\pgfusepath{stroke}%
\end{pgfscope}%
\begin{pgfscope}%
\pgfsetrectcap%
\pgfsetmiterjoin%
\pgfsetlinewidth{0.803000pt}%
\definecolor{currentstroke}{rgb}{0.000000,0.000000,0.000000}%
\pgfsetstrokecolor{currentstroke}%
\pgfsetdash{}{0pt}%
\pgfpathmoveto{\pgfqpoint{5.760000in}{0.960000in}}%
\pgfpathlineto{\pgfqpoint{5.760000in}{4.224000in}}%
\pgfusepath{stroke}%
\end{pgfscope}%
\begin{pgfscope}%
\pgfsetrectcap%
\pgfsetmiterjoin%
\pgfsetlinewidth{0.803000pt}%
\definecolor{currentstroke}{rgb}{0.000000,0.000000,0.000000}%
\pgfsetstrokecolor{currentstroke}%
\pgfsetdash{}{0pt}%
\pgfpathmoveto{\pgfqpoint{0.800000in}{0.960000in}}%
\pgfpathlineto{\pgfqpoint{5.760000in}{0.960000in}}%
\pgfusepath{stroke}%
\end{pgfscope}%
\begin{pgfscope}%
\pgfsetrectcap%
\pgfsetmiterjoin%
\pgfsetlinewidth{0.803000pt}%
\definecolor{currentstroke}{rgb}{0.000000,0.000000,0.000000}%
\pgfsetstrokecolor{currentstroke}%
\pgfsetdash{}{0pt}%
\pgfpathmoveto{\pgfqpoint{0.800000in}{4.224000in}}%
\pgfpathlineto{\pgfqpoint{5.760000in}{4.224000in}}%
\pgfusepath{stroke}%
\end{pgfscope}%
\begin{pgfscope}%
\definecolor{textcolor}{rgb}{0.000000,0.000000,0.000000}%
\pgfsetstrokecolor{textcolor}%
\pgfsetfillcolor{textcolor}%
%\pgftext[x=3.280000in,y=4.307333in,,base]{\color{textcolor}\rmfamily\fontsize{12.000000}{14.400000}\selectfont Évolution de la température dans les tests sans step-down et sans ventilateur}%
\end{pgfscope}%
\begin{pgfscope}%
\pgfsetbuttcap%
\pgfsetmiterjoin%
\definecolor{currentfill}{rgb}{1.000000,1.000000,1.000000}%
\pgfsetfillcolor{currentfill}%
\pgfsetfillopacity{0.800000}%
\pgfsetlinewidth{1.003750pt}%
\definecolor{currentstroke}{rgb}{0.800000,0.800000,0.800000}%
\pgfsetstrokecolor{currentstroke}%
\pgfsetstrokeopacity{0.800000}%
\pgfsetdash{}{0pt}%
\pgfpathmoveto{\pgfqpoint{0.897222in}{2.280657in}}%
\pgfpathlineto{\pgfqpoint{2.489045in}{2.280657in}}%
\pgfpathquadraticcurveto{\pgfqpoint{2.516823in}{2.280657in}}{\pgfqpoint{2.516823in}{2.308435in}}%
\pgfpathlineto{\pgfqpoint{2.516823in}{2.875565in}}%
\pgfpathquadraticcurveto{\pgfqpoint{2.516823in}{2.903342in}}{\pgfqpoint{2.489045in}{2.903342in}}%
\pgfpathlineto{\pgfqpoint{0.897222in}{2.903342in}}%
\pgfpathquadraticcurveto{\pgfqpoint{0.869444in}{2.903342in}}{\pgfqpoint{0.869444in}{2.875565in}}%
\pgfpathlineto{\pgfqpoint{0.869444in}{2.308435in}}%
\pgfpathquadraticcurveto{\pgfqpoint{0.869444in}{2.280657in}}{\pgfqpoint{0.897222in}{2.280657in}}%
\pgfpathclose%
\pgfusepath{stroke,fill}%
\end{pgfscope}%
\begin{pgfscope}%
\pgfsetrectcap%
\pgfsetroundjoin%
\pgfsetlinewidth{1.505625pt}%
\definecolor{currentstroke}{rgb}{0.121569,0.466667,0.705882}%
\pgfsetstrokecolor{currentstroke}%
\pgfsetdash{}{0pt}%
\pgfpathmoveto{\pgfqpoint{0.925000in}{2.799176in}}%
\pgfpathlineto{\pgfqpoint{1.202778in}{2.799176in}}%
\pgfusepath{stroke}%
\end{pgfscope}%
\begin{pgfscope}%
\definecolor{textcolor}{rgb}{0.000000,0.000000,0.000000}%
\pgfsetstrokecolor{textcolor}%
\pgfsetfillcolor{textcolor}%
\pgftext[x=1.313889in,y=2.750565in,left,base]{\color{textcolor}\rmfamily\fontsize{10.000000}{12.000000}\selectfont CPU}%
\end{pgfscope}%
\begin{pgfscope}%
\pgfsetrectcap%
\pgfsetroundjoin%
\pgfsetlinewidth{1.505625pt}%
\definecolor{currentstroke}{rgb}{1.000000,0.498039,0.054902}%
\pgfsetstrokecolor{currentstroke}%
\pgfsetdash{}{0pt}%
\pgfpathmoveto{\pgfqpoint{0.925000in}{2.605503in}}%
\pgfpathlineto{\pgfqpoint{1.202778in}{2.605503in}}%
\pgfusepath{stroke}%
\end{pgfscope}%
\begin{pgfscope}%
\definecolor{textcolor}{rgb}{0.000000,0.000000,0.000000}%
\pgfsetstrokecolor{textcolor}%
\pgfsetfillcolor{textcolor}%
\pgftext[x=1.313889in,y=2.556892in,left,base]{\color{textcolor}\rmfamily\fontsize{10.000000}{12.000000}\selectfont Zone Raspberry Pi}%
\end{pgfscope}%
\begin{pgfscope}%
\pgfsetrectcap%
\pgfsetroundjoin%
\pgfsetlinewidth{1.505625pt}%
\definecolor{currentstroke}{rgb}{0.172549,0.627451,0.172549}%
\pgfsetstrokecolor{currentstroke}%
\pgfsetdash{}{0pt}%
\pgfpathmoveto{\pgfqpoint{0.925000in}{2.411830in}}%
\pgfpathlineto{\pgfqpoint{1.202778in}{2.411830in}}%
\pgfusepath{stroke}%
\end{pgfscope}%
\begin{pgfscope}%
\definecolor{textcolor}{rgb}{0.000000,0.000000,0.000000}%
\pgfsetstrokecolor{textcolor}%
\pgfsetfillcolor{textcolor}%
\pgftext[x=1.313889in,y=2.363219in,left,base]{\color{textcolor}\rmfamily\fontsize{10.000000}{12.000000}\selectfont Zone step-down}%
\end{pgfscope}%
\end{pgfpicture}%
\makeatother%
\endgroup%

  \label{fig:test_2}
  \vspace{-1cm}
  \caption{\textbf{Test 2 :} Évolution de la température dans les tests sans step-down et sans ventilateur}
\end{figure}

~

\noindent
Dans le troisième test, le step-down se trouvait à nouveau à l'intérieur du boîtier, mais un ventilateur était utilisé pour souffler de l'air vers l'intérieur de la boîte. Dans ce test, l'objectif était de comprendre à quel point un ventilateur améliore la solution de refroidissement du boîtier. De plus, l'idée était aussi de vérifier si la température a un impact sur les performances du modem.

\begin{figure}[ht!]
  \centering
  %% Creator: Matplotlib, PGF backend
%%
%% To include the figure in your LaTeX document, write
%%   \input{<filename>.pgf}
%%
%% Make sure the required packages are loaded in your preamble
%%   \usepackage{pgf}
%%
%% Figures using additional raster images can only be included by \input if
%% they are in the same directory as the main LaTeX file. For loading figures
%% from other directories you can use the `import` package
%%   \usepackage{import}
%% and then include the figures with
%%   \import{<path to file>}{<filename>.pgf}
%%
%% Matplotlib used the following preamble
%%
\begingroup%
\makeatletter%
\begin{pgfpicture}%
\pgfpathrectangle{\pgfpointorigin}{\pgfqpoint{6.400000in}{4.800000in}}%
\pgfusepath{use as bounding box, clip}%
\begin{pgfscope}%
\pgfsetbuttcap%
\pgfsetmiterjoin%
\definecolor{currentfill}{rgb}{1.000000,1.000000,1.000000}%
\pgfsetfillcolor{currentfill}%
\pgfsetlinewidth{0.000000pt}%
\definecolor{currentstroke}{rgb}{1.000000,1.000000,1.000000}%
\pgfsetstrokecolor{currentstroke}%
\pgfsetdash{}{0pt}%
\pgfpathmoveto{\pgfqpoint{0.000000in}{0.000000in}}%
\pgfpathlineto{\pgfqpoint{6.400000in}{0.000000in}}%
\pgfpathlineto{\pgfqpoint{6.400000in}{4.800000in}}%
\pgfpathlineto{\pgfqpoint{0.000000in}{4.800000in}}%
\pgfpathclose%
\pgfusepath{fill}%
\end{pgfscope}%
\begin{pgfscope}%
\pgfsetbuttcap%
\pgfsetmiterjoin%
\definecolor{currentfill}{rgb}{1.000000,1.000000,1.000000}%
\pgfsetfillcolor{currentfill}%
\pgfsetlinewidth{0.000000pt}%
\definecolor{currentstroke}{rgb}{0.000000,0.000000,0.000000}%
\pgfsetstrokecolor{currentstroke}%
\pgfsetstrokeopacity{0.000000}%
\pgfsetdash{}{0pt}%
\pgfpathmoveto{\pgfqpoint{0.800000in}{0.960000in}}%
\pgfpathlineto{\pgfqpoint{5.760000in}{0.960000in}}%
\pgfpathlineto{\pgfqpoint{5.760000in}{4.224000in}}%
\pgfpathlineto{\pgfqpoint{0.800000in}{4.224000in}}%
\pgfpathclose%
\pgfusepath{fill}%
\end{pgfscope}%
\begin{pgfscope}%
\pgfsetbuttcap%
\pgfsetroundjoin%
\definecolor{currentfill}{rgb}{0.000000,0.000000,0.000000}%
\pgfsetfillcolor{currentfill}%
\pgfsetlinewidth{0.803000pt}%
\definecolor{currentstroke}{rgb}{0.000000,0.000000,0.000000}%
\pgfsetstrokecolor{currentstroke}%
\pgfsetdash{}{0pt}%
\pgfsys@defobject{currentmarker}{\pgfqpoint{0.000000in}{-0.048611in}}{\pgfqpoint{0.000000in}{0.000000in}}{%
\pgfpathmoveto{\pgfqpoint{0.000000in}{0.000000in}}%
\pgfpathlineto{\pgfqpoint{0.000000in}{-0.048611in}}%
\pgfusepath{stroke,fill}%
}%
\begin{pgfscope}%
\pgfsys@transformshift{0.800000in}{0.960000in}%
\pgfsys@useobject{currentmarker}{}%
\end{pgfscope}%
\end{pgfscope}%
\begin{pgfscope}%
\definecolor{textcolor}{rgb}{0.000000,0.000000,0.000000}%
\pgfsetstrokecolor{textcolor}%
\pgfsetfillcolor{textcolor}%
\pgftext[x=0.512522in,y=0.621070in,left,base,rotate=30.000000]{\color{textcolor}\rmfamily\fontsize{10.000000}{12.000000}\selectfont 00:00}%
\end{pgfscope}%
\begin{pgfscope}%
\pgfsetbuttcap%
\pgfsetroundjoin%
\definecolor{currentfill}{rgb}{0.000000,0.000000,0.000000}%
\pgfsetfillcolor{currentfill}%
\pgfsetlinewidth{0.803000pt}%
\definecolor{currentstroke}{rgb}{0.000000,0.000000,0.000000}%
\pgfsetstrokecolor{currentstroke}%
\pgfsetdash{}{0pt}%
\pgfsys@defobject{currentmarker}{\pgfqpoint{0.000000in}{-0.048611in}}{\pgfqpoint{0.000000in}{0.000000in}}{%
\pgfpathmoveto{\pgfqpoint{0.000000in}{0.000000in}}%
\pgfpathlineto{\pgfqpoint{0.000000in}{-0.048611in}}%
\pgfusepath{stroke,fill}%
}%
\begin{pgfscope}%
\pgfsys@transformshift{1.791914in}{0.960000in}%
\pgfsys@useobject{currentmarker}{}%
\end{pgfscope}%
\end{pgfscope}%
\begin{pgfscope}%
\definecolor{textcolor}{rgb}{0.000000,0.000000,0.000000}%
\pgfsetstrokecolor{textcolor}%
\pgfsetfillcolor{textcolor}%
\pgftext[x=1.504436in,y=0.621070in,left,base,rotate=30.000000]{\color{textcolor}\rmfamily\fontsize{10.000000}{12.000000}\selectfont 01:00}%
\end{pgfscope}%
\begin{pgfscope}%
\pgfsetbuttcap%
\pgfsetroundjoin%
\definecolor{currentfill}{rgb}{0.000000,0.000000,0.000000}%
\pgfsetfillcolor{currentfill}%
\pgfsetlinewidth{0.803000pt}%
\definecolor{currentstroke}{rgb}{0.000000,0.000000,0.000000}%
\pgfsetstrokecolor{currentstroke}%
\pgfsetdash{}{0pt}%
\pgfsys@defobject{currentmarker}{\pgfqpoint{0.000000in}{-0.048611in}}{\pgfqpoint{0.000000in}{0.000000in}}{%
\pgfpathmoveto{\pgfqpoint{0.000000in}{0.000000in}}%
\pgfpathlineto{\pgfqpoint{0.000000in}{-0.048611in}}%
\pgfusepath{stroke,fill}%
}%
\begin{pgfscope}%
\pgfsys@transformshift{2.783828in}{0.960000in}%
\pgfsys@useobject{currentmarker}{}%
\end{pgfscope}%
\end{pgfscope}%
\begin{pgfscope}%
\definecolor{textcolor}{rgb}{0.000000,0.000000,0.000000}%
\pgfsetstrokecolor{textcolor}%
\pgfsetfillcolor{textcolor}%
\pgftext[x=2.496350in,y=0.621070in,left,base,rotate=30.000000]{\color{textcolor}\rmfamily\fontsize{10.000000}{12.000000}\selectfont 02:00}%
\end{pgfscope}%
\begin{pgfscope}%
\pgfsetbuttcap%
\pgfsetroundjoin%
\definecolor{currentfill}{rgb}{0.000000,0.000000,0.000000}%
\pgfsetfillcolor{currentfill}%
\pgfsetlinewidth{0.803000pt}%
\definecolor{currentstroke}{rgb}{0.000000,0.000000,0.000000}%
\pgfsetstrokecolor{currentstroke}%
\pgfsetdash{}{0pt}%
\pgfsys@defobject{currentmarker}{\pgfqpoint{0.000000in}{-0.048611in}}{\pgfqpoint{0.000000in}{0.000000in}}{%
\pgfpathmoveto{\pgfqpoint{0.000000in}{0.000000in}}%
\pgfpathlineto{\pgfqpoint{0.000000in}{-0.048611in}}%
\pgfusepath{stroke,fill}%
}%
\begin{pgfscope}%
\pgfsys@transformshift{3.775742in}{0.960000in}%
\pgfsys@useobject{currentmarker}{}%
\end{pgfscope}%
\end{pgfscope}%
\begin{pgfscope}%
\definecolor{textcolor}{rgb}{0.000000,0.000000,0.000000}%
\pgfsetstrokecolor{textcolor}%
\pgfsetfillcolor{textcolor}%
\pgftext[x=3.488264in,y=0.621070in,left,base,rotate=30.000000]{\color{textcolor}\rmfamily\fontsize{10.000000}{12.000000}\selectfont 03:00}%
\end{pgfscope}%
\begin{pgfscope}%
\pgfsetbuttcap%
\pgfsetroundjoin%
\definecolor{currentfill}{rgb}{0.000000,0.000000,0.000000}%
\pgfsetfillcolor{currentfill}%
\pgfsetlinewidth{0.803000pt}%
\definecolor{currentstroke}{rgb}{0.000000,0.000000,0.000000}%
\pgfsetstrokecolor{currentstroke}%
\pgfsetdash{}{0pt}%
\pgfsys@defobject{currentmarker}{\pgfqpoint{0.000000in}{-0.048611in}}{\pgfqpoint{0.000000in}{0.000000in}}{%
\pgfpathmoveto{\pgfqpoint{0.000000in}{0.000000in}}%
\pgfpathlineto{\pgfqpoint{0.000000in}{-0.048611in}}%
\pgfusepath{stroke,fill}%
}%
\begin{pgfscope}%
\pgfsys@transformshift{4.767657in}{0.960000in}%
\pgfsys@useobject{currentmarker}{}%
\end{pgfscope}%
\end{pgfscope}%
\begin{pgfscope}%
\definecolor{textcolor}{rgb}{0.000000,0.000000,0.000000}%
\pgfsetstrokecolor{textcolor}%
\pgfsetfillcolor{textcolor}%
\pgftext[x=4.480178in,y=0.621070in,left,base,rotate=30.000000]{\color{textcolor}\rmfamily\fontsize{10.000000}{12.000000}\selectfont 04:00}%
\end{pgfscope}%
\begin{pgfscope}%
\pgfsetbuttcap%
\pgfsetroundjoin%
\definecolor{currentfill}{rgb}{0.000000,0.000000,0.000000}%
\pgfsetfillcolor{currentfill}%
\pgfsetlinewidth{0.803000pt}%
\definecolor{currentstroke}{rgb}{0.000000,0.000000,0.000000}%
\pgfsetstrokecolor{currentstroke}%
\pgfsetdash{}{0pt}%
\pgfsys@defobject{currentmarker}{\pgfqpoint{0.000000in}{-0.048611in}}{\pgfqpoint{0.000000in}{0.000000in}}{%
\pgfpathmoveto{\pgfqpoint{0.000000in}{0.000000in}}%
\pgfpathlineto{\pgfqpoint{0.000000in}{-0.048611in}}%
\pgfusepath{stroke,fill}%
}%
\begin{pgfscope}%
\pgfsys@transformshift{5.759571in}{0.960000in}%
\pgfsys@useobject{currentmarker}{}%
\end{pgfscope}%
\end{pgfscope}%
\begin{pgfscope}%
\definecolor{textcolor}{rgb}{0.000000,0.000000,0.000000}%
\pgfsetstrokecolor{textcolor}%
\pgfsetfillcolor{textcolor}%
\pgftext[x=5.472093in,y=0.621070in,left,base,rotate=30.000000]{\color{textcolor}\rmfamily\fontsize{10.000000}{12.000000}\selectfont 05:00}%
\end{pgfscope}%
\begin{pgfscope}%
\definecolor{textcolor}{rgb}{0.000000,0.000000,0.000000}%
\pgfsetstrokecolor{textcolor}%
\pgfsetfillcolor{textcolor}%
\pgftext[x=3.280000in,y=0.542126in,,top]{\color{textcolor}\rmfamily\fontsize{10.000000}{12.000000}\selectfont Temps (hh:mm)}%
\end{pgfscope}%
\begin{pgfscope}%
\pgfsetbuttcap%
\pgfsetroundjoin%
\definecolor{currentfill}{rgb}{0.000000,0.000000,0.000000}%
\pgfsetfillcolor{currentfill}%
\pgfsetlinewidth{0.803000pt}%
\definecolor{currentstroke}{rgb}{0.000000,0.000000,0.000000}%
\pgfsetstrokecolor{currentstroke}%
\pgfsetdash{}{0pt}%
\pgfsys@defobject{currentmarker}{\pgfqpoint{-0.048611in}{0.000000in}}{\pgfqpoint{0.000000in}{0.000000in}}{%
\pgfpathmoveto{\pgfqpoint{0.000000in}{0.000000in}}%
\pgfpathlineto{\pgfqpoint{-0.048611in}{0.000000in}}%
\pgfusepath{stroke,fill}%
}%
\begin{pgfscope}%
\pgfsys@transformshift{0.800000in}{1.015011in}%
\pgfsys@useobject{currentmarker}{}%
\end{pgfscope}%
\end{pgfscope}%
\begin{pgfscope}%
\definecolor{textcolor}{rgb}{0.000000,0.000000,0.000000}%
\pgfsetstrokecolor{textcolor}%
\pgfsetfillcolor{textcolor}%
\pgftext[x=0.563888in,y=0.966786in,left,base]{\color{textcolor}\rmfamily\fontsize{10.000000}{12.000000}\selectfont \(\displaystyle 20\)}%
\end{pgfscope}%
\begin{pgfscope}%
\pgfsetbuttcap%
\pgfsetroundjoin%
\definecolor{currentfill}{rgb}{0.000000,0.000000,0.000000}%
\pgfsetfillcolor{currentfill}%
\pgfsetlinewidth{0.803000pt}%
\definecolor{currentstroke}{rgb}{0.000000,0.000000,0.000000}%
\pgfsetstrokecolor{currentstroke}%
\pgfsetdash{}{0pt}%
\pgfsys@defobject{currentmarker}{\pgfqpoint{-0.048611in}{0.000000in}}{\pgfqpoint{0.000000in}{0.000000in}}{%
\pgfpathmoveto{\pgfqpoint{0.000000in}{0.000000in}}%
\pgfpathlineto{\pgfqpoint{-0.048611in}{0.000000in}}%
\pgfusepath{stroke,fill}%
}%
\begin{pgfscope}%
\pgfsys@transformshift{0.800000in}{1.473438in}%
\pgfsys@useobject{currentmarker}{}%
\end{pgfscope}%
\end{pgfscope}%
\begin{pgfscope}%
\definecolor{textcolor}{rgb}{0.000000,0.000000,0.000000}%
\pgfsetstrokecolor{textcolor}%
\pgfsetfillcolor{textcolor}%
\pgftext[x=0.563888in,y=1.425213in,left,base]{\color{textcolor}\rmfamily\fontsize{10.000000}{12.000000}\selectfont \(\displaystyle 25\)}%
\end{pgfscope}%
\begin{pgfscope}%
\pgfsetbuttcap%
\pgfsetroundjoin%
\definecolor{currentfill}{rgb}{0.000000,0.000000,0.000000}%
\pgfsetfillcolor{currentfill}%
\pgfsetlinewidth{0.803000pt}%
\definecolor{currentstroke}{rgb}{0.000000,0.000000,0.000000}%
\pgfsetstrokecolor{currentstroke}%
\pgfsetdash{}{0pt}%
\pgfsys@defobject{currentmarker}{\pgfqpoint{-0.048611in}{0.000000in}}{\pgfqpoint{0.000000in}{0.000000in}}{%
\pgfpathmoveto{\pgfqpoint{0.000000in}{0.000000in}}%
\pgfpathlineto{\pgfqpoint{-0.048611in}{0.000000in}}%
\pgfusepath{stroke,fill}%
}%
\begin{pgfscope}%
\pgfsys@transformshift{0.800000in}{1.931865in}%
\pgfsys@useobject{currentmarker}{}%
\end{pgfscope}%
\end{pgfscope}%
\begin{pgfscope}%
\definecolor{textcolor}{rgb}{0.000000,0.000000,0.000000}%
\pgfsetstrokecolor{textcolor}%
\pgfsetfillcolor{textcolor}%
\pgftext[x=0.563888in,y=1.883640in,left,base]{\color{textcolor}\rmfamily\fontsize{10.000000}{12.000000}\selectfont \(\displaystyle 30\)}%
\end{pgfscope}%
\begin{pgfscope}%
\pgfsetbuttcap%
\pgfsetroundjoin%
\definecolor{currentfill}{rgb}{0.000000,0.000000,0.000000}%
\pgfsetfillcolor{currentfill}%
\pgfsetlinewidth{0.803000pt}%
\definecolor{currentstroke}{rgb}{0.000000,0.000000,0.000000}%
\pgfsetstrokecolor{currentstroke}%
\pgfsetdash{}{0pt}%
\pgfsys@defobject{currentmarker}{\pgfqpoint{-0.048611in}{0.000000in}}{\pgfqpoint{0.000000in}{0.000000in}}{%
\pgfpathmoveto{\pgfqpoint{0.000000in}{0.000000in}}%
\pgfpathlineto{\pgfqpoint{-0.048611in}{0.000000in}}%
\pgfusepath{stroke,fill}%
}%
\begin{pgfscope}%
\pgfsys@transformshift{0.800000in}{2.390292in}%
\pgfsys@useobject{currentmarker}{}%
\end{pgfscope}%
\end{pgfscope}%
\begin{pgfscope}%
\definecolor{textcolor}{rgb}{0.000000,0.000000,0.000000}%
\pgfsetstrokecolor{textcolor}%
\pgfsetfillcolor{textcolor}%
\pgftext[x=0.563888in,y=2.342067in,left,base]{\color{textcolor}\rmfamily\fontsize{10.000000}{12.000000}\selectfont \(\displaystyle 35\)}%
\end{pgfscope}%
\begin{pgfscope}%
\pgfsetbuttcap%
\pgfsetroundjoin%
\definecolor{currentfill}{rgb}{0.000000,0.000000,0.000000}%
\pgfsetfillcolor{currentfill}%
\pgfsetlinewidth{0.803000pt}%
\definecolor{currentstroke}{rgb}{0.000000,0.000000,0.000000}%
\pgfsetstrokecolor{currentstroke}%
\pgfsetdash{}{0pt}%
\pgfsys@defobject{currentmarker}{\pgfqpoint{-0.048611in}{0.000000in}}{\pgfqpoint{0.000000in}{0.000000in}}{%
\pgfpathmoveto{\pgfqpoint{0.000000in}{0.000000in}}%
\pgfpathlineto{\pgfqpoint{-0.048611in}{0.000000in}}%
\pgfusepath{stroke,fill}%
}%
\begin{pgfscope}%
\pgfsys@transformshift{0.800000in}{2.848719in}%
\pgfsys@useobject{currentmarker}{}%
\end{pgfscope}%
\end{pgfscope}%
\begin{pgfscope}%
\definecolor{textcolor}{rgb}{0.000000,0.000000,0.000000}%
\pgfsetstrokecolor{textcolor}%
\pgfsetfillcolor{textcolor}%
\pgftext[x=0.563888in,y=2.800494in,left,base]{\color{textcolor}\rmfamily\fontsize{10.000000}{12.000000}\selectfont \(\displaystyle 40\)}%
\end{pgfscope}%
\begin{pgfscope}%
\pgfsetbuttcap%
\pgfsetroundjoin%
\definecolor{currentfill}{rgb}{0.000000,0.000000,0.000000}%
\pgfsetfillcolor{currentfill}%
\pgfsetlinewidth{0.803000pt}%
\definecolor{currentstroke}{rgb}{0.000000,0.000000,0.000000}%
\pgfsetstrokecolor{currentstroke}%
\pgfsetdash{}{0pt}%
\pgfsys@defobject{currentmarker}{\pgfqpoint{-0.048611in}{0.000000in}}{\pgfqpoint{0.000000in}{0.000000in}}{%
\pgfpathmoveto{\pgfqpoint{0.000000in}{0.000000in}}%
\pgfpathlineto{\pgfqpoint{-0.048611in}{0.000000in}}%
\pgfusepath{stroke,fill}%
}%
\begin{pgfscope}%
\pgfsys@transformshift{0.800000in}{3.307146in}%
\pgfsys@useobject{currentmarker}{}%
\end{pgfscope}%
\end{pgfscope}%
\begin{pgfscope}%
\definecolor{textcolor}{rgb}{0.000000,0.000000,0.000000}%
\pgfsetstrokecolor{textcolor}%
\pgfsetfillcolor{textcolor}%
\pgftext[x=0.563888in,y=3.258921in,left,base]{\color{textcolor}\rmfamily\fontsize{10.000000}{12.000000}\selectfont \(\displaystyle 45\)}%
\end{pgfscope}%
\begin{pgfscope}%
\pgfsetbuttcap%
\pgfsetroundjoin%
\definecolor{currentfill}{rgb}{0.000000,0.000000,0.000000}%
\pgfsetfillcolor{currentfill}%
\pgfsetlinewidth{0.803000pt}%
\definecolor{currentstroke}{rgb}{0.000000,0.000000,0.000000}%
\pgfsetstrokecolor{currentstroke}%
\pgfsetdash{}{0pt}%
\pgfsys@defobject{currentmarker}{\pgfqpoint{-0.048611in}{0.000000in}}{\pgfqpoint{0.000000in}{0.000000in}}{%
\pgfpathmoveto{\pgfqpoint{0.000000in}{0.000000in}}%
\pgfpathlineto{\pgfqpoint{-0.048611in}{0.000000in}}%
\pgfusepath{stroke,fill}%
}%
\begin{pgfscope}%
\pgfsys@transformshift{0.800000in}{3.765573in}%
\pgfsys@useobject{currentmarker}{}%
\end{pgfscope}%
\end{pgfscope}%
\begin{pgfscope}%
\definecolor{textcolor}{rgb}{0.000000,0.000000,0.000000}%
\pgfsetstrokecolor{textcolor}%
\pgfsetfillcolor{textcolor}%
\pgftext[x=0.563888in,y=3.717348in,left,base]{\color{textcolor}\rmfamily\fontsize{10.000000}{12.000000}\selectfont \(\displaystyle 50\)}%
\end{pgfscope}%
\begin{pgfscope}%
\pgfsetbuttcap%
\pgfsetroundjoin%
\definecolor{currentfill}{rgb}{0.000000,0.000000,0.000000}%
\pgfsetfillcolor{currentfill}%
\pgfsetlinewidth{0.803000pt}%
\definecolor{currentstroke}{rgb}{0.000000,0.000000,0.000000}%
\pgfsetstrokecolor{currentstroke}%
\pgfsetdash{}{0pt}%
\pgfsys@defobject{currentmarker}{\pgfqpoint{-0.048611in}{0.000000in}}{\pgfqpoint{0.000000in}{0.000000in}}{%
\pgfpathmoveto{\pgfqpoint{0.000000in}{0.000000in}}%
\pgfpathlineto{\pgfqpoint{-0.048611in}{0.000000in}}%
\pgfusepath{stroke,fill}%
}%
\begin{pgfscope}%
\pgfsys@transformshift{0.800000in}{4.224000in}%
\pgfsys@useobject{currentmarker}{}%
\end{pgfscope}%
\end{pgfscope}%
\begin{pgfscope}%
\definecolor{textcolor}{rgb}{0.000000,0.000000,0.000000}%
\pgfsetstrokecolor{textcolor}%
\pgfsetfillcolor{textcolor}%
\pgftext[x=0.563888in,y=4.175775in,left,base]{\color{textcolor}\rmfamily\fontsize{10.000000}{12.000000}\selectfont \(\displaystyle 55\)}%
\end{pgfscope}%
\begin{pgfscope}%
\definecolor{textcolor}{rgb}{0.000000,0.000000,0.000000}%
\pgfsetstrokecolor{textcolor}%
\pgfsetfillcolor{textcolor}%
\pgftext[x=0.508333in,y=2.592000in,,bottom,rotate=90.000000]{\color{textcolor}\rmfamily\fontsize{10.000000}{12.000000}\selectfont Température (\textdegree{}C)}%
\end{pgfscope}%
\begin{pgfscope}%
\pgfpathrectangle{\pgfqpoint{0.800000in}{0.960000in}}{\pgfqpoint{4.960000in}{3.264000in}}%
\pgfusepath{clip}%
\pgfsetrectcap%
\pgfsetroundjoin%
\pgfsetlinewidth{1.505625pt}%
\definecolor{currentstroke}{rgb}{0.121569,0.466667,0.705882}%
\pgfsetstrokecolor{currentstroke}%
\pgfsetdash{}{0pt}%
\pgfpathmoveto{\pgfqpoint{0.808275in}{2.295551in}}%
\pgfpathlineto{\pgfqpoint{0.816550in}{2.329169in}}%
\pgfpathlineto{\pgfqpoint{0.824827in}{2.396404in}}%
\pgfpathlineto{\pgfqpoint{0.833104in}{2.411685in}}%
\pgfpathlineto{\pgfqpoint{0.841378in}{2.430022in}}%
\pgfpathlineto{\pgfqpoint{0.849653in}{2.445303in}}%
\pgfpathlineto{\pgfqpoint{0.857923in}{2.430022in}}%
\pgfpathlineto{\pgfqpoint{0.866200in}{2.460584in}}%
\pgfpathlineto{\pgfqpoint{0.874478in}{2.463640in}}%
\pgfpathlineto{\pgfqpoint{0.882755in}{2.445303in}}%
\pgfpathlineto{\pgfqpoint{0.891030in}{2.445303in}}%
\pgfpathlineto{\pgfqpoint{0.899309in}{2.460584in}}%
\pgfpathlineto{\pgfqpoint{0.907586in}{2.478921in}}%
\pgfpathlineto{\pgfqpoint{0.915864in}{2.478921in}}%
\pgfpathlineto{\pgfqpoint{0.924137in}{2.494202in}}%
\pgfpathlineto{\pgfqpoint{0.932409in}{2.445303in}}%
\pgfpathlineto{\pgfqpoint{0.940683in}{2.430022in}}%
\pgfpathlineto{\pgfqpoint{0.948961in}{2.463640in}}%
\pgfpathlineto{\pgfqpoint{0.957238in}{2.478921in}}%
\pgfpathlineto{\pgfqpoint{0.965514in}{2.463640in}}%
\pgfpathlineto{\pgfqpoint{0.973788in}{2.463640in}}%
\pgfpathlineto{\pgfqpoint{0.982065in}{2.478921in}}%
\pgfpathlineto{\pgfqpoint{0.990336in}{2.478921in}}%
\pgfpathlineto{\pgfqpoint{0.998610in}{2.497258in}}%
\pgfpathlineto{\pgfqpoint{1.006884in}{2.445303in}}%
\pgfpathlineto{\pgfqpoint{1.015159in}{2.463640in}}%
\pgfpathlineto{\pgfqpoint{1.023433in}{2.497258in}}%
\pgfpathlineto{\pgfqpoint{1.031707in}{2.445303in}}%
\pgfpathlineto{\pgfqpoint{1.039981in}{2.478921in}}%
\pgfpathlineto{\pgfqpoint{1.048256in}{2.478921in}}%
\pgfpathlineto{\pgfqpoint{1.056530in}{2.460584in}}%
\pgfpathlineto{\pgfqpoint{1.064804in}{2.497258in}}%
\pgfpathlineto{\pgfqpoint{1.073078in}{2.445303in}}%
\pgfpathlineto{\pgfqpoint{1.081356in}{2.463640in}}%
\pgfpathlineto{\pgfqpoint{1.089632in}{2.478921in}}%
\pgfpathlineto{\pgfqpoint{1.097909in}{2.445303in}}%
\pgfpathlineto{\pgfqpoint{1.106186in}{2.445303in}}%
\pgfpathlineto{\pgfqpoint{1.114463in}{2.430022in}}%
\pgfpathlineto{\pgfqpoint{1.122740in}{2.478921in}}%
\pgfpathlineto{\pgfqpoint{1.131018in}{2.411685in}}%
\pgfpathlineto{\pgfqpoint{1.139292in}{2.463640in}}%
\pgfpathlineto{\pgfqpoint{1.147566in}{2.430022in}}%
\pgfpathlineto{\pgfqpoint{1.164110in}{2.460584in}}%
\pgfpathlineto{\pgfqpoint{1.172383in}{2.396404in}}%
\pgfpathlineto{\pgfqpoint{1.180655in}{2.460584in}}%
\pgfpathlineto{\pgfqpoint{1.188931in}{2.430022in}}%
\pgfpathlineto{\pgfqpoint{1.205474in}{2.430022in}}%
\pgfpathlineto{\pgfqpoint{1.213747in}{2.426966in}}%
\pgfpathlineto{\pgfqpoint{1.222019in}{2.497258in}}%
\pgfpathlineto{\pgfqpoint{1.230290in}{2.445303in}}%
\pgfpathlineto{\pgfqpoint{1.238565in}{2.430022in}}%
\pgfpathlineto{\pgfqpoint{1.255110in}{2.430022in}}%
\pgfpathlineto{\pgfqpoint{1.263388in}{2.445303in}}%
\pgfpathlineto{\pgfqpoint{1.271665in}{2.463640in}}%
\pgfpathlineto{\pgfqpoint{1.279941in}{2.460584in}}%
\pgfpathlineto{\pgfqpoint{1.288218in}{2.430022in}}%
\pgfpathlineto{\pgfqpoint{1.296496in}{2.445303in}}%
\pgfpathlineto{\pgfqpoint{1.304766in}{2.445303in}}%
\pgfpathlineto{\pgfqpoint{1.313043in}{2.396404in}}%
\pgfpathlineto{\pgfqpoint{1.321322in}{2.445303in}}%
\pgfpathlineto{\pgfqpoint{1.329593in}{2.411685in}}%
\pgfpathlineto{\pgfqpoint{1.337871in}{2.445303in}}%
\pgfpathlineto{\pgfqpoint{1.346143in}{2.430022in}}%
\pgfpathlineto{\pgfqpoint{1.354418in}{2.445303in}}%
\pgfpathlineto{\pgfqpoint{1.362696in}{2.411685in}}%
\pgfpathlineto{\pgfqpoint{1.370974in}{2.445303in}}%
\pgfpathlineto{\pgfqpoint{1.379244in}{2.463640in}}%
\pgfpathlineto{\pgfqpoint{1.387520in}{2.430022in}}%
\pgfpathlineto{\pgfqpoint{1.395797in}{2.430022in}}%
\pgfpathlineto{\pgfqpoint{1.404069in}{2.426966in}}%
\pgfpathlineto{\pgfqpoint{1.420624in}{2.463640in}}%
\pgfpathlineto{\pgfqpoint{1.428895in}{2.445303in}}%
\pgfpathlineto{\pgfqpoint{1.437173in}{2.463640in}}%
\pgfpathlineto{\pgfqpoint{1.445448in}{2.445303in}}%
\pgfpathlineto{\pgfqpoint{1.453725in}{2.445303in}}%
\pgfpathlineto{\pgfqpoint{1.462002in}{2.426966in}}%
\pgfpathlineto{\pgfqpoint{1.470279in}{2.393348in}}%
\pgfpathlineto{\pgfqpoint{1.478556in}{2.426966in}}%
\pgfpathlineto{\pgfqpoint{1.486833in}{2.430022in}}%
\pgfpathlineto{\pgfqpoint{1.495110in}{2.411685in}}%
\pgfpathlineto{\pgfqpoint{1.503387in}{2.430022in}}%
\pgfpathlineto{\pgfqpoint{1.511664in}{2.442247in}}%
\pgfpathlineto{\pgfqpoint{1.519944in}{2.426966in}}%
\pgfpathlineto{\pgfqpoint{1.528219in}{2.362787in}}%
\pgfpathlineto{\pgfqpoint{1.536497in}{2.411685in}}%
\pgfpathlineto{\pgfqpoint{1.544773in}{2.378067in}}%
\pgfpathlineto{\pgfqpoint{1.553049in}{2.426966in}}%
\pgfpathlineto{\pgfqpoint{1.561326in}{2.430022in}}%
\pgfpathlineto{\pgfqpoint{1.569603in}{2.411685in}}%
\pgfpathlineto{\pgfqpoint{1.577880in}{2.396404in}}%
\pgfpathlineto{\pgfqpoint{1.619263in}{2.396404in}}%
\pgfpathlineto{\pgfqpoint{1.627537in}{2.411685in}}%
\pgfpathlineto{\pgfqpoint{1.635813in}{2.411685in}}%
\pgfpathlineto{\pgfqpoint{1.644090in}{2.396404in}}%
\pgfpathlineto{\pgfqpoint{1.652360in}{2.411685in}}%
\pgfpathlineto{\pgfqpoint{1.660630in}{2.430022in}}%
\pgfpathlineto{\pgfqpoint{1.668907in}{2.411685in}}%
\pgfpathlineto{\pgfqpoint{1.677184in}{2.426966in}}%
\pgfpathlineto{\pgfqpoint{1.693735in}{2.396404in}}%
\pgfpathlineto{\pgfqpoint{1.702012in}{2.396404in}}%
\pgfpathlineto{\pgfqpoint{1.710289in}{2.411685in}}%
\pgfpathlineto{\pgfqpoint{1.718566in}{2.396404in}}%
\pgfpathlineto{\pgfqpoint{1.735126in}{2.396404in}}%
\pgfpathlineto{\pgfqpoint{1.743398in}{2.411685in}}%
\pgfpathlineto{\pgfqpoint{1.751674in}{2.396404in}}%
\pgfpathlineto{\pgfqpoint{1.759950in}{2.329169in}}%
\pgfpathlineto{\pgfqpoint{1.768223in}{2.396404in}}%
\pgfpathlineto{\pgfqpoint{1.776500in}{2.396404in}}%
\pgfpathlineto{\pgfqpoint{1.784775in}{2.411685in}}%
\pgfpathlineto{\pgfqpoint{1.793051in}{2.396404in}}%
\pgfpathlineto{\pgfqpoint{1.817880in}{2.396404in}}%
\pgfpathlineto{\pgfqpoint{1.826156in}{2.381124in}}%
\pgfpathlineto{\pgfqpoint{1.834433in}{2.396404in}}%
\pgfpathlineto{\pgfqpoint{1.842706in}{2.381124in}}%
\pgfpathlineto{\pgfqpoint{1.850981in}{2.411685in}}%
\pgfpathlineto{\pgfqpoint{1.859257in}{2.396404in}}%
\pgfpathlineto{\pgfqpoint{1.867529in}{2.411685in}}%
\pgfpathlineto{\pgfqpoint{1.875803in}{2.396404in}}%
\pgfpathlineto{\pgfqpoint{1.884081in}{2.426966in}}%
\pgfpathlineto{\pgfqpoint{1.892358in}{2.362787in}}%
\pgfpathlineto{\pgfqpoint{1.900630in}{2.411685in}}%
\pgfpathlineto{\pgfqpoint{1.908905in}{2.381124in}}%
\pgfpathlineto{\pgfqpoint{1.917181in}{2.381124in}}%
\pgfpathlineto{\pgfqpoint{1.925458in}{2.411685in}}%
\pgfpathlineto{\pgfqpoint{1.933727in}{2.381124in}}%
\pgfpathlineto{\pgfqpoint{1.942003in}{2.396404in}}%
\pgfpathlineto{\pgfqpoint{1.950280in}{2.365843in}}%
\pgfpathlineto{\pgfqpoint{1.958555in}{2.396404in}}%
\pgfpathlineto{\pgfqpoint{1.966828in}{2.381124in}}%
\pgfpathlineto{\pgfqpoint{1.975103in}{2.378067in}}%
\pgfpathlineto{\pgfqpoint{1.983380in}{2.396404in}}%
\pgfpathlineto{\pgfqpoint{1.991652in}{2.381124in}}%
\pgfpathlineto{\pgfqpoint{1.999924in}{2.381124in}}%
\pgfpathlineto{\pgfqpoint{2.008200in}{2.365843in}}%
\pgfpathlineto{\pgfqpoint{2.024754in}{2.396404in}}%
\pgfpathlineto{\pgfqpoint{2.033031in}{2.381124in}}%
\pgfpathlineto{\pgfqpoint{2.041308in}{2.411685in}}%
\pgfpathlineto{\pgfqpoint{2.049583in}{2.381124in}}%
\pgfpathlineto{\pgfqpoint{2.057852in}{2.381124in}}%
\pgfpathlineto{\pgfqpoint{2.066122in}{2.347506in}}%
\pgfpathlineto{\pgfqpoint{2.074399in}{2.411685in}}%
\pgfpathlineto{\pgfqpoint{2.082676in}{2.365843in}}%
\pgfpathlineto{\pgfqpoint{2.090954in}{2.396404in}}%
\pgfpathlineto{\pgfqpoint{2.099230in}{2.378067in}}%
\pgfpathlineto{\pgfqpoint{2.107504in}{2.365843in}}%
\pgfpathlineto{\pgfqpoint{2.115780in}{2.396404in}}%
\pgfpathlineto{\pgfqpoint{2.124056in}{2.381124in}}%
\pgfpathlineto{\pgfqpoint{2.132334in}{2.381124in}}%
\pgfpathlineto{\pgfqpoint{2.140612in}{2.396404in}}%
\pgfpathlineto{\pgfqpoint{2.148881in}{2.381124in}}%
\pgfpathlineto{\pgfqpoint{2.157159in}{2.396404in}}%
\pgfpathlineto{\pgfqpoint{2.165435in}{2.381124in}}%
\pgfpathlineto{\pgfqpoint{2.173707in}{2.396404in}}%
\pgfpathlineto{\pgfqpoint{2.181985in}{2.381124in}}%
\pgfpathlineto{\pgfqpoint{2.190260in}{2.381124in}}%
\pgfpathlineto{\pgfqpoint{2.198536in}{2.396404in}}%
\pgfpathlineto{\pgfqpoint{2.206813in}{2.381124in}}%
\pgfpathlineto{\pgfqpoint{2.215087in}{2.347506in}}%
\pgfpathlineto{\pgfqpoint{2.223358in}{2.381124in}}%
\pgfpathlineto{\pgfqpoint{2.231631in}{2.396404in}}%
\pgfpathlineto{\pgfqpoint{2.248183in}{2.396404in}}%
\pgfpathlineto{\pgfqpoint{2.256459in}{2.381124in}}%
\pgfpathlineto{\pgfqpoint{2.264737in}{2.329169in}}%
\pgfpathlineto{\pgfqpoint{2.273008in}{2.396404in}}%
\pgfpathlineto{\pgfqpoint{2.281283in}{2.396404in}}%
\pgfpathlineto{\pgfqpoint{2.289555in}{2.381124in}}%
\pgfpathlineto{\pgfqpoint{2.297832in}{2.396404in}}%
\pgfpathlineto{\pgfqpoint{2.306106in}{2.381124in}}%
\pgfpathlineto{\pgfqpoint{2.314379in}{2.411685in}}%
\pgfpathlineto{\pgfqpoint{2.322651in}{2.381124in}}%
\pgfpathlineto{\pgfqpoint{2.330927in}{2.381124in}}%
\pgfpathlineto{\pgfqpoint{2.339202in}{2.378067in}}%
\pgfpathlineto{\pgfqpoint{2.347479in}{2.381124in}}%
\pgfpathlineto{\pgfqpoint{2.355757in}{2.362787in}}%
\pgfpathlineto{\pgfqpoint{2.364032in}{2.396404in}}%
\pgfpathlineto{\pgfqpoint{2.372304in}{2.381124in}}%
\pgfpathlineto{\pgfqpoint{2.380582in}{2.411685in}}%
\pgfpathlineto{\pgfqpoint{2.388860in}{2.396404in}}%
\pgfpathlineto{\pgfqpoint{2.397141in}{2.396404in}}%
\pgfpathlineto{\pgfqpoint{2.405417in}{2.365843in}}%
\pgfpathlineto{\pgfqpoint{2.413695in}{2.362787in}}%
\pgfpathlineto{\pgfqpoint{2.421971in}{2.396404in}}%
\pgfpathlineto{\pgfqpoint{2.430248in}{2.362787in}}%
\pgfpathlineto{\pgfqpoint{2.438526in}{2.430022in}}%
\pgfpathlineto{\pgfqpoint{2.446803in}{2.396404in}}%
\pgfpathlineto{\pgfqpoint{2.455081in}{2.411685in}}%
\pgfpathlineto{\pgfqpoint{2.463354in}{2.430022in}}%
\pgfpathlineto{\pgfqpoint{2.471627in}{2.381124in}}%
\pgfpathlineto{\pgfqpoint{2.479899in}{2.362787in}}%
\pgfpathlineto{\pgfqpoint{2.488176in}{2.411685in}}%
\pgfpathlineto{\pgfqpoint{2.496453in}{2.378067in}}%
\pgfpathlineto{\pgfqpoint{2.504730in}{2.381124in}}%
\pgfpathlineto{\pgfqpoint{2.513007in}{2.396404in}}%
\pgfpathlineto{\pgfqpoint{2.529558in}{2.396404in}}%
\pgfpathlineto{\pgfqpoint{2.537830in}{2.430022in}}%
\pgfpathlineto{\pgfqpoint{2.546106in}{2.414742in}}%
\pgfpathlineto{\pgfqpoint{2.554383in}{2.347506in}}%
\pgfpathlineto{\pgfqpoint{2.562661in}{2.362787in}}%
\pgfpathlineto{\pgfqpoint{2.579212in}{2.430022in}}%
\pgfpathlineto{\pgfqpoint{2.595767in}{2.393348in}}%
\pgfpathlineto{\pgfqpoint{2.604043in}{2.430022in}}%
\pgfpathlineto{\pgfqpoint{2.612319in}{2.362787in}}%
\pgfpathlineto{\pgfqpoint{2.620596in}{2.396404in}}%
\pgfpathlineto{\pgfqpoint{2.628874in}{2.347506in}}%
\pgfpathlineto{\pgfqpoint{2.637150in}{2.362787in}}%
\pgfpathlineto{\pgfqpoint{2.645427in}{2.411685in}}%
\pgfpathlineto{\pgfqpoint{2.661981in}{2.381124in}}%
\pgfpathlineto{\pgfqpoint{2.670258in}{2.378067in}}%
\pgfpathlineto{\pgfqpoint{2.678535in}{2.430022in}}%
\pgfpathlineto{\pgfqpoint{2.686812in}{2.396404in}}%
\pgfpathlineto{\pgfqpoint{2.695089in}{2.381124in}}%
\pgfpathlineto{\pgfqpoint{2.703366in}{2.396404in}}%
\pgfpathlineto{\pgfqpoint{2.711644in}{2.381124in}}%
\pgfpathlineto{\pgfqpoint{2.719921in}{2.393348in}}%
\pgfpathlineto{\pgfqpoint{2.728199in}{2.414742in}}%
\pgfpathlineto{\pgfqpoint{2.736478in}{2.414742in}}%
\pgfpathlineto{\pgfqpoint{2.744755in}{2.393348in}}%
\pgfpathlineto{\pgfqpoint{2.753033in}{2.378067in}}%
\pgfpathlineto{\pgfqpoint{2.761311in}{2.396404in}}%
\pgfpathlineto{\pgfqpoint{2.769585in}{2.411685in}}%
\pgfpathlineto{\pgfqpoint{2.777859in}{2.396404in}}%
\pgfpathlineto{\pgfqpoint{2.786131in}{2.430022in}}%
\pgfpathlineto{\pgfqpoint{2.794408in}{2.430022in}}%
\pgfpathlineto{\pgfqpoint{2.802686in}{2.411685in}}%
\pgfpathlineto{\pgfqpoint{2.810964in}{2.414742in}}%
\pgfpathlineto{\pgfqpoint{2.819238in}{2.430022in}}%
\pgfpathlineto{\pgfqpoint{2.827515in}{2.414742in}}%
\pgfpathlineto{\pgfqpoint{2.835791in}{2.362787in}}%
\pgfpathlineto{\pgfqpoint{2.844066in}{2.411685in}}%
\pgfpathlineto{\pgfqpoint{2.860615in}{2.381124in}}%
\pgfpathlineto{\pgfqpoint{2.868892in}{2.396404in}}%
\pgfpathlineto{\pgfqpoint{2.877169in}{2.378067in}}%
\pgfpathlineto{\pgfqpoint{2.885446in}{2.381124in}}%
\pgfpathlineto{\pgfqpoint{2.893723in}{2.396404in}}%
\pgfpathlineto{\pgfqpoint{2.902000in}{2.430022in}}%
\pgfpathlineto{\pgfqpoint{2.910277in}{2.378067in}}%
\pgfpathlineto{\pgfqpoint{2.918554in}{2.381124in}}%
\pgfpathlineto{\pgfqpoint{2.926831in}{2.396404in}}%
\pgfpathlineto{\pgfqpoint{2.935109in}{2.426966in}}%
\pgfpathlineto{\pgfqpoint{2.943385in}{2.396404in}}%
\pgfpathlineto{\pgfqpoint{2.951662in}{2.426966in}}%
\pgfpathlineto{\pgfqpoint{2.959939in}{2.396404in}}%
\pgfpathlineto{\pgfqpoint{2.984769in}{2.396404in}}%
\pgfpathlineto{\pgfqpoint{2.993044in}{2.381124in}}%
\pgfpathlineto{\pgfqpoint{3.001321in}{2.411685in}}%
\pgfpathlineto{\pgfqpoint{3.009598in}{2.396404in}}%
\pgfpathlineto{\pgfqpoint{3.026153in}{2.396404in}}%
\pgfpathlineto{\pgfqpoint{3.034430in}{2.381124in}}%
\pgfpathlineto{\pgfqpoint{3.042703in}{2.396404in}}%
\pgfpathlineto{\pgfqpoint{3.050981in}{2.414742in}}%
\pgfpathlineto{\pgfqpoint{3.059258in}{2.381124in}}%
\pgfpathlineto{\pgfqpoint{3.067531in}{2.378067in}}%
\pgfpathlineto{\pgfqpoint{3.075808in}{2.381124in}}%
\pgfpathlineto{\pgfqpoint{3.084085in}{2.396404in}}%
\pgfpathlineto{\pgfqpoint{3.092362in}{2.378067in}}%
\pgfpathlineto{\pgfqpoint{3.100640in}{2.411685in}}%
\pgfpathlineto{\pgfqpoint{3.108918in}{2.381124in}}%
\pgfpathlineto{\pgfqpoint{3.117193in}{2.362787in}}%
\pgfpathlineto{\pgfqpoint{3.125471in}{2.381124in}}%
\pgfpathlineto{\pgfqpoint{3.133749in}{2.362787in}}%
\pgfpathlineto{\pgfqpoint{3.142019in}{2.329169in}}%
\pgfpathlineto{\pgfqpoint{3.150297in}{2.365843in}}%
\pgfpathlineto{\pgfqpoint{3.158574in}{2.396404in}}%
\pgfpathlineto{\pgfqpoint{3.183404in}{2.396404in}}%
\pgfpathlineto{\pgfqpoint{3.191681in}{2.381124in}}%
\pgfpathlineto{\pgfqpoint{3.199959in}{2.381124in}}%
\pgfpathlineto{\pgfqpoint{3.208236in}{2.396404in}}%
\pgfpathlineto{\pgfqpoint{3.216514in}{2.396404in}}%
\pgfpathlineto{\pgfqpoint{3.224792in}{2.381124in}}%
\pgfpathlineto{\pgfqpoint{3.233069in}{2.347506in}}%
\pgfpathlineto{\pgfqpoint{3.241347in}{2.396404in}}%
\pgfpathlineto{\pgfqpoint{3.249624in}{2.396404in}}%
\pgfpathlineto{\pgfqpoint{3.257902in}{2.381124in}}%
\pgfpathlineto{\pgfqpoint{3.266180in}{2.362787in}}%
\pgfpathlineto{\pgfqpoint{3.282735in}{2.362787in}}%
\pgfpathlineto{\pgfqpoint{3.291009in}{2.381124in}}%
\pgfpathlineto{\pgfqpoint{3.299286in}{2.381124in}}%
\pgfpathlineto{\pgfqpoint{3.307559in}{2.396404in}}%
\pgfpathlineto{\pgfqpoint{3.315835in}{2.396404in}}%
\pgfpathlineto{\pgfqpoint{3.324112in}{2.381124in}}%
\pgfpathlineto{\pgfqpoint{3.332389in}{2.381124in}}%
\pgfpathlineto{\pgfqpoint{3.340664in}{2.411685in}}%
\pgfpathlineto{\pgfqpoint{3.348936in}{2.381124in}}%
\pgfpathlineto{\pgfqpoint{3.357206in}{2.381124in}}%
\pgfpathlineto{\pgfqpoint{3.365484in}{2.396404in}}%
\pgfpathlineto{\pgfqpoint{3.373761in}{2.378067in}}%
\pgfpathlineto{\pgfqpoint{3.382039in}{2.381124in}}%
\pgfpathlineto{\pgfqpoint{3.390312in}{2.362787in}}%
\pgfpathlineto{\pgfqpoint{3.398590in}{2.381124in}}%
\pgfpathlineto{\pgfqpoint{3.406864in}{2.378067in}}%
\pgfpathlineto{\pgfqpoint{3.423417in}{2.313888in}}%
\pgfpathlineto{\pgfqpoint{3.431690in}{2.362787in}}%
\pgfpathlineto{\pgfqpoint{3.439967in}{2.381124in}}%
\pgfpathlineto{\pgfqpoint{3.448245in}{2.381124in}}%
\pgfpathlineto{\pgfqpoint{3.456522in}{2.347506in}}%
\pgfpathlineto{\pgfqpoint{3.464800in}{2.381124in}}%
\pgfpathlineto{\pgfqpoint{3.473077in}{2.362787in}}%
\pgfpathlineto{\pgfqpoint{3.481350in}{2.381124in}}%
\pgfpathlineto{\pgfqpoint{3.489627in}{2.381124in}}%
\pgfpathlineto{\pgfqpoint{3.497897in}{2.362787in}}%
\pgfpathlineto{\pgfqpoint{3.506169in}{2.280270in}}%
\pgfpathlineto{\pgfqpoint{3.514443in}{2.362787in}}%
\pgfpathlineto{\pgfqpoint{3.522721in}{2.365843in}}%
\pgfpathlineto{\pgfqpoint{3.530999in}{2.365843in}}%
\pgfpathlineto{\pgfqpoint{3.539274in}{2.332225in}}%
\pgfpathlineto{\pgfqpoint{3.547551in}{2.396404in}}%
\pgfpathlineto{\pgfqpoint{3.555829in}{2.347506in}}%
\pgfpathlineto{\pgfqpoint{3.564102in}{2.378067in}}%
\pgfpathlineto{\pgfqpoint{3.572379in}{2.365843in}}%
\pgfpathlineto{\pgfqpoint{3.580657in}{2.381124in}}%
\pgfpathlineto{\pgfqpoint{3.588935in}{2.329169in}}%
\pgfpathlineto{\pgfqpoint{3.597213in}{2.329169in}}%
\pgfpathlineto{\pgfqpoint{3.605487in}{2.362787in}}%
\pgfpathlineto{\pgfqpoint{3.613765in}{2.332225in}}%
\pgfpathlineto{\pgfqpoint{3.622041in}{2.362787in}}%
\pgfpathlineto{\pgfqpoint{3.630318in}{2.381124in}}%
\pgfpathlineto{\pgfqpoint{3.638595in}{2.347506in}}%
\pgfpathlineto{\pgfqpoint{3.646872in}{2.362787in}}%
\pgfpathlineto{\pgfqpoint{3.655150in}{2.332225in}}%
\pgfpathlineto{\pgfqpoint{3.663426in}{2.344449in}}%
\pgfpathlineto{\pgfqpoint{3.671698in}{2.362787in}}%
\pgfpathlineto{\pgfqpoint{3.679975in}{2.332225in}}%
\pgfpathlineto{\pgfqpoint{3.688247in}{2.347506in}}%
\pgfpathlineto{\pgfqpoint{3.696523in}{2.313888in}}%
\pgfpathlineto{\pgfqpoint{3.704801in}{2.347506in}}%
\pgfpathlineto{\pgfqpoint{3.713075in}{2.362787in}}%
\pgfpathlineto{\pgfqpoint{3.721352in}{2.310831in}}%
\pgfpathlineto{\pgfqpoint{3.729629in}{2.347506in}}%
\pgfpathlineto{\pgfqpoint{3.746173in}{2.347506in}}%
\pgfpathlineto{\pgfqpoint{3.754450in}{2.313888in}}%
\pgfpathlineto{\pgfqpoint{3.762728in}{2.332225in}}%
\pgfpathlineto{\pgfqpoint{3.771006in}{2.313888in}}%
\pgfpathlineto{\pgfqpoint{3.779283in}{2.332225in}}%
\pgfpathlineto{\pgfqpoint{3.787559in}{2.329169in}}%
\pgfpathlineto{\pgfqpoint{3.795836in}{2.313888in}}%
\pgfpathlineto{\pgfqpoint{3.804112in}{2.347506in}}%
\pgfpathlineto{\pgfqpoint{3.812387in}{2.332225in}}%
\pgfpathlineto{\pgfqpoint{3.820665in}{2.329169in}}%
\pgfpathlineto{\pgfqpoint{3.828942in}{2.332225in}}%
\pgfpathlineto{\pgfqpoint{3.837218in}{2.378067in}}%
\pgfpathlineto{\pgfqpoint{3.845496in}{2.347506in}}%
\pgfpathlineto{\pgfqpoint{3.862043in}{2.347506in}}%
\pgfpathlineto{\pgfqpoint{3.870318in}{2.313888in}}%
\pgfpathlineto{\pgfqpoint{3.878592in}{2.313888in}}%
\pgfpathlineto{\pgfqpoint{3.886870in}{2.347506in}}%
\pgfpathlineto{\pgfqpoint{3.895144in}{2.313888in}}%
\pgfpathlineto{\pgfqpoint{3.903422in}{2.344449in}}%
\pgfpathlineto{\pgfqpoint{3.911699in}{2.347506in}}%
\pgfpathlineto{\pgfqpoint{3.919975in}{2.362787in}}%
\pgfpathlineto{\pgfqpoint{3.928252in}{2.313888in}}%
\pgfpathlineto{\pgfqpoint{3.936528in}{2.344449in}}%
\pgfpathlineto{\pgfqpoint{3.944805in}{2.313888in}}%
\pgfpathlineto{\pgfqpoint{3.953083in}{2.332225in}}%
\pgfpathlineto{\pgfqpoint{3.961359in}{2.329169in}}%
\pgfpathlineto{\pgfqpoint{3.969633in}{2.347506in}}%
\pgfpathlineto{\pgfqpoint{3.977911in}{2.347506in}}%
\pgfpathlineto{\pgfqpoint{3.986185in}{2.329169in}}%
\pgfpathlineto{\pgfqpoint{3.994458in}{2.332225in}}%
\pgfpathlineto{\pgfqpoint{4.002736in}{2.313888in}}%
\pgfpathlineto{\pgfqpoint{4.011010in}{2.347506in}}%
\pgfpathlineto{\pgfqpoint{4.019281in}{2.347506in}}%
\pgfpathlineto{\pgfqpoint{4.027558in}{2.332225in}}%
\pgfpathlineto{\pgfqpoint{4.035835in}{2.329169in}}%
\pgfpathlineto{\pgfqpoint{4.044109in}{2.347506in}}%
\pgfpathlineto{\pgfqpoint{4.052387in}{2.347506in}}%
\pgfpathlineto{\pgfqpoint{4.060663in}{2.313888in}}%
\pgfpathlineto{\pgfqpoint{4.068939in}{2.347506in}}%
\pgfpathlineto{\pgfqpoint{4.077217in}{2.362787in}}%
\pgfpathlineto{\pgfqpoint{4.085494in}{2.298607in}}%
\pgfpathlineto{\pgfqpoint{4.093765in}{2.329169in}}%
\pgfpathlineto{\pgfqpoint{4.102040in}{2.332225in}}%
\pgfpathlineto{\pgfqpoint{4.110316in}{2.313888in}}%
\pgfpathlineto{\pgfqpoint{4.118590in}{2.316944in}}%
\pgfpathlineto{\pgfqpoint{4.126865in}{2.332225in}}%
\pgfpathlineto{\pgfqpoint{4.135143in}{2.313888in}}%
\pgfpathlineto{\pgfqpoint{4.143420in}{2.329169in}}%
\pgfpathlineto{\pgfqpoint{4.159974in}{2.298607in}}%
\pgfpathlineto{\pgfqpoint{4.168252in}{2.298607in}}%
\pgfpathlineto{\pgfqpoint{4.176524in}{2.313888in}}%
\pgfpathlineto{\pgfqpoint{4.193079in}{2.313888in}}%
\pgfpathlineto{\pgfqpoint{4.201357in}{2.316944in}}%
\pgfpathlineto{\pgfqpoint{4.209627in}{2.332225in}}%
\pgfpathlineto{\pgfqpoint{4.217904in}{2.313888in}}%
\pgfpathlineto{\pgfqpoint{4.226182in}{2.298607in}}%
\pgfpathlineto{\pgfqpoint{4.242730in}{2.329169in}}%
\pgfpathlineto{\pgfqpoint{4.251008in}{2.313888in}}%
\pgfpathlineto{\pgfqpoint{4.259278in}{2.347506in}}%
\pgfpathlineto{\pgfqpoint{4.267556in}{2.313888in}}%
\pgfpathlineto{\pgfqpoint{4.284111in}{2.313888in}}%
\pgfpathlineto{\pgfqpoint{4.292380in}{2.329169in}}%
\pgfpathlineto{\pgfqpoint{4.308934in}{2.329169in}}%
\pgfpathlineto{\pgfqpoint{4.317208in}{2.313888in}}%
\pgfpathlineto{\pgfqpoint{4.325486in}{2.344449in}}%
\pgfpathlineto{\pgfqpoint{4.333763in}{2.329169in}}%
\pgfpathlineto{\pgfqpoint{4.342033in}{2.329169in}}%
\pgfpathlineto{\pgfqpoint{4.350305in}{2.313888in}}%
\pgfpathlineto{\pgfqpoint{4.358580in}{2.329169in}}%
\pgfpathlineto{\pgfqpoint{4.375156in}{2.329169in}}%
\pgfpathlineto{\pgfqpoint{4.383430in}{2.344449in}}%
\pgfpathlineto{\pgfqpoint{4.391708in}{2.329169in}}%
\pgfpathlineto{\pgfqpoint{4.399985in}{2.347506in}}%
\pgfpathlineto{\pgfqpoint{4.408263in}{2.329169in}}%
\pgfpathlineto{\pgfqpoint{4.449644in}{2.329169in}}%
\pgfpathlineto{\pgfqpoint{4.457916in}{2.332225in}}%
\pgfpathlineto{\pgfqpoint{4.466193in}{2.347506in}}%
\pgfpathlineto{\pgfqpoint{4.474471in}{2.295551in}}%
\pgfpathlineto{\pgfqpoint{4.482747in}{2.329169in}}%
\pgfpathlineto{\pgfqpoint{4.491024in}{2.344449in}}%
\pgfpathlineto{\pgfqpoint{4.499301in}{2.313888in}}%
\pgfpathlineto{\pgfqpoint{4.507578in}{2.329169in}}%
\pgfpathlineto{\pgfqpoint{4.515855in}{2.329169in}}%
\pgfpathlineto{\pgfqpoint{4.524132in}{2.362787in}}%
\pgfpathlineto{\pgfqpoint{4.532405in}{2.329169in}}%
\pgfpathlineto{\pgfqpoint{4.540676in}{2.362787in}}%
\pgfpathlineto{\pgfqpoint{4.548950in}{2.329169in}}%
\pgfpathlineto{\pgfqpoint{4.557221in}{2.332225in}}%
\pgfpathlineto{\pgfqpoint{4.565498in}{2.329169in}}%
\pgfpathlineto{\pgfqpoint{4.573775in}{2.362787in}}%
\pgfpathlineto{\pgfqpoint{4.582050in}{2.347506in}}%
\pgfpathlineto{\pgfqpoint{4.590324in}{2.313888in}}%
\pgfpathlineto{\pgfqpoint{4.598597in}{2.347506in}}%
\pgfpathlineto{\pgfqpoint{4.606872in}{2.347506in}}%
\pgfpathlineto{\pgfqpoint{4.615149in}{2.362787in}}%
\pgfpathlineto{\pgfqpoint{4.623426in}{2.329169in}}%
\pgfpathlineto{\pgfqpoint{4.631703in}{2.347506in}}%
\pgfpathlineto{\pgfqpoint{4.639979in}{2.329169in}}%
\pgfpathlineto{\pgfqpoint{4.648257in}{2.381124in}}%
\pgfpathlineto{\pgfqpoint{4.664803in}{2.344449in}}%
\pgfpathlineto{\pgfqpoint{4.673075in}{2.362787in}}%
\pgfpathlineto{\pgfqpoint{4.681350in}{2.362787in}}%
\pgfpathlineto{\pgfqpoint{4.689626in}{2.347506in}}%
\pgfpathlineto{\pgfqpoint{4.697904in}{2.362787in}}%
\pgfpathlineto{\pgfqpoint{4.706181in}{2.381124in}}%
\pgfpathlineto{\pgfqpoint{4.714459in}{2.347506in}}%
\pgfpathlineto{\pgfqpoint{4.722737in}{2.381124in}}%
\pgfpathlineto{\pgfqpoint{4.731007in}{2.362787in}}%
\pgfpathlineto{\pgfqpoint{4.739282in}{2.381124in}}%
\pgfpathlineto{\pgfqpoint{4.747559in}{2.329169in}}%
\pgfpathlineto{\pgfqpoint{4.755836in}{2.329169in}}%
\pgfpathlineto{\pgfqpoint{4.764113in}{2.362787in}}%
\pgfpathlineto{\pgfqpoint{4.772391in}{2.347506in}}%
\pgfpathlineto{\pgfqpoint{4.780668in}{2.362787in}}%
\pgfpathlineto{\pgfqpoint{4.788946in}{2.362787in}}%
\pgfpathlineto{\pgfqpoint{4.797223in}{2.347506in}}%
\pgfpathlineto{\pgfqpoint{4.822045in}{2.347506in}}%
\pgfpathlineto{\pgfqpoint{4.830319in}{2.381124in}}%
\pgfpathlineto{\pgfqpoint{4.838598in}{2.381124in}}%
\pgfpathlineto{\pgfqpoint{4.846870in}{2.378067in}}%
\pgfpathlineto{\pgfqpoint{4.855144in}{2.381124in}}%
\pgfpathlineto{\pgfqpoint{4.863414in}{2.344449in}}%
\pgfpathlineto{\pgfqpoint{4.879960in}{2.381124in}}%
\pgfpathlineto{\pgfqpoint{4.888233in}{2.381124in}}%
\pgfpathlineto{\pgfqpoint{4.896508in}{2.347506in}}%
\pgfpathlineto{\pgfqpoint{4.904785in}{2.378067in}}%
\pgfpathlineto{\pgfqpoint{4.913060in}{2.362787in}}%
\pgfpathlineto{\pgfqpoint{4.921336in}{2.365843in}}%
\pgfpathlineto{\pgfqpoint{4.929608in}{2.362787in}}%
\pgfpathlineto{\pgfqpoint{4.937878in}{2.362787in}}%
\pgfpathlineto{\pgfqpoint{4.946153in}{2.347506in}}%
\pgfpathlineto{\pgfqpoint{4.954428in}{2.378067in}}%
\pgfpathlineto{\pgfqpoint{4.962705in}{2.362787in}}%
\pgfpathlineto{\pgfqpoint{4.970981in}{2.396404in}}%
\pgfpathlineto{\pgfqpoint{4.979259in}{2.399461in}}%
\pgfpathlineto{\pgfqpoint{4.987536in}{2.381124in}}%
\pgfpathlineto{\pgfqpoint{4.995814in}{2.365843in}}%
\pgfpathlineto{\pgfqpoint{5.004092in}{2.347506in}}%
\pgfpathlineto{\pgfqpoint{5.012370in}{2.362787in}}%
\pgfpathlineto{\pgfqpoint{5.020648in}{2.329169in}}%
\pgfpathlineto{\pgfqpoint{5.028923in}{2.365843in}}%
\pgfpathlineto{\pgfqpoint{5.037199in}{2.344449in}}%
\pgfpathlineto{\pgfqpoint{5.045477in}{2.362787in}}%
\pgfpathlineto{\pgfqpoint{5.053753in}{2.362787in}}%
\pgfpathlineto{\pgfqpoint{5.062031in}{2.347506in}}%
\pgfpathlineto{\pgfqpoint{5.070306in}{2.362787in}}%
\pgfpathlineto{\pgfqpoint{5.078578in}{2.362787in}}%
\pgfpathlineto{\pgfqpoint{5.086850in}{2.396404in}}%
\pgfpathlineto{\pgfqpoint{5.095123in}{2.347506in}}%
\pgfpathlineto{\pgfqpoint{5.103396in}{2.381124in}}%
\pgfpathlineto{\pgfqpoint{5.111668in}{2.396404in}}%
\pgfpathlineto{\pgfqpoint{5.119941in}{2.362787in}}%
\pgfpathlineto{\pgfqpoint{5.128211in}{2.381124in}}%
\pgfpathlineto{\pgfqpoint{5.136489in}{2.347506in}}%
\pgfpathlineto{\pgfqpoint{5.153045in}{2.378067in}}%
\pgfpathlineto{\pgfqpoint{5.161317in}{2.381124in}}%
\pgfpathlineto{\pgfqpoint{5.169591in}{2.381124in}}%
\pgfpathlineto{\pgfqpoint{5.177864in}{2.362787in}}%
\pgfpathlineto{\pgfqpoint{5.186136in}{2.381124in}}%
\pgfpathlineto{\pgfqpoint{5.202687in}{2.381124in}}%
\pgfpathlineto{\pgfqpoint{5.210965in}{2.411685in}}%
\pgfpathlineto{\pgfqpoint{5.219242in}{2.381124in}}%
\pgfpathlineto{\pgfqpoint{5.227519in}{2.396404in}}%
\pgfpathlineto{\pgfqpoint{5.235796in}{2.362787in}}%
\pgfpathlineto{\pgfqpoint{5.244073in}{2.347506in}}%
\pgfpathlineto{\pgfqpoint{5.252350in}{2.362787in}}%
\pgfpathlineto{\pgfqpoint{5.260627in}{2.381124in}}%
\pgfpathlineto{\pgfqpoint{5.277175in}{2.381124in}}%
\pgfpathlineto{\pgfqpoint{5.285453in}{2.362787in}}%
\pgfpathlineto{\pgfqpoint{5.293728in}{2.362787in}}%
\pgfpathlineto{\pgfqpoint{5.302005in}{2.365843in}}%
\pgfpathlineto{\pgfqpoint{5.310282in}{2.362787in}}%
\pgfpathlineto{\pgfqpoint{5.318557in}{2.396404in}}%
\pgfpathlineto{\pgfqpoint{5.335106in}{2.396404in}}%
\pgfpathlineto{\pgfqpoint{5.343383in}{2.381124in}}%
\pgfpathlineto{\pgfqpoint{5.351661in}{2.396404in}}%
\pgfpathlineto{\pgfqpoint{5.359931in}{2.362787in}}%
\pgfpathlineto{\pgfqpoint{5.368209in}{2.381124in}}%
\pgfpathlineto{\pgfqpoint{5.376486in}{2.414742in}}%
\pgfpathlineto{\pgfqpoint{5.384756in}{2.381124in}}%
\pgfpathlineto{\pgfqpoint{5.393034in}{2.378067in}}%
\pgfpathlineto{\pgfqpoint{5.401303in}{2.414742in}}%
\pgfpathlineto{\pgfqpoint{5.409581in}{2.414742in}}%
\pgfpathlineto{\pgfqpoint{5.417859in}{2.381124in}}%
\pgfpathlineto{\pgfqpoint{5.426137in}{2.381124in}}%
\pgfpathlineto{\pgfqpoint{5.434414in}{2.393348in}}%
\pgfpathlineto{\pgfqpoint{5.442692in}{2.414742in}}%
\pgfpathlineto{\pgfqpoint{5.450969in}{2.396404in}}%
\pgfpathlineto{\pgfqpoint{5.459247in}{2.396404in}}%
\pgfpathlineto{\pgfqpoint{5.467519in}{2.430022in}}%
\pgfpathlineto{\pgfqpoint{5.475797in}{2.396404in}}%
\pgfpathlineto{\pgfqpoint{5.484074in}{2.381124in}}%
\pgfpathlineto{\pgfqpoint{5.492352in}{2.396404in}}%
\pgfpathlineto{\pgfqpoint{5.500628in}{2.414742in}}%
\pgfpathlineto{\pgfqpoint{5.508906in}{2.396404in}}%
\pgfpathlineto{\pgfqpoint{5.517183in}{1.934921in}}%
\pgfpathlineto{\pgfqpoint{5.525462in}{1.852404in}}%
\pgfpathlineto{\pgfqpoint{5.533741in}{1.834067in}}%
\pgfpathlineto{\pgfqpoint{5.542020in}{1.769888in}}%
\pgfpathlineto{\pgfqpoint{5.550299in}{1.785169in}}%
\pgfpathlineto{\pgfqpoint{5.566857in}{1.754607in}}%
\pgfpathlineto{\pgfqpoint{5.575136in}{1.754607in}}%
\pgfpathlineto{\pgfqpoint{5.583415in}{1.736270in}}%
\pgfpathlineto{\pgfqpoint{5.591693in}{1.772944in}}%
\pgfpathlineto{\pgfqpoint{5.599972in}{1.769888in}}%
\pgfpathlineto{\pgfqpoint{5.608251in}{1.736270in}}%
\pgfpathlineto{\pgfqpoint{5.624809in}{1.736270in}}%
\pgfpathlineto{\pgfqpoint{5.633088in}{1.754607in}}%
\pgfpathlineto{\pgfqpoint{5.641367in}{1.754607in}}%
\pgfpathlineto{\pgfqpoint{5.649646in}{1.705708in}}%
\pgfpathlineto{\pgfqpoint{5.657924in}{1.736270in}}%
\pgfpathlineto{\pgfqpoint{5.666203in}{1.736270in}}%
\pgfpathlineto{\pgfqpoint{5.674482in}{1.717933in}}%
\pgfpathlineto{\pgfqpoint{5.682761in}{1.736270in}}%
\pgfpathlineto{\pgfqpoint{5.691040in}{1.702652in}}%
\pgfpathlineto{\pgfqpoint{5.699319in}{1.736270in}}%
\pgfpathlineto{\pgfqpoint{5.707597in}{1.754607in}}%
\pgfpathlineto{\pgfqpoint{5.715876in}{1.720989in}}%
\pgfpathlineto{\pgfqpoint{5.724155in}{1.736270in}}%
\pgfpathlineto{\pgfqpoint{5.732430in}{1.736270in}}%
\pgfpathlineto{\pgfqpoint{5.740702in}{1.754607in}}%
\pgfpathlineto{\pgfqpoint{5.748974in}{1.720989in}}%
\pgfpathlineto{\pgfqpoint{5.757245in}{1.754607in}}%
\pgfpathlineto{\pgfqpoint{5.757245in}{1.754607in}}%
\pgfusepath{stroke}%
\end{pgfscope}%
\begin{pgfscope}%
\pgfpathrectangle{\pgfqpoint{0.800000in}{0.960000in}}{\pgfqpoint{4.960000in}{3.264000in}}%
\pgfusepath{clip}%
\pgfsetrectcap%
\pgfsetroundjoin%
\pgfsetlinewidth{1.505625pt}%
\definecolor{currentstroke}{rgb}{1.000000,0.498039,0.054902}%
\pgfsetstrokecolor{currentstroke}%
\pgfsetdash{}{0pt}%
\pgfpathmoveto{\pgfqpoint{0.790000in}{1.188831in}}%
\pgfpathlineto{\pgfqpoint{0.800964in}{1.188831in}}%
\pgfpathlineto{\pgfqpoint{0.809737in}{1.186921in}}%
\pgfpathlineto{\pgfqpoint{0.844822in}{1.194562in}}%
\pgfpathlineto{\pgfqpoint{0.853579in}{1.194562in}}%
\pgfpathlineto{\pgfqpoint{0.862374in}{1.196472in}}%
\pgfpathlineto{\pgfqpoint{0.871141in}{1.196472in}}%
\pgfpathlineto{\pgfqpoint{0.879895in}{1.200292in}}%
\pgfpathlineto{\pgfqpoint{0.888671in}{1.200292in}}%
\pgfpathlineto{\pgfqpoint{0.897437in}{1.202202in}}%
\pgfpathlineto{\pgfqpoint{0.906195in}{1.202202in}}%
\pgfpathlineto{\pgfqpoint{0.914968in}{1.206022in}}%
\pgfpathlineto{\pgfqpoint{0.932514in}{1.206022in}}%
\pgfpathlineto{\pgfqpoint{0.941287in}{1.207933in}}%
\pgfpathlineto{\pgfqpoint{0.950059in}{1.206022in}}%
\pgfpathlineto{\pgfqpoint{0.958832in}{1.206022in}}%
\pgfpathlineto{\pgfqpoint{0.967605in}{1.207933in}}%
\pgfpathlineto{\pgfqpoint{0.993924in}{1.207933in}}%
\pgfpathlineto{\pgfqpoint{1.011448in}{1.211753in}}%
\pgfpathlineto{\pgfqpoint{1.020221in}{1.211753in}}%
\pgfpathlineto{\pgfqpoint{1.028994in}{1.209843in}}%
\pgfpathlineto{\pgfqpoint{1.046540in}{1.209843in}}%
\pgfpathlineto{\pgfqpoint{1.055313in}{1.211753in}}%
\pgfpathlineto{\pgfqpoint{1.064082in}{1.211753in}}%
\pgfpathlineto{\pgfqpoint{1.072858in}{1.209843in}}%
\pgfpathlineto{\pgfqpoint{1.081631in}{1.209843in}}%
\pgfpathlineto{\pgfqpoint{1.090404in}{1.207933in}}%
\pgfpathlineto{\pgfqpoint{1.116723in}{1.207933in}}%
\pgfpathlineto{\pgfqpoint{1.125496in}{1.206022in}}%
\pgfpathlineto{\pgfqpoint{1.134269in}{1.206022in}}%
\pgfpathlineto{\pgfqpoint{1.151793in}{1.202202in}}%
\pgfpathlineto{\pgfqpoint{1.160566in}{1.202202in}}%
\pgfpathlineto{\pgfqpoint{1.169339in}{1.198382in}}%
\pgfpathlineto{\pgfqpoint{1.186884in}{1.198382in}}%
\pgfpathlineto{\pgfqpoint{1.195652in}{1.194562in}}%
\pgfpathlineto{\pgfqpoint{1.204408in}{1.192652in}}%
\pgfpathlineto{\pgfqpoint{1.213181in}{1.192652in}}%
\pgfpathlineto{\pgfqpoint{1.221954in}{1.190742in}}%
\pgfpathlineto{\pgfqpoint{1.248273in}{1.190742in}}%
\pgfpathlineto{\pgfqpoint{1.257046in}{1.188831in}}%
\pgfpathlineto{\pgfqpoint{1.283387in}{1.188831in}}%
\pgfpathlineto{\pgfqpoint{1.292160in}{1.190742in}}%
\pgfpathlineto{\pgfqpoint{1.300933in}{1.190742in}}%
\pgfpathlineto{\pgfqpoint{1.309705in}{1.188831in}}%
\pgfpathlineto{\pgfqpoint{1.327251in}{1.188831in}}%
\pgfpathlineto{\pgfqpoint{1.336024in}{1.190742in}}%
\pgfpathlineto{\pgfqpoint{1.344797in}{1.188831in}}%
\pgfpathlineto{\pgfqpoint{1.353570in}{1.188831in}}%
\pgfpathlineto{\pgfqpoint{1.362343in}{1.186921in}}%
\pgfpathlineto{\pgfqpoint{1.371116in}{1.186921in}}%
\pgfpathlineto{\pgfqpoint{1.397413in}{1.181191in}}%
\pgfpathlineto{\pgfqpoint{1.406186in}{1.181191in}}%
\pgfpathlineto{\pgfqpoint{1.414959in}{1.177371in}}%
\pgfpathlineto{\pgfqpoint{1.441273in}{1.177371in}}%
\pgfpathlineto{\pgfqpoint{1.450028in}{1.175461in}}%
\pgfpathlineto{\pgfqpoint{1.476339in}{1.175461in}}%
\pgfpathlineto{\pgfqpoint{1.485092in}{1.171640in}}%
\pgfpathlineto{\pgfqpoint{1.493849in}{1.169730in}}%
\pgfpathlineto{\pgfqpoint{1.511395in}{1.169730in}}%
\pgfpathlineto{\pgfqpoint{1.520168in}{1.165910in}}%
\pgfpathlineto{\pgfqpoint{1.528941in}{1.165910in}}%
\pgfpathlineto{\pgfqpoint{1.537713in}{1.162090in}}%
\pgfpathlineto{\pgfqpoint{1.572805in}{1.162090in}}%
\pgfpathlineto{\pgfqpoint{1.590351in}{1.158270in}}%
\pgfpathlineto{\pgfqpoint{1.616670in}{1.158270in}}%
\pgfpathlineto{\pgfqpoint{1.625443in}{1.156360in}}%
\pgfpathlineto{\pgfqpoint{1.634216in}{1.158270in}}%
\pgfpathlineto{\pgfqpoint{1.651762in}{1.154449in}}%
\pgfpathlineto{\pgfqpoint{1.669309in}{1.154449in}}%
\pgfpathlineto{\pgfqpoint{1.678080in}{1.152539in}}%
\pgfpathlineto{\pgfqpoint{1.713172in}{1.152539in}}%
\pgfpathlineto{\pgfqpoint{1.721945in}{1.148719in}}%
\pgfpathlineto{\pgfqpoint{1.730718in}{1.148719in}}%
\pgfpathlineto{\pgfqpoint{1.739491in}{1.150629in}}%
\pgfpathlineto{\pgfqpoint{1.748264in}{1.148719in}}%
\pgfpathlineto{\pgfqpoint{1.765806in}{1.148719in}}%
\pgfpathlineto{\pgfqpoint{1.774583in}{1.146809in}}%
\pgfpathlineto{\pgfqpoint{1.783356in}{1.148719in}}%
\pgfpathlineto{\pgfqpoint{1.792128in}{1.152539in}}%
\pgfpathlineto{\pgfqpoint{1.800901in}{1.150629in}}%
\pgfpathlineto{\pgfqpoint{1.809674in}{1.152539in}}%
\pgfpathlineto{\pgfqpoint{1.827220in}{1.148719in}}%
\pgfpathlineto{\pgfqpoint{1.844766in}{1.148719in}}%
\pgfpathlineto{\pgfqpoint{1.853539in}{1.150629in}}%
\pgfpathlineto{\pgfqpoint{1.871085in}{1.150629in}}%
\pgfpathlineto{\pgfqpoint{1.879857in}{1.146809in}}%
\pgfpathlineto{\pgfqpoint{1.888631in}{1.148719in}}%
\pgfpathlineto{\pgfqpoint{1.897400in}{1.146809in}}%
\pgfpathlineto{\pgfqpoint{1.906177in}{1.146809in}}%
\pgfpathlineto{\pgfqpoint{1.914943in}{1.148719in}}%
\pgfpathlineto{\pgfqpoint{1.923700in}{1.148719in}}%
\pgfpathlineto{\pgfqpoint{1.941246in}{1.144899in}}%
\pgfpathlineto{\pgfqpoint{1.950019in}{1.144899in}}%
\pgfpathlineto{\pgfqpoint{1.958792in}{1.141079in}}%
\pgfpathlineto{\pgfqpoint{1.967565in}{1.139169in}}%
\pgfpathlineto{\pgfqpoint{1.985375in}{1.139169in}}%
\pgfpathlineto{\pgfqpoint{1.994148in}{1.137258in}}%
\pgfpathlineto{\pgfqpoint{2.020706in}{1.137258in}}%
\pgfpathlineto{\pgfqpoint{2.029483in}{1.139169in}}%
\pgfpathlineto{\pgfqpoint{2.047006in}{1.139169in}}%
\pgfpathlineto{\pgfqpoint{2.056022in}{1.142989in}}%
\pgfpathlineto{\pgfqpoint{2.064795in}{1.142989in}}%
\pgfpathlineto{\pgfqpoint{2.091113in}{1.137258in}}%
\pgfpathlineto{\pgfqpoint{2.099886in}{1.137258in}}%
\pgfpathlineto{\pgfqpoint{2.108659in}{1.135348in}}%
\pgfpathlineto{\pgfqpoint{2.117432in}{1.135348in}}%
\pgfpathlineto{\pgfqpoint{2.126205in}{1.133438in}}%
\pgfpathlineto{\pgfqpoint{2.143751in}{1.133438in}}%
\pgfpathlineto{\pgfqpoint{2.152524in}{1.135348in}}%
\pgfpathlineto{\pgfqpoint{2.161294in}{1.131528in}}%
\pgfpathlineto{\pgfqpoint{2.170066in}{1.131528in}}%
\pgfpathlineto{\pgfqpoint{2.178843in}{1.129618in}}%
\pgfpathlineto{\pgfqpoint{2.205139in}{1.129618in}}%
\pgfpathlineto{\pgfqpoint{2.213912in}{1.131528in}}%
\pgfpathlineto{\pgfqpoint{2.222685in}{1.129618in}}%
\pgfpathlineto{\pgfqpoint{2.275323in}{1.129618in}}%
\pgfpathlineto{\pgfqpoint{2.284096in}{1.131528in}}%
\pgfpathlineto{\pgfqpoint{2.292869in}{1.129618in}}%
\pgfpathlineto{\pgfqpoint{2.301642in}{1.129618in}}%
\pgfpathlineto{\pgfqpoint{2.310415in}{1.127708in}}%
\pgfpathlineto{\pgfqpoint{2.319188in}{1.127708in}}%
\pgfpathlineto{\pgfqpoint{2.327957in}{1.129618in}}%
\pgfpathlineto{\pgfqpoint{2.336976in}{1.127708in}}%
\pgfpathlineto{\pgfqpoint{2.354522in}{1.131528in}}%
\pgfpathlineto{\pgfqpoint{2.363295in}{1.131528in}}%
\pgfpathlineto{\pgfqpoint{2.372061in}{1.129618in}}%
\pgfpathlineto{\pgfqpoint{2.380819in}{1.131528in}}%
\pgfpathlineto{\pgfqpoint{2.389591in}{1.131528in}}%
\pgfpathlineto{\pgfqpoint{2.398364in}{1.133438in}}%
\pgfpathlineto{\pgfqpoint{2.407137in}{1.133438in}}%
\pgfpathlineto{\pgfqpoint{2.415910in}{1.135348in}}%
\pgfpathlineto{\pgfqpoint{2.424677in}{1.135348in}}%
\pgfpathlineto{\pgfqpoint{2.433434in}{1.137258in}}%
\pgfpathlineto{\pgfqpoint{2.442207in}{1.135348in}}%
\pgfpathlineto{\pgfqpoint{2.450980in}{1.135348in}}%
\pgfpathlineto{\pgfqpoint{2.459753in}{1.139169in}}%
\pgfpathlineto{\pgfqpoint{2.486072in}{1.139169in}}%
\pgfpathlineto{\pgfqpoint{2.494845in}{1.141079in}}%
\pgfpathlineto{\pgfqpoint{2.512390in}{1.141079in}}%
\pgfpathlineto{\pgfqpoint{2.521163in}{1.142989in}}%
\pgfpathlineto{\pgfqpoint{2.538709in}{1.142989in}}%
\pgfpathlineto{\pgfqpoint{2.547482in}{1.144899in}}%
\pgfpathlineto{\pgfqpoint{2.565028in}{1.144899in}}%
\pgfpathlineto{\pgfqpoint{2.573801in}{1.146809in}}%
\pgfpathlineto{\pgfqpoint{2.582574in}{1.146809in}}%
\pgfpathlineto{\pgfqpoint{2.591341in}{1.148719in}}%
\pgfpathlineto{\pgfqpoint{2.600098in}{1.146809in}}%
\pgfpathlineto{\pgfqpoint{2.608871in}{1.146809in}}%
\pgfpathlineto{\pgfqpoint{2.617644in}{1.148719in}}%
\pgfpathlineto{\pgfqpoint{2.626416in}{1.146809in}}%
\pgfpathlineto{\pgfqpoint{2.643962in}{1.146809in}}%
\pgfpathlineto{\pgfqpoint{2.652735in}{1.142989in}}%
\pgfpathlineto{\pgfqpoint{2.661508in}{1.141079in}}%
\pgfpathlineto{\pgfqpoint{2.670281in}{1.141079in}}%
\pgfpathlineto{\pgfqpoint{2.679054in}{1.142989in}}%
\pgfpathlineto{\pgfqpoint{2.687849in}{1.141079in}}%
\pgfpathlineto{\pgfqpoint{2.696616in}{1.141079in}}%
\pgfpathlineto{\pgfqpoint{2.714146in}{1.144899in}}%
\pgfpathlineto{\pgfqpoint{2.722919in}{1.142989in}}%
\pgfpathlineto{\pgfqpoint{2.731692in}{1.144899in}}%
\pgfpathlineto{\pgfqpoint{2.758004in}{1.144899in}}%
\pgfpathlineto{\pgfqpoint{2.767004in}{1.146809in}}%
\pgfpathlineto{\pgfqpoint{2.810868in}{1.146809in}}%
\pgfpathlineto{\pgfqpoint{2.819641in}{1.144899in}}%
\pgfpathlineto{\pgfqpoint{2.828414in}{1.144899in}}%
\pgfpathlineto{\pgfqpoint{2.837187in}{1.146809in}}%
\pgfpathlineto{\pgfqpoint{2.872279in}{1.146809in}}%
\pgfpathlineto{\pgfqpoint{2.881052in}{1.148719in}}%
\pgfpathlineto{\pgfqpoint{2.889825in}{1.146809in}}%
\pgfpathlineto{\pgfqpoint{2.924916in}{1.146809in}}%
\pgfpathlineto{\pgfqpoint{2.933689in}{1.144899in}}%
\pgfpathlineto{\pgfqpoint{2.942462in}{1.146809in}}%
\pgfpathlineto{\pgfqpoint{2.951235in}{1.146809in}}%
\pgfpathlineto{\pgfqpoint{2.960008in}{1.142989in}}%
\pgfpathlineto{\pgfqpoint{2.986327in}{1.142989in}}%
\pgfpathlineto{\pgfqpoint{2.995100in}{1.141079in}}%
\pgfpathlineto{\pgfqpoint{3.012646in}{1.141079in}}%
\pgfpathlineto{\pgfqpoint{3.021419in}{1.139169in}}%
\pgfpathlineto{\pgfqpoint{3.047735in}{1.139169in}}%
\pgfpathlineto{\pgfqpoint{3.056510in}{1.137258in}}%
\pgfpathlineto{\pgfqpoint{3.074052in}{1.141079in}}%
\pgfpathlineto{\pgfqpoint{3.082808in}{1.137258in}}%
\pgfpathlineto{\pgfqpoint{3.091580in}{1.137258in}}%
\pgfpathlineto{\pgfqpoint{3.100353in}{1.135348in}}%
\pgfpathlineto{\pgfqpoint{3.135445in}{1.135348in}}%
\pgfpathlineto{\pgfqpoint{3.144212in}{1.137258in}}%
\pgfpathlineto{\pgfqpoint{3.161735in}{1.137258in}}%
\pgfpathlineto{\pgfqpoint{3.170492in}{1.135348in}}%
\pgfpathlineto{\pgfqpoint{3.205584in}{1.135348in}}%
\pgfpathlineto{\pgfqpoint{3.214354in}{1.133438in}}%
\pgfpathlineto{\pgfqpoint{3.240676in}{1.133438in}}%
\pgfpathlineto{\pgfqpoint{3.249449in}{1.131528in}}%
\pgfpathlineto{\pgfqpoint{3.293314in}{1.131528in}}%
\pgfpathlineto{\pgfqpoint{3.310859in}{1.127708in}}%
\pgfpathlineto{\pgfqpoint{3.319632in}{1.129618in}}%
\pgfpathlineto{\pgfqpoint{3.328405in}{1.129618in}}%
\pgfpathlineto{\pgfqpoint{3.337178in}{1.127708in}}%
\pgfpathlineto{\pgfqpoint{3.345951in}{1.127708in}}%
\pgfpathlineto{\pgfqpoint{3.354724in}{1.125798in}}%
\pgfpathlineto{\pgfqpoint{3.363497in}{1.127708in}}%
\pgfpathlineto{\pgfqpoint{3.416399in}{1.127708in}}%
\pgfpathlineto{\pgfqpoint{3.433945in}{1.123888in}}%
\pgfpathlineto{\pgfqpoint{3.451491in}{1.123888in}}%
\pgfpathlineto{\pgfqpoint{3.460264in}{1.121978in}}%
\pgfpathlineto{\pgfqpoint{3.469036in}{1.121978in}}%
\pgfpathlineto{\pgfqpoint{3.486582in}{1.118157in}}%
\pgfpathlineto{\pgfqpoint{3.495355in}{1.118157in}}%
\pgfpathlineto{\pgfqpoint{3.512901in}{1.114337in}}%
\pgfpathlineto{\pgfqpoint{3.530447in}{1.114337in}}%
\pgfpathlineto{\pgfqpoint{3.539462in}{1.112427in}}%
\pgfpathlineto{\pgfqpoint{3.548235in}{1.114337in}}%
\pgfpathlineto{\pgfqpoint{3.583327in}{1.114337in}}%
\pgfpathlineto{\pgfqpoint{3.592100in}{1.116247in}}%
\pgfpathlineto{\pgfqpoint{3.600873in}{1.114337in}}%
\pgfpathlineto{\pgfqpoint{3.609646in}{1.116247in}}%
\pgfpathlineto{\pgfqpoint{3.627192in}{1.112427in}}%
\pgfpathlineto{\pgfqpoint{3.635965in}{1.114337in}}%
\pgfpathlineto{\pgfqpoint{3.653510in}{1.110517in}}%
\pgfpathlineto{\pgfqpoint{3.662283in}{1.112427in}}%
\pgfpathlineto{\pgfqpoint{3.671053in}{1.110517in}}%
\pgfpathlineto{\pgfqpoint{3.679829in}{1.110517in}}%
\pgfpathlineto{\pgfqpoint{3.697375in}{1.106697in}}%
\pgfpathlineto{\pgfqpoint{3.706147in}{1.106697in}}%
\pgfpathlineto{\pgfqpoint{3.714921in}{1.104787in}}%
\pgfpathlineto{\pgfqpoint{3.741240in}{1.104787in}}%
\pgfpathlineto{\pgfqpoint{3.750013in}{1.100966in}}%
\pgfpathlineto{\pgfqpoint{3.785104in}{1.100966in}}%
\pgfpathlineto{\pgfqpoint{3.793877in}{1.097146in}}%
\pgfpathlineto{\pgfqpoint{3.846515in}{1.097146in}}%
\pgfpathlineto{\pgfqpoint{3.855288in}{1.099056in}}%
\pgfpathlineto{\pgfqpoint{3.872856in}{1.099056in}}%
\pgfpathlineto{\pgfqpoint{3.881629in}{1.100966in}}%
\pgfpathlineto{\pgfqpoint{3.890402in}{1.099056in}}%
\pgfpathlineto{\pgfqpoint{3.899167in}{1.100966in}}%
\pgfpathlineto{\pgfqpoint{3.907925in}{1.099056in}}%
\pgfpathlineto{\pgfqpoint{3.925471in}{1.099056in}}%
\pgfpathlineto{\pgfqpoint{3.943017in}{1.095236in}}%
\pgfpathlineto{\pgfqpoint{3.960563in}{1.099056in}}%
\pgfpathlineto{\pgfqpoint{3.969336in}{1.097146in}}%
\pgfpathlineto{\pgfqpoint{3.995655in}{1.097146in}}%
\pgfpathlineto{\pgfqpoint{4.004428in}{1.095236in}}%
\pgfpathlineto{\pgfqpoint{4.021973in}{1.095236in}}%
\pgfpathlineto{\pgfqpoint{4.030746in}{1.091416in}}%
\pgfpathlineto{\pgfqpoint{4.057065in}{1.091416in}}%
\pgfpathlineto{\pgfqpoint{4.065838in}{1.089506in}}%
\pgfpathlineto{\pgfqpoint{4.074611in}{1.091416in}}%
\pgfpathlineto{\pgfqpoint{4.092157in}{1.087596in}}%
\pgfpathlineto{\pgfqpoint{4.100930in}{1.087596in}}%
\pgfpathlineto{\pgfqpoint{4.109703in}{1.085685in}}%
\pgfpathlineto{\pgfqpoint{4.118476in}{1.085685in}}%
\pgfpathlineto{\pgfqpoint{4.127249in}{1.083775in}}%
\pgfpathlineto{\pgfqpoint{4.136015in}{1.083775in}}%
\pgfpathlineto{\pgfqpoint{4.144772in}{1.079955in}}%
\pgfpathlineto{\pgfqpoint{4.179864in}{1.079955in}}%
\pgfpathlineto{\pgfqpoint{4.188636in}{1.081865in}}%
\pgfpathlineto{\pgfqpoint{4.197410in}{1.079955in}}%
\pgfpathlineto{\pgfqpoint{4.206183in}{1.079955in}}%
\pgfpathlineto{\pgfqpoint{4.214956in}{1.081865in}}%
\pgfpathlineto{\pgfqpoint{4.223729in}{1.079955in}}%
\pgfpathlineto{\pgfqpoint{4.285161in}{1.079955in}}%
\pgfpathlineto{\pgfqpoint{4.293934in}{1.081865in}}%
\pgfpathlineto{\pgfqpoint{4.302707in}{1.081865in}}%
\pgfpathlineto{\pgfqpoint{4.311480in}{1.079955in}}%
\pgfpathlineto{\pgfqpoint{4.320253in}{1.081865in}}%
\pgfpathlineto{\pgfqpoint{4.329026in}{1.081865in}}%
\pgfpathlineto{\pgfqpoint{4.337799in}{1.083775in}}%
\pgfpathlineto{\pgfqpoint{4.364118in}{1.083775in}}%
\pgfpathlineto{\pgfqpoint{4.372891in}{1.085685in}}%
\pgfpathlineto{\pgfqpoint{4.381664in}{1.083775in}}%
\pgfpathlineto{\pgfqpoint{4.407982in}{1.089506in}}%
\pgfpathlineto{\pgfqpoint{4.425522in}{1.089506in}}%
\pgfpathlineto{\pgfqpoint{4.434279in}{1.091416in}}%
\pgfpathlineto{\pgfqpoint{4.443052in}{1.089506in}}%
\pgfpathlineto{\pgfqpoint{4.451825in}{1.091416in}}%
\pgfpathlineto{\pgfqpoint{4.486917in}{1.091416in}}%
\pgfpathlineto{\pgfqpoint{4.495690in}{1.093326in}}%
\pgfpathlineto{\pgfqpoint{4.504462in}{1.091416in}}%
\pgfpathlineto{\pgfqpoint{4.513235in}{1.095236in}}%
\pgfpathlineto{\pgfqpoint{4.522008in}{1.093326in}}%
\pgfpathlineto{\pgfqpoint{4.530775in}{1.095236in}}%
\pgfpathlineto{\pgfqpoint{4.565851in}{1.095236in}}%
\pgfpathlineto{\pgfqpoint{4.583397in}{1.099056in}}%
\pgfpathlineto{\pgfqpoint{4.592170in}{1.099056in}}%
\pgfpathlineto{\pgfqpoint{4.600943in}{1.097146in}}%
\pgfpathlineto{\pgfqpoint{4.627261in}{1.102876in}}%
\pgfpathlineto{\pgfqpoint{4.636028in}{1.102876in}}%
\pgfpathlineto{\pgfqpoint{4.644785in}{1.100966in}}%
\pgfpathlineto{\pgfqpoint{4.662326in}{1.104787in}}%
\pgfpathlineto{\pgfqpoint{4.671082in}{1.104787in}}%
\pgfpathlineto{\pgfqpoint{4.688628in}{1.108607in}}%
\pgfpathlineto{\pgfqpoint{4.706174in}{1.108607in}}%
\pgfpathlineto{\pgfqpoint{4.714947in}{1.110517in}}%
\pgfpathlineto{\pgfqpoint{4.723716in}{1.110517in}}%
\pgfpathlineto{\pgfqpoint{4.741265in}{1.114337in}}%
\pgfpathlineto{\pgfqpoint{4.750038in}{1.114337in}}%
\pgfpathlineto{\pgfqpoint{4.758811in}{1.116247in}}%
\pgfpathlineto{\pgfqpoint{4.767584in}{1.116247in}}%
\pgfpathlineto{\pgfqpoint{4.785130in}{1.120067in}}%
\pgfpathlineto{\pgfqpoint{4.793903in}{1.120067in}}%
\pgfpathlineto{\pgfqpoint{4.802676in}{1.121978in}}%
\pgfpathlineto{\pgfqpoint{4.811449in}{1.121978in}}%
\pgfpathlineto{\pgfqpoint{4.820221in}{1.123888in}}%
\pgfpathlineto{\pgfqpoint{4.855336in}{1.123888in}}%
\pgfpathlineto{\pgfqpoint{4.864109in}{1.125798in}}%
\pgfpathlineto{\pgfqpoint{4.872881in}{1.125798in}}%
\pgfpathlineto{\pgfqpoint{4.890427in}{1.129618in}}%
\pgfpathlineto{\pgfqpoint{4.899200in}{1.127708in}}%
\pgfpathlineto{\pgfqpoint{4.907973in}{1.129618in}}%
\pgfpathlineto{\pgfqpoint{4.934291in}{1.129618in}}%
\pgfpathlineto{\pgfqpoint{4.943065in}{1.127708in}}%
\pgfpathlineto{\pgfqpoint{4.951838in}{1.127708in}}%
\pgfpathlineto{\pgfqpoint{4.960605in}{1.125798in}}%
\pgfpathlineto{\pgfqpoint{4.969362in}{1.127708in}}%
\pgfpathlineto{\pgfqpoint{5.039545in}{1.127708in}}%
\pgfpathlineto{\pgfqpoint{5.048318in}{1.125798in}}%
\pgfpathlineto{\pgfqpoint{5.057091in}{1.125798in}}%
\pgfpathlineto{\pgfqpoint{5.065858in}{1.127708in}}%
\pgfpathlineto{\pgfqpoint{5.074614in}{1.127708in}}%
\pgfpathlineto{\pgfqpoint{5.083388in}{1.131528in}}%
\pgfpathlineto{\pgfqpoint{5.092161in}{1.129618in}}%
\pgfpathlineto{\pgfqpoint{5.118479in}{1.129618in}}%
\pgfpathlineto{\pgfqpoint{5.127252in}{1.131528in}}%
\pgfpathlineto{\pgfqpoint{5.136025in}{1.129618in}}%
\pgfpathlineto{\pgfqpoint{5.144798in}{1.131528in}}%
\pgfpathlineto{\pgfqpoint{5.162344in}{1.131528in}}%
\pgfpathlineto{\pgfqpoint{5.171115in}{1.135348in}}%
\pgfpathlineto{\pgfqpoint{5.179890in}{1.135348in}}%
\pgfpathlineto{\pgfqpoint{5.188663in}{1.133438in}}%
\pgfpathlineto{\pgfqpoint{5.206209in}{1.133438in}}%
\pgfpathlineto{\pgfqpoint{5.214982in}{1.135348in}}%
\pgfpathlineto{\pgfqpoint{5.232539in}{1.135348in}}%
\pgfpathlineto{\pgfqpoint{5.241300in}{1.133438in}}%
\pgfpathlineto{\pgfqpoint{5.258846in}{1.133438in}}%
\pgfpathlineto{\pgfqpoint{5.293938in}{1.141079in}}%
\pgfpathlineto{\pgfqpoint{5.302711in}{1.139169in}}%
\pgfpathlineto{\pgfqpoint{5.311484in}{1.141079in}}%
\pgfpathlineto{\pgfqpoint{5.320251in}{1.141079in}}%
\pgfpathlineto{\pgfqpoint{5.329250in}{1.144899in}}%
\pgfpathlineto{\pgfqpoint{5.338023in}{1.142989in}}%
\pgfpathlineto{\pgfqpoint{5.346796in}{1.144899in}}%
\pgfpathlineto{\pgfqpoint{5.373115in}{1.144899in}}%
\pgfpathlineto{\pgfqpoint{5.381888in}{1.146809in}}%
\pgfpathlineto{\pgfqpoint{5.390661in}{1.144899in}}%
\pgfpathlineto{\pgfqpoint{5.425752in}{1.152539in}}%
\pgfpathlineto{\pgfqpoint{5.434525in}{1.152539in}}%
\pgfpathlineto{\pgfqpoint{5.443298in}{1.154449in}}%
\pgfpathlineto{\pgfqpoint{5.452071in}{1.154449in}}%
\pgfpathlineto{\pgfqpoint{5.460844in}{1.156360in}}%
\pgfpathlineto{\pgfqpoint{5.478389in}{1.156360in}}%
\pgfpathlineto{\pgfqpoint{5.487163in}{1.160180in}}%
\pgfpathlineto{\pgfqpoint{5.495936in}{1.160180in}}%
\pgfpathlineto{\pgfqpoint{5.504709in}{1.162090in}}%
\pgfpathlineto{\pgfqpoint{5.513482in}{1.162090in}}%
\pgfpathlineto{\pgfqpoint{5.522255in}{1.164000in}}%
\pgfpathlineto{\pgfqpoint{5.539800in}{1.164000in}}%
\pgfpathlineto{\pgfqpoint{5.557340in}{1.167820in}}%
\pgfpathlineto{\pgfqpoint{5.583643in}{1.167820in}}%
\pgfpathlineto{\pgfqpoint{5.592416in}{1.164000in}}%
\pgfpathlineto{\pgfqpoint{5.601186in}{1.162090in}}%
\pgfpathlineto{\pgfqpoint{5.618977in}{1.162090in}}%
\pgfpathlineto{\pgfqpoint{5.636523in}{1.158270in}}%
\pgfpathlineto{\pgfqpoint{5.654069in}{1.158270in}}%
\pgfpathlineto{\pgfqpoint{5.662842in}{1.154449in}}%
\pgfpathlineto{\pgfqpoint{5.697934in}{1.154449in}}%
\pgfpathlineto{\pgfqpoint{5.706703in}{1.150629in}}%
\pgfpathlineto{\pgfqpoint{5.724252in}{1.150629in}}%
\pgfpathlineto{\pgfqpoint{5.733025in}{1.152539in}}%
\pgfpathlineto{\pgfqpoint{5.750810in}{1.152539in}}%
\pgfpathlineto{\pgfqpoint{5.759587in}{1.148719in}}%
\pgfpathlineto{\pgfqpoint{5.770000in}{1.150272in}}%
\pgfpathlineto{\pgfqpoint{5.770000in}{1.150272in}}%
\pgfusepath{stroke}%
\end{pgfscope}%
\begin{pgfscope}%
\pgfpathrectangle{\pgfqpoint{0.800000in}{0.960000in}}{\pgfqpoint{4.960000in}{3.264000in}}%
\pgfusepath{clip}%
\pgfsetrectcap%
\pgfsetroundjoin%
\pgfsetlinewidth{1.505625pt}%
\definecolor{currentstroke}{rgb}{0.172549,0.627451,0.172549}%
\pgfsetstrokecolor{currentstroke}%
\pgfsetdash{}{0pt}%
\pgfpathmoveto{\pgfqpoint{0.790000in}{1.144422in}}%
\pgfpathlineto{\pgfqpoint{0.792191in}{1.144899in}}%
\pgfpathlineto{\pgfqpoint{0.800964in}{1.141079in}}%
\pgfpathlineto{\pgfqpoint{0.809737in}{1.139169in}}%
\pgfpathlineto{\pgfqpoint{0.818510in}{1.141079in}}%
\pgfpathlineto{\pgfqpoint{0.827283in}{1.144899in}}%
\pgfpathlineto{\pgfqpoint{0.836055in}{1.144899in}}%
\pgfpathlineto{\pgfqpoint{0.844822in}{1.148719in}}%
\pgfpathlineto{\pgfqpoint{0.853579in}{1.148719in}}%
\pgfpathlineto{\pgfqpoint{0.871141in}{1.152539in}}%
\pgfpathlineto{\pgfqpoint{0.879895in}{1.152539in}}%
\pgfpathlineto{\pgfqpoint{0.888671in}{1.156360in}}%
\pgfpathlineto{\pgfqpoint{0.897437in}{1.156360in}}%
\pgfpathlineto{\pgfqpoint{0.906195in}{1.158270in}}%
\pgfpathlineto{\pgfqpoint{0.914968in}{1.158270in}}%
\pgfpathlineto{\pgfqpoint{0.932514in}{1.162090in}}%
\pgfpathlineto{\pgfqpoint{0.941287in}{1.162090in}}%
\pgfpathlineto{\pgfqpoint{0.950059in}{1.164000in}}%
\pgfpathlineto{\pgfqpoint{0.967605in}{1.164000in}}%
\pgfpathlineto{\pgfqpoint{0.976378in}{1.165910in}}%
\pgfpathlineto{\pgfqpoint{1.002691in}{1.165910in}}%
\pgfpathlineto{\pgfqpoint{1.011448in}{1.164000in}}%
\pgfpathlineto{\pgfqpoint{1.020221in}{1.164000in}}%
\pgfpathlineto{\pgfqpoint{1.028994in}{1.165910in}}%
\pgfpathlineto{\pgfqpoint{1.072858in}{1.165910in}}%
\pgfpathlineto{\pgfqpoint{1.081631in}{1.164000in}}%
\pgfpathlineto{\pgfqpoint{1.099177in}{1.164000in}}%
\pgfpathlineto{\pgfqpoint{1.116723in}{1.160180in}}%
\pgfpathlineto{\pgfqpoint{1.125496in}{1.162090in}}%
\pgfpathlineto{\pgfqpoint{1.134269in}{1.162090in}}%
\pgfpathlineto{\pgfqpoint{1.143034in}{1.156360in}}%
\pgfpathlineto{\pgfqpoint{1.151793in}{1.156360in}}%
\pgfpathlineto{\pgfqpoint{1.169339in}{1.152539in}}%
\pgfpathlineto{\pgfqpoint{1.178112in}{1.152539in}}%
\pgfpathlineto{\pgfqpoint{1.213181in}{1.144899in}}%
\pgfpathlineto{\pgfqpoint{1.257046in}{1.144899in}}%
\pgfpathlineto{\pgfqpoint{1.265841in}{1.142989in}}%
\pgfpathlineto{\pgfqpoint{1.309705in}{1.142989in}}%
\pgfpathlineto{\pgfqpoint{1.318478in}{1.141079in}}%
\pgfpathlineto{\pgfqpoint{1.344797in}{1.141079in}}%
\pgfpathlineto{\pgfqpoint{1.353570in}{1.142989in}}%
\pgfpathlineto{\pgfqpoint{1.379883in}{1.137258in}}%
\pgfpathlineto{\pgfqpoint{1.388640in}{1.133438in}}%
\pgfpathlineto{\pgfqpoint{1.406186in}{1.133438in}}%
\pgfpathlineto{\pgfqpoint{1.414959in}{1.131528in}}%
\pgfpathlineto{\pgfqpoint{1.423732in}{1.131528in}}%
\pgfpathlineto{\pgfqpoint{1.432504in}{1.127708in}}%
\pgfpathlineto{\pgfqpoint{1.441273in}{1.125798in}}%
\pgfpathlineto{\pgfqpoint{1.467574in}{1.125798in}}%
\pgfpathlineto{\pgfqpoint{1.485092in}{1.121978in}}%
\pgfpathlineto{\pgfqpoint{1.493849in}{1.121978in}}%
\pgfpathlineto{\pgfqpoint{1.502622in}{1.120067in}}%
\pgfpathlineto{\pgfqpoint{1.511395in}{1.116247in}}%
\pgfpathlineto{\pgfqpoint{1.520168in}{1.116247in}}%
\pgfpathlineto{\pgfqpoint{1.528941in}{1.114337in}}%
\pgfpathlineto{\pgfqpoint{1.537713in}{1.114337in}}%
\pgfpathlineto{\pgfqpoint{1.546486in}{1.112427in}}%
\pgfpathlineto{\pgfqpoint{1.572805in}{1.112427in}}%
\pgfpathlineto{\pgfqpoint{1.581578in}{1.110517in}}%
\pgfpathlineto{\pgfqpoint{1.590351in}{1.110517in}}%
\pgfpathlineto{\pgfqpoint{1.599124in}{1.108607in}}%
\pgfpathlineto{\pgfqpoint{1.616670in}{1.108607in}}%
\pgfpathlineto{\pgfqpoint{1.634216in}{1.104787in}}%
\pgfpathlineto{\pgfqpoint{1.669309in}{1.104787in}}%
\pgfpathlineto{\pgfqpoint{1.686853in}{1.100966in}}%
\pgfpathlineto{\pgfqpoint{1.721945in}{1.100966in}}%
\pgfpathlineto{\pgfqpoint{1.730718in}{1.099056in}}%
\pgfpathlineto{\pgfqpoint{1.739491in}{1.100966in}}%
\pgfpathlineto{\pgfqpoint{1.748264in}{1.097146in}}%
\pgfpathlineto{\pgfqpoint{1.765806in}{1.097146in}}%
\pgfpathlineto{\pgfqpoint{1.783356in}{1.100966in}}%
\pgfpathlineto{\pgfqpoint{1.792128in}{1.100966in}}%
\pgfpathlineto{\pgfqpoint{1.800901in}{1.099056in}}%
\pgfpathlineto{\pgfqpoint{1.809674in}{1.100966in}}%
\pgfpathlineto{\pgfqpoint{1.818447in}{1.099056in}}%
\pgfpathlineto{\pgfqpoint{1.835993in}{1.099056in}}%
\pgfpathlineto{\pgfqpoint{1.844766in}{1.100966in}}%
\pgfpathlineto{\pgfqpoint{1.853539in}{1.099056in}}%
\pgfpathlineto{\pgfqpoint{1.862312in}{1.099056in}}%
\pgfpathlineto{\pgfqpoint{1.871085in}{1.097146in}}%
\pgfpathlineto{\pgfqpoint{1.906177in}{1.097146in}}%
\pgfpathlineto{\pgfqpoint{1.914943in}{1.099056in}}%
\pgfpathlineto{\pgfqpoint{1.923700in}{1.099056in}}%
\pgfpathlineto{\pgfqpoint{1.950019in}{1.093326in}}%
\pgfpathlineto{\pgfqpoint{1.958792in}{1.089506in}}%
\pgfpathlineto{\pgfqpoint{1.976602in}{1.089506in}}%
\pgfpathlineto{\pgfqpoint{1.985375in}{1.087596in}}%
\pgfpathlineto{\pgfqpoint{1.994148in}{1.087596in}}%
\pgfpathlineto{\pgfqpoint{2.002921in}{1.085685in}}%
\pgfpathlineto{\pgfqpoint{2.011694in}{1.085685in}}%
\pgfpathlineto{\pgfqpoint{2.020706in}{1.087596in}}%
\pgfpathlineto{\pgfqpoint{2.029483in}{1.087596in}}%
\pgfpathlineto{\pgfqpoint{2.038248in}{1.085685in}}%
\pgfpathlineto{\pgfqpoint{2.056022in}{1.089506in}}%
\pgfpathlineto{\pgfqpoint{2.073568in}{1.089506in}}%
\pgfpathlineto{\pgfqpoint{2.082341in}{1.087596in}}%
\pgfpathlineto{\pgfqpoint{2.091113in}{1.087596in}}%
\pgfpathlineto{\pgfqpoint{2.099886in}{1.083775in}}%
\pgfpathlineto{\pgfqpoint{2.108659in}{1.083775in}}%
\pgfpathlineto{\pgfqpoint{2.117432in}{1.085685in}}%
\pgfpathlineto{\pgfqpoint{2.134978in}{1.081865in}}%
\pgfpathlineto{\pgfqpoint{2.143751in}{1.081865in}}%
\pgfpathlineto{\pgfqpoint{2.152524in}{1.079955in}}%
\pgfpathlineto{\pgfqpoint{2.196367in}{1.079955in}}%
\pgfpathlineto{\pgfqpoint{2.205139in}{1.078045in}}%
\pgfpathlineto{\pgfqpoint{2.249004in}{1.078045in}}%
\pgfpathlineto{\pgfqpoint{2.257777in}{1.076135in}}%
\pgfpathlineto{\pgfqpoint{2.266550in}{1.078045in}}%
\pgfpathlineto{\pgfqpoint{2.275323in}{1.078045in}}%
\pgfpathlineto{\pgfqpoint{2.284096in}{1.079955in}}%
\pgfpathlineto{\pgfqpoint{2.301642in}{1.076135in}}%
\pgfpathlineto{\pgfqpoint{2.310415in}{1.076135in}}%
\pgfpathlineto{\pgfqpoint{2.319188in}{1.078045in}}%
\pgfpathlineto{\pgfqpoint{2.336976in}{1.078045in}}%
\pgfpathlineto{\pgfqpoint{2.345749in}{1.079955in}}%
\pgfpathlineto{\pgfqpoint{2.354522in}{1.078045in}}%
\pgfpathlineto{\pgfqpoint{2.363295in}{1.078045in}}%
\pgfpathlineto{\pgfqpoint{2.372061in}{1.079955in}}%
\pgfpathlineto{\pgfqpoint{2.389591in}{1.079955in}}%
\pgfpathlineto{\pgfqpoint{2.398364in}{1.081865in}}%
\pgfpathlineto{\pgfqpoint{2.407137in}{1.081865in}}%
\pgfpathlineto{\pgfqpoint{2.415910in}{1.083775in}}%
\pgfpathlineto{\pgfqpoint{2.433434in}{1.083775in}}%
\pgfpathlineto{\pgfqpoint{2.450980in}{1.087596in}}%
\pgfpathlineto{\pgfqpoint{2.459753in}{1.085685in}}%
\pgfpathlineto{\pgfqpoint{2.468526in}{1.085685in}}%
\pgfpathlineto{\pgfqpoint{2.477299in}{1.089506in}}%
\pgfpathlineto{\pgfqpoint{2.486072in}{1.087596in}}%
\pgfpathlineto{\pgfqpoint{2.494845in}{1.091416in}}%
\pgfpathlineto{\pgfqpoint{2.512390in}{1.091416in}}%
\pgfpathlineto{\pgfqpoint{2.521163in}{1.093326in}}%
\pgfpathlineto{\pgfqpoint{2.538709in}{1.093326in}}%
\pgfpathlineto{\pgfqpoint{2.547482in}{1.095236in}}%
\pgfpathlineto{\pgfqpoint{2.573801in}{1.095236in}}%
\pgfpathlineto{\pgfqpoint{2.582574in}{1.097146in}}%
\pgfpathlineto{\pgfqpoint{2.591341in}{1.095236in}}%
\pgfpathlineto{\pgfqpoint{2.600098in}{1.095236in}}%
\pgfpathlineto{\pgfqpoint{2.608871in}{1.097146in}}%
\pgfpathlineto{\pgfqpoint{2.617644in}{1.095236in}}%
\pgfpathlineto{\pgfqpoint{2.626416in}{1.095236in}}%
\pgfpathlineto{\pgfqpoint{2.635189in}{1.097146in}}%
\pgfpathlineto{\pgfqpoint{2.643962in}{1.095236in}}%
\pgfpathlineto{\pgfqpoint{2.652735in}{1.091416in}}%
\pgfpathlineto{\pgfqpoint{2.661508in}{1.093326in}}%
\pgfpathlineto{\pgfqpoint{2.670281in}{1.093326in}}%
\pgfpathlineto{\pgfqpoint{2.679054in}{1.091416in}}%
\pgfpathlineto{\pgfqpoint{2.687849in}{1.091416in}}%
\pgfpathlineto{\pgfqpoint{2.696616in}{1.093326in}}%
\pgfpathlineto{\pgfqpoint{2.705373in}{1.091416in}}%
\pgfpathlineto{\pgfqpoint{2.714146in}{1.093326in}}%
\pgfpathlineto{\pgfqpoint{2.749237in}{1.093326in}}%
\pgfpathlineto{\pgfqpoint{2.758004in}{1.097146in}}%
\pgfpathlineto{\pgfqpoint{2.784550in}{1.097146in}}%
\pgfpathlineto{\pgfqpoint{2.793323in}{1.095236in}}%
\pgfpathlineto{\pgfqpoint{2.810868in}{1.095236in}}%
\pgfpathlineto{\pgfqpoint{2.819641in}{1.097146in}}%
\pgfpathlineto{\pgfqpoint{2.828414in}{1.097146in}}%
\pgfpathlineto{\pgfqpoint{2.837187in}{1.095236in}}%
\pgfpathlineto{\pgfqpoint{2.845957in}{1.097146in}}%
\pgfpathlineto{\pgfqpoint{2.854733in}{1.095236in}}%
\pgfpathlineto{\pgfqpoint{2.872279in}{1.095236in}}%
\pgfpathlineto{\pgfqpoint{2.881052in}{1.093326in}}%
\pgfpathlineto{\pgfqpoint{2.889825in}{1.095236in}}%
\pgfpathlineto{\pgfqpoint{2.898598in}{1.095236in}}%
\pgfpathlineto{\pgfqpoint{2.916144in}{1.091416in}}%
\pgfpathlineto{\pgfqpoint{2.924916in}{1.093326in}}%
\pgfpathlineto{\pgfqpoint{2.933689in}{1.091416in}}%
\pgfpathlineto{\pgfqpoint{2.942462in}{1.093326in}}%
\pgfpathlineto{\pgfqpoint{2.951235in}{1.091416in}}%
\pgfpathlineto{\pgfqpoint{2.960008in}{1.091416in}}%
\pgfpathlineto{\pgfqpoint{2.968781in}{1.089506in}}%
\pgfpathlineto{\pgfqpoint{2.986327in}{1.089506in}}%
\pgfpathlineto{\pgfqpoint{2.995100in}{1.091416in}}%
\pgfpathlineto{\pgfqpoint{3.003874in}{1.089506in}}%
\pgfpathlineto{\pgfqpoint{3.038965in}{1.089506in}}%
\pgfpathlineto{\pgfqpoint{3.056510in}{1.085685in}}%
\pgfpathlineto{\pgfqpoint{3.065283in}{1.087596in}}%
\pgfpathlineto{\pgfqpoint{3.074052in}{1.087596in}}%
\pgfpathlineto{\pgfqpoint{3.082808in}{1.085685in}}%
\pgfpathlineto{\pgfqpoint{3.135445in}{1.085685in}}%
\pgfpathlineto{\pgfqpoint{3.144212in}{1.087596in}}%
\pgfpathlineto{\pgfqpoint{3.188038in}{1.087596in}}%
\pgfpathlineto{\pgfqpoint{3.196811in}{1.083775in}}%
\pgfpathlineto{\pgfqpoint{3.205584in}{1.085685in}}%
\pgfpathlineto{\pgfqpoint{3.214354in}{1.081865in}}%
\pgfpathlineto{\pgfqpoint{3.223130in}{1.083775in}}%
\pgfpathlineto{\pgfqpoint{3.231903in}{1.081865in}}%
\pgfpathlineto{\pgfqpoint{3.240676in}{1.081865in}}%
\pgfpathlineto{\pgfqpoint{3.258222in}{1.078045in}}%
\pgfpathlineto{\pgfqpoint{3.284541in}{1.078045in}}%
\pgfpathlineto{\pgfqpoint{3.293314in}{1.079955in}}%
\pgfpathlineto{\pgfqpoint{3.302086in}{1.078045in}}%
\pgfpathlineto{\pgfqpoint{3.328405in}{1.078045in}}%
\pgfpathlineto{\pgfqpoint{3.337178in}{1.076135in}}%
\pgfpathlineto{\pgfqpoint{3.372270in}{1.076135in}}%
\pgfpathlineto{\pgfqpoint{3.381285in}{1.074225in}}%
\pgfpathlineto{\pgfqpoint{3.390058in}{1.076135in}}%
\pgfpathlineto{\pgfqpoint{3.398853in}{1.074225in}}%
\pgfpathlineto{\pgfqpoint{3.407626in}{1.076135in}}%
\pgfpathlineto{\pgfqpoint{3.416399in}{1.072315in}}%
\pgfpathlineto{\pgfqpoint{3.425172in}{1.072315in}}%
\pgfpathlineto{\pgfqpoint{3.433945in}{1.070404in}}%
\pgfpathlineto{\pgfqpoint{3.451491in}{1.070404in}}%
\pgfpathlineto{\pgfqpoint{3.469036in}{1.066584in}}%
\pgfpathlineto{\pgfqpoint{3.477809in}{1.066584in}}%
\pgfpathlineto{\pgfqpoint{3.486582in}{1.062764in}}%
\pgfpathlineto{\pgfqpoint{3.504128in}{1.062764in}}%
\pgfpathlineto{\pgfqpoint{3.512901in}{1.058944in}}%
\pgfpathlineto{\pgfqpoint{3.521674in}{1.058944in}}%
\pgfpathlineto{\pgfqpoint{3.530447in}{1.060854in}}%
\pgfpathlineto{\pgfqpoint{3.539462in}{1.060854in}}%
\pgfpathlineto{\pgfqpoint{3.548235in}{1.058944in}}%
\pgfpathlineto{\pgfqpoint{3.557008in}{1.058944in}}%
\pgfpathlineto{\pgfqpoint{3.565781in}{1.057034in}}%
\pgfpathlineto{\pgfqpoint{3.574554in}{1.060854in}}%
\pgfpathlineto{\pgfqpoint{3.583327in}{1.058944in}}%
\pgfpathlineto{\pgfqpoint{3.627192in}{1.058944in}}%
\pgfpathlineto{\pgfqpoint{3.635965in}{1.057034in}}%
\pgfpathlineto{\pgfqpoint{3.662283in}{1.057034in}}%
\pgfpathlineto{\pgfqpoint{3.679829in}{1.053213in}}%
\pgfpathlineto{\pgfqpoint{3.706147in}{1.053213in}}%
\pgfpathlineto{\pgfqpoint{3.714921in}{1.049393in}}%
\pgfpathlineto{\pgfqpoint{3.750013in}{1.049393in}}%
\pgfpathlineto{\pgfqpoint{3.758782in}{1.045573in}}%
\pgfpathlineto{\pgfqpoint{3.767558in}{1.045573in}}%
\pgfpathlineto{\pgfqpoint{3.776331in}{1.043663in}}%
\pgfpathlineto{\pgfqpoint{3.785104in}{1.045573in}}%
\pgfpathlineto{\pgfqpoint{3.793877in}{1.045573in}}%
\pgfpathlineto{\pgfqpoint{3.802650in}{1.043663in}}%
\pgfpathlineto{\pgfqpoint{3.820196in}{1.043663in}}%
\pgfpathlineto{\pgfqpoint{3.828969in}{1.041753in}}%
\pgfpathlineto{\pgfqpoint{3.846515in}{1.041753in}}%
\pgfpathlineto{\pgfqpoint{3.855288in}{1.043663in}}%
\pgfpathlineto{\pgfqpoint{3.899167in}{1.043663in}}%
\pgfpathlineto{\pgfqpoint{3.907925in}{1.045573in}}%
\pgfpathlineto{\pgfqpoint{3.916698in}{1.043663in}}%
\pgfpathlineto{\pgfqpoint{3.925471in}{1.043663in}}%
\pgfpathlineto{\pgfqpoint{3.943017in}{1.039843in}}%
\pgfpathlineto{\pgfqpoint{3.951790in}{1.041753in}}%
\pgfpathlineto{\pgfqpoint{3.960563in}{1.039843in}}%
\pgfpathlineto{\pgfqpoint{3.995655in}{1.039843in}}%
\pgfpathlineto{\pgfqpoint{4.004428in}{1.037933in}}%
\pgfpathlineto{\pgfqpoint{4.021973in}{1.037933in}}%
\pgfpathlineto{\pgfqpoint{4.030746in}{1.036022in}}%
\pgfpathlineto{\pgfqpoint{4.065838in}{1.036022in}}%
\pgfpathlineto{\pgfqpoint{4.100930in}{1.028382in}}%
\pgfpathlineto{\pgfqpoint{4.109703in}{1.028382in}}%
\pgfpathlineto{\pgfqpoint{4.144772in}{1.020742in}}%
\pgfpathlineto{\pgfqpoint{4.153545in}{1.020742in}}%
\pgfpathlineto{\pgfqpoint{4.162318in}{1.022652in}}%
\pgfpathlineto{\pgfqpoint{4.197410in}{1.022652in}}%
\pgfpathlineto{\pgfqpoint{4.206183in}{1.024562in}}%
\pgfpathlineto{\pgfqpoint{4.214956in}{1.022652in}}%
\pgfpathlineto{\pgfqpoint{4.223729in}{1.024562in}}%
\pgfpathlineto{\pgfqpoint{4.232502in}{1.022652in}}%
\pgfpathlineto{\pgfqpoint{4.241275in}{1.022652in}}%
\pgfpathlineto{\pgfqpoint{4.250048in}{1.024562in}}%
\pgfpathlineto{\pgfqpoint{4.258820in}{1.022652in}}%
\pgfpathlineto{\pgfqpoint{4.285161in}{1.022652in}}%
\pgfpathlineto{\pgfqpoint{4.293934in}{1.020742in}}%
\pgfpathlineto{\pgfqpoint{4.302707in}{1.022652in}}%
\pgfpathlineto{\pgfqpoint{4.311480in}{1.022652in}}%
\pgfpathlineto{\pgfqpoint{4.329026in}{1.026472in}}%
\pgfpathlineto{\pgfqpoint{4.372891in}{1.026472in}}%
\pgfpathlineto{\pgfqpoint{4.390436in}{1.030292in}}%
\pgfpathlineto{\pgfqpoint{4.399209in}{1.028382in}}%
\pgfpathlineto{\pgfqpoint{4.407982in}{1.032202in}}%
\pgfpathlineto{\pgfqpoint{4.434279in}{1.032202in}}%
\pgfpathlineto{\pgfqpoint{4.443052in}{1.034112in}}%
\pgfpathlineto{\pgfqpoint{4.460598in}{1.034112in}}%
\pgfpathlineto{\pgfqpoint{4.469371in}{1.036022in}}%
\pgfpathlineto{\pgfqpoint{4.478144in}{1.036022in}}%
\pgfpathlineto{\pgfqpoint{4.486917in}{1.034112in}}%
\pgfpathlineto{\pgfqpoint{4.495690in}{1.036022in}}%
\pgfpathlineto{\pgfqpoint{4.513235in}{1.036022in}}%
\pgfpathlineto{\pgfqpoint{4.522008in}{1.037933in}}%
\pgfpathlineto{\pgfqpoint{4.530775in}{1.037933in}}%
\pgfpathlineto{\pgfqpoint{4.539532in}{1.039843in}}%
\pgfpathlineto{\pgfqpoint{4.548305in}{1.037933in}}%
\pgfpathlineto{\pgfqpoint{4.557078in}{1.041753in}}%
\pgfpathlineto{\pgfqpoint{4.565851in}{1.039843in}}%
\pgfpathlineto{\pgfqpoint{4.574624in}{1.041753in}}%
\pgfpathlineto{\pgfqpoint{4.583397in}{1.041753in}}%
\pgfpathlineto{\pgfqpoint{4.600943in}{1.045573in}}%
\pgfpathlineto{\pgfqpoint{4.627261in}{1.045573in}}%
\pgfpathlineto{\pgfqpoint{4.644785in}{1.049393in}}%
\pgfpathlineto{\pgfqpoint{4.653558in}{1.049393in}}%
\pgfpathlineto{\pgfqpoint{4.662326in}{1.051303in}}%
\pgfpathlineto{\pgfqpoint{4.679855in}{1.051303in}}%
\pgfpathlineto{\pgfqpoint{4.697401in}{1.055124in}}%
\pgfpathlineto{\pgfqpoint{4.706174in}{1.055124in}}%
\pgfpathlineto{\pgfqpoint{4.714947in}{1.057034in}}%
\pgfpathlineto{\pgfqpoint{4.732493in}{1.057034in}}%
\pgfpathlineto{\pgfqpoint{4.741265in}{1.058944in}}%
\pgfpathlineto{\pgfqpoint{4.758811in}{1.058944in}}%
\pgfpathlineto{\pgfqpoint{4.802676in}{1.068494in}}%
\pgfpathlineto{\pgfqpoint{4.811449in}{1.068494in}}%
\pgfpathlineto{\pgfqpoint{4.820221in}{1.070404in}}%
\pgfpathlineto{\pgfqpoint{4.828995in}{1.070404in}}%
\pgfpathlineto{\pgfqpoint{4.837786in}{1.072315in}}%
\pgfpathlineto{\pgfqpoint{4.846563in}{1.072315in}}%
\pgfpathlineto{\pgfqpoint{4.855336in}{1.074225in}}%
\pgfpathlineto{\pgfqpoint{4.864109in}{1.074225in}}%
\pgfpathlineto{\pgfqpoint{4.872881in}{1.072315in}}%
\pgfpathlineto{\pgfqpoint{4.881654in}{1.074225in}}%
\pgfpathlineto{\pgfqpoint{4.899200in}{1.074225in}}%
\pgfpathlineto{\pgfqpoint{4.907973in}{1.076135in}}%
\pgfpathlineto{\pgfqpoint{4.925519in}{1.076135in}}%
\pgfpathlineto{\pgfqpoint{4.934291in}{1.074225in}}%
\pgfpathlineto{\pgfqpoint{4.943065in}{1.076135in}}%
\pgfpathlineto{\pgfqpoint{4.951838in}{1.074225in}}%
\pgfpathlineto{\pgfqpoint{4.960605in}{1.076135in}}%
\pgfpathlineto{\pgfqpoint{4.969362in}{1.074225in}}%
\pgfpathlineto{\pgfqpoint{4.978135in}{1.076135in}}%
\pgfpathlineto{\pgfqpoint{4.995680in}{1.076135in}}%
\pgfpathlineto{\pgfqpoint{5.004453in}{1.074225in}}%
\pgfpathlineto{\pgfqpoint{5.057091in}{1.074225in}}%
\pgfpathlineto{\pgfqpoint{5.074614in}{1.078045in}}%
\pgfpathlineto{\pgfqpoint{5.083388in}{1.078045in}}%
\pgfpathlineto{\pgfqpoint{5.092161in}{1.076135in}}%
\pgfpathlineto{\pgfqpoint{5.109706in}{1.076135in}}%
\pgfpathlineto{\pgfqpoint{5.118479in}{1.079955in}}%
\pgfpathlineto{\pgfqpoint{5.127252in}{1.079955in}}%
\pgfpathlineto{\pgfqpoint{5.136025in}{1.081865in}}%
\pgfpathlineto{\pgfqpoint{5.144798in}{1.079955in}}%
\pgfpathlineto{\pgfqpoint{5.153571in}{1.081865in}}%
\pgfpathlineto{\pgfqpoint{5.162344in}{1.081865in}}%
\pgfpathlineto{\pgfqpoint{5.171115in}{1.079955in}}%
\pgfpathlineto{\pgfqpoint{5.179890in}{1.083775in}}%
\pgfpathlineto{\pgfqpoint{5.267619in}{1.083775in}}%
\pgfpathlineto{\pgfqpoint{5.276392in}{1.087596in}}%
\pgfpathlineto{\pgfqpoint{5.285165in}{1.089506in}}%
\pgfpathlineto{\pgfqpoint{5.293938in}{1.089506in}}%
\pgfpathlineto{\pgfqpoint{5.302711in}{1.093326in}}%
\pgfpathlineto{\pgfqpoint{5.338023in}{1.093326in}}%
\pgfpathlineto{\pgfqpoint{5.355569in}{1.097146in}}%
\pgfpathlineto{\pgfqpoint{5.364342in}{1.095236in}}%
\pgfpathlineto{\pgfqpoint{5.373115in}{1.095236in}}%
\pgfpathlineto{\pgfqpoint{5.381888in}{1.097146in}}%
\pgfpathlineto{\pgfqpoint{5.408206in}{1.097146in}}%
\pgfpathlineto{\pgfqpoint{5.416979in}{1.100966in}}%
\pgfpathlineto{\pgfqpoint{5.425752in}{1.100966in}}%
\pgfpathlineto{\pgfqpoint{5.443298in}{1.104787in}}%
\pgfpathlineto{\pgfqpoint{5.452071in}{1.104787in}}%
\pgfpathlineto{\pgfqpoint{5.460844in}{1.106697in}}%
\pgfpathlineto{\pgfqpoint{5.469617in}{1.106697in}}%
\pgfpathlineto{\pgfqpoint{5.513482in}{1.116247in}}%
\pgfpathlineto{\pgfqpoint{5.522255in}{1.116247in}}%
\pgfpathlineto{\pgfqpoint{5.531027in}{1.118157in}}%
\pgfpathlineto{\pgfqpoint{5.539800in}{1.118157in}}%
\pgfpathlineto{\pgfqpoint{5.548573in}{1.116247in}}%
\pgfpathlineto{\pgfqpoint{5.557340in}{1.120067in}}%
\pgfpathlineto{\pgfqpoint{5.574870in}{1.116247in}}%
\pgfpathlineto{\pgfqpoint{5.583643in}{1.118157in}}%
\pgfpathlineto{\pgfqpoint{5.601186in}{1.114337in}}%
\pgfpathlineto{\pgfqpoint{5.609962in}{1.110517in}}%
\pgfpathlineto{\pgfqpoint{5.618977in}{1.110517in}}%
\pgfpathlineto{\pgfqpoint{5.636523in}{1.106697in}}%
\pgfpathlineto{\pgfqpoint{5.662842in}{1.106697in}}%
\pgfpathlineto{\pgfqpoint{5.680384in}{1.102876in}}%
\pgfpathlineto{\pgfqpoint{5.689161in}{1.102876in}}%
\pgfpathlineto{\pgfqpoint{5.697934in}{1.100966in}}%
\pgfpathlineto{\pgfqpoint{5.706703in}{1.100966in}}%
\pgfpathlineto{\pgfqpoint{5.715479in}{1.099056in}}%
\pgfpathlineto{\pgfqpoint{5.724252in}{1.099056in}}%
\pgfpathlineto{\pgfqpoint{5.733025in}{1.100966in}}%
\pgfpathlineto{\pgfqpoint{5.741798in}{1.099056in}}%
\pgfpathlineto{\pgfqpoint{5.750810in}{1.095236in}}%
\pgfpathlineto{\pgfqpoint{5.759587in}{1.093326in}}%
\pgfpathlineto{\pgfqpoint{5.770000in}{1.094879in}}%
\pgfpathlineto{\pgfqpoint{5.770000in}{1.094879in}}%
\pgfusepath{stroke}%
\end{pgfscope}%
\begin{pgfscope}%
\pgfsetrectcap%
\pgfsetmiterjoin%
\pgfsetlinewidth{0.803000pt}%
\definecolor{currentstroke}{rgb}{0.000000,0.000000,0.000000}%
\pgfsetstrokecolor{currentstroke}%
\pgfsetdash{}{0pt}%
\pgfpathmoveto{\pgfqpoint{0.800000in}{0.960000in}}%
\pgfpathlineto{\pgfqpoint{0.800000in}{4.224000in}}%
\pgfusepath{stroke}%
\end{pgfscope}%
\begin{pgfscope}%
\pgfsetrectcap%
\pgfsetmiterjoin%
\pgfsetlinewidth{0.803000pt}%
\definecolor{currentstroke}{rgb}{0.000000,0.000000,0.000000}%
\pgfsetstrokecolor{currentstroke}%
\pgfsetdash{}{0pt}%
\pgfpathmoveto{\pgfqpoint{5.760000in}{0.960000in}}%
\pgfpathlineto{\pgfqpoint{5.760000in}{4.224000in}}%
\pgfusepath{stroke}%
\end{pgfscope}%
\begin{pgfscope}%
\pgfsetrectcap%
\pgfsetmiterjoin%
\pgfsetlinewidth{0.803000pt}%
\definecolor{currentstroke}{rgb}{0.000000,0.000000,0.000000}%
\pgfsetstrokecolor{currentstroke}%
\pgfsetdash{}{0pt}%
\pgfpathmoveto{\pgfqpoint{0.800000in}{0.960000in}}%
\pgfpathlineto{\pgfqpoint{5.760000in}{0.960000in}}%
\pgfusepath{stroke}%
\end{pgfscope}%
\begin{pgfscope}%
\pgfsetrectcap%
\pgfsetmiterjoin%
\pgfsetlinewidth{0.803000pt}%
\definecolor{currentstroke}{rgb}{0.000000,0.000000,0.000000}%
\pgfsetstrokecolor{currentstroke}%
\pgfsetdash{}{0pt}%
\pgfpathmoveto{\pgfqpoint{0.800000in}{4.224000in}}%
\pgfpathlineto{\pgfqpoint{5.760000in}{4.224000in}}%
\pgfusepath{stroke}%
\end{pgfscope}%
\begin{pgfscope}%
\definecolor{textcolor}{rgb}{0.000000,0.000000,0.000000}%
\pgfsetstrokecolor{textcolor}%
\pgfsetfillcolor{textcolor}%
%\pgftext[x=3.280000in,y=4.307333in,,base]{\color{textcolor}\rmfamily\fontsize{12.000000}{14.400000}\selectfont Évolution de la température dans les tests avec step-down et avec ventilateur}%
\end{pgfscope}%
\begin{pgfscope}%
\pgfsetbuttcap%
\pgfsetmiterjoin%
\definecolor{currentfill}{rgb}{1.000000,1.000000,1.000000}%
\pgfsetfillcolor{currentfill}%
\pgfsetfillopacity{0.800000}%
\pgfsetlinewidth{1.003750pt}%
\definecolor{currentstroke}{rgb}{0.800000,0.800000,0.800000}%
\pgfsetstrokecolor{currentstroke}%
\pgfsetstrokeopacity{0.800000}%
\pgfsetdash{}{0pt}%
\pgfpathmoveto{\pgfqpoint{4.070955in}{3.531871in}}%
\pgfpathlineto{\pgfqpoint{5.662778in}{3.531871in}}%
\pgfpathquadraticcurveto{\pgfqpoint{5.690556in}{3.531871in}}{\pgfqpoint{5.690556in}{3.559648in}}%
\pgfpathlineto{\pgfqpoint{5.690556in}{4.126778in}}%
\pgfpathquadraticcurveto{\pgfqpoint{5.690556in}{4.154556in}}{\pgfqpoint{5.662778in}{4.154556in}}%
\pgfpathlineto{\pgfqpoint{4.070955in}{4.154556in}}%
\pgfpathquadraticcurveto{\pgfqpoint{4.043177in}{4.154556in}}{\pgfqpoint{4.043177in}{4.126778in}}%
\pgfpathlineto{\pgfqpoint{4.043177in}{3.559648in}}%
\pgfpathquadraticcurveto{\pgfqpoint{4.043177in}{3.531871in}}{\pgfqpoint{4.070955in}{3.531871in}}%
\pgfpathclose%
\pgfusepath{stroke,fill}%
\end{pgfscope}%
\begin{pgfscope}%
\pgfsetrectcap%
\pgfsetroundjoin%
\pgfsetlinewidth{1.505625pt}%
\definecolor{currentstroke}{rgb}{0.121569,0.466667,0.705882}%
\pgfsetstrokecolor{currentstroke}%
\pgfsetdash{}{0pt}%
\pgfpathmoveto{\pgfqpoint{4.098733in}{4.050389in}}%
\pgfpathlineto{\pgfqpoint{4.376510in}{4.050389in}}%
\pgfusepath{stroke}%
\end{pgfscope}%
\begin{pgfscope}%
\definecolor{textcolor}{rgb}{0.000000,0.000000,0.000000}%
\pgfsetstrokecolor{textcolor}%
\pgfsetfillcolor{textcolor}%
\pgftext[x=4.487622in,y=4.001778in,left,base]{\color{textcolor}\rmfamily\fontsize{10.000000}{12.000000}\selectfont CPU}%
\end{pgfscope}%
\begin{pgfscope}%
\pgfsetrectcap%
\pgfsetroundjoin%
\pgfsetlinewidth{1.505625pt}%
\definecolor{currentstroke}{rgb}{1.000000,0.498039,0.054902}%
\pgfsetstrokecolor{currentstroke}%
\pgfsetdash{}{0pt}%
\pgfpathmoveto{\pgfqpoint{4.098733in}{3.856716in}}%
\pgfpathlineto{\pgfqpoint{4.376510in}{3.856716in}}%
\pgfusepath{stroke}%
\end{pgfscope}%
\begin{pgfscope}%
\definecolor{textcolor}{rgb}{0.000000,0.000000,0.000000}%
\pgfsetstrokecolor{textcolor}%
\pgfsetfillcolor{textcolor}%
\pgftext[x=4.487622in,y=3.808105in,left,base]{\color{textcolor}\rmfamily\fontsize{10.000000}{12.000000}\selectfont Zone Raspberry Pi}%
\end{pgfscope}%
\begin{pgfscope}%
\pgfsetrectcap%
\pgfsetroundjoin%
\pgfsetlinewidth{1.505625pt}%
\definecolor{currentstroke}{rgb}{0.172549,0.627451,0.172549}%
\pgfsetstrokecolor{currentstroke}%
\pgfsetdash{}{0pt}%
\pgfpathmoveto{\pgfqpoint{4.098733in}{3.663043in}}%
\pgfpathlineto{\pgfqpoint{4.376510in}{3.663043in}}%
\pgfusepath{stroke}%
\end{pgfscope}%
\begin{pgfscope}%
\definecolor{textcolor}{rgb}{0.000000,0.000000,0.000000}%
\pgfsetstrokecolor{textcolor}%
\pgfsetfillcolor{textcolor}%
\pgftext[x=4.487622in,y=3.614432in,left,base]{\color{textcolor}\rmfamily\fontsize{10.000000}{12.000000}\selectfont Zone step-down}%
\end{pgfscope}%
\end{pgfpicture}%
\makeatother%
\endgroup%

  \label{fig:test_3}
  \vspace{-1cm}
  \caption{\textbf{Test 3 :} Évolution de la température dans les tests avec step-down et avec ventilateur}
\end{figure}

~

\noindent
Finalement, dans le quatrième test, le step-down a été entouré en 10 couches de feuille d'aluminium et placé à l'intérieur de la boîte. Les couches de feuille d'aluminium permettent de créer un blindage magnétique pour protéger le modem.


\begin{figure}[h!]
  \centering
  %% Creator: Matplotlib, PGF backend
%%
%% To include the figure in your LaTeX document, write
%%   \input{<filename>.pgf}
%%
%% Make sure the required packages are loaded in your preamble
%%   \usepackage{pgf}
%%
%% Figures using additional raster images can only be included by \input if
%% they are in the same directory as the main LaTeX file. For loading figures
%% from other directories you can use the `import` package
%%   \usepackage{import}
%% and then include the figures with
%%   \import{<path to file>}{<filename>.pgf}
%%
%% Matplotlib used the following preamble
%%
\begingroup%
\makeatletter%
\begin{pgfpicture}%
\pgfpathrectangle{\pgfpointorigin}{\pgfqpoint{5.590556in}{4.068988in}}%
\pgfusepath{use as bounding box, clip}%
\begin{pgfscope}%
\pgfsetbuttcap%
\pgfsetmiterjoin%
\definecolor{currentfill}{rgb}{1.000000,1.000000,1.000000}%
\pgfsetfillcolor{currentfill}%
\pgfsetlinewidth{0.000000pt}%
\definecolor{currentstroke}{rgb}{1.000000,1.000000,1.000000}%
\pgfsetstrokecolor{currentstroke}%
\pgfsetdash{}{0pt}%
\pgfpathmoveto{\pgfqpoint{0.000000in}{0.000000in}}%
\pgfpathlineto{\pgfqpoint{5.590556in}{0.000000in}}%
\pgfpathlineto{\pgfqpoint{5.590556in}{4.068988in}}%
\pgfpathlineto{\pgfqpoint{0.000000in}{4.068988in}}%
\pgfpathclose%
\pgfusepath{fill}%
\end{pgfscope}%
\begin{pgfscope}%
\pgfsetbuttcap%
\pgfsetmiterjoin%
\definecolor{currentfill}{rgb}{1.000000,1.000000,1.000000}%
\pgfsetfillcolor{currentfill}%
\pgfsetlinewidth{0.000000pt}%
\definecolor{currentstroke}{rgb}{0.000000,0.000000,0.000000}%
\pgfsetstrokecolor{currentstroke}%
\pgfsetstrokeopacity{0.000000}%
\pgfsetdash{}{0pt}%
\pgfpathmoveto{\pgfqpoint{0.530556in}{0.656763in}}%
\pgfpathlineto{\pgfqpoint{5.490556in}{0.656763in}}%
\pgfpathlineto{\pgfqpoint{5.490556in}{3.920763in}}%
\pgfpathlineto{\pgfqpoint{0.530556in}{3.920763in}}%
\pgfpathclose%
\pgfusepath{fill}%
\end{pgfscope}%
\begin{pgfscope}%
\pgfsetbuttcap%
\pgfsetroundjoin%
\definecolor{currentfill}{rgb}{0.000000,0.000000,0.000000}%
\pgfsetfillcolor{currentfill}%
\pgfsetlinewidth{0.803000pt}%
\definecolor{currentstroke}{rgb}{0.000000,0.000000,0.000000}%
\pgfsetstrokecolor{currentstroke}%
\pgfsetdash{}{0pt}%
\pgfsys@defobject{currentmarker}{\pgfqpoint{0.000000in}{-0.048611in}}{\pgfqpoint{0.000000in}{0.000000in}}{%
\pgfpathmoveto{\pgfqpoint{0.000000in}{0.000000in}}%
\pgfpathlineto{\pgfqpoint{0.000000in}{-0.048611in}}%
\pgfusepath{stroke,fill}%
}%
\begin{pgfscope}%
\pgfsys@transformshift{0.530556in}{0.656763in}%
\pgfsys@useobject{currentmarker}{}%
\end{pgfscope}%
\end{pgfscope}%
\begin{pgfscope}%
\definecolor{textcolor}{rgb}{0.000000,0.000000,0.000000}%
\pgfsetstrokecolor{textcolor}%
\pgfsetfillcolor{textcolor}%
\pgftext[x=0.243078in,y=0.317832in,left,base,rotate=30.000000]{\color{textcolor}\rmfamily\fontsize{10.000000}{12.000000}\selectfont 00:00}%
\end{pgfscope}%
\begin{pgfscope}%
\pgfsetbuttcap%
\pgfsetroundjoin%
\definecolor{currentfill}{rgb}{0.000000,0.000000,0.000000}%
\pgfsetfillcolor{currentfill}%
\pgfsetlinewidth{0.803000pt}%
\definecolor{currentstroke}{rgb}{0.000000,0.000000,0.000000}%
\pgfsetstrokecolor{currentstroke}%
\pgfsetdash{}{0pt}%
\pgfsys@defobject{currentmarker}{\pgfqpoint{0.000000in}{-0.048611in}}{\pgfqpoint{0.000000in}{0.000000in}}{%
\pgfpathmoveto{\pgfqpoint{0.000000in}{0.000000in}}%
\pgfpathlineto{\pgfqpoint{0.000000in}{-0.048611in}}%
\pgfusepath{stroke,fill}%
}%
\begin{pgfscope}%
\pgfsys@transformshift{1.522481in}{0.656763in}%
\pgfsys@useobject{currentmarker}{}%
\end{pgfscope}%
\end{pgfscope}%
\begin{pgfscope}%
\definecolor{textcolor}{rgb}{0.000000,0.000000,0.000000}%
\pgfsetstrokecolor{textcolor}%
\pgfsetfillcolor{textcolor}%
\pgftext[x=1.235003in,y=0.317832in,left,base,rotate=30.000000]{\color{textcolor}\rmfamily\fontsize{10.000000}{12.000000}\selectfont 01:00}%
\end{pgfscope}%
\begin{pgfscope}%
\pgfsetbuttcap%
\pgfsetroundjoin%
\definecolor{currentfill}{rgb}{0.000000,0.000000,0.000000}%
\pgfsetfillcolor{currentfill}%
\pgfsetlinewidth{0.803000pt}%
\definecolor{currentstroke}{rgb}{0.000000,0.000000,0.000000}%
\pgfsetstrokecolor{currentstroke}%
\pgfsetdash{}{0pt}%
\pgfsys@defobject{currentmarker}{\pgfqpoint{0.000000in}{-0.048611in}}{\pgfqpoint{0.000000in}{0.000000in}}{%
\pgfpathmoveto{\pgfqpoint{0.000000in}{0.000000in}}%
\pgfpathlineto{\pgfqpoint{0.000000in}{-0.048611in}}%
\pgfusepath{stroke,fill}%
}%
\begin{pgfscope}%
\pgfsys@transformshift{2.514406in}{0.656763in}%
\pgfsys@useobject{currentmarker}{}%
\end{pgfscope}%
\end{pgfscope}%
\begin{pgfscope}%
\definecolor{textcolor}{rgb}{0.000000,0.000000,0.000000}%
\pgfsetstrokecolor{textcolor}%
\pgfsetfillcolor{textcolor}%
\pgftext[x=2.226928in,y=0.317832in,left,base,rotate=30.000000]{\color{textcolor}\rmfamily\fontsize{10.000000}{12.000000}\selectfont 02:00}%
\end{pgfscope}%
\begin{pgfscope}%
\pgfsetbuttcap%
\pgfsetroundjoin%
\definecolor{currentfill}{rgb}{0.000000,0.000000,0.000000}%
\pgfsetfillcolor{currentfill}%
\pgfsetlinewidth{0.803000pt}%
\definecolor{currentstroke}{rgb}{0.000000,0.000000,0.000000}%
\pgfsetstrokecolor{currentstroke}%
\pgfsetdash{}{0pt}%
\pgfsys@defobject{currentmarker}{\pgfqpoint{0.000000in}{-0.048611in}}{\pgfqpoint{0.000000in}{0.000000in}}{%
\pgfpathmoveto{\pgfqpoint{0.000000in}{0.000000in}}%
\pgfpathlineto{\pgfqpoint{0.000000in}{-0.048611in}}%
\pgfusepath{stroke,fill}%
}%
\begin{pgfscope}%
\pgfsys@transformshift{3.506331in}{0.656763in}%
\pgfsys@useobject{currentmarker}{}%
\end{pgfscope}%
\end{pgfscope}%
\begin{pgfscope}%
\definecolor{textcolor}{rgb}{0.000000,0.000000,0.000000}%
\pgfsetstrokecolor{textcolor}%
\pgfsetfillcolor{textcolor}%
\pgftext[x=3.218853in,y=0.317832in,left,base,rotate=30.000000]{\color{textcolor}\rmfamily\fontsize{10.000000}{12.000000}\selectfont 03:00}%
\end{pgfscope}%
\begin{pgfscope}%
\pgfsetbuttcap%
\pgfsetroundjoin%
\definecolor{currentfill}{rgb}{0.000000,0.000000,0.000000}%
\pgfsetfillcolor{currentfill}%
\pgfsetlinewidth{0.803000pt}%
\definecolor{currentstroke}{rgb}{0.000000,0.000000,0.000000}%
\pgfsetstrokecolor{currentstroke}%
\pgfsetdash{}{0pt}%
\pgfsys@defobject{currentmarker}{\pgfqpoint{0.000000in}{-0.048611in}}{\pgfqpoint{0.000000in}{0.000000in}}{%
\pgfpathmoveto{\pgfqpoint{0.000000in}{0.000000in}}%
\pgfpathlineto{\pgfqpoint{0.000000in}{-0.048611in}}%
\pgfusepath{stroke,fill}%
}%
\begin{pgfscope}%
\pgfsys@transformshift{4.498256in}{0.656763in}%
\pgfsys@useobject{currentmarker}{}%
\end{pgfscope}%
\end{pgfscope}%
\begin{pgfscope}%
\definecolor{textcolor}{rgb}{0.000000,0.000000,0.000000}%
\pgfsetstrokecolor{textcolor}%
\pgfsetfillcolor{textcolor}%
\pgftext[x=4.210778in,y=0.317832in,left,base,rotate=30.000000]{\color{textcolor}\rmfamily\fontsize{10.000000}{12.000000}\selectfont 04:00}%
\end{pgfscope}%
\begin{pgfscope}%
\pgfsetbuttcap%
\pgfsetroundjoin%
\definecolor{currentfill}{rgb}{0.000000,0.000000,0.000000}%
\pgfsetfillcolor{currentfill}%
\pgfsetlinewidth{0.803000pt}%
\definecolor{currentstroke}{rgb}{0.000000,0.000000,0.000000}%
\pgfsetstrokecolor{currentstroke}%
\pgfsetdash{}{0pt}%
\pgfsys@defobject{currentmarker}{\pgfqpoint{0.000000in}{-0.048611in}}{\pgfqpoint{0.000000in}{0.000000in}}{%
\pgfpathmoveto{\pgfqpoint{0.000000in}{0.000000in}}%
\pgfpathlineto{\pgfqpoint{0.000000in}{-0.048611in}}%
\pgfusepath{stroke,fill}%
}%
\begin{pgfscope}%
\pgfsys@transformshift{5.490181in}{0.656763in}%
\pgfsys@useobject{currentmarker}{}%
\end{pgfscope}%
\end{pgfscope}%
\begin{pgfscope}%
\definecolor{textcolor}{rgb}{0.000000,0.000000,0.000000}%
\pgfsetstrokecolor{textcolor}%
\pgfsetfillcolor{textcolor}%
\pgftext[x=5.202703in,y=0.317832in,left,base,rotate=30.000000]{\color{textcolor}\rmfamily\fontsize{10.000000}{12.000000}\selectfont 05:00}%
\end{pgfscope}%
\begin{pgfscope}%
\definecolor{textcolor}{rgb}{0.000000,0.000000,0.000000}%
\pgfsetstrokecolor{textcolor}%
\pgfsetfillcolor{textcolor}%
\pgftext[x=3.010556in,y=0.238889in,,top]{\color{textcolor}\rmfamily\fontsize{10.000000}{12.000000}\selectfont Temps (hh:mm)}%
\end{pgfscope}%
\begin{pgfscope}%
\pgfsetbuttcap%
\pgfsetroundjoin%
\definecolor{currentfill}{rgb}{0.000000,0.000000,0.000000}%
\pgfsetfillcolor{currentfill}%
\pgfsetlinewidth{0.803000pt}%
\definecolor{currentstroke}{rgb}{0.000000,0.000000,0.000000}%
\pgfsetstrokecolor{currentstroke}%
\pgfsetdash{}{0pt}%
\pgfsys@defobject{currentmarker}{\pgfqpoint{-0.048611in}{0.000000in}}{\pgfqpoint{0.000000in}{0.000000in}}{%
\pgfpathmoveto{\pgfqpoint{0.000000in}{0.000000in}}%
\pgfpathlineto{\pgfqpoint{-0.048611in}{0.000000in}}%
\pgfusepath{stroke,fill}%
}%
\begin{pgfscope}%
\pgfsys@transformshift{0.530556in}{0.656763in}%
\pgfsys@useobject{currentmarker}{}%
\end{pgfscope}%
\end{pgfscope}%
\begin{pgfscope}%
\definecolor{textcolor}{rgb}{0.000000,0.000000,0.000000}%
\pgfsetstrokecolor{textcolor}%
\pgfsetfillcolor{textcolor}%
\pgftext[x=0.294444in,y=0.608538in,left,base]{\color{textcolor}\rmfamily\fontsize{10.000000}{12.000000}\selectfont \(\displaystyle 20\)}%
\end{pgfscope}%
\begin{pgfscope}%
\pgfsetbuttcap%
\pgfsetroundjoin%
\definecolor{currentfill}{rgb}{0.000000,0.000000,0.000000}%
\pgfsetfillcolor{currentfill}%
\pgfsetlinewidth{0.803000pt}%
\definecolor{currentstroke}{rgb}{0.000000,0.000000,0.000000}%
\pgfsetstrokecolor{currentstroke}%
\pgfsetdash{}{0pt}%
\pgfsys@defobject{currentmarker}{\pgfqpoint{-0.048611in}{0.000000in}}{\pgfqpoint{0.000000in}{0.000000in}}{%
\pgfpathmoveto{\pgfqpoint{0.000000in}{0.000000in}}%
\pgfpathlineto{\pgfqpoint{-0.048611in}{0.000000in}}%
\pgfusepath{stroke,fill}%
}%
\begin{pgfscope}%
\pgfsys@transformshift{0.530556in}{1.123049in}%
\pgfsys@useobject{currentmarker}{}%
\end{pgfscope}%
\end{pgfscope}%
\begin{pgfscope}%
\definecolor{textcolor}{rgb}{0.000000,0.000000,0.000000}%
\pgfsetstrokecolor{textcolor}%
\pgfsetfillcolor{textcolor}%
\pgftext[x=0.294444in,y=1.074823in,left,base]{\color{textcolor}\rmfamily\fontsize{10.000000}{12.000000}\selectfont \(\displaystyle 25\)}%
\end{pgfscope}%
\begin{pgfscope}%
\pgfsetbuttcap%
\pgfsetroundjoin%
\definecolor{currentfill}{rgb}{0.000000,0.000000,0.000000}%
\pgfsetfillcolor{currentfill}%
\pgfsetlinewidth{0.803000pt}%
\definecolor{currentstroke}{rgb}{0.000000,0.000000,0.000000}%
\pgfsetstrokecolor{currentstroke}%
\pgfsetdash{}{0pt}%
\pgfsys@defobject{currentmarker}{\pgfqpoint{-0.048611in}{0.000000in}}{\pgfqpoint{0.000000in}{0.000000in}}{%
\pgfpathmoveto{\pgfqpoint{0.000000in}{0.000000in}}%
\pgfpathlineto{\pgfqpoint{-0.048611in}{0.000000in}}%
\pgfusepath{stroke,fill}%
}%
\begin{pgfscope}%
\pgfsys@transformshift{0.530556in}{1.589334in}%
\pgfsys@useobject{currentmarker}{}%
\end{pgfscope}%
\end{pgfscope}%
\begin{pgfscope}%
\definecolor{textcolor}{rgb}{0.000000,0.000000,0.000000}%
\pgfsetstrokecolor{textcolor}%
\pgfsetfillcolor{textcolor}%
\pgftext[x=0.294444in,y=1.541109in,left,base]{\color{textcolor}\rmfamily\fontsize{10.000000}{12.000000}\selectfont \(\displaystyle 30\)}%
\end{pgfscope}%
\begin{pgfscope}%
\pgfsetbuttcap%
\pgfsetroundjoin%
\definecolor{currentfill}{rgb}{0.000000,0.000000,0.000000}%
\pgfsetfillcolor{currentfill}%
\pgfsetlinewidth{0.803000pt}%
\definecolor{currentstroke}{rgb}{0.000000,0.000000,0.000000}%
\pgfsetstrokecolor{currentstroke}%
\pgfsetdash{}{0pt}%
\pgfsys@defobject{currentmarker}{\pgfqpoint{-0.048611in}{0.000000in}}{\pgfqpoint{0.000000in}{0.000000in}}{%
\pgfpathmoveto{\pgfqpoint{0.000000in}{0.000000in}}%
\pgfpathlineto{\pgfqpoint{-0.048611in}{0.000000in}}%
\pgfusepath{stroke,fill}%
}%
\begin{pgfscope}%
\pgfsys@transformshift{0.530556in}{2.055620in}%
\pgfsys@useobject{currentmarker}{}%
\end{pgfscope}%
\end{pgfscope}%
\begin{pgfscope}%
\definecolor{textcolor}{rgb}{0.000000,0.000000,0.000000}%
\pgfsetstrokecolor{textcolor}%
\pgfsetfillcolor{textcolor}%
\pgftext[x=0.294444in,y=2.007395in,left,base]{\color{textcolor}\rmfamily\fontsize{10.000000}{12.000000}\selectfont \(\displaystyle 35\)}%
\end{pgfscope}%
\begin{pgfscope}%
\pgfsetbuttcap%
\pgfsetroundjoin%
\definecolor{currentfill}{rgb}{0.000000,0.000000,0.000000}%
\pgfsetfillcolor{currentfill}%
\pgfsetlinewidth{0.803000pt}%
\definecolor{currentstroke}{rgb}{0.000000,0.000000,0.000000}%
\pgfsetstrokecolor{currentstroke}%
\pgfsetdash{}{0pt}%
\pgfsys@defobject{currentmarker}{\pgfqpoint{-0.048611in}{0.000000in}}{\pgfqpoint{0.000000in}{0.000000in}}{%
\pgfpathmoveto{\pgfqpoint{0.000000in}{0.000000in}}%
\pgfpathlineto{\pgfqpoint{-0.048611in}{0.000000in}}%
\pgfusepath{stroke,fill}%
}%
\begin{pgfscope}%
\pgfsys@transformshift{0.530556in}{2.521906in}%
\pgfsys@useobject{currentmarker}{}%
\end{pgfscope}%
\end{pgfscope}%
\begin{pgfscope}%
\definecolor{textcolor}{rgb}{0.000000,0.000000,0.000000}%
\pgfsetstrokecolor{textcolor}%
\pgfsetfillcolor{textcolor}%
\pgftext[x=0.294444in,y=2.473680in,left,base]{\color{textcolor}\rmfamily\fontsize{10.000000}{12.000000}\selectfont \(\displaystyle 40\)}%
\end{pgfscope}%
\begin{pgfscope}%
\pgfsetbuttcap%
\pgfsetroundjoin%
\definecolor{currentfill}{rgb}{0.000000,0.000000,0.000000}%
\pgfsetfillcolor{currentfill}%
\pgfsetlinewidth{0.803000pt}%
\definecolor{currentstroke}{rgb}{0.000000,0.000000,0.000000}%
\pgfsetstrokecolor{currentstroke}%
\pgfsetdash{}{0pt}%
\pgfsys@defobject{currentmarker}{\pgfqpoint{-0.048611in}{0.000000in}}{\pgfqpoint{0.000000in}{0.000000in}}{%
\pgfpathmoveto{\pgfqpoint{0.000000in}{0.000000in}}%
\pgfpathlineto{\pgfqpoint{-0.048611in}{0.000000in}}%
\pgfusepath{stroke,fill}%
}%
\begin{pgfscope}%
\pgfsys@transformshift{0.530556in}{2.988191in}%
\pgfsys@useobject{currentmarker}{}%
\end{pgfscope}%
\end{pgfscope}%
\begin{pgfscope}%
\definecolor{textcolor}{rgb}{0.000000,0.000000,0.000000}%
\pgfsetstrokecolor{textcolor}%
\pgfsetfillcolor{textcolor}%
\pgftext[x=0.294444in,y=2.939966in,left,base]{\color{textcolor}\rmfamily\fontsize{10.000000}{12.000000}\selectfont \(\displaystyle 45\)}%
\end{pgfscope}%
\begin{pgfscope}%
\pgfsetbuttcap%
\pgfsetroundjoin%
\definecolor{currentfill}{rgb}{0.000000,0.000000,0.000000}%
\pgfsetfillcolor{currentfill}%
\pgfsetlinewidth{0.803000pt}%
\definecolor{currentstroke}{rgb}{0.000000,0.000000,0.000000}%
\pgfsetstrokecolor{currentstroke}%
\pgfsetdash{}{0pt}%
\pgfsys@defobject{currentmarker}{\pgfqpoint{-0.048611in}{0.000000in}}{\pgfqpoint{0.000000in}{0.000000in}}{%
\pgfpathmoveto{\pgfqpoint{0.000000in}{0.000000in}}%
\pgfpathlineto{\pgfqpoint{-0.048611in}{0.000000in}}%
\pgfusepath{stroke,fill}%
}%
\begin{pgfscope}%
\pgfsys@transformshift{0.530556in}{3.454477in}%
\pgfsys@useobject{currentmarker}{}%
\end{pgfscope}%
\end{pgfscope}%
\begin{pgfscope}%
\definecolor{textcolor}{rgb}{0.000000,0.000000,0.000000}%
\pgfsetstrokecolor{textcolor}%
\pgfsetfillcolor{textcolor}%
\pgftext[x=0.294444in,y=3.406252in,left,base]{\color{textcolor}\rmfamily\fontsize{10.000000}{12.000000}\selectfont \(\displaystyle 50\)}%
\end{pgfscope}%
\begin{pgfscope}%
\pgfsetbuttcap%
\pgfsetroundjoin%
\definecolor{currentfill}{rgb}{0.000000,0.000000,0.000000}%
\pgfsetfillcolor{currentfill}%
\pgfsetlinewidth{0.803000pt}%
\definecolor{currentstroke}{rgb}{0.000000,0.000000,0.000000}%
\pgfsetstrokecolor{currentstroke}%
\pgfsetdash{}{0pt}%
\pgfsys@defobject{currentmarker}{\pgfqpoint{-0.048611in}{0.000000in}}{\pgfqpoint{0.000000in}{0.000000in}}{%
\pgfpathmoveto{\pgfqpoint{0.000000in}{0.000000in}}%
\pgfpathlineto{\pgfqpoint{-0.048611in}{0.000000in}}%
\pgfusepath{stroke,fill}%
}%
\begin{pgfscope}%
\pgfsys@transformshift{0.530556in}{3.920763in}%
\pgfsys@useobject{currentmarker}{}%
\end{pgfscope}%
\end{pgfscope}%
\begin{pgfscope}%
\definecolor{textcolor}{rgb}{0.000000,0.000000,0.000000}%
\pgfsetstrokecolor{textcolor}%
\pgfsetfillcolor{textcolor}%
\pgftext[x=0.294444in,y=3.872538in,left,base]{\color{textcolor}\rmfamily\fontsize{10.000000}{12.000000}\selectfont \(\displaystyle 55\)}%
\end{pgfscope}%
\begin{pgfscope}%
\definecolor{textcolor}{rgb}{0.000000,0.000000,0.000000}%
\pgfsetstrokecolor{textcolor}%
\pgfsetfillcolor{textcolor}%
\pgftext[x=0.238889in,y=2.288763in,,bottom,rotate=90.000000]{\color{textcolor}\rmfamily\fontsize{10.000000}{12.000000}\selectfont Température (\textdegree{}C)}%
\end{pgfscope}%
\begin{pgfscope}%
\pgfpathrectangle{\pgfqpoint{0.530556in}{0.656763in}}{\pgfqpoint{4.960000in}{3.264000in}}%
\pgfusepath{clip}%
\pgfsetrectcap%
\pgfsetroundjoin%
\pgfsetlinewidth{1.505625pt}%
\definecolor{currentstroke}{rgb}{0.121569,0.466667,0.705882}%
\pgfsetstrokecolor{currentstroke}%
\pgfsetdash{}{0pt}%
\pgfpathmoveto{\pgfqpoint{0.538865in}{2.929129in}}%
\pgfpathlineto{\pgfqpoint{0.547142in}{2.997517in}}%
\pgfpathlineto{\pgfqpoint{0.563697in}{3.096991in}}%
\pgfpathlineto{\pgfqpoint{0.571975in}{3.184031in}}%
\pgfpathlineto{\pgfqpoint{0.580253in}{3.165380in}}%
\pgfpathlineto{\pgfqpoint{0.588531in}{3.215117in}}%
\pgfpathlineto{\pgfqpoint{0.596808in}{3.230660in}}%
\pgfpathlineto{\pgfqpoint{0.613364in}{3.302157in}}%
\pgfpathlineto{\pgfqpoint{0.621641in}{3.280397in}}%
\pgfpathlineto{\pgfqpoint{0.629918in}{3.302157in}}%
\pgfpathlineto{\pgfqpoint{0.638196in}{3.317700in}}%
\pgfpathlineto{\pgfqpoint{0.646472in}{3.348786in}}%
\pgfpathlineto{\pgfqpoint{0.654749in}{3.317700in}}%
\pgfpathlineto{\pgfqpoint{0.663024in}{3.367437in}}%
\pgfpathlineto{\pgfqpoint{0.671295in}{3.367437in}}%
\pgfpathlineto{\pgfqpoint{0.679569in}{3.398523in}}%
\pgfpathlineto{\pgfqpoint{0.687846in}{3.417174in}}%
\pgfpathlineto{\pgfqpoint{0.696123in}{3.398523in}}%
\pgfpathlineto{\pgfqpoint{0.704400in}{3.398523in}}%
\pgfpathlineto{\pgfqpoint{0.712677in}{3.432717in}}%
\pgfpathlineto{\pgfqpoint{0.720954in}{3.432717in}}%
\pgfpathlineto{\pgfqpoint{0.729231in}{3.448260in}}%
\pgfpathlineto{\pgfqpoint{0.737509in}{3.451369in}}%
\pgfpathlineto{\pgfqpoint{0.745786in}{3.448260in}}%
\pgfpathlineto{\pgfqpoint{0.754063in}{3.451369in}}%
\pgfpathlineto{\pgfqpoint{0.762340in}{3.448260in}}%
\pgfpathlineto{\pgfqpoint{0.770617in}{3.448260in}}%
\pgfpathlineto{\pgfqpoint{0.778894in}{3.501106in}}%
\pgfpathlineto{\pgfqpoint{0.812002in}{3.501106in}}%
\pgfpathlineto{\pgfqpoint{0.828557in}{3.532191in}}%
\pgfpathlineto{\pgfqpoint{0.836834in}{3.516649in}}%
\pgfpathlineto{\pgfqpoint{0.845110in}{3.532191in}}%
\pgfpathlineto{\pgfqpoint{0.853387in}{3.516649in}}%
\pgfpathlineto{\pgfqpoint{0.861664in}{3.516649in}}%
\pgfpathlineto{\pgfqpoint{0.869942in}{3.563277in}}%
\pgfpathlineto{\pgfqpoint{0.878219in}{3.550843in}}%
\pgfpathlineto{\pgfqpoint{0.886496in}{3.547734in}}%
\pgfpathlineto{\pgfqpoint{0.894773in}{3.547734in}}%
\pgfpathlineto{\pgfqpoint{0.903050in}{3.563277in}}%
\pgfpathlineto{\pgfqpoint{0.911327in}{3.563277in}}%
\pgfpathlineto{\pgfqpoint{0.919604in}{3.581929in}}%
\pgfpathlineto{\pgfqpoint{0.927882in}{3.566386in}}%
\pgfpathlineto{\pgfqpoint{0.936159in}{3.581929in}}%
\pgfpathlineto{\pgfqpoint{0.944435in}{3.563277in}}%
\pgfpathlineto{\pgfqpoint{0.952712in}{3.597471in}}%
\pgfpathlineto{\pgfqpoint{0.960989in}{3.581929in}}%
\pgfpathlineto{\pgfqpoint{0.977543in}{3.581929in}}%
\pgfpathlineto{\pgfqpoint{0.985820in}{3.600580in}}%
\pgfpathlineto{\pgfqpoint{0.994098in}{3.581929in}}%
\pgfpathlineto{\pgfqpoint{1.002375in}{3.585037in}}%
\pgfpathlineto{\pgfqpoint{1.010652in}{3.600580in}}%
\pgfpathlineto{\pgfqpoint{1.018930in}{3.600580in}}%
\pgfpathlineto{\pgfqpoint{1.027203in}{3.581929in}}%
\pgfpathlineto{\pgfqpoint{1.043758in}{3.581929in}}%
\pgfpathlineto{\pgfqpoint{1.052030in}{3.600580in}}%
\pgfpathlineto{\pgfqpoint{1.060306in}{3.581929in}}%
\pgfpathlineto{\pgfqpoint{1.068583in}{3.581929in}}%
\pgfpathlineto{\pgfqpoint{1.076860in}{3.597471in}}%
\pgfpathlineto{\pgfqpoint{1.085137in}{3.600580in}}%
\pgfpathlineto{\pgfqpoint{1.093414in}{3.631666in}}%
\pgfpathlineto{\pgfqpoint{1.101691in}{3.634774in}}%
\pgfpathlineto{\pgfqpoint{1.109968in}{3.616123in}}%
\pgfpathlineto{\pgfqpoint{1.118246in}{3.616123in}}%
\pgfpathlineto{\pgfqpoint{1.126522in}{3.650317in}}%
\pgfpathlineto{\pgfqpoint{1.143076in}{3.650317in}}%
\pgfpathlineto{\pgfqpoint{1.151354in}{3.634774in}}%
\pgfpathlineto{\pgfqpoint{1.159630in}{3.634774in}}%
\pgfpathlineto{\pgfqpoint{1.167901in}{3.650317in}}%
\pgfpathlineto{\pgfqpoint{1.176176in}{3.634774in}}%
\pgfpathlineto{\pgfqpoint{1.184449in}{3.634774in}}%
\pgfpathlineto{\pgfqpoint{1.192725in}{3.650317in}}%
\pgfpathlineto{\pgfqpoint{1.200999in}{3.650317in}}%
\pgfpathlineto{\pgfqpoint{1.209277in}{3.616123in}}%
\pgfpathlineto{\pgfqpoint{1.217553in}{3.616123in}}%
\pgfpathlineto{\pgfqpoint{1.225830in}{3.653426in}}%
\pgfpathlineto{\pgfqpoint{1.234105in}{3.631666in}}%
\pgfpathlineto{\pgfqpoint{1.250660in}{3.631666in}}%
\pgfpathlineto{\pgfqpoint{1.258930in}{3.665860in}}%
\pgfpathlineto{\pgfqpoint{1.267208in}{3.631666in}}%
\pgfpathlineto{\pgfqpoint{1.275485in}{3.631666in}}%
\pgfpathlineto{\pgfqpoint{1.283762in}{3.668969in}}%
\pgfpathlineto{\pgfqpoint{1.292033in}{3.650317in}}%
\pgfpathlineto{\pgfqpoint{1.300310in}{3.665860in}}%
\pgfpathlineto{\pgfqpoint{1.308583in}{3.616123in}}%
\pgfpathlineto{\pgfqpoint{1.316860in}{3.634774in}}%
\pgfpathlineto{\pgfqpoint{1.325135in}{3.631666in}}%
\pgfpathlineto{\pgfqpoint{1.333407in}{3.668969in}}%
\pgfpathlineto{\pgfqpoint{1.341679in}{3.650317in}}%
\pgfpathlineto{\pgfqpoint{1.349957in}{3.650317in}}%
\pgfpathlineto{\pgfqpoint{1.358233in}{3.665860in}}%
\pgfpathlineto{\pgfqpoint{1.366510in}{3.650317in}}%
\pgfpathlineto{\pgfqpoint{1.374785in}{3.665860in}}%
\pgfpathlineto{\pgfqpoint{1.383057in}{3.650317in}}%
\pgfpathlineto{\pgfqpoint{1.391335in}{3.650317in}}%
\pgfpathlineto{\pgfqpoint{1.399612in}{3.665860in}}%
\pgfpathlineto{\pgfqpoint{1.407883in}{3.700054in}}%
\pgfpathlineto{\pgfqpoint{1.416161in}{3.684511in}}%
\pgfpathlineto{\pgfqpoint{1.424434in}{3.665860in}}%
\pgfpathlineto{\pgfqpoint{1.432711in}{3.684511in}}%
\pgfpathlineto{\pgfqpoint{1.440989in}{3.665860in}}%
\pgfpathlineto{\pgfqpoint{1.449267in}{3.684511in}}%
\pgfpathlineto{\pgfqpoint{1.457539in}{3.650317in}}%
\pgfpathlineto{\pgfqpoint{1.465816in}{3.684511in}}%
\pgfpathlineto{\pgfqpoint{1.474093in}{3.700054in}}%
\pgfpathlineto{\pgfqpoint{1.482371in}{3.650317in}}%
\pgfpathlineto{\pgfqpoint{1.490643in}{3.650317in}}%
\pgfpathlineto{\pgfqpoint{1.498915in}{3.718706in}}%
\pgfpathlineto{\pgfqpoint{1.507190in}{3.684511in}}%
\pgfpathlineto{\pgfqpoint{1.523737in}{3.684511in}}%
\pgfpathlineto{\pgfqpoint{1.532013in}{3.700054in}}%
\pgfpathlineto{\pgfqpoint{1.540284in}{3.684511in}}%
\pgfpathlineto{\pgfqpoint{1.548559in}{3.665860in}}%
\pgfpathlineto{\pgfqpoint{1.556835in}{3.700054in}}%
\pgfpathlineto{\pgfqpoint{1.565113in}{3.684511in}}%
\pgfpathlineto{\pgfqpoint{1.573390in}{3.665860in}}%
\pgfpathlineto{\pgfqpoint{1.581668in}{3.684511in}}%
\pgfpathlineto{\pgfqpoint{1.589941in}{3.684511in}}%
\pgfpathlineto{\pgfqpoint{1.598219in}{3.700054in}}%
\pgfpathlineto{\pgfqpoint{1.614767in}{3.700054in}}%
\pgfpathlineto{\pgfqpoint{1.623041in}{3.684511in}}%
\pgfpathlineto{\pgfqpoint{1.639596in}{3.684511in}}%
\pgfpathlineto{\pgfqpoint{1.647872in}{3.665860in}}%
\pgfpathlineto{\pgfqpoint{1.656150in}{3.700054in}}%
\pgfpathlineto{\pgfqpoint{1.664427in}{3.684511in}}%
\pgfpathlineto{\pgfqpoint{1.672704in}{3.700054in}}%
\pgfpathlineto{\pgfqpoint{1.680981in}{3.700054in}}%
\pgfpathlineto{\pgfqpoint{1.689258in}{3.718706in}}%
\pgfpathlineto{\pgfqpoint{1.697535in}{3.718706in}}%
\pgfpathlineto{\pgfqpoint{1.705812in}{3.700054in}}%
\pgfpathlineto{\pgfqpoint{1.714087in}{3.684511in}}%
\pgfpathlineto{\pgfqpoint{1.722357in}{3.684511in}}%
\pgfpathlineto{\pgfqpoint{1.730635in}{3.700054in}}%
\pgfpathlineto{\pgfqpoint{1.747181in}{3.700054in}}%
\pgfpathlineto{\pgfqpoint{1.755458in}{3.718706in}}%
\pgfpathlineto{\pgfqpoint{1.763733in}{3.700054in}}%
\pgfpathlineto{\pgfqpoint{1.772008in}{3.715597in}}%
\pgfpathlineto{\pgfqpoint{1.780283in}{3.700054in}}%
\pgfpathlineto{\pgfqpoint{1.796837in}{3.700054in}}%
\pgfpathlineto{\pgfqpoint{1.805114in}{3.718706in}}%
\pgfpathlineto{\pgfqpoint{1.813391in}{3.700054in}}%
\pgfpathlineto{\pgfqpoint{1.846490in}{3.700054in}}%
\pgfpathlineto{\pgfqpoint{1.854766in}{3.718706in}}%
\pgfpathlineto{\pgfqpoint{1.863043in}{3.718706in}}%
\pgfpathlineto{\pgfqpoint{1.871321in}{3.700054in}}%
\pgfpathlineto{\pgfqpoint{1.879593in}{3.700054in}}%
\pgfpathlineto{\pgfqpoint{1.887864in}{3.731140in}}%
\pgfpathlineto{\pgfqpoint{1.904419in}{3.700054in}}%
\pgfpathlineto{\pgfqpoint{1.912694in}{3.734249in}}%
\pgfpathlineto{\pgfqpoint{1.920972in}{3.700054in}}%
\pgfpathlineto{\pgfqpoint{1.929249in}{3.715597in}}%
\pgfpathlineto{\pgfqpoint{1.937527in}{3.718706in}}%
\pgfpathlineto{\pgfqpoint{1.945803in}{3.700054in}}%
\pgfpathlineto{\pgfqpoint{1.954080in}{3.700054in}}%
\pgfpathlineto{\pgfqpoint{1.962357in}{3.715597in}}%
\pgfpathlineto{\pgfqpoint{1.970634in}{3.684511in}}%
\pgfpathlineto{\pgfqpoint{1.978911in}{3.734249in}}%
\pgfpathlineto{\pgfqpoint{1.987188in}{3.718706in}}%
\pgfpathlineto{\pgfqpoint{2.003743in}{3.749791in}}%
\pgfpathlineto{\pgfqpoint{2.012021in}{3.700054in}}%
\pgfpathlineto{\pgfqpoint{2.020298in}{3.749791in}}%
\pgfpathlineto{\pgfqpoint{2.028576in}{3.734249in}}%
\pgfpathlineto{\pgfqpoint{2.036848in}{3.734249in}}%
\pgfpathlineto{\pgfqpoint{2.045126in}{3.718706in}}%
\pgfpathlineto{\pgfqpoint{2.053403in}{3.734249in}}%
\pgfpathlineto{\pgfqpoint{2.061682in}{3.734249in}}%
\pgfpathlineto{\pgfqpoint{2.069959in}{3.715597in}}%
\pgfpathlineto{\pgfqpoint{2.078233in}{3.734249in}}%
\pgfpathlineto{\pgfqpoint{2.086510in}{3.700054in}}%
\pgfpathlineto{\pgfqpoint{2.094787in}{3.734249in}}%
\pgfpathlineto{\pgfqpoint{2.103065in}{3.734249in}}%
\pgfpathlineto{\pgfqpoint{2.111343in}{3.749791in}}%
\pgfpathlineto{\pgfqpoint{2.119620in}{3.734249in}}%
\pgfpathlineto{\pgfqpoint{2.127891in}{3.749791in}}%
\pgfpathlineto{\pgfqpoint{2.136169in}{3.715597in}}%
\pgfpathlineto{\pgfqpoint{2.144440in}{3.715597in}}%
\pgfpathlineto{\pgfqpoint{2.152716in}{3.734249in}}%
\pgfpathlineto{\pgfqpoint{2.177542in}{3.734249in}}%
\pgfpathlineto{\pgfqpoint{2.185816in}{3.684511in}}%
\pgfpathlineto{\pgfqpoint{2.194089in}{3.715597in}}%
\pgfpathlineto{\pgfqpoint{2.202361in}{3.734249in}}%
\pgfpathlineto{\pgfqpoint{2.210638in}{3.715597in}}%
\pgfpathlineto{\pgfqpoint{2.218915in}{3.734249in}}%
\pgfpathlineto{\pgfqpoint{2.235469in}{3.734249in}}%
\pgfpathlineto{\pgfqpoint{2.243747in}{3.700054in}}%
\pgfpathlineto{\pgfqpoint{2.252023in}{3.734249in}}%
\pgfpathlineto{\pgfqpoint{2.260301in}{3.752900in}}%
\pgfpathlineto{\pgfqpoint{2.268572in}{3.700054in}}%
\pgfpathlineto{\pgfqpoint{2.276850in}{3.749791in}}%
\pgfpathlineto{\pgfqpoint{2.285123in}{3.700054in}}%
\pgfpathlineto{\pgfqpoint{2.301675in}{3.768443in}}%
\pgfpathlineto{\pgfqpoint{2.309953in}{3.734249in}}%
\pgfpathlineto{\pgfqpoint{2.318231in}{3.715597in}}%
\pgfpathlineto{\pgfqpoint{2.326509in}{3.752900in}}%
\pgfpathlineto{\pgfqpoint{2.334787in}{3.718706in}}%
\pgfpathlineto{\pgfqpoint{2.343066in}{3.734249in}}%
\pgfpathlineto{\pgfqpoint{2.351342in}{3.715597in}}%
\pgfpathlineto{\pgfqpoint{2.359618in}{3.752900in}}%
\pgfpathlineto{\pgfqpoint{2.367896in}{3.749791in}}%
\pgfpathlineto{\pgfqpoint{2.376173in}{3.765334in}}%
\pgfpathlineto{\pgfqpoint{2.384451in}{3.731140in}}%
\pgfpathlineto{\pgfqpoint{2.392728in}{3.715597in}}%
\pgfpathlineto{\pgfqpoint{2.400998in}{3.734249in}}%
\pgfpathlineto{\pgfqpoint{2.409275in}{3.715597in}}%
\pgfpathlineto{\pgfqpoint{2.417551in}{3.734249in}}%
\pgfpathlineto{\pgfqpoint{2.425828in}{3.768443in}}%
\pgfpathlineto{\pgfqpoint{2.434100in}{3.715597in}}%
\pgfpathlineto{\pgfqpoint{2.442377in}{3.715597in}}%
\pgfpathlineto{\pgfqpoint{2.450657in}{3.734249in}}%
\pgfpathlineto{\pgfqpoint{2.458934in}{3.734249in}}%
\pgfpathlineto{\pgfqpoint{2.467211in}{3.715597in}}%
\pgfpathlineto{\pgfqpoint{2.475488in}{3.715597in}}%
\pgfpathlineto{\pgfqpoint{2.483765in}{3.734249in}}%
\pgfpathlineto{\pgfqpoint{2.492042in}{3.715597in}}%
\pgfpathlineto{\pgfqpoint{2.508597in}{3.752900in}}%
\pgfpathlineto{\pgfqpoint{2.516874in}{3.734249in}}%
\pgfpathlineto{\pgfqpoint{2.525148in}{3.749791in}}%
\pgfpathlineto{\pgfqpoint{2.533426in}{3.734249in}}%
\pgfpathlineto{\pgfqpoint{2.541704in}{3.715597in}}%
\pgfpathlineto{\pgfqpoint{2.549981in}{3.718706in}}%
\pgfpathlineto{\pgfqpoint{2.558258in}{3.734249in}}%
\pgfpathlineto{\pgfqpoint{2.574813in}{3.734249in}}%
\pgfpathlineto{\pgfqpoint{2.583090in}{3.715597in}}%
\pgfpathlineto{\pgfqpoint{2.591367in}{3.734249in}}%
\pgfpathlineto{\pgfqpoint{2.599644in}{3.768443in}}%
\pgfpathlineto{\pgfqpoint{2.607920in}{3.752900in}}%
\pgfpathlineto{\pgfqpoint{2.616198in}{3.734249in}}%
\pgfpathlineto{\pgfqpoint{2.632752in}{3.734249in}}%
\pgfpathlineto{\pgfqpoint{2.641029in}{3.752900in}}%
\pgfpathlineto{\pgfqpoint{2.649306in}{3.715597in}}%
\pgfpathlineto{\pgfqpoint{2.657583in}{3.715597in}}%
\pgfpathlineto{\pgfqpoint{2.674138in}{3.752900in}}%
\pgfpathlineto{\pgfqpoint{2.682416in}{3.700054in}}%
\pgfpathlineto{\pgfqpoint{2.690694in}{3.715597in}}%
\pgfpathlineto{\pgfqpoint{2.698971in}{3.768443in}}%
\pgfpathlineto{\pgfqpoint{2.707243in}{3.715597in}}%
\pgfpathlineto{\pgfqpoint{2.715520in}{3.731140in}}%
\pgfpathlineto{\pgfqpoint{2.723797in}{3.734249in}}%
\pgfpathlineto{\pgfqpoint{2.732074in}{3.715597in}}%
\pgfpathlineto{\pgfqpoint{2.740349in}{3.768443in}}%
\pgfpathlineto{\pgfqpoint{2.748621in}{3.734249in}}%
\pgfpathlineto{\pgfqpoint{2.756893in}{3.715597in}}%
\pgfpathlineto{\pgfqpoint{2.765166in}{3.768443in}}%
\pgfpathlineto{\pgfqpoint{2.773438in}{3.734249in}}%
\pgfpathlineto{\pgfqpoint{2.781711in}{3.715597in}}%
\pgfpathlineto{\pgfqpoint{2.798260in}{3.715597in}}%
\pgfpathlineto{\pgfqpoint{2.806538in}{3.752900in}}%
\pgfpathlineto{\pgfqpoint{2.814812in}{3.768443in}}%
\pgfpathlineto{\pgfqpoint{2.823085in}{3.752900in}}%
\pgfpathlineto{\pgfqpoint{2.831357in}{3.749791in}}%
\pgfpathlineto{\pgfqpoint{2.839629in}{3.752900in}}%
\pgfpathlineto{\pgfqpoint{2.847906in}{3.718706in}}%
\pgfpathlineto{\pgfqpoint{2.856180in}{3.768443in}}%
\pgfpathlineto{\pgfqpoint{2.864452in}{3.752900in}}%
\pgfpathlineto{\pgfqpoint{2.872724in}{3.731140in}}%
\pgfpathlineto{\pgfqpoint{2.880996in}{3.700054in}}%
\pgfpathlineto{\pgfqpoint{2.889269in}{3.768443in}}%
\pgfpathlineto{\pgfqpoint{2.897541in}{3.765334in}}%
\pgfpathlineto{\pgfqpoint{2.905813in}{3.734249in}}%
\pgfpathlineto{\pgfqpoint{2.914087in}{3.749791in}}%
\pgfpathlineto{\pgfqpoint{2.922364in}{3.731140in}}%
\pgfpathlineto{\pgfqpoint{2.930642in}{3.752900in}}%
\pgfpathlineto{\pgfqpoint{2.938920in}{3.749791in}}%
\pgfpathlineto{\pgfqpoint{2.947195in}{3.734249in}}%
\pgfpathlineto{\pgfqpoint{2.955465in}{3.731140in}}%
\pgfpathlineto{\pgfqpoint{2.963741in}{3.715597in}}%
\pgfpathlineto{\pgfqpoint{2.972018in}{3.749791in}}%
\pgfpathlineto{\pgfqpoint{2.980295in}{3.752900in}}%
\pgfpathlineto{\pgfqpoint{2.988571in}{3.749791in}}%
\pgfpathlineto{\pgfqpoint{2.996844in}{3.752900in}}%
\pgfpathlineto{\pgfqpoint{3.005119in}{3.749791in}}%
\pgfpathlineto{\pgfqpoint{3.013395in}{3.734249in}}%
\pgfpathlineto{\pgfqpoint{3.021672in}{3.734249in}}%
\pgfpathlineto{\pgfqpoint{3.029949in}{3.731140in}}%
\pgfpathlineto{\pgfqpoint{3.038227in}{3.731140in}}%
\pgfpathlineto{\pgfqpoint{3.046504in}{3.734249in}}%
\pgfpathlineto{\pgfqpoint{3.054774in}{3.749791in}}%
\pgfpathlineto{\pgfqpoint{3.063052in}{3.749791in}}%
\pgfpathlineto{\pgfqpoint{3.071323in}{3.734249in}}%
\pgfpathlineto{\pgfqpoint{3.079598in}{3.734249in}}%
\pgfpathlineto{\pgfqpoint{3.087875in}{3.731140in}}%
\pgfpathlineto{\pgfqpoint{3.096148in}{3.734249in}}%
\pgfpathlineto{\pgfqpoint{3.104423in}{3.749791in}}%
\pgfpathlineto{\pgfqpoint{3.112700in}{3.752900in}}%
\pgfpathlineto{\pgfqpoint{3.120974in}{3.734249in}}%
\pgfpathlineto{\pgfqpoint{3.129248in}{3.734249in}}%
\pgfpathlineto{\pgfqpoint{3.137526in}{3.700054in}}%
\pgfpathlineto{\pgfqpoint{3.145803in}{3.734249in}}%
\pgfpathlineto{\pgfqpoint{3.154080in}{3.715597in}}%
\pgfpathlineto{\pgfqpoint{3.162357in}{3.734249in}}%
\pgfpathlineto{\pgfqpoint{3.170634in}{3.715597in}}%
\pgfpathlineto{\pgfqpoint{3.178912in}{3.715597in}}%
\pgfpathlineto{\pgfqpoint{3.187189in}{3.734249in}}%
\pgfpathlineto{\pgfqpoint{3.212015in}{3.734249in}}%
\pgfpathlineto{\pgfqpoint{3.220292in}{3.752900in}}%
\pgfpathlineto{\pgfqpoint{3.228561in}{3.765334in}}%
\pgfpathlineto{\pgfqpoint{3.236838in}{3.734249in}}%
\pgfpathlineto{\pgfqpoint{3.261666in}{3.734249in}}%
\pgfpathlineto{\pgfqpoint{3.269943in}{3.752900in}}%
\pgfpathlineto{\pgfqpoint{3.278220in}{3.715597in}}%
\pgfpathlineto{\pgfqpoint{3.286497in}{3.718706in}}%
\pgfpathlineto{\pgfqpoint{3.294770in}{3.749791in}}%
\pgfpathlineto{\pgfqpoint{3.303047in}{3.749791in}}%
\pgfpathlineto{\pgfqpoint{3.311324in}{3.734249in}}%
\pgfpathlineto{\pgfqpoint{3.319601in}{3.768443in}}%
\pgfpathlineto{\pgfqpoint{3.327878in}{3.734249in}}%
\pgfpathlineto{\pgfqpoint{3.336153in}{3.734249in}}%
\pgfpathlineto{\pgfqpoint{3.344428in}{3.768443in}}%
\pgfpathlineto{\pgfqpoint{3.352700in}{3.749791in}}%
\pgfpathlineto{\pgfqpoint{3.360975in}{3.765334in}}%
\pgfpathlineto{\pgfqpoint{3.369247in}{3.749791in}}%
\pgfpathlineto{\pgfqpoint{3.385797in}{3.749791in}}%
\pgfpathlineto{\pgfqpoint{3.394068in}{3.700054in}}%
\pgfpathlineto{\pgfqpoint{3.410624in}{3.768443in}}%
\pgfpathlineto{\pgfqpoint{3.418901in}{3.749791in}}%
\pgfpathlineto{\pgfqpoint{3.427176in}{3.768443in}}%
\pgfpathlineto{\pgfqpoint{3.435448in}{3.783986in}}%
\pgfpathlineto{\pgfqpoint{3.443725in}{3.765334in}}%
\pgfpathlineto{\pgfqpoint{3.452002in}{3.749791in}}%
\pgfpathlineto{\pgfqpoint{3.460279in}{3.768443in}}%
\pgfpathlineto{\pgfqpoint{3.468557in}{3.749791in}}%
\pgfpathlineto{\pgfqpoint{3.476829in}{3.783986in}}%
\pgfpathlineto{\pgfqpoint{3.485107in}{3.768443in}}%
\pgfpathlineto{\pgfqpoint{3.493385in}{3.783986in}}%
\pgfpathlineto{\pgfqpoint{3.509932in}{3.752900in}}%
\pgfpathlineto{\pgfqpoint{3.518208in}{3.752900in}}%
\pgfpathlineto{\pgfqpoint{3.526485in}{3.731140in}}%
\pgfpathlineto{\pgfqpoint{3.534757in}{3.749791in}}%
\pgfpathlineto{\pgfqpoint{3.543035in}{3.749791in}}%
\pgfpathlineto{\pgfqpoint{3.551312in}{3.765334in}}%
\pgfpathlineto{\pgfqpoint{3.559590in}{3.715597in}}%
\pgfpathlineto{\pgfqpoint{3.567867in}{3.783986in}}%
\pgfpathlineto{\pgfqpoint{3.576144in}{3.783986in}}%
\pgfpathlineto{\pgfqpoint{3.584421in}{3.787094in}}%
\pgfpathlineto{\pgfqpoint{3.592699in}{3.765334in}}%
\pgfpathlineto{\pgfqpoint{3.600976in}{3.749791in}}%
\pgfpathlineto{\pgfqpoint{3.609253in}{3.749791in}}%
\pgfpathlineto{\pgfqpoint{3.617530in}{3.768443in}}%
\pgfpathlineto{\pgfqpoint{3.634078in}{3.768443in}}%
\pgfpathlineto{\pgfqpoint{3.642355in}{3.783986in}}%
\pgfpathlineto{\pgfqpoint{3.650625in}{3.768443in}}%
\pgfpathlineto{\pgfqpoint{3.658901in}{3.802637in}}%
\pgfpathlineto{\pgfqpoint{3.667177in}{3.787094in}}%
\pgfpathlineto{\pgfqpoint{3.675455in}{3.768443in}}%
\pgfpathlineto{\pgfqpoint{3.683729in}{3.765334in}}%
\pgfpathlineto{\pgfqpoint{3.692003in}{3.768443in}}%
\pgfpathlineto{\pgfqpoint{3.700275in}{3.802637in}}%
\pgfpathlineto{\pgfqpoint{3.716829in}{3.802637in}}%
\pgfpathlineto{\pgfqpoint{3.725107in}{3.783986in}}%
\pgfpathlineto{\pgfqpoint{3.733384in}{3.749791in}}%
\pgfpathlineto{\pgfqpoint{3.741661in}{3.802637in}}%
\pgfpathlineto{\pgfqpoint{3.749938in}{3.765334in}}%
\pgfpathlineto{\pgfqpoint{3.758215in}{3.768443in}}%
\pgfpathlineto{\pgfqpoint{3.766487in}{3.765334in}}%
\pgfpathlineto{\pgfqpoint{3.783042in}{3.802637in}}%
\pgfpathlineto{\pgfqpoint{3.791312in}{3.802637in}}%
\pgfpathlineto{\pgfqpoint{3.799592in}{3.783986in}}%
\pgfpathlineto{\pgfqpoint{3.807870in}{3.768443in}}%
\pgfpathlineto{\pgfqpoint{3.816147in}{3.802637in}}%
\pgfpathlineto{\pgfqpoint{3.824421in}{3.802637in}}%
\pgfpathlineto{\pgfqpoint{3.832695in}{3.765334in}}%
\pgfpathlineto{\pgfqpoint{3.849246in}{3.802637in}}%
\pgfpathlineto{\pgfqpoint{3.874072in}{3.802637in}}%
\pgfpathlineto{\pgfqpoint{3.882349in}{3.749791in}}%
\pgfpathlineto{\pgfqpoint{3.890627in}{3.799529in}}%
\pgfpathlineto{\pgfqpoint{3.898905in}{3.799529in}}%
\pgfpathlineto{\pgfqpoint{3.907175in}{3.783986in}}%
\pgfpathlineto{\pgfqpoint{3.915452in}{3.783986in}}%
\pgfpathlineto{\pgfqpoint{3.923730in}{3.802637in}}%
\pgfpathlineto{\pgfqpoint{3.932000in}{3.783986in}}%
\pgfpathlineto{\pgfqpoint{3.948552in}{3.783986in}}%
\pgfpathlineto{\pgfqpoint{3.956829in}{3.802637in}}%
\pgfpathlineto{\pgfqpoint{3.965106in}{3.799529in}}%
\pgfpathlineto{\pgfqpoint{3.973383in}{3.802637in}}%
\pgfpathlineto{\pgfqpoint{3.981655in}{3.799529in}}%
\pgfpathlineto{\pgfqpoint{3.989927in}{3.783986in}}%
\pgfpathlineto{\pgfqpoint{3.998201in}{3.802637in}}%
\pgfpathlineto{\pgfqpoint{4.006473in}{3.818180in}}%
\pgfpathlineto{\pgfqpoint{4.014749in}{3.799529in}}%
\pgfpathlineto{\pgfqpoint{4.023026in}{3.802637in}}%
\pgfpathlineto{\pgfqpoint{4.031303in}{3.783986in}}%
\pgfpathlineto{\pgfqpoint{4.039580in}{3.780877in}}%
\pgfpathlineto{\pgfqpoint{4.047860in}{3.818180in}}%
\pgfpathlineto{\pgfqpoint{4.056138in}{3.818180in}}%
\pgfpathlineto{\pgfqpoint{4.064415in}{3.799529in}}%
\pgfpathlineto{\pgfqpoint{4.089246in}{3.799529in}}%
\pgfpathlineto{\pgfqpoint{4.097516in}{3.783986in}}%
\pgfpathlineto{\pgfqpoint{4.105793in}{3.818180in}}%
\pgfpathlineto{\pgfqpoint{4.114069in}{3.799529in}}%
\pgfpathlineto{\pgfqpoint{4.122341in}{3.818180in}}%
\pgfpathlineto{\pgfqpoint{4.130618in}{3.783986in}}%
\pgfpathlineto{\pgfqpoint{4.138893in}{3.799529in}}%
\pgfpathlineto{\pgfqpoint{4.147167in}{3.802637in}}%
\pgfpathlineto{\pgfqpoint{4.155443in}{3.818180in}}%
\pgfpathlineto{\pgfqpoint{4.163715in}{3.836831in}}%
\pgfpathlineto{\pgfqpoint{4.171991in}{3.818180in}}%
\pgfpathlineto{\pgfqpoint{4.180261in}{3.818180in}}%
\pgfpathlineto{\pgfqpoint{4.188538in}{3.836831in}}%
\pgfpathlineto{\pgfqpoint{4.196817in}{3.836831in}}%
\pgfpathlineto{\pgfqpoint{4.205089in}{3.799529in}}%
\pgfpathlineto{\pgfqpoint{4.213366in}{3.802637in}}%
\pgfpathlineto{\pgfqpoint{4.221638in}{3.799529in}}%
\pgfpathlineto{\pgfqpoint{4.229916in}{3.818180in}}%
\pgfpathlineto{\pgfqpoint{4.238187in}{3.818180in}}%
\pgfpathlineto{\pgfqpoint{4.246462in}{3.799529in}}%
\pgfpathlineto{\pgfqpoint{4.254734in}{3.802637in}}%
\pgfpathlineto{\pgfqpoint{4.263006in}{3.818180in}}%
\pgfpathlineto{\pgfqpoint{4.271279in}{3.836831in}}%
\pgfpathlineto{\pgfqpoint{4.279551in}{3.799529in}}%
\pgfpathlineto{\pgfqpoint{4.287826in}{3.818180in}}%
\pgfpathlineto{\pgfqpoint{4.296105in}{3.783986in}}%
\pgfpathlineto{\pgfqpoint{4.304382in}{3.836831in}}%
\pgfpathlineto{\pgfqpoint{4.312660in}{3.799529in}}%
\pgfpathlineto{\pgfqpoint{4.320938in}{3.799529in}}%
\pgfpathlineto{\pgfqpoint{4.337482in}{3.836831in}}%
\pgfpathlineto{\pgfqpoint{4.354037in}{3.836831in}}%
\pgfpathlineto{\pgfqpoint{4.362314in}{3.818180in}}%
\pgfpathlineto{\pgfqpoint{4.387147in}{3.818180in}}%
\pgfpathlineto{\pgfqpoint{4.395422in}{3.799529in}}%
\pgfpathlineto{\pgfqpoint{4.403695in}{3.818180in}}%
\pgfpathlineto{\pgfqpoint{4.411968in}{3.818180in}}%
\pgfpathlineto{\pgfqpoint{4.420241in}{3.852374in}}%
\pgfpathlineto{\pgfqpoint{4.428517in}{3.818180in}}%
\pgfpathlineto{\pgfqpoint{4.436794in}{3.836831in}}%
\pgfpathlineto{\pgfqpoint{4.445071in}{3.833723in}}%
\pgfpathlineto{\pgfqpoint{4.453348in}{3.836831in}}%
\pgfpathlineto{\pgfqpoint{4.478180in}{3.836831in}}%
\pgfpathlineto{\pgfqpoint{4.486453in}{3.818180in}}%
\pgfpathlineto{\pgfqpoint{4.511276in}{3.818180in}}%
\pgfpathlineto{\pgfqpoint{4.519550in}{3.780877in}}%
\pgfpathlineto{\pgfqpoint{4.544382in}{3.836831in}}%
\pgfpathlineto{\pgfqpoint{4.552659in}{3.799529in}}%
\pgfpathlineto{\pgfqpoint{4.560937in}{3.818180in}}%
\pgfpathlineto{\pgfqpoint{4.569214in}{3.818180in}}%
\pgfpathlineto{\pgfqpoint{4.577492in}{3.836831in}}%
\pgfpathlineto{\pgfqpoint{4.585770in}{3.852374in}}%
\pgfpathlineto{\pgfqpoint{4.594047in}{3.836831in}}%
\pgfpathlineto{\pgfqpoint{4.602323in}{3.818180in}}%
\pgfpathlineto{\pgfqpoint{4.610601in}{3.833723in}}%
\pgfpathlineto{\pgfqpoint{4.618872in}{3.836831in}}%
\pgfpathlineto{\pgfqpoint{4.627147in}{3.836831in}}%
\pgfpathlineto{\pgfqpoint{4.643694in}{3.799529in}}%
\pgfpathlineto{\pgfqpoint{4.651968in}{3.852374in}}%
\pgfpathlineto{\pgfqpoint{4.660245in}{3.818180in}}%
\pgfpathlineto{\pgfqpoint{4.668516in}{3.836831in}}%
\pgfpathlineto{\pgfqpoint{4.676794in}{3.836831in}}%
\pgfpathlineto{\pgfqpoint{4.685066in}{3.818180in}}%
\pgfpathlineto{\pgfqpoint{4.693344in}{3.836831in}}%
\pgfpathlineto{\pgfqpoint{4.718165in}{3.836831in}}%
\pgfpathlineto{\pgfqpoint{4.726443in}{3.818180in}}%
\pgfpathlineto{\pgfqpoint{4.734721in}{3.836831in}}%
\pgfpathlineto{\pgfqpoint{4.742993in}{3.836831in}}%
\pgfpathlineto{\pgfqpoint{4.751265in}{3.833723in}}%
\pgfpathlineto{\pgfqpoint{4.759541in}{3.836831in}}%
\pgfpathlineto{\pgfqpoint{4.776084in}{3.836831in}}%
\pgfpathlineto{\pgfqpoint{4.784362in}{3.852374in}}%
\pgfpathlineto{\pgfqpoint{4.792638in}{3.818180in}}%
\pgfpathlineto{\pgfqpoint{4.800911in}{3.799529in}}%
\pgfpathlineto{\pgfqpoint{4.809188in}{3.799529in}}%
\pgfpathlineto{\pgfqpoint{4.817460in}{3.836831in}}%
\pgfpathlineto{\pgfqpoint{4.825732in}{3.836831in}}%
\pgfpathlineto{\pgfqpoint{4.834005in}{3.818180in}}%
\pgfpathlineto{\pgfqpoint{4.842278in}{3.836831in}}%
\pgfpathlineto{\pgfqpoint{4.850550in}{3.818180in}}%
\pgfpathlineto{\pgfqpoint{4.858822in}{3.836831in}}%
\pgfpathlineto{\pgfqpoint{4.867094in}{3.818180in}}%
\pgfpathlineto{\pgfqpoint{4.875366in}{3.836831in}}%
\pgfpathlineto{\pgfqpoint{4.883642in}{3.818180in}}%
\pgfpathlineto{\pgfqpoint{4.891915in}{3.818180in}}%
\pgfpathlineto{\pgfqpoint{4.900193in}{3.799529in}}%
\pgfpathlineto{\pgfqpoint{4.908470in}{3.836831in}}%
\pgfpathlineto{\pgfqpoint{4.916748in}{3.818180in}}%
\pgfpathlineto{\pgfqpoint{4.925025in}{3.836831in}}%
\pgfpathlineto{\pgfqpoint{4.933299in}{3.836831in}}%
\pgfpathlineto{\pgfqpoint{4.941576in}{3.818180in}}%
\pgfpathlineto{\pgfqpoint{4.949854in}{3.836831in}}%
\pgfpathlineto{\pgfqpoint{4.958125in}{3.818180in}}%
\pgfpathlineto{\pgfqpoint{4.966398in}{3.783986in}}%
\pgfpathlineto{\pgfqpoint{4.974674in}{3.836831in}}%
\pgfpathlineto{\pgfqpoint{4.982944in}{3.852374in}}%
\pgfpathlineto{\pgfqpoint{4.991217in}{3.836831in}}%
\pgfpathlineto{\pgfqpoint{4.999494in}{3.818180in}}%
\pgfpathlineto{\pgfqpoint{5.007767in}{3.836831in}}%
\pgfpathlineto{\pgfqpoint{5.016045in}{3.852374in}}%
\pgfpathlineto{\pgfqpoint{5.024322in}{3.852374in}}%
\pgfpathlineto{\pgfqpoint{5.032599in}{3.818180in}}%
\pgfpathlineto{\pgfqpoint{5.040873in}{3.852374in}}%
\pgfpathlineto{\pgfqpoint{5.049150in}{3.833723in}}%
\pgfpathlineto{\pgfqpoint{5.057428in}{3.836831in}}%
\pgfpathlineto{\pgfqpoint{5.065704in}{3.818180in}}%
\pgfpathlineto{\pgfqpoint{5.073979in}{3.836831in}}%
\pgfpathlineto{\pgfqpoint{5.082252in}{3.836831in}}%
\pgfpathlineto{\pgfqpoint{5.090528in}{3.780877in}}%
\pgfpathlineto{\pgfqpoint{5.098805in}{3.818180in}}%
\pgfpathlineto{\pgfqpoint{5.107081in}{3.818180in}}%
\pgfpathlineto{\pgfqpoint{5.115351in}{3.836831in}}%
\pgfpathlineto{\pgfqpoint{5.123627in}{3.833723in}}%
\pgfpathlineto{\pgfqpoint{5.131904in}{3.836831in}}%
\pgfpathlineto{\pgfqpoint{5.140177in}{3.818180in}}%
\pgfpathlineto{\pgfqpoint{5.148452in}{3.836831in}}%
\pgfpathlineto{\pgfqpoint{5.156721in}{3.852374in}}%
\pgfpathlineto{\pgfqpoint{5.164999in}{3.818180in}}%
\pgfpathlineto{\pgfqpoint{5.173273in}{3.833723in}}%
\pgfpathlineto{\pgfqpoint{5.181544in}{3.836831in}}%
\pgfpathlineto{\pgfqpoint{5.189818in}{3.818180in}}%
\pgfpathlineto{\pgfqpoint{5.198096in}{3.818180in}}%
\pgfpathlineto{\pgfqpoint{5.206373in}{3.833723in}}%
\pgfpathlineto{\pgfqpoint{5.214651in}{3.818180in}}%
\pgfpathlineto{\pgfqpoint{5.222929in}{3.836831in}}%
\pgfpathlineto{\pgfqpoint{5.231200in}{3.818180in}}%
\pgfpathlineto{\pgfqpoint{5.239477in}{3.836831in}}%
\pgfpathlineto{\pgfqpoint{5.247754in}{3.333243in}}%
\pgfpathlineto{\pgfqpoint{5.256025in}{3.184031in}}%
\pgfpathlineto{\pgfqpoint{5.272583in}{3.065906in}}%
\pgfpathlineto{\pgfqpoint{5.289141in}{3.034820in}}%
\pgfpathlineto{\pgfqpoint{5.297420in}{2.978866in}}%
\pgfpathlineto{\pgfqpoint{5.305698in}{2.978866in}}%
\pgfpathlineto{\pgfqpoint{5.313972in}{2.932237in}}%
\pgfpathlineto{\pgfqpoint{5.347087in}{2.932237in}}%
\pgfpathlineto{\pgfqpoint{5.355366in}{2.898043in}}%
\pgfpathlineto{\pgfqpoint{5.363645in}{2.898043in}}%
\pgfpathlineto{\pgfqpoint{5.380203in}{2.866957in}}%
\pgfpathlineto{\pgfqpoint{5.388481in}{2.866957in}}%
\pgfpathlineto{\pgfqpoint{5.396756in}{2.848306in}}%
\pgfpathlineto{\pgfqpoint{5.405035in}{2.866957in}}%
\pgfpathlineto{\pgfqpoint{5.413314in}{2.848306in}}%
\pgfpathlineto{\pgfqpoint{5.421590in}{2.814111in}}%
\pgfpathlineto{\pgfqpoint{5.429866in}{2.829654in}}%
\pgfpathlineto{\pgfqpoint{5.438143in}{2.814111in}}%
\pgfpathlineto{\pgfqpoint{5.446419in}{2.814111in}}%
\pgfpathlineto{\pgfqpoint{5.462974in}{2.783026in}}%
\pgfpathlineto{\pgfqpoint{5.471248in}{2.814111in}}%
\pgfpathlineto{\pgfqpoint{5.479524in}{2.814111in}}%
\pgfpathlineto{\pgfqpoint{5.487801in}{2.783026in}}%
\pgfpathlineto{\pgfqpoint{5.487801in}{2.783026in}}%
\pgfusepath{stroke}%
\end{pgfscope}%
\begin{pgfscope}%
\pgfpathrectangle{\pgfqpoint{0.530556in}{0.656763in}}{\pgfqpoint{4.960000in}{3.264000in}}%
\pgfusepath{clip}%
\pgfsetrectcap%
\pgfsetroundjoin%
\pgfsetlinewidth{1.505625pt}%
\definecolor{currentstroke}{rgb}{1.000000,0.498039,0.054902}%
\pgfsetstrokecolor{currentstroke}%
\pgfsetdash{}{0pt}%
\pgfpathmoveto{\pgfqpoint{0.520556in}{0.939754in}}%
\pgfpathlineto{\pgfqpoint{0.523563in}{0.940420in}}%
\pgfpathlineto{\pgfqpoint{0.541109in}{0.940420in}}%
\pgfpathlineto{\pgfqpoint{0.549883in}{0.946249in}}%
\pgfpathlineto{\pgfqpoint{0.558656in}{0.946249in}}%
\pgfpathlineto{\pgfqpoint{0.567429in}{0.950134in}}%
\pgfpathlineto{\pgfqpoint{0.576202in}{0.950134in}}%
\pgfpathlineto{\pgfqpoint{0.584975in}{0.952077in}}%
\pgfpathlineto{\pgfqpoint{0.593748in}{0.955963in}}%
\pgfpathlineto{\pgfqpoint{0.620067in}{0.961791in}}%
\pgfpathlineto{\pgfqpoint{0.628840in}{0.965677in}}%
\pgfpathlineto{\pgfqpoint{0.637613in}{0.965677in}}%
\pgfpathlineto{\pgfqpoint{0.655159in}{0.969563in}}%
\pgfpathlineto{\pgfqpoint{0.663932in}{0.973449in}}%
\pgfpathlineto{\pgfqpoint{0.672705in}{0.973449in}}%
\pgfpathlineto{\pgfqpoint{0.716570in}{0.983163in}}%
\pgfpathlineto{\pgfqpoint{0.725343in}{0.987049in}}%
\pgfpathlineto{\pgfqpoint{0.742889in}{0.987049in}}%
\pgfpathlineto{\pgfqpoint{0.751662in}{0.992877in}}%
\pgfpathlineto{\pgfqpoint{0.760435in}{0.992877in}}%
\pgfpathlineto{\pgfqpoint{0.777981in}{0.996763in}}%
\pgfpathlineto{\pgfqpoint{0.795527in}{0.996763in}}%
\pgfpathlineto{\pgfqpoint{0.804300in}{1.000649in}}%
\pgfpathlineto{\pgfqpoint{0.874484in}{1.016191in}}%
\pgfpathlineto{\pgfqpoint{0.883257in}{1.016191in}}%
\pgfpathlineto{\pgfqpoint{0.892031in}{1.020077in}}%
\pgfpathlineto{\pgfqpoint{0.909577in}{1.020077in}}%
\pgfpathlineto{\pgfqpoint{0.918350in}{1.022020in}}%
\pgfpathlineto{\pgfqpoint{0.935896in}{1.022020in}}%
\pgfpathlineto{\pgfqpoint{0.944669in}{1.025906in}}%
\pgfpathlineto{\pgfqpoint{0.953442in}{1.025906in}}%
\pgfpathlineto{\pgfqpoint{0.962215in}{1.027849in}}%
\pgfpathlineto{\pgfqpoint{0.970983in}{1.027849in}}%
\pgfpathlineto{\pgfqpoint{0.979739in}{1.029791in}}%
\pgfpathlineto{\pgfqpoint{0.997285in}{1.029791in}}%
\pgfpathlineto{\pgfqpoint{1.006058in}{1.033677in}}%
\pgfpathlineto{\pgfqpoint{1.014831in}{1.035620in}}%
\pgfpathlineto{\pgfqpoint{1.032377in}{1.035620in}}%
\pgfpathlineto{\pgfqpoint{1.041150in}{1.037563in}}%
\pgfpathlineto{\pgfqpoint{1.049923in}{1.035620in}}%
\pgfpathlineto{\pgfqpoint{1.058696in}{1.037563in}}%
\pgfpathlineto{\pgfqpoint{1.067469in}{1.037563in}}%
\pgfpathlineto{\pgfqpoint{1.076242in}{1.039506in}}%
\pgfpathlineto{\pgfqpoint{1.085015in}{1.039506in}}%
\pgfpathlineto{\pgfqpoint{1.093788in}{1.041449in}}%
\pgfpathlineto{\pgfqpoint{1.111334in}{1.041449in}}%
\pgfpathlineto{\pgfqpoint{1.120107in}{1.043391in}}%
\pgfpathlineto{\pgfqpoint{1.128880in}{1.043391in}}%
\pgfpathlineto{\pgfqpoint{1.137653in}{1.045334in}}%
\pgfpathlineto{\pgfqpoint{1.216610in}{1.045334in}}%
\pgfpathlineto{\pgfqpoint{1.225383in}{1.047277in}}%
\pgfpathlineto{\pgfqpoint{1.242929in}{1.047277in}}%
\pgfpathlineto{\pgfqpoint{1.251702in}{1.049220in}}%
\pgfpathlineto{\pgfqpoint{1.260476in}{1.049220in}}%
\pgfpathlineto{\pgfqpoint{1.269249in}{1.051163in}}%
\pgfpathlineto{\pgfqpoint{1.286795in}{1.051163in}}%
\pgfpathlineto{\pgfqpoint{1.295590in}{1.053106in}}%
\pgfpathlineto{\pgfqpoint{1.321909in}{1.053106in}}%
\pgfpathlineto{\pgfqpoint{1.339455in}{1.056991in}}%
\pgfpathlineto{\pgfqpoint{1.462277in}{1.056991in}}%
\pgfpathlineto{\pgfqpoint{1.471050in}{1.058934in}}%
\pgfpathlineto{\pgfqpoint{1.523688in}{1.058934in}}%
\pgfpathlineto{\pgfqpoint{1.532461in}{1.060877in}}%
\pgfpathlineto{\pgfqpoint{1.567553in}{1.060877in}}%
\pgfpathlineto{\pgfqpoint{1.576326in}{1.062820in}}%
\pgfpathlineto{\pgfqpoint{1.602646in}{1.062820in}}%
\pgfpathlineto{\pgfqpoint{1.611419in}{1.064763in}}%
\pgfpathlineto{\pgfqpoint{1.628965in}{1.064763in}}%
\pgfpathlineto{\pgfqpoint{1.637738in}{1.066706in}}%
\pgfpathlineto{\pgfqpoint{1.655284in}{1.066706in}}%
\pgfpathlineto{\pgfqpoint{1.664057in}{1.070591in}}%
\pgfpathlineto{\pgfqpoint{1.672830in}{1.068649in}}%
\pgfpathlineto{\pgfqpoint{1.681603in}{1.070591in}}%
\pgfpathlineto{\pgfqpoint{1.690376in}{1.070591in}}%
\pgfpathlineto{\pgfqpoint{1.699149in}{1.072534in}}%
\pgfpathlineto{\pgfqpoint{1.707922in}{1.070591in}}%
\pgfpathlineto{\pgfqpoint{1.725468in}{1.074477in}}%
\pgfpathlineto{\pgfqpoint{1.743014in}{1.074477in}}%
\pgfpathlineto{\pgfqpoint{1.751787in}{1.076420in}}%
\pgfpathlineto{\pgfqpoint{1.778106in}{1.076420in}}%
\pgfpathlineto{\pgfqpoint{1.786879in}{1.078363in}}%
\pgfpathlineto{\pgfqpoint{1.813198in}{1.078363in}}%
\pgfpathlineto{\pgfqpoint{1.821971in}{1.080306in}}%
\pgfpathlineto{\pgfqpoint{1.857063in}{1.080306in}}%
\pgfpathlineto{\pgfqpoint{1.865836in}{1.082249in}}%
\pgfpathlineto{\pgfqpoint{1.883382in}{1.082249in}}%
\pgfpathlineto{\pgfqpoint{1.892155in}{1.084191in}}%
\pgfpathlineto{\pgfqpoint{1.944794in}{1.084191in}}%
\pgfpathlineto{\pgfqpoint{1.953567in}{1.086134in}}%
\pgfpathlineto{\pgfqpoint{1.971113in}{1.086134in}}%
\pgfpathlineto{\pgfqpoint{1.979886in}{1.088077in}}%
\pgfpathlineto{\pgfqpoint{2.032524in}{1.088077in}}%
\pgfpathlineto{\pgfqpoint{2.041297in}{1.090020in}}%
\pgfpathlineto{\pgfqpoint{2.050070in}{1.090020in}}%
\pgfpathlineto{\pgfqpoint{2.058843in}{1.088077in}}%
\pgfpathlineto{\pgfqpoint{2.067616in}{1.088077in}}%
\pgfpathlineto{\pgfqpoint{2.076389in}{1.090020in}}%
\pgfpathlineto{\pgfqpoint{2.129027in}{1.090020in}}%
\pgfpathlineto{\pgfqpoint{2.137800in}{1.091963in}}%
\pgfpathlineto{\pgfqpoint{2.146573in}{1.090020in}}%
\pgfpathlineto{\pgfqpoint{2.155346in}{1.090020in}}%
\pgfpathlineto{\pgfqpoint{2.164119in}{1.091963in}}%
\pgfpathlineto{\pgfqpoint{2.172892in}{1.091963in}}%
\pgfpathlineto{\pgfqpoint{2.181665in}{1.090020in}}%
\pgfpathlineto{\pgfqpoint{2.190438in}{1.091963in}}%
\pgfpathlineto{\pgfqpoint{2.199211in}{1.091963in}}%
\pgfpathlineto{\pgfqpoint{2.207984in}{1.093906in}}%
\pgfpathlineto{\pgfqpoint{2.216757in}{1.090020in}}%
\pgfpathlineto{\pgfqpoint{2.225530in}{1.091963in}}%
\pgfpathlineto{\pgfqpoint{2.234303in}{1.090020in}}%
\pgfpathlineto{\pgfqpoint{2.251849in}{1.090020in}}%
\pgfpathlineto{\pgfqpoint{2.260614in}{1.091963in}}%
\pgfpathlineto{\pgfqpoint{2.322012in}{1.091963in}}%
\pgfpathlineto{\pgfqpoint{2.330785in}{1.093906in}}%
\pgfpathlineto{\pgfqpoint{2.339558in}{1.091963in}}%
\pgfpathlineto{\pgfqpoint{2.348331in}{1.091963in}}%
\pgfpathlineto{\pgfqpoint{2.357104in}{1.093906in}}%
\pgfpathlineto{\pgfqpoint{2.365877in}{1.093906in}}%
\pgfpathlineto{\pgfqpoint{2.374650in}{1.091963in}}%
\pgfpathlineto{\pgfqpoint{2.409742in}{1.091963in}}%
\pgfpathlineto{\pgfqpoint{2.418534in}{1.088077in}}%
\pgfpathlineto{\pgfqpoint{2.427310in}{1.091963in}}%
\pgfpathlineto{\pgfqpoint{2.436083in}{1.090020in}}%
\pgfpathlineto{\pgfqpoint{2.444855in}{1.090020in}}%
\pgfpathlineto{\pgfqpoint{2.453623in}{1.088077in}}%
\pgfpathlineto{\pgfqpoint{2.479926in}{1.088077in}}%
\pgfpathlineto{\pgfqpoint{2.488699in}{1.086134in}}%
\pgfpathlineto{\pgfqpoint{2.497472in}{1.088077in}}%
\pgfpathlineto{\pgfqpoint{2.506245in}{1.086134in}}%
\pgfpathlineto{\pgfqpoint{2.523788in}{1.086134in}}%
\pgfpathlineto{\pgfqpoint{2.532564in}{1.088077in}}%
\pgfpathlineto{\pgfqpoint{2.541337in}{1.088077in}}%
\pgfpathlineto{\pgfqpoint{2.550110in}{1.086134in}}%
\pgfpathlineto{\pgfqpoint{2.558884in}{1.086134in}}%
\pgfpathlineto{\pgfqpoint{2.567656in}{1.088077in}}%
\pgfpathlineto{\pgfqpoint{2.646614in}{1.088077in}}%
\pgfpathlineto{\pgfqpoint{2.655387in}{1.090020in}}%
\pgfpathlineto{\pgfqpoint{2.672927in}{1.090020in}}%
\pgfpathlineto{\pgfqpoint{2.681684in}{1.088077in}}%
\pgfpathlineto{\pgfqpoint{2.690457in}{1.088077in}}%
\pgfpathlineto{\pgfqpoint{2.699230in}{1.090020in}}%
\pgfpathlineto{\pgfqpoint{2.708003in}{1.090020in}}%
\pgfpathlineto{\pgfqpoint{2.716776in}{1.088077in}}%
\pgfpathlineto{\pgfqpoint{2.734322in}{1.088077in}}%
\pgfpathlineto{\pgfqpoint{2.743095in}{1.086134in}}%
\pgfpathlineto{\pgfqpoint{2.751868in}{1.088077in}}%
\pgfpathlineto{\pgfqpoint{2.857122in}{1.088077in}}%
\pgfpathlineto{\pgfqpoint{2.865895in}{1.090020in}}%
\pgfpathlineto{\pgfqpoint{2.892192in}{1.090020in}}%
\pgfpathlineto{\pgfqpoint{2.900962in}{1.088077in}}%
\pgfpathlineto{\pgfqpoint{2.909738in}{1.090020in}}%
\pgfpathlineto{\pgfqpoint{2.936057in}{1.090020in}}%
\pgfpathlineto{\pgfqpoint{2.944830in}{1.088077in}}%
\pgfpathlineto{\pgfqpoint{2.953846in}{1.090020in}}%
\pgfpathlineto{\pgfqpoint{2.971392in}{1.090020in}}%
\pgfpathlineto{\pgfqpoint{2.980165in}{1.091963in}}%
\pgfpathlineto{\pgfqpoint{3.024030in}{1.091963in}}%
\pgfpathlineto{\pgfqpoint{3.032803in}{1.093906in}}%
\pgfpathlineto{\pgfqpoint{3.120533in}{1.093906in}}%
\pgfpathlineto{\pgfqpoint{3.129306in}{1.095849in}}%
\pgfpathlineto{\pgfqpoint{3.155603in}{1.095849in}}%
\pgfpathlineto{\pgfqpoint{3.164376in}{1.097791in}}%
\pgfpathlineto{\pgfqpoint{3.190695in}{1.097791in}}%
\pgfpathlineto{\pgfqpoint{3.199468in}{1.099734in}}%
\pgfpathlineto{\pgfqpoint{3.225787in}{1.099734in}}%
\pgfpathlineto{\pgfqpoint{3.243334in}{1.103620in}}%
\pgfpathlineto{\pgfqpoint{3.269653in}{1.103620in}}%
\pgfpathlineto{\pgfqpoint{3.278426in}{1.105563in}}%
\pgfpathlineto{\pgfqpoint{3.287199in}{1.103620in}}%
\pgfpathlineto{\pgfqpoint{3.295972in}{1.105563in}}%
\pgfpathlineto{\pgfqpoint{3.313518in}{1.105563in}}%
\pgfpathlineto{\pgfqpoint{3.322291in}{1.107506in}}%
\pgfpathlineto{\pgfqpoint{3.348610in}{1.107506in}}%
\pgfpathlineto{\pgfqpoint{3.357383in}{1.109449in}}%
\pgfpathlineto{\pgfqpoint{3.366156in}{1.109449in}}%
\pgfpathlineto{\pgfqpoint{3.383702in}{1.113334in}}%
\pgfpathlineto{\pgfqpoint{3.418794in}{1.113334in}}%
\pgfpathlineto{\pgfqpoint{3.427567in}{1.117220in}}%
\pgfpathlineto{\pgfqpoint{3.453886in}{1.117220in}}%
\pgfpathlineto{\pgfqpoint{3.471432in}{1.121106in}}%
\pgfpathlineto{\pgfqpoint{3.497751in}{1.121106in}}%
\pgfpathlineto{\pgfqpoint{3.506524in}{1.123049in}}%
\pgfpathlineto{\pgfqpoint{3.515297in}{1.123049in}}%
\pgfpathlineto{\pgfqpoint{3.524070in}{1.124991in}}%
\pgfpathlineto{\pgfqpoint{3.532843in}{1.124991in}}%
\pgfpathlineto{\pgfqpoint{3.541616in}{1.126934in}}%
\pgfpathlineto{\pgfqpoint{3.559162in}{1.126934in}}%
\pgfpathlineto{\pgfqpoint{3.567936in}{1.128877in}}%
\pgfpathlineto{\pgfqpoint{3.585482in}{1.128877in}}%
\pgfpathlineto{\pgfqpoint{3.594255in}{1.130820in}}%
\pgfpathlineto{\pgfqpoint{3.611801in}{1.130820in}}%
\pgfpathlineto{\pgfqpoint{3.620574in}{1.134706in}}%
\pgfpathlineto{\pgfqpoint{3.629347in}{1.134706in}}%
\pgfpathlineto{\pgfqpoint{3.638120in}{1.136649in}}%
\pgfpathlineto{\pgfqpoint{3.646893in}{1.136649in}}%
\pgfpathlineto{\pgfqpoint{3.673212in}{1.142477in}}%
\pgfpathlineto{\pgfqpoint{3.681985in}{1.142477in}}%
\pgfpathlineto{\pgfqpoint{3.690780in}{1.146363in}}%
\pgfpathlineto{\pgfqpoint{3.699553in}{1.144420in}}%
\pgfpathlineto{\pgfqpoint{3.708326in}{1.144420in}}%
\pgfpathlineto{\pgfqpoint{3.734645in}{1.150249in}}%
\pgfpathlineto{\pgfqpoint{3.752191in}{1.150249in}}%
\pgfpathlineto{\pgfqpoint{3.760964in}{1.152191in}}%
\pgfpathlineto{\pgfqpoint{3.769737in}{1.156077in}}%
\pgfpathlineto{\pgfqpoint{3.787283in}{1.156077in}}%
\pgfpathlineto{\pgfqpoint{3.796056in}{1.158020in}}%
\pgfpathlineto{\pgfqpoint{3.813602in}{1.158020in}}%
\pgfpathlineto{\pgfqpoint{3.831148in}{1.161906in}}%
\pgfpathlineto{\pgfqpoint{3.857489in}{1.161906in}}%
\pgfpathlineto{\pgfqpoint{3.866262in}{1.163849in}}%
\pgfpathlineto{\pgfqpoint{3.883808in}{1.163849in}}%
\pgfpathlineto{\pgfqpoint{3.892582in}{1.167734in}}%
\pgfpathlineto{\pgfqpoint{3.901355in}{1.167734in}}%
\pgfpathlineto{\pgfqpoint{3.910128in}{1.169677in}}%
\pgfpathlineto{\pgfqpoint{3.927674in}{1.169677in}}%
\pgfpathlineto{\pgfqpoint{3.936447in}{1.171620in}}%
\pgfpathlineto{\pgfqpoint{3.953993in}{1.171620in}}%
\pgfpathlineto{\pgfqpoint{3.971533in}{1.175506in}}%
\pgfpathlineto{\pgfqpoint{3.989063in}{1.175506in}}%
\pgfpathlineto{\pgfqpoint{4.006609in}{1.179391in}}%
\pgfpathlineto{\pgfqpoint{4.024155in}{1.179391in}}%
\pgfpathlineto{\pgfqpoint{4.032928in}{1.181334in}}%
\pgfpathlineto{\pgfqpoint{4.059247in}{1.181334in}}%
\pgfpathlineto{\pgfqpoint{4.076793in}{1.185220in}}%
\pgfpathlineto{\pgfqpoint{4.111885in}{1.185220in}}%
\pgfpathlineto{\pgfqpoint{4.120658in}{1.187163in}}%
\pgfpathlineto{\pgfqpoint{4.138204in}{1.187163in}}%
\pgfpathlineto{\pgfqpoint{4.155750in}{1.191049in}}%
\pgfpathlineto{\pgfqpoint{4.234707in}{1.191049in}}%
\pgfpathlineto{\pgfqpoint{4.243481in}{1.192991in}}%
\pgfpathlineto{\pgfqpoint{4.339984in}{1.192991in}}%
\pgfpathlineto{\pgfqpoint{4.348757in}{1.194934in}}%
\pgfpathlineto{\pgfqpoint{4.436509in}{1.194934in}}%
\pgfpathlineto{\pgfqpoint{4.445282in}{1.196877in}}%
\pgfpathlineto{\pgfqpoint{4.454055in}{1.194934in}}%
\pgfpathlineto{\pgfqpoint{4.462828in}{1.196877in}}%
\pgfpathlineto{\pgfqpoint{4.471601in}{1.194934in}}%
\pgfpathlineto{\pgfqpoint{4.480374in}{1.194934in}}%
\pgfpathlineto{\pgfqpoint{4.489147in}{1.196877in}}%
\pgfpathlineto{\pgfqpoint{4.568104in}{1.196877in}}%
\pgfpathlineto{\pgfqpoint{4.576877in}{1.194934in}}%
\pgfpathlineto{\pgfqpoint{4.585651in}{1.198820in}}%
\pgfpathlineto{\pgfqpoint{4.611970in}{1.198820in}}%
\pgfpathlineto{\pgfqpoint{4.620743in}{1.196877in}}%
\pgfpathlineto{\pgfqpoint{4.629516in}{1.198820in}}%
\pgfpathlineto{\pgfqpoint{4.647062in}{1.198820in}}%
\pgfpathlineto{\pgfqpoint{4.655835in}{1.196877in}}%
\pgfpathlineto{\pgfqpoint{4.664608in}{1.198820in}}%
\pgfpathlineto{\pgfqpoint{4.682154in}{1.198820in}}%
\pgfpathlineto{\pgfqpoint{4.690927in}{1.196877in}}%
\pgfpathlineto{\pgfqpoint{4.708473in}{1.196877in}}%
\pgfpathlineto{\pgfqpoint{4.717246in}{1.194934in}}%
\pgfpathlineto{\pgfqpoint{4.726019in}{1.196877in}}%
\pgfpathlineto{\pgfqpoint{4.743565in}{1.196877in}}%
\pgfpathlineto{\pgfqpoint{4.752338in}{1.194934in}}%
\pgfpathlineto{\pgfqpoint{4.769884in}{1.194934in}}%
\pgfpathlineto{\pgfqpoint{4.778657in}{1.196877in}}%
\pgfpathlineto{\pgfqpoint{4.883933in}{1.196877in}}%
\pgfpathlineto{\pgfqpoint{4.892728in}{1.194934in}}%
\pgfpathlineto{\pgfqpoint{4.927821in}{1.194934in}}%
\pgfpathlineto{\pgfqpoint{4.936594in}{1.196877in}}%
\pgfpathlineto{\pgfqpoint{4.945367in}{1.194934in}}%
\pgfpathlineto{\pgfqpoint{4.962913in}{1.194934in}}%
\pgfpathlineto{\pgfqpoint{4.971686in}{1.196877in}}%
\pgfpathlineto{\pgfqpoint{4.980459in}{1.194934in}}%
\pgfpathlineto{\pgfqpoint{5.006778in}{1.194934in}}%
\pgfpathlineto{\pgfqpoint{5.015551in}{1.196877in}}%
\pgfpathlineto{\pgfqpoint{5.024324in}{1.196877in}}%
\pgfpathlineto{\pgfqpoint{5.033097in}{1.192991in}}%
\pgfpathlineto{\pgfqpoint{5.041870in}{1.196877in}}%
\pgfpathlineto{\pgfqpoint{5.059438in}{1.196877in}}%
\pgfpathlineto{\pgfqpoint{5.068211in}{1.194934in}}%
\pgfpathlineto{\pgfqpoint{5.076984in}{1.194934in}}%
\pgfpathlineto{\pgfqpoint{5.085757in}{1.196877in}}%
\pgfpathlineto{\pgfqpoint{5.103303in}{1.192991in}}%
\pgfpathlineto{\pgfqpoint{5.112076in}{1.194934in}}%
\pgfpathlineto{\pgfqpoint{5.155941in}{1.194934in}}%
\pgfpathlineto{\pgfqpoint{5.164714in}{1.192991in}}%
\pgfpathlineto{\pgfqpoint{5.243672in}{1.192991in}}%
\pgfpathlineto{\pgfqpoint{5.261218in}{1.189106in}}%
\pgfpathlineto{\pgfqpoint{5.269991in}{1.189106in}}%
\pgfpathlineto{\pgfqpoint{5.278764in}{1.187163in}}%
\pgfpathlineto{\pgfqpoint{5.287537in}{1.187163in}}%
\pgfpathlineto{\pgfqpoint{5.296310in}{1.185220in}}%
\pgfpathlineto{\pgfqpoint{5.313856in}{1.185220in}}%
\pgfpathlineto{\pgfqpoint{5.322629in}{1.183277in}}%
\pgfpathlineto{\pgfqpoint{5.331402in}{1.183277in}}%
\pgfpathlineto{\pgfqpoint{5.340175in}{1.179391in}}%
\pgfpathlineto{\pgfqpoint{5.348948in}{1.177449in}}%
\pgfpathlineto{\pgfqpoint{5.357721in}{1.177449in}}%
\pgfpathlineto{\pgfqpoint{5.392813in}{1.169677in}}%
\pgfpathlineto{\pgfqpoint{5.401586in}{1.169677in}}%
\pgfpathlineto{\pgfqpoint{5.410359in}{1.167734in}}%
\pgfpathlineto{\pgfqpoint{5.419132in}{1.163849in}}%
\pgfpathlineto{\pgfqpoint{5.427905in}{1.163849in}}%
\pgfpathlineto{\pgfqpoint{5.471770in}{1.154134in}}%
\pgfpathlineto{\pgfqpoint{5.480543in}{1.154134in}}%
\pgfpathlineto{\pgfqpoint{5.500556in}{1.149702in}}%
\pgfpathlineto{\pgfqpoint{5.500556in}{1.149702in}}%
\pgfusepath{stroke}%
\end{pgfscope}%
\begin{pgfscope}%
\pgfpathrectangle{\pgfqpoint{0.530556in}{0.656763in}}{\pgfqpoint{4.960000in}{3.264000in}}%
\pgfusepath{clip}%
\pgfsetrectcap%
\pgfsetroundjoin%
\pgfsetlinewidth{1.505625pt}%
\definecolor{currentstroke}{rgb}{0.172549,0.627451,0.172549}%
\pgfsetstrokecolor{currentstroke}%
\pgfsetdash{}{0pt}%
\pgfpathmoveto{\pgfqpoint{0.520556in}{1.301070in}}%
\pgfpathlineto{\pgfqpoint{0.523563in}{1.303734in}}%
\pgfpathlineto{\pgfqpoint{0.576202in}{1.338706in}}%
\pgfpathlineto{\pgfqpoint{0.593748in}{1.354249in}}%
\pgfpathlineto{\pgfqpoint{0.602521in}{1.360077in}}%
\pgfpathlineto{\pgfqpoint{0.611294in}{1.369791in}}%
\pgfpathlineto{\pgfqpoint{0.620067in}{1.377563in}}%
\pgfpathlineto{\pgfqpoint{0.637613in}{1.396991in}}%
\pgfpathlineto{\pgfqpoint{0.646386in}{1.404763in}}%
\pgfpathlineto{\pgfqpoint{0.663932in}{1.424191in}}%
\pgfpathlineto{\pgfqpoint{0.672705in}{1.431963in}}%
\pgfpathlineto{\pgfqpoint{0.681478in}{1.437791in}}%
\pgfpathlineto{\pgfqpoint{0.690251in}{1.449449in}}%
\pgfpathlineto{\pgfqpoint{0.699024in}{1.457220in}}%
\pgfpathlineto{\pgfqpoint{0.707797in}{1.466934in}}%
\pgfpathlineto{\pgfqpoint{0.725343in}{1.478591in}}%
\pgfpathlineto{\pgfqpoint{0.734116in}{1.486363in}}%
\pgfpathlineto{\pgfqpoint{0.742889in}{1.496077in}}%
\pgfpathlineto{\pgfqpoint{0.751662in}{1.503849in}}%
\pgfpathlineto{\pgfqpoint{0.769208in}{1.515506in}}%
\pgfpathlineto{\pgfqpoint{0.777981in}{1.523277in}}%
\pgfpathlineto{\pgfqpoint{0.786754in}{1.529106in}}%
\pgfpathlineto{\pgfqpoint{0.795527in}{1.536877in}}%
\pgfpathlineto{\pgfqpoint{0.821846in}{1.554363in}}%
\pgfpathlineto{\pgfqpoint{0.830619in}{1.558249in}}%
\pgfpathlineto{\pgfqpoint{0.839392in}{1.564077in}}%
\pgfpathlineto{\pgfqpoint{0.848165in}{1.567963in}}%
\pgfpathlineto{\pgfqpoint{0.856938in}{1.573791in}}%
\pgfpathlineto{\pgfqpoint{0.865711in}{1.577677in}}%
\pgfpathlineto{\pgfqpoint{0.874484in}{1.583506in}}%
\pgfpathlineto{\pgfqpoint{0.883257in}{1.585449in}}%
\pgfpathlineto{\pgfqpoint{0.892031in}{1.591277in}}%
\pgfpathlineto{\pgfqpoint{0.900804in}{1.593220in}}%
\pgfpathlineto{\pgfqpoint{0.909577in}{1.599049in}}%
\pgfpathlineto{\pgfqpoint{0.927123in}{1.606820in}}%
\pgfpathlineto{\pgfqpoint{0.935896in}{1.608763in}}%
\pgfpathlineto{\pgfqpoint{0.962215in}{1.620420in}}%
\pgfpathlineto{\pgfqpoint{0.970983in}{1.622363in}}%
\pgfpathlineto{\pgfqpoint{0.979739in}{1.626249in}}%
\pgfpathlineto{\pgfqpoint{0.988512in}{1.628191in}}%
\pgfpathlineto{\pgfqpoint{0.997285in}{1.632077in}}%
\pgfpathlineto{\pgfqpoint{1.023604in}{1.637906in}}%
\pgfpathlineto{\pgfqpoint{1.041150in}{1.645677in}}%
\pgfpathlineto{\pgfqpoint{1.049923in}{1.647620in}}%
\pgfpathlineto{\pgfqpoint{1.058696in}{1.647620in}}%
\pgfpathlineto{\pgfqpoint{1.067469in}{1.649563in}}%
\pgfpathlineto{\pgfqpoint{1.076242in}{1.653449in}}%
\pgfpathlineto{\pgfqpoint{1.172742in}{1.674820in}}%
\pgfpathlineto{\pgfqpoint{1.181518in}{1.674820in}}%
\pgfpathlineto{\pgfqpoint{1.207837in}{1.680649in}}%
\pgfpathlineto{\pgfqpoint{1.216610in}{1.680649in}}%
\pgfpathlineto{\pgfqpoint{1.225383in}{1.682591in}}%
\pgfpathlineto{\pgfqpoint{1.234156in}{1.686477in}}%
\pgfpathlineto{\pgfqpoint{1.242929in}{1.684534in}}%
\pgfpathlineto{\pgfqpoint{1.251702in}{1.686477in}}%
\pgfpathlineto{\pgfqpoint{1.260476in}{1.686477in}}%
\pgfpathlineto{\pgfqpoint{1.278022in}{1.690363in}}%
\pgfpathlineto{\pgfqpoint{1.304363in}{1.690363in}}%
\pgfpathlineto{\pgfqpoint{1.321909in}{1.694249in}}%
\pgfpathlineto{\pgfqpoint{1.339455in}{1.694249in}}%
\pgfpathlineto{\pgfqpoint{1.348228in}{1.696191in}}%
\pgfpathlineto{\pgfqpoint{1.365773in}{1.696191in}}%
\pgfpathlineto{\pgfqpoint{1.383320in}{1.700077in}}%
\pgfpathlineto{\pgfqpoint{1.400866in}{1.700077in}}%
\pgfpathlineto{\pgfqpoint{1.418412in}{1.703963in}}%
\pgfpathlineto{\pgfqpoint{1.444731in}{1.703963in}}%
\pgfpathlineto{\pgfqpoint{1.453504in}{1.707849in}}%
\pgfpathlineto{\pgfqpoint{1.462277in}{1.705906in}}%
\pgfpathlineto{\pgfqpoint{1.471050in}{1.707849in}}%
\pgfpathlineto{\pgfqpoint{1.479823in}{1.707849in}}%
\pgfpathlineto{\pgfqpoint{1.488596in}{1.711734in}}%
\pgfpathlineto{\pgfqpoint{1.497369in}{1.711734in}}%
\pgfpathlineto{\pgfqpoint{1.506142in}{1.713677in}}%
\pgfpathlineto{\pgfqpoint{1.532461in}{1.713677in}}%
\pgfpathlineto{\pgfqpoint{1.541234in}{1.717563in}}%
\pgfpathlineto{\pgfqpoint{1.567553in}{1.717563in}}%
\pgfpathlineto{\pgfqpoint{1.576326in}{1.719506in}}%
\pgfpathlineto{\pgfqpoint{1.585100in}{1.717563in}}%
\pgfpathlineto{\pgfqpoint{1.593873in}{1.721449in}}%
\pgfpathlineto{\pgfqpoint{1.611419in}{1.721449in}}%
\pgfpathlineto{\pgfqpoint{1.620192in}{1.725334in}}%
\pgfpathlineto{\pgfqpoint{1.628965in}{1.723391in}}%
\pgfpathlineto{\pgfqpoint{1.637738in}{1.725334in}}%
\pgfpathlineto{\pgfqpoint{1.655284in}{1.725334in}}%
\pgfpathlineto{\pgfqpoint{1.664057in}{1.727277in}}%
\pgfpathlineto{\pgfqpoint{1.672830in}{1.727277in}}%
\pgfpathlineto{\pgfqpoint{1.681603in}{1.729220in}}%
\pgfpathlineto{\pgfqpoint{1.690376in}{1.729220in}}%
\pgfpathlineto{\pgfqpoint{1.699149in}{1.731163in}}%
\pgfpathlineto{\pgfqpoint{1.707922in}{1.729220in}}%
\pgfpathlineto{\pgfqpoint{1.716695in}{1.731163in}}%
\pgfpathlineto{\pgfqpoint{1.725468in}{1.731163in}}%
\pgfpathlineto{\pgfqpoint{1.734241in}{1.733106in}}%
\pgfpathlineto{\pgfqpoint{1.751787in}{1.733106in}}%
\pgfpathlineto{\pgfqpoint{1.760560in}{1.736991in}}%
\pgfpathlineto{\pgfqpoint{1.778106in}{1.736991in}}%
\pgfpathlineto{\pgfqpoint{1.786879in}{1.738934in}}%
\pgfpathlineto{\pgfqpoint{1.813198in}{1.738934in}}%
\pgfpathlineto{\pgfqpoint{1.821971in}{1.740877in}}%
\pgfpathlineto{\pgfqpoint{1.848290in}{1.740877in}}%
\pgfpathlineto{\pgfqpoint{1.865836in}{1.744763in}}%
\pgfpathlineto{\pgfqpoint{1.874609in}{1.742820in}}%
\pgfpathlineto{\pgfqpoint{1.883382in}{1.744763in}}%
\pgfpathlineto{\pgfqpoint{1.892155in}{1.742820in}}%
\pgfpathlineto{\pgfqpoint{1.900928in}{1.746706in}}%
\pgfpathlineto{\pgfqpoint{1.909701in}{1.748649in}}%
\pgfpathlineto{\pgfqpoint{1.927248in}{1.748649in}}%
\pgfpathlineto{\pgfqpoint{1.936021in}{1.750591in}}%
\pgfpathlineto{\pgfqpoint{1.944794in}{1.748649in}}%
\pgfpathlineto{\pgfqpoint{1.962340in}{1.748649in}}%
\pgfpathlineto{\pgfqpoint{1.971113in}{1.750591in}}%
\pgfpathlineto{\pgfqpoint{1.979886in}{1.750591in}}%
\pgfpathlineto{\pgfqpoint{1.988659in}{1.752534in}}%
\pgfpathlineto{\pgfqpoint{1.997432in}{1.752534in}}%
\pgfpathlineto{\pgfqpoint{2.006205in}{1.750591in}}%
\pgfpathlineto{\pgfqpoint{2.014978in}{1.754477in}}%
\pgfpathlineto{\pgfqpoint{2.023751in}{1.754477in}}%
\pgfpathlineto{\pgfqpoint{2.032524in}{1.752534in}}%
\pgfpathlineto{\pgfqpoint{2.041297in}{1.754477in}}%
\pgfpathlineto{\pgfqpoint{2.058843in}{1.754477in}}%
\pgfpathlineto{\pgfqpoint{2.067616in}{1.756420in}}%
\pgfpathlineto{\pgfqpoint{2.093935in}{1.756420in}}%
\pgfpathlineto{\pgfqpoint{2.102708in}{1.754477in}}%
\pgfpathlineto{\pgfqpoint{2.120254in}{1.758363in}}%
\pgfpathlineto{\pgfqpoint{2.190438in}{1.758363in}}%
\pgfpathlineto{\pgfqpoint{2.199211in}{1.760306in}}%
\pgfpathlineto{\pgfqpoint{2.216757in}{1.760306in}}%
\pgfpathlineto{\pgfqpoint{2.225530in}{1.762249in}}%
\pgfpathlineto{\pgfqpoint{2.234303in}{1.760306in}}%
\pgfpathlineto{\pgfqpoint{2.243076in}{1.760306in}}%
\pgfpathlineto{\pgfqpoint{2.251849in}{1.762249in}}%
\pgfpathlineto{\pgfqpoint{2.260614in}{1.762249in}}%
\pgfpathlineto{\pgfqpoint{2.269373in}{1.760306in}}%
\pgfpathlineto{\pgfqpoint{2.278146in}{1.760306in}}%
\pgfpathlineto{\pgfqpoint{2.286919in}{1.764191in}}%
\pgfpathlineto{\pgfqpoint{2.295693in}{1.762249in}}%
\pgfpathlineto{\pgfqpoint{2.304466in}{1.764191in}}%
\pgfpathlineto{\pgfqpoint{2.313239in}{1.762249in}}%
\pgfpathlineto{\pgfqpoint{2.409742in}{1.762249in}}%
\pgfpathlineto{\pgfqpoint{2.418534in}{1.764191in}}%
\pgfpathlineto{\pgfqpoint{2.436083in}{1.764191in}}%
\pgfpathlineto{\pgfqpoint{2.444855in}{1.762249in}}%
\pgfpathlineto{\pgfqpoint{2.453623in}{1.764191in}}%
\pgfpathlineto{\pgfqpoint{2.479926in}{1.764191in}}%
\pgfpathlineto{\pgfqpoint{2.488699in}{1.766134in}}%
\pgfpathlineto{\pgfqpoint{2.497472in}{1.764191in}}%
\pgfpathlineto{\pgfqpoint{2.532564in}{1.764191in}}%
\pgfpathlineto{\pgfqpoint{2.541337in}{1.766134in}}%
\pgfpathlineto{\pgfqpoint{2.550110in}{1.764191in}}%
\pgfpathlineto{\pgfqpoint{2.558884in}{1.764191in}}%
\pgfpathlineto{\pgfqpoint{2.567656in}{1.766134in}}%
\pgfpathlineto{\pgfqpoint{2.576429in}{1.764191in}}%
\pgfpathlineto{\pgfqpoint{2.593975in}{1.764191in}}%
\pgfpathlineto{\pgfqpoint{2.602748in}{1.762249in}}%
\pgfpathlineto{\pgfqpoint{2.620294in}{1.762249in}}%
\pgfpathlineto{\pgfqpoint{2.637841in}{1.766134in}}%
\pgfpathlineto{\pgfqpoint{2.646614in}{1.766134in}}%
\pgfpathlineto{\pgfqpoint{2.655387in}{1.764191in}}%
\pgfpathlineto{\pgfqpoint{2.664160in}{1.764191in}}%
\pgfpathlineto{\pgfqpoint{2.672927in}{1.766134in}}%
\pgfpathlineto{\pgfqpoint{2.699230in}{1.766134in}}%
\pgfpathlineto{\pgfqpoint{2.708003in}{1.764191in}}%
\pgfpathlineto{\pgfqpoint{2.716776in}{1.764191in}}%
\pgfpathlineto{\pgfqpoint{2.725549in}{1.766134in}}%
\pgfpathlineto{\pgfqpoint{2.734322in}{1.764191in}}%
\pgfpathlineto{\pgfqpoint{2.743095in}{1.764191in}}%
\pgfpathlineto{\pgfqpoint{2.751868in}{1.766134in}}%
\pgfpathlineto{\pgfqpoint{2.760641in}{1.766134in}}%
\pgfpathlineto{\pgfqpoint{2.769414in}{1.768077in}}%
\pgfpathlineto{\pgfqpoint{2.778182in}{1.766134in}}%
\pgfpathlineto{\pgfqpoint{2.786938in}{1.768077in}}%
\pgfpathlineto{\pgfqpoint{2.813257in}{1.768077in}}%
\pgfpathlineto{\pgfqpoint{2.822030in}{1.766134in}}%
\pgfpathlineto{\pgfqpoint{2.830803in}{1.768077in}}%
\pgfpathlineto{\pgfqpoint{2.839576in}{1.766134in}}%
\pgfpathlineto{\pgfqpoint{2.865895in}{1.766134in}}%
\pgfpathlineto{\pgfqpoint{2.874668in}{1.764191in}}%
\pgfpathlineto{\pgfqpoint{2.883436in}{1.766134in}}%
\pgfpathlineto{\pgfqpoint{2.892192in}{1.764191in}}%
\pgfpathlineto{\pgfqpoint{2.900962in}{1.766134in}}%
\pgfpathlineto{\pgfqpoint{2.909738in}{1.764191in}}%
\pgfpathlineto{\pgfqpoint{3.006484in}{1.764191in}}%
\pgfpathlineto{\pgfqpoint{3.015257in}{1.762249in}}%
\pgfpathlineto{\pgfqpoint{3.024030in}{1.764191in}}%
\pgfpathlineto{\pgfqpoint{3.032803in}{1.764191in}}%
\pgfpathlineto{\pgfqpoint{3.041576in}{1.762249in}}%
\pgfpathlineto{\pgfqpoint{3.050349in}{1.762249in}}%
\pgfpathlineto{\pgfqpoint{3.067895in}{1.766134in}}%
\pgfpathlineto{\pgfqpoint{3.076668in}{1.762249in}}%
\pgfpathlineto{\pgfqpoint{3.094214in}{1.766134in}}%
\pgfpathlineto{\pgfqpoint{3.111760in}{1.762249in}}%
\pgfpathlineto{\pgfqpoint{3.120533in}{1.764191in}}%
\pgfpathlineto{\pgfqpoint{3.129306in}{1.762249in}}%
\pgfpathlineto{\pgfqpoint{3.138079in}{1.762249in}}%
\pgfpathlineto{\pgfqpoint{3.146847in}{1.766134in}}%
\pgfpathlineto{\pgfqpoint{3.164376in}{1.766134in}}%
\pgfpathlineto{\pgfqpoint{3.173149in}{1.768077in}}%
\pgfpathlineto{\pgfqpoint{3.181922in}{1.766134in}}%
\pgfpathlineto{\pgfqpoint{3.190695in}{1.768077in}}%
\pgfpathlineto{\pgfqpoint{3.208241in}{1.768077in}}%
\pgfpathlineto{\pgfqpoint{3.217014in}{1.770020in}}%
\pgfpathlineto{\pgfqpoint{3.243334in}{1.770020in}}%
\pgfpathlineto{\pgfqpoint{3.252107in}{1.771963in}}%
\pgfpathlineto{\pgfqpoint{3.269653in}{1.771963in}}%
\pgfpathlineto{\pgfqpoint{3.278426in}{1.773906in}}%
\pgfpathlineto{\pgfqpoint{3.287199in}{1.771963in}}%
\pgfpathlineto{\pgfqpoint{3.295972in}{1.775849in}}%
\pgfpathlineto{\pgfqpoint{3.331064in}{1.775849in}}%
\pgfpathlineto{\pgfqpoint{3.339837in}{1.777791in}}%
\pgfpathlineto{\pgfqpoint{3.366156in}{1.777791in}}%
\pgfpathlineto{\pgfqpoint{3.374929in}{1.781677in}}%
\pgfpathlineto{\pgfqpoint{3.401248in}{1.781677in}}%
\pgfpathlineto{\pgfqpoint{3.410021in}{1.783620in}}%
\pgfpathlineto{\pgfqpoint{3.436340in}{1.783620in}}%
\pgfpathlineto{\pgfqpoint{3.453886in}{1.787506in}}%
\pgfpathlineto{\pgfqpoint{3.488978in}{1.787506in}}%
\pgfpathlineto{\pgfqpoint{3.497751in}{1.785563in}}%
\pgfpathlineto{\pgfqpoint{3.506524in}{1.787506in}}%
\pgfpathlineto{\pgfqpoint{3.515297in}{1.791391in}}%
\pgfpathlineto{\pgfqpoint{3.524070in}{1.789449in}}%
\pgfpathlineto{\pgfqpoint{3.532843in}{1.789449in}}%
\pgfpathlineto{\pgfqpoint{3.541616in}{1.791391in}}%
\pgfpathlineto{\pgfqpoint{3.550389in}{1.791391in}}%
\pgfpathlineto{\pgfqpoint{3.559162in}{1.793334in}}%
\pgfpathlineto{\pgfqpoint{3.594255in}{1.793334in}}%
\pgfpathlineto{\pgfqpoint{3.611801in}{1.797220in}}%
\pgfpathlineto{\pgfqpoint{3.638120in}{1.797220in}}%
\pgfpathlineto{\pgfqpoint{3.646893in}{1.799163in}}%
\pgfpathlineto{\pgfqpoint{3.673212in}{1.799163in}}%
\pgfpathlineto{\pgfqpoint{3.690780in}{1.803049in}}%
\pgfpathlineto{\pgfqpoint{3.708326in}{1.803049in}}%
\pgfpathlineto{\pgfqpoint{3.717099in}{1.804991in}}%
\pgfpathlineto{\pgfqpoint{3.725872in}{1.804991in}}%
\pgfpathlineto{\pgfqpoint{3.734645in}{1.806934in}}%
\pgfpathlineto{\pgfqpoint{3.760964in}{1.806934in}}%
\pgfpathlineto{\pgfqpoint{3.769737in}{1.810820in}}%
\pgfpathlineto{\pgfqpoint{3.796056in}{1.810820in}}%
\pgfpathlineto{\pgfqpoint{3.804829in}{1.812763in}}%
\pgfpathlineto{\pgfqpoint{3.813602in}{1.812763in}}%
\pgfpathlineto{\pgfqpoint{3.822375in}{1.814706in}}%
\pgfpathlineto{\pgfqpoint{3.839921in}{1.814706in}}%
\pgfpathlineto{\pgfqpoint{3.848694in}{1.816649in}}%
\pgfpathlineto{\pgfqpoint{3.866262in}{1.816649in}}%
\pgfpathlineto{\pgfqpoint{3.875035in}{1.818591in}}%
\pgfpathlineto{\pgfqpoint{3.892582in}{1.818591in}}%
\pgfpathlineto{\pgfqpoint{3.901355in}{1.822477in}}%
\pgfpathlineto{\pgfqpoint{3.910128in}{1.820534in}}%
\pgfpathlineto{\pgfqpoint{3.918901in}{1.822477in}}%
\pgfpathlineto{\pgfqpoint{3.927674in}{1.822477in}}%
\pgfpathlineto{\pgfqpoint{3.936447in}{1.824420in}}%
\pgfpathlineto{\pgfqpoint{3.945220in}{1.824420in}}%
\pgfpathlineto{\pgfqpoint{3.953993in}{1.826363in}}%
\pgfpathlineto{\pgfqpoint{3.971533in}{1.826363in}}%
\pgfpathlineto{\pgfqpoint{3.980290in}{1.828306in}}%
\pgfpathlineto{\pgfqpoint{4.006609in}{1.828306in}}%
\pgfpathlineto{\pgfqpoint{4.015382in}{1.832191in}}%
\pgfpathlineto{\pgfqpoint{4.024155in}{1.830249in}}%
\pgfpathlineto{\pgfqpoint{4.041701in}{1.834134in}}%
\pgfpathlineto{\pgfqpoint{4.068020in}{1.834134in}}%
\pgfpathlineto{\pgfqpoint{4.076793in}{1.836077in}}%
\pgfpathlineto{\pgfqpoint{4.085566in}{1.836077in}}%
\pgfpathlineto{\pgfqpoint{4.094339in}{1.838020in}}%
\pgfpathlineto{\pgfqpoint{4.103112in}{1.836077in}}%
\pgfpathlineto{\pgfqpoint{4.120658in}{1.839963in}}%
\pgfpathlineto{\pgfqpoint{4.129431in}{1.839963in}}%
\pgfpathlineto{\pgfqpoint{4.138204in}{1.838020in}}%
\pgfpathlineto{\pgfqpoint{4.146977in}{1.843849in}}%
\pgfpathlineto{\pgfqpoint{4.199615in}{1.843849in}}%
\pgfpathlineto{\pgfqpoint{4.208388in}{1.845791in}}%
\pgfpathlineto{\pgfqpoint{4.243481in}{1.845791in}}%
\pgfpathlineto{\pgfqpoint{4.252254in}{1.847734in}}%
\pgfpathlineto{\pgfqpoint{4.278573in}{1.847734in}}%
\pgfpathlineto{\pgfqpoint{4.287346in}{1.849677in}}%
\pgfpathlineto{\pgfqpoint{4.296119in}{1.847734in}}%
\pgfpathlineto{\pgfqpoint{4.304891in}{1.849677in}}%
\pgfpathlineto{\pgfqpoint{4.339984in}{1.849677in}}%
\pgfpathlineto{\pgfqpoint{4.348757in}{1.851620in}}%
\pgfpathlineto{\pgfqpoint{4.410168in}{1.851620in}}%
\pgfpathlineto{\pgfqpoint{4.418963in}{1.853563in}}%
\pgfpathlineto{\pgfqpoint{4.462828in}{1.853563in}}%
\pgfpathlineto{\pgfqpoint{4.471601in}{1.855506in}}%
\pgfpathlineto{\pgfqpoint{4.480374in}{1.855506in}}%
\pgfpathlineto{\pgfqpoint{4.489147in}{1.857449in}}%
\pgfpathlineto{\pgfqpoint{4.576877in}{1.857449in}}%
\pgfpathlineto{\pgfqpoint{4.585651in}{1.859391in}}%
\pgfpathlineto{\pgfqpoint{4.594424in}{1.857449in}}%
\pgfpathlineto{\pgfqpoint{4.603197in}{1.859391in}}%
\pgfpathlineto{\pgfqpoint{4.638289in}{1.859391in}}%
\pgfpathlineto{\pgfqpoint{4.647062in}{1.857449in}}%
\pgfpathlineto{\pgfqpoint{4.655835in}{1.859391in}}%
\pgfpathlineto{\pgfqpoint{4.699700in}{1.859391in}}%
\pgfpathlineto{\pgfqpoint{4.708473in}{1.857449in}}%
\pgfpathlineto{\pgfqpoint{4.717246in}{1.859391in}}%
\pgfpathlineto{\pgfqpoint{4.734792in}{1.859391in}}%
\pgfpathlineto{\pgfqpoint{4.743565in}{1.857449in}}%
\pgfpathlineto{\pgfqpoint{4.752338in}{1.859391in}}%
\pgfpathlineto{\pgfqpoint{4.769884in}{1.859391in}}%
\pgfpathlineto{\pgfqpoint{4.778657in}{1.861334in}}%
\pgfpathlineto{\pgfqpoint{4.787430in}{1.859391in}}%
\pgfpathlineto{\pgfqpoint{4.796203in}{1.859391in}}%
\pgfpathlineto{\pgfqpoint{4.804976in}{1.861334in}}%
\pgfpathlineto{\pgfqpoint{4.813749in}{1.859391in}}%
\pgfpathlineto{\pgfqpoint{4.822522in}{1.859391in}}%
\pgfpathlineto{\pgfqpoint{4.831295in}{1.861334in}}%
\pgfpathlineto{\pgfqpoint{4.840068in}{1.861334in}}%
\pgfpathlineto{\pgfqpoint{4.848841in}{1.859391in}}%
\pgfpathlineto{\pgfqpoint{4.866387in}{1.859391in}}%
\pgfpathlineto{\pgfqpoint{4.875160in}{1.861334in}}%
\pgfpathlineto{\pgfqpoint{4.883933in}{1.861334in}}%
\pgfpathlineto{\pgfqpoint{4.892728in}{1.859391in}}%
\pgfpathlineto{\pgfqpoint{4.980459in}{1.859391in}}%
\pgfpathlineto{\pgfqpoint{4.989232in}{1.861334in}}%
\pgfpathlineto{\pgfqpoint{4.998005in}{1.859391in}}%
\pgfpathlineto{\pgfqpoint{5.006778in}{1.861334in}}%
\pgfpathlineto{\pgfqpoint{5.015551in}{1.859391in}}%
\pgfpathlineto{\pgfqpoint{5.024324in}{1.861334in}}%
\pgfpathlineto{\pgfqpoint{5.033097in}{1.859391in}}%
\pgfpathlineto{\pgfqpoint{5.050643in}{1.859391in}}%
\pgfpathlineto{\pgfqpoint{5.059438in}{1.861334in}}%
\pgfpathlineto{\pgfqpoint{5.068211in}{1.859391in}}%
\pgfpathlineto{\pgfqpoint{5.076984in}{1.861334in}}%
\pgfpathlineto{\pgfqpoint{5.085757in}{1.859391in}}%
\pgfpathlineto{\pgfqpoint{5.094530in}{1.861334in}}%
\pgfpathlineto{\pgfqpoint{5.103303in}{1.859391in}}%
\pgfpathlineto{\pgfqpoint{5.243672in}{1.859391in}}%
\pgfpathlineto{\pgfqpoint{5.252445in}{1.857449in}}%
\pgfpathlineto{\pgfqpoint{5.261218in}{1.859391in}}%
\pgfpathlineto{\pgfqpoint{5.305083in}{1.849677in}}%
\pgfpathlineto{\pgfqpoint{5.313856in}{1.845791in}}%
\pgfpathlineto{\pgfqpoint{5.322629in}{1.839963in}}%
\pgfpathlineto{\pgfqpoint{5.331402in}{1.838020in}}%
\pgfpathlineto{\pgfqpoint{5.340175in}{1.834134in}}%
\pgfpathlineto{\pgfqpoint{5.410359in}{1.787506in}}%
\pgfpathlineto{\pgfqpoint{5.419132in}{1.779734in}}%
\pgfpathlineto{\pgfqpoint{5.427905in}{1.775849in}}%
\pgfpathlineto{\pgfqpoint{5.436678in}{1.770020in}}%
\pgfpathlineto{\pgfqpoint{5.445451in}{1.762249in}}%
\pgfpathlineto{\pgfqpoint{5.462997in}{1.754477in}}%
\pgfpathlineto{\pgfqpoint{5.471770in}{1.748649in}}%
\pgfpathlineto{\pgfqpoint{5.480543in}{1.744763in}}%
\pgfpathlineto{\pgfqpoint{5.489316in}{1.738934in}}%
\pgfpathlineto{\pgfqpoint{5.500556in}{1.733956in}}%
\pgfpathlineto{\pgfqpoint{5.500556in}{1.733956in}}%
\pgfusepath{stroke}%
\end{pgfscope}%
\begin{pgfscope}%
\pgfsetrectcap%
\pgfsetmiterjoin%
\pgfsetlinewidth{0.803000pt}%
\definecolor{currentstroke}{rgb}{0.000000,0.000000,0.000000}%
\pgfsetstrokecolor{currentstroke}%
\pgfsetdash{}{0pt}%
\pgfpathmoveto{\pgfqpoint{0.530556in}{0.656763in}}%
\pgfpathlineto{\pgfqpoint{0.530556in}{3.920763in}}%
\pgfusepath{stroke}%
\end{pgfscope}%
\begin{pgfscope}%
\pgfsetrectcap%
\pgfsetmiterjoin%
\pgfsetlinewidth{0.803000pt}%
\definecolor{currentstroke}{rgb}{0.000000,0.000000,0.000000}%
\pgfsetstrokecolor{currentstroke}%
\pgfsetdash{}{0pt}%
\pgfpathmoveto{\pgfqpoint{5.490556in}{0.656763in}}%
\pgfpathlineto{\pgfqpoint{5.490556in}{3.920763in}}%
\pgfusepath{stroke}%
\end{pgfscope}%
\begin{pgfscope}%
\pgfsetrectcap%
\pgfsetmiterjoin%
\pgfsetlinewidth{0.803000pt}%
\definecolor{currentstroke}{rgb}{0.000000,0.000000,0.000000}%
\pgfsetstrokecolor{currentstroke}%
\pgfsetdash{}{0pt}%
\pgfpathmoveto{\pgfqpoint{0.530556in}{0.656763in}}%
\pgfpathlineto{\pgfqpoint{5.490556in}{0.656763in}}%
\pgfusepath{stroke}%
\end{pgfscope}%
\begin{pgfscope}%
\pgfsetrectcap%
\pgfsetmiterjoin%
\pgfsetlinewidth{0.803000pt}%
\definecolor{currentstroke}{rgb}{0.000000,0.000000,0.000000}%
\pgfsetstrokecolor{currentstroke}%
\pgfsetdash{}{0pt}%
\pgfpathmoveto{\pgfqpoint{0.530556in}{3.920763in}}%
\pgfpathlineto{\pgfqpoint{5.490556in}{3.920763in}}%
\pgfusepath{stroke}%
\end{pgfscope}%
\begin{pgfscope}%
\pgfsetbuttcap%
\pgfsetmiterjoin%
\definecolor{currentfill}{rgb}{1.000000,1.000000,1.000000}%
\pgfsetfillcolor{currentfill}%
\pgfsetfillopacity{0.800000}%
\pgfsetlinewidth{1.003750pt}%
\definecolor{currentstroke}{rgb}{0.800000,0.800000,0.800000}%
\pgfsetstrokecolor{currentstroke}%
\pgfsetstrokeopacity{0.800000}%
\pgfsetdash{}{0pt}%
\pgfpathmoveto{\pgfqpoint{3.801511in}{1.977420in}}%
\pgfpathlineto{\pgfqpoint{5.393334in}{1.977420in}}%
\pgfpathquadraticcurveto{\pgfqpoint{5.421112in}{1.977420in}}{\pgfqpoint{5.421112in}{2.005198in}}%
\pgfpathlineto{\pgfqpoint{5.421112in}{2.572328in}}%
\pgfpathquadraticcurveto{\pgfqpoint{5.421112in}{2.600105in}}{\pgfqpoint{5.393334in}{2.600105in}}%
\pgfpathlineto{\pgfqpoint{3.801511in}{2.600105in}}%
\pgfpathquadraticcurveto{\pgfqpoint{3.773733in}{2.600105in}}{\pgfqpoint{3.773733in}{2.572328in}}%
\pgfpathlineto{\pgfqpoint{3.773733in}{2.005198in}}%
\pgfpathquadraticcurveto{\pgfqpoint{3.773733in}{1.977420in}}{\pgfqpoint{3.801511in}{1.977420in}}%
\pgfpathclose%
\pgfusepath{stroke,fill}%
\end{pgfscope}%
\begin{pgfscope}%
\pgfsetrectcap%
\pgfsetroundjoin%
\pgfsetlinewidth{1.505625pt}%
\definecolor{currentstroke}{rgb}{0.121569,0.466667,0.705882}%
\pgfsetstrokecolor{currentstroke}%
\pgfsetdash{}{0pt}%
\pgfpathmoveto{\pgfqpoint{3.829289in}{2.495939in}}%
\pgfpathlineto{\pgfqpoint{4.107066in}{2.495939in}}%
\pgfusepath{stroke}%
\end{pgfscope}%
\begin{pgfscope}%
\definecolor{textcolor}{rgb}{0.000000,0.000000,0.000000}%
\pgfsetstrokecolor{textcolor}%
\pgfsetfillcolor{textcolor}%
\pgftext[x=4.218177in,y=2.447328in,left,base]{\color{textcolor}\rmfamily\fontsize{10.000000}{12.000000}\selectfont CPU}%
\end{pgfscope}%
\begin{pgfscope}%
\pgfsetrectcap%
\pgfsetroundjoin%
\pgfsetlinewidth{1.505625pt}%
\definecolor{currentstroke}{rgb}{1.000000,0.498039,0.054902}%
\pgfsetstrokecolor{currentstroke}%
\pgfsetdash{}{0pt}%
\pgfpathmoveto{\pgfqpoint{3.829289in}{2.302266in}}%
\pgfpathlineto{\pgfqpoint{4.107066in}{2.302266in}}%
\pgfusepath{stroke}%
\end{pgfscope}%
\begin{pgfscope}%
\definecolor{textcolor}{rgb}{0.000000,0.000000,0.000000}%
\pgfsetstrokecolor{textcolor}%
\pgfsetfillcolor{textcolor}%
\pgftext[x=4.218177in,y=2.253655in,left,base]{\color{textcolor}\rmfamily\fontsize{10.000000}{12.000000}\selectfont Zone Raspberry Pi}%
\end{pgfscope}%
\begin{pgfscope}%
\pgfsetrectcap%
\pgfsetroundjoin%
\pgfsetlinewidth{1.505625pt}%
\definecolor{currentstroke}{rgb}{0.172549,0.627451,0.172549}%
\pgfsetstrokecolor{currentstroke}%
\pgfsetdash{}{0pt}%
\pgfpathmoveto{\pgfqpoint{3.829289in}{2.108593in}}%
\pgfpathlineto{\pgfqpoint{4.107066in}{2.108593in}}%
\pgfusepath{stroke}%
\end{pgfscope}%
\begin{pgfscope}%
\definecolor{textcolor}{rgb}{0.000000,0.000000,0.000000}%
\pgfsetstrokecolor{textcolor}%
\pgfsetfillcolor{textcolor}%
\pgftext[x=4.218177in,y=2.059982in,left,base]{\color{textcolor}\rmfamily\fontsize{10.000000}{12.000000}\selectfont Zone step-down}%
\end{pgfscope}%
\end{pgfpicture}%
\makeatother%
\endgroup%

  \label{fig:test_4}
  \vspace{-0.2cm}
  \caption{\textbf{Test 4 :} Évolution de la température dans les tests avec step-down et blindage magnétique mais sans ventilateur}
\end{figure}


\begin{table}[ht]
  \centering
%\resizebox{\textwidth}{!}{%
\begin{tabular}{@{}lllll@{}}
\toprule
                                      & \textbf{Test 1}                                                        & \textbf{Test 2}                                                         & \textbf{Test 3}                                                          & \textbf{Test 4}                                                     \\ \midrule
\textbf{Protocol:SendPackets Error}    & \begin{tabular}[c]{@{}l@{}}min : 6\\ max : 9\\ moy : 7,33\end{tabular} & \begin{tabular}[c]{@{}l@{}}min : 2\\ max : 3 \\ moy : 2,33\end{tabular} & \begin{tabular}[c]{@{}l@{}}min : 8\\ max : 13\\ moy : 10,33\end{tabular} & \begin{tabular}[c]{@{}l@{}}min : 2\\ max : 6\\ moy : 4\end{tabular} \\ \midrule
\textbf{Receive:parseMessage Warning} & \begin{tabular}[c]{@{}l@{}}min : 4\\ max : 5\\ moy : 4,33\end{tabular} & \begin{tabular}[c]{@{}l@{}}min : 0\\ max : 0\\ moy : 0\end{tabular}     & \begin{tabular}[c]{@{}l@{}}min : 4\\ max : 9\\ moy : 6,33\end{tabular}   & \begin{tabular}[c]{@{}l@{}}min : 0\\ max : 0\\ moy : 0\end{tabular} \\ \bottomrule
\end{tabular}%
%}
\caption{Statistiques loggés par le SDK de Thingstream pour les 4 premiers tests}
\label{tab:res2}
\end{table}

~

\noindent
Le troisième test montre que le ventilateur a un grand impact sur la température à l'intérieur du boîtier. En effet, la température du CPU chute entre 14°C à 15°C après l'ajout du ventilateur. De plus, la température dans le boîtier devient plus homogène. Il est tout à fait prévisible que la température du XL4015 diminue aussi avec l'ajout du ventilateur, ce qui est bénéfique pour la durée de vie du composant. Cependant, le nombre élevé de warnings et d'erreurs pour le troisième test indique que la température n'a pas d'impact sur les performances du SIM800L. (tableau \ref{tab:res2})

~

\noindent
Les résultats des tests 2 et 4 permettent de conclure que le champ magnétique produit par le step-down est la cause des défaillances au niveau du SIM800L. En effet, lors de ces deux tests, le Raspberry Pi et le modem n'ont plus eu de problème à échanger des messages. De plus, le nombre d'erreurs pendant de la transmission de paquets diminue aussi considérablement.

~

\noindent
Les résultats des tests ci-dessous permettent de tirer les conclusions suivantes :

~

\begin{itemize}
  \item Le prochain boîtier devrait avoir un ventilateur intégré afin de mieux gérer la température des composants.
  \item Dans la mesure du possible, le step-down doit être éloigné du SIM800L. Un blindage magnétique peut être utilisé en supplément lorsque la distance entre les deux composants n'est pas suffisante.
\end{itemize}

~

\noindent
Les deux points présentés ci-dessous définissent la base du nouveau boîtier qui a été spécifiquement créé pour loger les différents composants de ce projet. Le boîter sera présenté dans la section suivante.


\subsection{Résultat}
\label{sec:protores}

\noindent
À la suite des choix présentés ci-dessus, les principaux composants du prototype sont :

~

\begin{itemize}
  \item Raspberry Pi 2B
  \item SIM800L
  \item XL4015
  \item Circuit imprimé réalisé l'année précédente
\end{itemize}

~

\noindent
 Comme expliqué précédemment, une boîte a été conçue pour abriter les composants. Pour ce faire, le boîtier a d'abord été modélisé en 3D sur Fusion 360. Ensuite, il a été produit en ayant recours à l'impression en 3D. Ce procédé de fabrication a permis de concevoir le boîtier en plus ou moins une demi-journée.

 ~

 \noindent
Dans cette nouvelle boîte, le step-down a été éloigné le plus possible du SIM800L et une paroi entre les deux composants a aussi été ajoutée. Un matériau d'isolation magnétique peut être additionné sur la paroi afin de limiter l'impact du champ magnétique sur le modem. Un ventilateur de 10mm de diamètre a également été ajouté afin d'avoir une meilleure solution de refroidissement. Veuillez noter que le ventilateur est placé de façon à aspirer l'air vers l'intérieur. Le flux d'air quittant le ventilateur est turbulent et cela permet qu'il se propage dans un peu près toutes les zones à l'intérieur de la boîte \cite{cooling}. Le résultat de la modélisation 3D est visible sur les figures \ref{fig:box1}, \ref{fig:box2} et \ref{fig:box3}.



\begin{figure}[ht!]
  \centering
  \includegraphics[scale=0.4]{img/el_prototype/box1}
  \caption{Vue des faces avant et supérieure avec le couvercle}
  \label{fig:box1}
\end{figure}


\begin{figure}[ht!]
  \centering
  \includegraphics[scale=0.4]{img/el_prototype/box2}
  \caption{Vue de la face supérieure sans le couvercle}
  \label{fig:box2}
\end{figure}

\begin{figure}[ht!]
  \centering
  \includegraphics[scale=0.4]{img/el_prototype/box3}
  \caption{Vue de la face arrière sans le couvercle}
  \label{fig:box3}
\end{figure}

~

\noindent
Afin de valider le boîtier, le protocole de test présenté dans la section \ref{sec:prototests} a été à nouveau employé. Dans le premier test (test 5), aucun matériau de blindage magnétique n'a été utilisé. Toutefois, la distance accrue entre le step-down et le SIM800L suffit déjà pour que la communication entre le Raspberry et le modem ait lieu sans aucun problème. Ajouter des feuilles de papier d'aluminium sur la paroi permet de diminuer encore plus l'impact du champ magnétique puisque le nombre d'erreurs est divisé par deux. (voir test 6 sur \ref{tab:res22}). En particulier, cette dernière solution s'est montrée presque aussi efficace qu'éloigner le step-down 50 cm du boîtier.

~

\noindent
La solution de refroidissement de ce boîtier est également très performante puisque la température dans les deux zones (Raspberry Pi et step-down) est tout à fait identique. Ceci montre donc que l'importance du boîtier ne peut pas être négligée. En effet, avoir un boîtier adapté aux besoins des composants est essentiel afin d'assurer le bon fonctionnement de ces derniers et de prolonger leur durée de vie.


\begin{figure}[ht!]
  \centering
  %% Creator: Matplotlib, PGF backend
%%
%% To include the figure in your LaTeX document, write
%%   \input{<filename>.pgf}
%%
%% Make sure the required packages are loaded in your preamble
%%   \usepackage{pgf}
%%
%% Figures using additional raster images can only be included by \input if
%% they are in the same directory as the main LaTeX file. For loading figures
%% from other directories you can use the `import` package
%%   \usepackage{import}
%% and then include the figures with
%%   \import{<path to file>}{<filename>.pgf}
%%
%% Matplotlib used the following preamble
%%
\begingroup%
\makeatletter%
\begin{pgfpicture}%
\pgfpathrectangle{\pgfpointorigin}{\pgfqpoint{5.590556in}{4.068988in}}%
\pgfusepath{use as bounding box, clip}%
\begin{pgfscope}%
\pgfsetbuttcap%
\pgfsetmiterjoin%
\definecolor{currentfill}{rgb}{1.000000,1.000000,1.000000}%
\pgfsetfillcolor{currentfill}%
\pgfsetlinewidth{0.000000pt}%
\definecolor{currentstroke}{rgb}{1.000000,1.000000,1.000000}%
\pgfsetstrokecolor{currentstroke}%
\pgfsetdash{}{0pt}%
\pgfpathmoveto{\pgfqpoint{0.000000in}{0.000000in}}%
\pgfpathlineto{\pgfqpoint{5.590556in}{0.000000in}}%
\pgfpathlineto{\pgfqpoint{5.590556in}{4.068988in}}%
\pgfpathlineto{\pgfqpoint{0.000000in}{4.068988in}}%
\pgfpathclose%
\pgfusepath{fill}%
\end{pgfscope}%
\begin{pgfscope}%
\pgfsetbuttcap%
\pgfsetmiterjoin%
\definecolor{currentfill}{rgb}{1.000000,1.000000,1.000000}%
\pgfsetfillcolor{currentfill}%
\pgfsetlinewidth{0.000000pt}%
\definecolor{currentstroke}{rgb}{0.000000,0.000000,0.000000}%
\pgfsetstrokecolor{currentstroke}%
\pgfsetstrokeopacity{0.000000}%
\pgfsetdash{}{0pt}%
\pgfpathmoveto{\pgfqpoint{0.530556in}{0.656763in}}%
\pgfpathlineto{\pgfqpoint{5.490556in}{0.656763in}}%
\pgfpathlineto{\pgfqpoint{5.490556in}{3.920763in}}%
\pgfpathlineto{\pgfqpoint{0.530556in}{3.920763in}}%
\pgfpathclose%
\pgfusepath{fill}%
\end{pgfscope}%
\begin{pgfscope}%
\pgfsetbuttcap%
\pgfsetroundjoin%
\definecolor{currentfill}{rgb}{0.000000,0.000000,0.000000}%
\pgfsetfillcolor{currentfill}%
\pgfsetlinewidth{0.803000pt}%
\definecolor{currentstroke}{rgb}{0.000000,0.000000,0.000000}%
\pgfsetstrokecolor{currentstroke}%
\pgfsetdash{}{0pt}%
\pgfsys@defobject{currentmarker}{\pgfqpoint{0.000000in}{-0.048611in}}{\pgfqpoint{0.000000in}{0.000000in}}{%
\pgfpathmoveto{\pgfqpoint{0.000000in}{0.000000in}}%
\pgfpathlineto{\pgfqpoint{0.000000in}{-0.048611in}}%
\pgfusepath{stroke,fill}%
}%
\begin{pgfscope}%
\pgfsys@transformshift{0.530556in}{0.656763in}%
\pgfsys@useobject{currentmarker}{}%
\end{pgfscope}%
\end{pgfscope}%
\begin{pgfscope}%
\definecolor{textcolor}{rgb}{0.000000,0.000000,0.000000}%
\pgfsetstrokecolor{textcolor}%
\pgfsetfillcolor{textcolor}%
\pgftext[x=0.243078in,y=0.317832in,left,base,rotate=30.000000]{\color{textcolor}\rmfamily\fontsize{10.000000}{12.000000}\selectfont 00:00}%
\end{pgfscope}%
\begin{pgfscope}%
\pgfsetbuttcap%
\pgfsetroundjoin%
\definecolor{currentfill}{rgb}{0.000000,0.000000,0.000000}%
\pgfsetfillcolor{currentfill}%
\pgfsetlinewidth{0.803000pt}%
\definecolor{currentstroke}{rgb}{0.000000,0.000000,0.000000}%
\pgfsetstrokecolor{currentstroke}%
\pgfsetdash{}{0pt}%
\pgfsys@defobject{currentmarker}{\pgfqpoint{0.000000in}{-0.048611in}}{\pgfqpoint{0.000000in}{0.000000in}}{%
\pgfpathmoveto{\pgfqpoint{0.000000in}{0.000000in}}%
\pgfpathlineto{\pgfqpoint{0.000000in}{-0.048611in}}%
\pgfusepath{stroke,fill}%
}%
\begin{pgfscope}%
\pgfsys@transformshift{1.522519in}{0.656763in}%
\pgfsys@useobject{currentmarker}{}%
\end{pgfscope}%
\end{pgfscope}%
\begin{pgfscope}%
\definecolor{textcolor}{rgb}{0.000000,0.000000,0.000000}%
\pgfsetstrokecolor{textcolor}%
\pgfsetfillcolor{textcolor}%
\pgftext[x=1.235041in,y=0.317832in,left,base,rotate=30.000000]{\color{textcolor}\rmfamily\fontsize{10.000000}{12.000000}\selectfont 01:00}%
\end{pgfscope}%
\begin{pgfscope}%
\pgfsetbuttcap%
\pgfsetroundjoin%
\definecolor{currentfill}{rgb}{0.000000,0.000000,0.000000}%
\pgfsetfillcolor{currentfill}%
\pgfsetlinewidth{0.803000pt}%
\definecolor{currentstroke}{rgb}{0.000000,0.000000,0.000000}%
\pgfsetstrokecolor{currentstroke}%
\pgfsetdash{}{0pt}%
\pgfsys@defobject{currentmarker}{\pgfqpoint{0.000000in}{-0.048611in}}{\pgfqpoint{0.000000in}{0.000000in}}{%
\pgfpathmoveto{\pgfqpoint{0.000000in}{0.000000in}}%
\pgfpathlineto{\pgfqpoint{0.000000in}{-0.048611in}}%
\pgfusepath{stroke,fill}%
}%
\begin{pgfscope}%
\pgfsys@transformshift{2.514481in}{0.656763in}%
\pgfsys@useobject{currentmarker}{}%
\end{pgfscope}%
\end{pgfscope}%
\begin{pgfscope}%
\definecolor{textcolor}{rgb}{0.000000,0.000000,0.000000}%
\pgfsetstrokecolor{textcolor}%
\pgfsetfillcolor{textcolor}%
\pgftext[x=2.227003in,y=0.317832in,left,base,rotate=30.000000]{\color{textcolor}\rmfamily\fontsize{10.000000}{12.000000}\selectfont 02:00}%
\end{pgfscope}%
\begin{pgfscope}%
\pgfsetbuttcap%
\pgfsetroundjoin%
\definecolor{currentfill}{rgb}{0.000000,0.000000,0.000000}%
\pgfsetfillcolor{currentfill}%
\pgfsetlinewidth{0.803000pt}%
\definecolor{currentstroke}{rgb}{0.000000,0.000000,0.000000}%
\pgfsetstrokecolor{currentstroke}%
\pgfsetdash{}{0pt}%
\pgfsys@defobject{currentmarker}{\pgfqpoint{0.000000in}{-0.048611in}}{\pgfqpoint{0.000000in}{0.000000in}}{%
\pgfpathmoveto{\pgfqpoint{0.000000in}{0.000000in}}%
\pgfpathlineto{\pgfqpoint{0.000000in}{-0.048611in}}%
\pgfusepath{stroke,fill}%
}%
\begin{pgfscope}%
\pgfsys@transformshift{3.506444in}{0.656763in}%
\pgfsys@useobject{currentmarker}{}%
\end{pgfscope}%
\end{pgfscope}%
\begin{pgfscope}%
\definecolor{textcolor}{rgb}{0.000000,0.000000,0.000000}%
\pgfsetstrokecolor{textcolor}%
\pgfsetfillcolor{textcolor}%
\pgftext[x=3.218966in,y=0.317832in,left,base,rotate=30.000000]{\color{textcolor}\rmfamily\fontsize{10.000000}{12.000000}\selectfont 03:00}%
\end{pgfscope}%
\begin{pgfscope}%
\pgfsetbuttcap%
\pgfsetroundjoin%
\definecolor{currentfill}{rgb}{0.000000,0.000000,0.000000}%
\pgfsetfillcolor{currentfill}%
\pgfsetlinewidth{0.803000pt}%
\definecolor{currentstroke}{rgb}{0.000000,0.000000,0.000000}%
\pgfsetstrokecolor{currentstroke}%
\pgfsetdash{}{0pt}%
\pgfsys@defobject{currentmarker}{\pgfqpoint{0.000000in}{-0.048611in}}{\pgfqpoint{0.000000in}{0.000000in}}{%
\pgfpathmoveto{\pgfqpoint{0.000000in}{0.000000in}}%
\pgfpathlineto{\pgfqpoint{0.000000in}{-0.048611in}}%
\pgfusepath{stroke,fill}%
}%
\begin{pgfscope}%
\pgfsys@transformshift{4.498407in}{0.656763in}%
\pgfsys@useobject{currentmarker}{}%
\end{pgfscope}%
\end{pgfscope}%
\begin{pgfscope}%
\definecolor{textcolor}{rgb}{0.000000,0.000000,0.000000}%
\pgfsetstrokecolor{textcolor}%
\pgfsetfillcolor{textcolor}%
\pgftext[x=4.210929in,y=0.317832in,left,base,rotate=30.000000]{\color{textcolor}\rmfamily\fontsize{10.000000}{12.000000}\selectfont 04:00}%
\end{pgfscope}%
\begin{pgfscope}%
\pgfsetbuttcap%
\pgfsetroundjoin%
\definecolor{currentfill}{rgb}{0.000000,0.000000,0.000000}%
\pgfsetfillcolor{currentfill}%
\pgfsetlinewidth{0.803000pt}%
\definecolor{currentstroke}{rgb}{0.000000,0.000000,0.000000}%
\pgfsetstrokecolor{currentstroke}%
\pgfsetdash{}{0pt}%
\pgfsys@defobject{currentmarker}{\pgfqpoint{0.000000in}{-0.048611in}}{\pgfqpoint{0.000000in}{0.000000in}}{%
\pgfpathmoveto{\pgfqpoint{0.000000in}{0.000000in}}%
\pgfpathlineto{\pgfqpoint{0.000000in}{-0.048611in}}%
\pgfusepath{stroke,fill}%
}%
\begin{pgfscope}%
\pgfsys@transformshift{5.490369in}{0.656763in}%
\pgfsys@useobject{currentmarker}{}%
\end{pgfscope}%
\end{pgfscope}%
\begin{pgfscope}%
\definecolor{textcolor}{rgb}{0.000000,0.000000,0.000000}%
\pgfsetstrokecolor{textcolor}%
\pgfsetfillcolor{textcolor}%
\pgftext[x=5.202891in,y=0.317832in,left,base,rotate=30.000000]{\color{textcolor}\rmfamily\fontsize{10.000000}{12.000000}\selectfont 05:00}%
\end{pgfscope}%
\begin{pgfscope}%
\definecolor{textcolor}{rgb}{0.000000,0.000000,0.000000}%
\pgfsetstrokecolor{textcolor}%
\pgfsetfillcolor{textcolor}%
\pgftext[x=3.010556in,y=0.238889in,,top]{\color{textcolor}\rmfamily\fontsize{10.000000}{12.000000}\selectfont Temps (hh:mm)}%
\end{pgfscope}%
\begin{pgfscope}%
\pgfsetbuttcap%
\pgfsetroundjoin%
\definecolor{currentfill}{rgb}{0.000000,0.000000,0.000000}%
\pgfsetfillcolor{currentfill}%
\pgfsetlinewidth{0.803000pt}%
\definecolor{currentstroke}{rgb}{0.000000,0.000000,0.000000}%
\pgfsetstrokecolor{currentstroke}%
\pgfsetdash{}{0pt}%
\pgfsys@defobject{currentmarker}{\pgfqpoint{-0.048611in}{0.000000in}}{\pgfqpoint{0.000000in}{0.000000in}}{%
\pgfpathmoveto{\pgfqpoint{0.000000in}{0.000000in}}%
\pgfpathlineto{\pgfqpoint{-0.048611in}{0.000000in}}%
\pgfusepath{stroke,fill}%
}%
\begin{pgfscope}%
\pgfsys@transformshift{0.530556in}{0.656763in}%
\pgfsys@useobject{currentmarker}{}%
\end{pgfscope}%
\end{pgfscope}%
\begin{pgfscope}%
\definecolor{textcolor}{rgb}{0.000000,0.000000,0.000000}%
\pgfsetstrokecolor{textcolor}%
\pgfsetfillcolor{textcolor}%
\pgftext[x=0.294444in,y=0.608538in,left,base]{\color{textcolor}\rmfamily\fontsize{10.000000}{12.000000}\selectfont \(\displaystyle 20\)}%
\end{pgfscope}%
\begin{pgfscope}%
\pgfsetbuttcap%
\pgfsetroundjoin%
\definecolor{currentfill}{rgb}{0.000000,0.000000,0.000000}%
\pgfsetfillcolor{currentfill}%
\pgfsetlinewidth{0.803000pt}%
\definecolor{currentstroke}{rgb}{0.000000,0.000000,0.000000}%
\pgfsetstrokecolor{currentstroke}%
\pgfsetdash{}{0pt}%
\pgfsys@defobject{currentmarker}{\pgfqpoint{-0.048611in}{0.000000in}}{\pgfqpoint{0.000000in}{0.000000in}}{%
\pgfpathmoveto{\pgfqpoint{0.000000in}{0.000000in}}%
\pgfpathlineto{\pgfqpoint{-0.048611in}{0.000000in}}%
\pgfusepath{stroke,fill}%
}%
\begin{pgfscope}%
\pgfsys@transformshift{0.530556in}{1.123049in}%
\pgfsys@useobject{currentmarker}{}%
\end{pgfscope}%
\end{pgfscope}%
\begin{pgfscope}%
\definecolor{textcolor}{rgb}{0.000000,0.000000,0.000000}%
\pgfsetstrokecolor{textcolor}%
\pgfsetfillcolor{textcolor}%
\pgftext[x=0.294444in,y=1.074823in,left,base]{\color{textcolor}\rmfamily\fontsize{10.000000}{12.000000}\selectfont \(\displaystyle 25\)}%
\end{pgfscope}%
\begin{pgfscope}%
\pgfsetbuttcap%
\pgfsetroundjoin%
\definecolor{currentfill}{rgb}{0.000000,0.000000,0.000000}%
\pgfsetfillcolor{currentfill}%
\pgfsetlinewidth{0.803000pt}%
\definecolor{currentstroke}{rgb}{0.000000,0.000000,0.000000}%
\pgfsetstrokecolor{currentstroke}%
\pgfsetdash{}{0pt}%
\pgfsys@defobject{currentmarker}{\pgfqpoint{-0.048611in}{0.000000in}}{\pgfqpoint{0.000000in}{0.000000in}}{%
\pgfpathmoveto{\pgfqpoint{0.000000in}{0.000000in}}%
\pgfpathlineto{\pgfqpoint{-0.048611in}{0.000000in}}%
\pgfusepath{stroke,fill}%
}%
\begin{pgfscope}%
\pgfsys@transformshift{0.530556in}{1.589334in}%
\pgfsys@useobject{currentmarker}{}%
\end{pgfscope}%
\end{pgfscope}%
\begin{pgfscope}%
\definecolor{textcolor}{rgb}{0.000000,0.000000,0.000000}%
\pgfsetstrokecolor{textcolor}%
\pgfsetfillcolor{textcolor}%
\pgftext[x=0.294444in,y=1.541109in,left,base]{\color{textcolor}\rmfamily\fontsize{10.000000}{12.000000}\selectfont \(\displaystyle 30\)}%
\end{pgfscope}%
\begin{pgfscope}%
\pgfsetbuttcap%
\pgfsetroundjoin%
\definecolor{currentfill}{rgb}{0.000000,0.000000,0.000000}%
\pgfsetfillcolor{currentfill}%
\pgfsetlinewidth{0.803000pt}%
\definecolor{currentstroke}{rgb}{0.000000,0.000000,0.000000}%
\pgfsetstrokecolor{currentstroke}%
\pgfsetdash{}{0pt}%
\pgfsys@defobject{currentmarker}{\pgfqpoint{-0.048611in}{0.000000in}}{\pgfqpoint{0.000000in}{0.000000in}}{%
\pgfpathmoveto{\pgfqpoint{0.000000in}{0.000000in}}%
\pgfpathlineto{\pgfqpoint{-0.048611in}{0.000000in}}%
\pgfusepath{stroke,fill}%
}%
\begin{pgfscope}%
\pgfsys@transformshift{0.530556in}{2.055620in}%
\pgfsys@useobject{currentmarker}{}%
\end{pgfscope}%
\end{pgfscope}%
\begin{pgfscope}%
\definecolor{textcolor}{rgb}{0.000000,0.000000,0.000000}%
\pgfsetstrokecolor{textcolor}%
\pgfsetfillcolor{textcolor}%
\pgftext[x=0.294444in,y=2.007395in,left,base]{\color{textcolor}\rmfamily\fontsize{10.000000}{12.000000}\selectfont \(\displaystyle 35\)}%
\end{pgfscope}%
\begin{pgfscope}%
\pgfsetbuttcap%
\pgfsetroundjoin%
\definecolor{currentfill}{rgb}{0.000000,0.000000,0.000000}%
\pgfsetfillcolor{currentfill}%
\pgfsetlinewidth{0.803000pt}%
\definecolor{currentstroke}{rgb}{0.000000,0.000000,0.000000}%
\pgfsetstrokecolor{currentstroke}%
\pgfsetdash{}{0pt}%
\pgfsys@defobject{currentmarker}{\pgfqpoint{-0.048611in}{0.000000in}}{\pgfqpoint{0.000000in}{0.000000in}}{%
\pgfpathmoveto{\pgfqpoint{0.000000in}{0.000000in}}%
\pgfpathlineto{\pgfqpoint{-0.048611in}{0.000000in}}%
\pgfusepath{stroke,fill}%
}%
\begin{pgfscope}%
\pgfsys@transformshift{0.530556in}{2.521906in}%
\pgfsys@useobject{currentmarker}{}%
\end{pgfscope}%
\end{pgfscope}%
\begin{pgfscope}%
\definecolor{textcolor}{rgb}{0.000000,0.000000,0.000000}%
\pgfsetstrokecolor{textcolor}%
\pgfsetfillcolor{textcolor}%
\pgftext[x=0.294444in,y=2.473680in,left,base]{\color{textcolor}\rmfamily\fontsize{10.000000}{12.000000}\selectfont \(\displaystyle 40\)}%
\end{pgfscope}%
\begin{pgfscope}%
\pgfsetbuttcap%
\pgfsetroundjoin%
\definecolor{currentfill}{rgb}{0.000000,0.000000,0.000000}%
\pgfsetfillcolor{currentfill}%
\pgfsetlinewidth{0.803000pt}%
\definecolor{currentstroke}{rgb}{0.000000,0.000000,0.000000}%
\pgfsetstrokecolor{currentstroke}%
\pgfsetdash{}{0pt}%
\pgfsys@defobject{currentmarker}{\pgfqpoint{-0.048611in}{0.000000in}}{\pgfqpoint{0.000000in}{0.000000in}}{%
\pgfpathmoveto{\pgfqpoint{0.000000in}{0.000000in}}%
\pgfpathlineto{\pgfqpoint{-0.048611in}{0.000000in}}%
\pgfusepath{stroke,fill}%
}%
\begin{pgfscope}%
\pgfsys@transformshift{0.530556in}{2.988191in}%
\pgfsys@useobject{currentmarker}{}%
\end{pgfscope}%
\end{pgfscope}%
\begin{pgfscope}%
\definecolor{textcolor}{rgb}{0.000000,0.000000,0.000000}%
\pgfsetstrokecolor{textcolor}%
\pgfsetfillcolor{textcolor}%
\pgftext[x=0.294444in,y=2.939966in,left,base]{\color{textcolor}\rmfamily\fontsize{10.000000}{12.000000}\selectfont \(\displaystyle 45\)}%
\end{pgfscope}%
\begin{pgfscope}%
\pgfsetbuttcap%
\pgfsetroundjoin%
\definecolor{currentfill}{rgb}{0.000000,0.000000,0.000000}%
\pgfsetfillcolor{currentfill}%
\pgfsetlinewidth{0.803000pt}%
\definecolor{currentstroke}{rgb}{0.000000,0.000000,0.000000}%
\pgfsetstrokecolor{currentstroke}%
\pgfsetdash{}{0pt}%
\pgfsys@defobject{currentmarker}{\pgfqpoint{-0.048611in}{0.000000in}}{\pgfqpoint{0.000000in}{0.000000in}}{%
\pgfpathmoveto{\pgfqpoint{0.000000in}{0.000000in}}%
\pgfpathlineto{\pgfqpoint{-0.048611in}{0.000000in}}%
\pgfusepath{stroke,fill}%
}%
\begin{pgfscope}%
\pgfsys@transformshift{0.530556in}{3.454477in}%
\pgfsys@useobject{currentmarker}{}%
\end{pgfscope}%
\end{pgfscope}%
\begin{pgfscope}%
\definecolor{textcolor}{rgb}{0.000000,0.000000,0.000000}%
\pgfsetstrokecolor{textcolor}%
\pgfsetfillcolor{textcolor}%
\pgftext[x=0.294444in,y=3.406252in,left,base]{\color{textcolor}\rmfamily\fontsize{10.000000}{12.000000}\selectfont \(\displaystyle 50\)}%
\end{pgfscope}%
\begin{pgfscope}%
\pgfsetbuttcap%
\pgfsetroundjoin%
\definecolor{currentfill}{rgb}{0.000000,0.000000,0.000000}%
\pgfsetfillcolor{currentfill}%
\pgfsetlinewidth{0.803000pt}%
\definecolor{currentstroke}{rgb}{0.000000,0.000000,0.000000}%
\pgfsetstrokecolor{currentstroke}%
\pgfsetdash{}{0pt}%
\pgfsys@defobject{currentmarker}{\pgfqpoint{-0.048611in}{0.000000in}}{\pgfqpoint{0.000000in}{0.000000in}}{%
\pgfpathmoveto{\pgfqpoint{0.000000in}{0.000000in}}%
\pgfpathlineto{\pgfqpoint{-0.048611in}{0.000000in}}%
\pgfusepath{stroke,fill}%
}%
\begin{pgfscope}%
\pgfsys@transformshift{0.530556in}{3.920763in}%
\pgfsys@useobject{currentmarker}{}%
\end{pgfscope}%
\end{pgfscope}%
\begin{pgfscope}%
\definecolor{textcolor}{rgb}{0.000000,0.000000,0.000000}%
\pgfsetstrokecolor{textcolor}%
\pgfsetfillcolor{textcolor}%
\pgftext[x=0.294444in,y=3.872538in,left,base]{\color{textcolor}\rmfamily\fontsize{10.000000}{12.000000}\selectfont \(\displaystyle 55\)}%
\end{pgfscope}%
\begin{pgfscope}%
\definecolor{textcolor}{rgb}{0.000000,0.000000,0.000000}%
\pgfsetstrokecolor{textcolor}%
\pgfsetfillcolor{textcolor}%
\pgftext[x=0.238889in,y=2.288763in,,bottom,rotate=90.000000]{\color{textcolor}\rmfamily\fontsize{10.000000}{12.000000}\selectfont Température (\textdegree{}C)}%
\end{pgfscope}%
\begin{pgfscope}%
\pgfpathrectangle{\pgfqpoint{0.530556in}{0.656763in}}{\pgfqpoint{4.960000in}{3.264000in}}%
\pgfusepath{clip}%
\pgfsetrectcap%
\pgfsetroundjoin%
\pgfsetlinewidth{1.505625pt}%
\definecolor{currentstroke}{rgb}{0.121569,0.466667,0.705882}%
\pgfsetstrokecolor{currentstroke}%
\pgfsetdash{}{0pt}%
\pgfpathmoveto{\pgfqpoint{0.538832in}{1.912626in}}%
\pgfpathlineto{\pgfqpoint{0.547107in}{1.946820in}}%
\pgfpathlineto{\pgfqpoint{0.555385in}{1.993449in}}%
\pgfpathlineto{\pgfqpoint{0.563663in}{1.993449in}}%
\pgfpathlineto{\pgfqpoint{0.571939in}{1.977906in}}%
\pgfpathlineto{\pgfqpoint{0.580217in}{1.977906in}}%
\pgfpathlineto{\pgfqpoint{0.596772in}{2.008991in}}%
\pgfpathlineto{\pgfqpoint{0.605042in}{2.012100in}}%
\pgfpathlineto{\pgfqpoint{0.613318in}{2.008991in}}%
\pgfpathlineto{\pgfqpoint{0.621592in}{2.043186in}}%
\pgfpathlineto{\pgfqpoint{0.629865in}{2.043186in}}%
\pgfpathlineto{\pgfqpoint{0.638139in}{2.061837in}}%
\pgfpathlineto{\pgfqpoint{0.646412in}{2.061837in}}%
\pgfpathlineto{\pgfqpoint{0.654687in}{2.027643in}}%
\pgfpathlineto{\pgfqpoint{0.662960in}{2.080489in}}%
\pgfpathlineto{\pgfqpoint{0.671232in}{2.043186in}}%
\pgfpathlineto{\pgfqpoint{0.679507in}{2.046294in}}%
\pgfpathlineto{\pgfqpoint{0.687780in}{2.027643in}}%
\pgfpathlineto{\pgfqpoint{0.696053in}{2.046294in}}%
\pgfpathlineto{\pgfqpoint{0.704327in}{2.061837in}}%
\pgfpathlineto{\pgfqpoint{0.712605in}{2.043186in}}%
\pgfpathlineto{\pgfqpoint{0.720880in}{2.046294in}}%
\pgfpathlineto{\pgfqpoint{0.729153in}{2.043186in}}%
\pgfpathlineto{\pgfqpoint{0.737425in}{2.080489in}}%
\pgfpathlineto{\pgfqpoint{0.753979in}{2.043186in}}%
\pgfpathlineto{\pgfqpoint{0.762255in}{2.080489in}}%
\pgfpathlineto{\pgfqpoint{0.770532in}{2.027643in}}%
\pgfpathlineto{\pgfqpoint{0.778805in}{2.061837in}}%
\pgfpathlineto{\pgfqpoint{0.787077in}{2.027643in}}%
\pgfpathlineto{\pgfqpoint{0.795351in}{2.061837in}}%
\pgfpathlineto{\pgfqpoint{0.803624in}{2.046294in}}%
\pgfpathlineto{\pgfqpoint{0.811897in}{2.027643in}}%
\pgfpathlineto{\pgfqpoint{0.820168in}{2.043186in}}%
\pgfpathlineto{\pgfqpoint{0.828443in}{2.043186in}}%
\pgfpathlineto{\pgfqpoint{0.836716in}{2.061837in}}%
\pgfpathlineto{\pgfqpoint{0.844990in}{2.043186in}}%
\pgfpathlineto{\pgfqpoint{0.853261in}{2.061837in}}%
\pgfpathlineto{\pgfqpoint{0.861536in}{2.043186in}}%
\pgfpathlineto{\pgfqpoint{0.869814in}{2.080489in}}%
\pgfpathlineto{\pgfqpoint{0.878087in}{2.061837in}}%
\pgfpathlineto{\pgfqpoint{0.894640in}{2.061837in}}%
\pgfpathlineto{\pgfqpoint{0.902918in}{2.027643in}}%
\pgfpathlineto{\pgfqpoint{0.911190in}{2.027643in}}%
\pgfpathlineto{\pgfqpoint{0.919466in}{2.061837in}}%
\pgfpathlineto{\pgfqpoint{0.927741in}{2.080489in}}%
\pgfpathlineto{\pgfqpoint{0.936017in}{2.061837in}}%
\pgfpathlineto{\pgfqpoint{0.944294in}{2.061837in}}%
\pgfpathlineto{\pgfqpoint{0.952572in}{2.096031in}}%
\pgfpathlineto{\pgfqpoint{0.960850in}{2.043186in}}%
\pgfpathlineto{\pgfqpoint{0.969127in}{2.043186in}}%
\pgfpathlineto{\pgfqpoint{0.985678in}{2.080489in}}%
\pgfpathlineto{\pgfqpoint{0.993958in}{2.080489in}}%
\pgfpathlineto{\pgfqpoint{1.002237in}{2.061837in}}%
\pgfpathlineto{\pgfqpoint{1.010514in}{2.061837in}}%
\pgfpathlineto{\pgfqpoint{1.018793in}{2.043186in}}%
\pgfpathlineto{\pgfqpoint{1.027071in}{2.080489in}}%
\pgfpathlineto{\pgfqpoint{1.035349in}{2.061837in}}%
\pgfpathlineto{\pgfqpoint{1.043627in}{2.080489in}}%
\pgfpathlineto{\pgfqpoint{1.051905in}{2.061837in}}%
\pgfpathlineto{\pgfqpoint{1.068461in}{2.061837in}}%
\pgfpathlineto{\pgfqpoint{1.076734in}{2.080489in}}%
\pgfpathlineto{\pgfqpoint{1.093279in}{2.080489in}}%
\pgfpathlineto{\pgfqpoint{1.101552in}{2.077380in}}%
\pgfpathlineto{\pgfqpoint{1.109830in}{2.080489in}}%
\pgfpathlineto{\pgfqpoint{1.176039in}{2.080489in}}%
\pgfpathlineto{\pgfqpoint{1.184316in}{2.046294in}}%
\pgfpathlineto{\pgfqpoint{1.192594in}{2.080489in}}%
\pgfpathlineto{\pgfqpoint{1.200871in}{2.096031in}}%
\pgfpathlineto{\pgfqpoint{1.209148in}{2.043186in}}%
\pgfpathlineto{\pgfqpoint{1.217426in}{2.008991in}}%
\pgfpathlineto{\pgfqpoint{1.225703in}{2.043186in}}%
\pgfpathlineto{\pgfqpoint{1.242258in}{2.080489in}}%
\pgfpathlineto{\pgfqpoint{1.250535in}{2.043186in}}%
\pgfpathlineto{\pgfqpoint{1.258813in}{2.080489in}}%
\pgfpathlineto{\pgfqpoint{1.267091in}{2.061837in}}%
\pgfpathlineto{\pgfqpoint{1.275367in}{2.077380in}}%
\pgfpathlineto{\pgfqpoint{1.283642in}{2.061837in}}%
\pgfpathlineto{\pgfqpoint{1.291917in}{2.061837in}}%
\pgfpathlineto{\pgfqpoint{1.308466in}{2.024534in}}%
\pgfpathlineto{\pgfqpoint{1.316741in}{2.046294in}}%
\pgfpathlineto{\pgfqpoint{1.325015in}{2.027643in}}%
\pgfpathlineto{\pgfqpoint{1.333290in}{2.043186in}}%
\pgfpathlineto{\pgfqpoint{1.341567in}{2.080489in}}%
\pgfpathlineto{\pgfqpoint{1.349845in}{2.024534in}}%
\pgfpathlineto{\pgfqpoint{1.358122in}{2.061837in}}%
\pgfpathlineto{\pgfqpoint{1.366399in}{1.993449in}}%
\pgfpathlineto{\pgfqpoint{1.374677in}{2.027643in}}%
\pgfpathlineto{\pgfqpoint{1.382954in}{2.012100in}}%
\pgfpathlineto{\pgfqpoint{1.391232in}{2.027643in}}%
\pgfpathlineto{\pgfqpoint{1.399510in}{2.046294in}}%
\pgfpathlineto{\pgfqpoint{1.407787in}{2.027643in}}%
\pgfpathlineto{\pgfqpoint{1.424344in}{2.027643in}}%
\pgfpathlineto{\pgfqpoint{1.432621in}{2.046294in}}%
\pgfpathlineto{\pgfqpoint{1.440897in}{2.046294in}}%
\pgfpathlineto{\pgfqpoint{1.449175in}{2.061837in}}%
\pgfpathlineto{\pgfqpoint{1.457452in}{2.046294in}}%
\pgfpathlineto{\pgfqpoint{1.465725in}{2.024534in}}%
\pgfpathlineto{\pgfqpoint{1.473999in}{2.027643in}}%
\pgfpathlineto{\pgfqpoint{1.482277in}{2.027643in}}%
\pgfpathlineto{\pgfqpoint{1.490551in}{2.043186in}}%
\pgfpathlineto{\pgfqpoint{1.498829in}{2.027643in}}%
\pgfpathlineto{\pgfqpoint{1.515383in}{2.027643in}}%
\pgfpathlineto{\pgfqpoint{1.523657in}{2.046294in}}%
\pgfpathlineto{\pgfqpoint{1.531935in}{2.061837in}}%
\pgfpathlineto{\pgfqpoint{1.540212in}{2.046294in}}%
\pgfpathlineto{\pgfqpoint{1.548489in}{2.043186in}}%
\pgfpathlineto{\pgfqpoint{1.556764in}{2.027643in}}%
\pgfpathlineto{\pgfqpoint{1.565040in}{2.027643in}}%
\pgfpathlineto{\pgfqpoint{1.573312in}{2.043186in}}%
\pgfpathlineto{\pgfqpoint{1.581586in}{2.027643in}}%
\pgfpathlineto{\pgfqpoint{1.589859in}{2.046294in}}%
\pgfpathlineto{\pgfqpoint{1.598137in}{2.024534in}}%
\pgfpathlineto{\pgfqpoint{1.606414in}{2.027643in}}%
\pgfpathlineto{\pgfqpoint{1.614691in}{2.046294in}}%
\pgfpathlineto{\pgfqpoint{1.622963in}{2.043186in}}%
\pgfpathlineto{\pgfqpoint{1.631240in}{2.027643in}}%
\pgfpathlineto{\pgfqpoint{1.639511in}{2.046294in}}%
\pgfpathlineto{\pgfqpoint{1.647788in}{2.027643in}}%
\pgfpathlineto{\pgfqpoint{1.656066in}{2.061837in}}%
\pgfpathlineto{\pgfqpoint{1.664344in}{2.043186in}}%
\pgfpathlineto{\pgfqpoint{1.672621in}{2.061837in}}%
\pgfpathlineto{\pgfqpoint{1.680899in}{2.027643in}}%
\pgfpathlineto{\pgfqpoint{1.689176in}{2.046294in}}%
\pgfpathlineto{\pgfqpoint{1.705726in}{2.046294in}}%
\pgfpathlineto{\pgfqpoint{1.714001in}{2.080489in}}%
\pgfpathlineto{\pgfqpoint{1.722277in}{2.043186in}}%
\pgfpathlineto{\pgfqpoint{1.730549in}{2.080489in}}%
\pgfpathlineto{\pgfqpoint{1.738826in}{2.080489in}}%
\pgfpathlineto{\pgfqpoint{1.747103in}{2.077380in}}%
\pgfpathlineto{\pgfqpoint{1.755375in}{2.061837in}}%
\pgfpathlineto{\pgfqpoint{1.763651in}{2.043186in}}%
\pgfpathlineto{\pgfqpoint{1.771925in}{2.077380in}}%
\pgfpathlineto{\pgfqpoint{1.780199in}{2.046294in}}%
\pgfpathlineto{\pgfqpoint{1.788477in}{2.061837in}}%
\pgfpathlineto{\pgfqpoint{1.796754in}{2.080489in}}%
\pgfpathlineto{\pgfqpoint{1.805031in}{2.061837in}}%
\pgfpathlineto{\pgfqpoint{1.813309in}{2.096031in}}%
\pgfpathlineto{\pgfqpoint{1.821586in}{2.027643in}}%
\pgfpathlineto{\pgfqpoint{1.829858in}{2.096031in}}%
\pgfpathlineto{\pgfqpoint{1.838130in}{2.046294in}}%
\pgfpathlineto{\pgfqpoint{1.846402in}{2.027643in}}%
\pgfpathlineto{\pgfqpoint{1.854679in}{2.061837in}}%
\pgfpathlineto{\pgfqpoint{1.862957in}{2.046294in}}%
\pgfpathlineto{\pgfqpoint{1.871232in}{2.061837in}}%
\pgfpathlineto{\pgfqpoint{1.896050in}{2.061837in}}%
\pgfpathlineto{\pgfqpoint{1.904328in}{2.030751in}}%
\pgfpathlineto{\pgfqpoint{1.912606in}{2.061837in}}%
\pgfpathlineto{\pgfqpoint{1.920884in}{2.027643in}}%
\pgfpathlineto{\pgfqpoint{1.929162in}{2.061837in}}%
\pgfpathlineto{\pgfqpoint{1.962273in}{2.061837in}}%
\pgfpathlineto{\pgfqpoint{1.970549in}{2.027643in}}%
\pgfpathlineto{\pgfqpoint{1.978827in}{2.027643in}}%
\pgfpathlineto{\pgfqpoint{1.987105in}{2.043186in}}%
\pgfpathlineto{\pgfqpoint{1.995382in}{2.046294in}}%
\pgfpathlineto{\pgfqpoint{2.003657in}{2.080489in}}%
\pgfpathlineto{\pgfqpoint{2.011930in}{2.027643in}}%
\pgfpathlineto{\pgfqpoint{2.020207in}{2.008991in}}%
\pgfpathlineto{\pgfqpoint{2.028485in}{2.061837in}}%
\pgfpathlineto{\pgfqpoint{2.036763in}{2.024534in}}%
\pgfpathlineto{\pgfqpoint{2.045039in}{2.027643in}}%
\pgfpathlineto{\pgfqpoint{2.053317in}{2.043186in}}%
\pgfpathlineto{\pgfqpoint{2.061595in}{2.080489in}}%
\pgfpathlineto{\pgfqpoint{2.069873in}{2.080489in}}%
\pgfpathlineto{\pgfqpoint{2.078150in}{2.043186in}}%
\pgfpathlineto{\pgfqpoint{2.086428in}{2.061837in}}%
\pgfpathlineto{\pgfqpoint{2.094706in}{2.027643in}}%
\pgfpathlineto{\pgfqpoint{2.102984in}{2.061837in}}%
\pgfpathlineto{\pgfqpoint{2.111262in}{2.080489in}}%
\pgfpathlineto{\pgfqpoint{2.127814in}{2.043186in}}%
\pgfpathlineto{\pgfqpoint{2.136091in}{2.096031in}}%
\pgfpathlineto{\pgfqpoint{2.144363in}{2.061837in}}%
\pgfpathlineto{\pgfqpoint{2.152641in}{2.080489in}}%
\pgfpathlineto{\pgfqpoint{2.160918in}{2.061837in}}%
\pgfpathlineto{\pgfqpoint{2.169195in}{2.080489in}}%
\pgfpathlineto{\pgfqpoint{2.177472in}{2.061837in}}%
\pgfpathlineto{\pgfqpoint{2.185749in}{2.080489in}}%
\pgfpathlineto{\pgfqpoint{2.194019in}{2.043186in}}%
\pgfpathlineto{\pgfqpoint{2.202297in}{2.061837in}}%
\pgfpathlineto{\pgfqpoint{2.210570in}{2.061837in}}%
\pgfpathlineto{\pgfqpoint{2.218848in}{2.080489in}}%
\pgfpathlineto{\pgfqpoint{2.227125in}{2.080489in}}%
\pgfpathlineto{\pgfqpoint{2.235403in}{2.061837in}}%
\pgfpathlineto{\pgfqpoint{2.243676in}{2.080489in}}%
\pgfpathlineto{\pgfqpoint{2.251948in}{2.061837in}}%
\pgfpathlineto{\pgfqpoint{2.276776in}{2.061837in}}%
\pgfpathlineto{\pgfqpoint{2.285050in}{2.080489in}}%
\pgfpathlineto{\pgfqpoint{2.293326in}{2.061837in}}%
\pgfpathlineto{\pgfqpoint{2.301603in}{2.080489in}}%
\pgfpathlineto{\pgfqpoint{2.309880in}{2.061837in}}%
\pgfpathlineto{\pgfqpoint{2.318158in}{2.080489in}}%
\pgfpathlineto{\pgfqpoint{2.326435in}{2.046294in}}%
\pgfpathlineto{\pgfqpoint{2.334712in}{2.061837in}}%
\pgfpathlineto{\pgfqpoint{2.342989in}{2.061837in}}%
\pgfpathlineto{\pgfqpoint{2.351266in}{2.080489in}}%
\pgfpathlineto{\pgfqpoint{2.359544in}{2.043186in}}%
\pgfpathlineto{\pgfqpoint{2.376092in}{2.080489in}}%
\pgfpathlineto{\pgfqpoint{2.384369in}{2.080489in}}%
\pgfpathlineto{\pgfqpoint{2.392641in}{2.096031in}}%
\pgfpathlineto{\pgfqpoint{2.400918in}{2.080489in}}%
\pgfpathlineto{\pgfqpoint{2.409192in}{2.061837in}}%
\pgfpathlineto{\pgfqpoint{2.417467in}{2.080489in}}%
\pgfpathlineto{\pgfqpoint{2.425740in}{2.080489in}}%
\pgfpathlineto{\pgfqpoint{2.434015in}{2.096031in}}%
\pgfpathlineto{\pgfqpoint{2.442290in}{2.080489in}}%
\pgfpathlineto{\pgfqpoint{2.450567in}{2.046294in}}%
\pgfpathlineto{\pgfqpoint{2.458843in}{2.080489in}}%
\pgfpathlineto{\pgfqpoint{2.467121in}{2.061837in}}%
\pgfpathlineto{\pgfqpoint{2.475392in}{2.061837in}}%
\pgfpathlineto{\pgfqpoint{2.483666in}{2.043186in}}%
\pgfpathlineto{\pgfqpoint{2.491938in}{2.080489in}}%
\pgfpathlineto{\pgfqpoint{2.500209in}{2.080489in}}%
\pgfpathlineto{\pgfqpoint{2.508488in}{2.061837in}}%
\pgfpathlineto{\pgfqpoint{2.525040in}{2.061837in}}%
\pgfpathlineto{\pgfqpoint{2.533319in}{2.080489in}}%
\pgfpathlineto{\pgfqpoint{2.566412in}{2.080489in}}%
\pgfpathlineto{\pgfqpoint{2.574689in}{2.061837in}}%
\pgfpathlineto{\pgfqpoint{2.582966in}{2.096031in}}%
\pgfpathlineto{\pgfqpoint{2.591242in}{2.080489in}}%
\pgfpathlineto{\pgfqpoint{2.599519in}{2.077380in}}%
\pgfpathlineto{\pgfqpoint{2.607797in}{2.096031in}}%
\pgfpathlineto{\pgfqpoint{2.616077in}{2.080489in}}%
\pgfpathlineto{\pgfqpoint{2.624351in}{2.061837in}}%
\pgfpathlineto{\pgfqpoint{2.632629in}{2.080489in}}%
\pgfpathlineto{\pgfqpoint{2.640903in}{2.080489in}}%
\pgfpathlineto{\pgfqpoint{2.649175in}{2.027643in}}%
\pgfpathlineto{\pgfqpoint{2.657453in}{2.080489in}}%
\pgfpathlineto{\pgfqpoint{2.665724in}{2.061837in}}%
\pgfpathlineto{\pgfqpoint{2.674002in}{2.080489in}}%
\pgfpathlineto{\pgfqpoint{2.682279in}{2.061837in}}%
\pgfpathlineto{\pgfqpoint{2.690557in}{2.061837in}}%
\pgfpathlineto{\pgfqpoint{2.698835in}{2.077380in}}%
\pgfpathlineto{\pgfqpoint{2.707113in}{2.080489in}}%
\pgfpathlineto{\pgfqpoint{2.715391in}{2.080489in}}%
\pgfpathlineto{\pgfqpoint{2.723669in}{2.043186in}}%
\pgfpathlineto{\pgfqpoint{2.731947in}{2.096031in}}%
\pgfpathlineto{\pgfqpoint{2.740227in}{2.080489in}}%
\pgfpathlineto{\pgfqpoint{2.748498in}{2.096031in}}%
\pgfpathlineto{\pgfqpoint{2.756776in}{2.080489in}}%
\pgfpathlineto{\pgfqpoint{2.773333in}{2.080489in}}%
\pgfpathlineto{\pgfqpoint{2.789889in}{2.043186in}}%
\pgfpathlineto{\pgfqpoint{2.798167in}{2.043186in}}%
\pgfpathlineto{\pgfqpoint{2.806438in}{2.080489in}}%
\pgfpathlineto{\pgfqpoint{2.822992in}{2.080489in}}%
\pgfpathlineto{\pgfqpoint{2.831264in}{2.077380in}}%
\pgfpathlineto{\pgfqpoint{2.839542in}{2.080489in}}%
\pgfpathlineto{\pgfqpoint{2.847820in}{2.096031in}}%
\pgfpathlineto{\pgfqpoint{2.856091in}{2.080489in}}%
\pgfpathlineto{\pgfqpoint{2.864369in}{2.096031in}}%
\pgfpathlineto{\pgfqpoint{2.872647in}{2.061837in}}%
\pgfpathlineto{\pgfqpoint{2.880922in}{2.080489in}}%
\pgfpathlineto{\pgfqpoint{2.897477in}{2.080489in}}%
\pgfpathlineto{\pgfqpoint{2.905755in}{2.096031in}}%
\pgfpathlineto{\pgfqpoint{2.922308in}{2.096031in}}%
\pgfpathlineto{\pgfqpoint{2.930583in}{2.111574in}}%
\pgfpathlineto{\pgfqpoint{2.938860in}{2.080489in}}%
\pgfpathlineto{\pgfqpoint{2.947136in}{2.077380in}}%
\pgfpathlineto{\pgfqpoint{2.955407in}{2.111574in}}%
\pgfpathlineto{\pgfqpoint{2.963680in}{2.080489in}}%
\pgfpathlineto{\pgfqpoint{2.980227in}{2.080489in}}%
\pgfpathlineto{\pgfqpoint{2.988503in}{2.096031in}}%
\pgfpathlineto{\pgfqpoint{3.005056in}{2.096031in}}%
\pgfpathlineto{\pgfqpoint{3.013333in}{2.043186in}}%
\pgfpathlineto{\pgfqpoint{3.021611in}{2.096031in}}%
\pgfpathlineto{\pgfqpoint{3.029888in}{2.080489in}}%
\pgfpathlineto{\pgfqpoint{3.038163in}{2.092923in}}%
\pgfpathlineto{\pgfqpoint{3.046436in}{2.096031in}}%
\pgfpathlineto{\pgfqpoint{3.054714in}{2.111574in}}%
\pgfpathlineto{\pgfqpoint{3.062992in}{2.096031in}}%
\pgfpathlineto{\pgfqpoint{3.071266in}{2.096031in}}%
\pgfpathlineto{\pgfqpoint{3.079538in}{2.080489in}}%
\pgfpathlineto{\pgfqpoint{3.087810in}{2.096031in}}%
\pgfpathlineto{\pgfqpoint{3.096084in}{2.096031in}}%
\pgfpathlineto{\pgfqpoint{3.104362in}{2.111574in}}%
\pgfpathlineto{\pgfqpoint{3.112635in}{2.080489in}}%
\pgfpathlineto{\pgfqpoint{3.120910in}{2.080489in}}%
\pgfpathlineto{\pgfqpoint{3.129187in}{2.077380in}}%
\pgfpathlineto{\pgfqpoint{3.137461in}{2.096031in}}%
\pgfpathlineto{\pgfqpoint{3.145737in}{2.080489in}}%
\pgfpathlineto{\pgfqpoint{3.154015in}{2.061837in}}%
\pgfpathlineto{\pgfqpoint{3.162292in}{2.096031in}}%
\pgfpathlineto{\pgfqpoint{3.170570in}{2.096031in}}%
\pgfpathlineto{\pgfqpoint{3.178841in}{2.077380in}}%
\pgfpathlineto{\pgfqpoint{3.187117in}{2.080489in}}%
\pgfpathlineto{\pgfqpoint{3.195394in}{2.061837in}}%
\pgfpathlineto{\pgfqpoint{3.203666in}{2.096031in}}%
\pgfpathlineto{\pgfqpoint{3.211944in}{2.077380in}}%
\pgfpathlineto{\pgfqpoint{3.220219in}{2.061837in}}%
\pgfpathlineto{\pgfqpoint{3.228495in}{2.096031in}}%
\pgfpathlineto{\pgfqpoint{3.236773in}{2.077380in}}%
\pgfpathlineto{\pgfqpoint{3.245046in}{2.096031in}}%
\pgfpathlineto{\pgfqpoint{3.278143in}{2.096031in}}%
\pgfpathlineto{\pgfqpoint{3.286416in}{2.077380in}}%
\pgfpathlineto{\pgfqpoint{3.294688in}{2.096031in}}%
\pgfpathlineto{\pgfqpoint{3.302963in}{2.061837in}}%
\pgfpathlineto{\pgfqpoint{3.311239in}{2.096031in}}%
\pgfpathlineto{\pgfqpoint{3.352609in}{2.096031in}}%
\pgfpathlineto{\pgfqpoint{3.360882in}{2.077380in}}%
\pgfpathlineto{\pgfqpoint{3.369160in}{2.096031in}}%
\pgfpathlineto{\pgfqpoint{3.377430in}{2.096031in}}%
\pgfpathlineto{\pgfqpoint{3.385708in}{2.077380in}}%
\pgfpathlineto{\pgfqpoint{3.393986in}{2.096031in}}%
\pgfpathlineto{\pgfqpoint{3.402257in}{2.096031in}}%
\pgfpathlineto{\pgfqpoint{3.410529in}{2.077380in}}%
\pgfpathlineto{\pgfqpoint{3.418804in}{2.096031in}}%
\pgfpathlineto{\pgfqpoint{3.427082in}{2.096031in}}%
\pgfpathlineto{\pgfqpoint{3.435360in}{2.077380in}}%
\pgfpathlineto{\pgfqpoint{3.443631in}{2.096031in}}%
\pgfpathlineto{\pgfqpoint{3.542931in}{2.096031in}}%
\pgfpathlineto{\pgfqpoint{3.551204in}{2.077380in}}%
\pgfpathlineto{\pgfqpoint{3.559478in}{2.096031in}}%
\pgfpathlineto{\pgfqpoint{3.567756in}{2.096031in}}%
\pgfpathlineto{\pgfqpoint{3.576034in}{2.080489in}}%
\pgfpathlineto{\pgfqpoint{3.584304in}{2.096031in}}%
\pgfpathlineto{\pgfqpoint{3.592576in}{2.080489in}}%
\pgfpathlineto{\pgfqpoint{3.600852in}{2.096031in}}%
\pgfpathlineto{\pgfqpoint{3.625676in}{2.096031in}}%
\pgfpathlineto{\pgfqpoint{3.633952in}{2.061837in}}%
\pgfpathlineto{\pgfqpoint{3.642231in}{2.096031in}}%
\pgfpathlineto{\pgfqpoint{3.675334in}{2.096031in}}%
\pgfpathlineto{\pgfqpoint{3.683610in}{2.061837in}}%
\pgfpathlineto{\pgfqpoint{3.691888in}{2.061837in}}%
\pgfpathlineto{\pgfqpoint{3.700160in}{2.096031in}}%
\pgfpathlineto{\pgfqpoint{3.733260in}{2.096031in}}%
\pgfpathlineto{\pgfqpoint{3.741534in}{2.077380in}}%
\pgfpathlineto{\pgfqpoint{3.749812in}{2.096031in}}%
\pgfpathlineto{\pgfqpoint{3.791201in}{2.096031in}}%
\pgfpathlineto{\pgfqpoint{3.799477in}{2.080489in}}%
\pgfpathlineto{\pgfqpoint{3.807753in}{2.061837in}}%
\pgfpathlineto{\pgfqpoint{3.816025in}{2.080489in}}%
\pgfpathlineto{\pgfqpoint{3.824299in}{2.096031in}}%
\pgfpathlineto{\pgfqpoint{3.857406in}{2.096031in}}%
\pgfpathlineto{\pgfqpoint{3.865684in}{2.127117in}}%
\pgfpathlineto{\pgfqpoint{3.873963in}{2.096031in}}%
\pgfpathlineto{\pgfqpoint{3.882237in}{2.111574in}}%
\pgfpathlineto{\pgfqpoint{3.890512in}{2.077380in}}%
\pgfpathlineto{\pgfqpoint{3.898787in}{2.096031in}}%
\pgfpathlineto{\pgfqpoint{3.907064in}{2.080489in}}%
\pgfpathlineto{\pgfqpoint{3.915342in}{2.096031in}}%
\pgfpathlineto{\pgfqpoint{3.923612in}{2.077380in}}%
\pgfpathlineto{\pgfqpoint{3.931890in}{2.111574in}}%
\pgfpathlineto{\pgfqpoint{3.940168in}{2.096031in}}%
\pgfpathlineto{\pgfqpoint{3.956722in}{2.096031in}}%
\pgfpathlineto{\pgfqpoint{3.964993in}{2.061837in}}%
\pgfpathlineto{\pgfqpoint{3.973267in}{2.080489in}}%
\pgfpathlineto{\pgfqpoint{3.981539in}{2.077380in}}%
\pgfpathlineto{\pgfqpoint{3.989809in}{2.111574in}}%
\pgfpathlineto{\pgfqpoint{3.998087in}{2.077380in}}%
\pgfpathlineto{\pgfqpoint{4.006362in}{2.077380in}}%
\pgfpathlineto{\pgfqpoint{4.014633in}{2.061837in}}%
\pgfpathlineto{\pgfqpoint{4.022912in}{2.096031in}}%
\pgfpathlineto{\pgfqpoint{4.039466in}{2.096031in}}%
\pgfpathlineto{\pgfqpoint{4.047743in}{2.077380in}}%
\pgfpathlineto{\pgfqpoint{4.056021in}{2.111574in}}%
\pgfpathlineto{\pgfqpoint{4.072576in}{2.080489in}}%
\pgfpathlineto{\pgfqpoint{4.089132in}{2.111574in}}%
\pgfpathlineto{\pgfqpoint{4.097411in}{2.080489in}}%
\pgfpathlineto{\pgfqpoint{4.105682in}{2.096031in}}%
\pgfpathlineto{\pgfqpoint{4.113960in}{2.077380in}}%
\pgfpathlineto{\pgfqpoint{4.122237in}{2.061837in}}%
\pgfpathlineto{\pgfqpoint{4.130514in}{2.077380in}}%
\pgfpathlineto{\pgfqpoint{4.138792in}{2.061837in}}%
\pgfpathlineto{\pgfqpoint{4.147069in}{2.061837in}}%
\pgfpathlineto{\pgfqpoint{4.155346in}{2.077380in}}%
\pgfpathlineto{\pgfqpoint{4.163624in}{2.096031in}}%
\pgfpathlineto{\pgfqpoint{4.171902in}{2.043186in}}%
\pgfpathlineto{\pgfqpoint{4.180180in}{2.096031in}}%
\pgfpathlineto{\pgfqpoint{4.213294in}{2.096031in}}%
\pgfpathlineto{\pgfqpoint{4.221572in}{2.043186in}}%
\pgfpathlineto{\pgfqpoint{4.238128in}{2.111574in}}%
\pgfpathlineto{\pgfqpoint{4.246403in}{2.111574in}}%
\pgfpathlineto{\pgfqpoint{4.254681in}{2.096031in}}%
\pgfpathlineto{\pgfqpoint{4.262954in}{2.061837in}}%
\pgfpathlineto{\pgfqpoint{4.271227in}{2.092923in}}%
\pgfpathlineto{\pgfqpoint{4.279500in}{2.061837in}}%
\pgfpathlineto{\pgfqpoint{4.287775in}{2.077380in}}%
\pgfpathlineto{\pgfqpoint{4.296053in}{2.096031in}}%
\pgfpathlineto{\pgfqpoint{4.304325in}{2.096031in}}%
\pgfpathlineto{\pgfqpoint{4.312598in}{2.061837in}}%
\pgfpathlineto{\pgfqpoint{4.320871in}{2.092923in}}%
\pgfpathlineto{\pgfqpoint{4.329147in}{2.096031in}}%
\pgfpathlineto{\pgfqpoint{4.337425in}{2.077380in}}%
\pgfpathlineto{\pgfqpoint{4.345702in}{2.111574in}}%
\pgfpathlineto{\pgfqpoint{4.353980in}{2.077380in}}%
\pgfpathlineto{\pgfqpoint{4.362257in}{2.114683in}}%
\pgfpathlineto{\pgfqpoint{4.370533in}{2.092923in}}%
\pgfpathlineto{\pgfqpoint{4.378811in}{2.096031in}}%
\pgfpathlineto{\pgfqpoint{4.387089in}{2.096031in}}%
\pgfpathlineto{\pgfqpoint{4.395360in}{2.077380in}}%
\pgfpathlineto{\pgfqpoint{4.403637in}{2.061837in}}%
\pgfpathlineto{\pgfqpoint{4.411915in}{2.130226in}}%
\pgfpathlineto{\pgfqpoint{4.420193in}{2.111574in}}%
\pgfpathlineto{\pgfqpoint{4.428470in}{2.061837in}}%
\pgfpathlineto{\pgfqpoint{4.436747in}{2.096031in}}%
\pgfpathlineto{\pgfqpoint{4.445024in}{2.092923in}}%
\pgfpathlineto{\pgfqpoint{4.453302in}{2.111574in}}%
\pgfpathlineto{\pgfqpoint{4.461579in}{2.077380in}}%
\pgfpathlineto{\pgfqpoint{4.469853in}{2.061837in}}%
\pgfpathlineto{\pgfqpoint{4.478131in}{2.096031in}}%
\pgfpathlineto{\pgfqpoint{4.486401in}{2.111574in}}%
\pgfpathlineto{\pgfqpoint{4.502947in}{2.080489in}}%
\pgfpathlineto{\pgfqpoint{4.511225in}{2.061837in}}%
\pgfpathlineto{\pgfqpoint{4.519503in}{2.077380in}}%
\pgfpathlineto{\pgfqpoint{4.527774in}{2.096031in}}%
\pgfpathlineto{\pgfqpoint{4.536051in}{2.096031in}}%
\pgfpathlineto{\pgfqpoint{4.544329in}{2.061837in}}%
\pgfpathlineto{\pgfqpoint{4.552607in}{2.077380in}}%
\pgfpathlineto{\pgfqpoint{4.560880in}{2.061837in}}%
\pgfpathlineto{\pgfqpoint{4.569158in}{2.111574in}}%
\pgfpathlineto{\pgfqpoint{4.577435in}{2.077380in}}%
\pgfpathlineto{\pgfqpoint{4.585706in}{2.111574in}}%
\pgfpathlineto{\pgfqpoint{4.593978in}{2.077380in}}%
\pgfpathlineto{\pgfqpoint{4.602252in}{2.111574in}}%
\pgfpathlineto{\pgfqpoint{4.618802in}{2.111574in}}%
\pgfpathlineto{\pgfqpoint{4.627074in}{2.096031in}}%
\pgfpathlineto{\pgfqpoint{4.635352in}{2.096031in}}%
\pgfpathlineto{\pgfqpoint{4.643630in}{2.077380in}}%
\pgfpathlineto{\pgfqpoint{4.651901in}{2.127117in}}%
\pgfpathlineto{\pgfqpoint{4.660179in}{2.127117in}}%
\pgfpathlineto{\pgfqpoint{4.668457in}{2.092923in}}%
\pgfpathlineto{\pgfqpoint{4.676732in}{2.077380in}}%
\pgfpathlineto{\pgfqpoint{4.685009in}{2.096031in}}%
\pgfpathlineto{\pgfqpoint{4.718108in}{2.096031in}}%
\pgfpathlineto{\pgfqpoint{4.726382in}{2.061837in}}%
\pgfpathlineto{\pgfqpoint{4.734659in}{2.096031in}}%
\pgfpathlineto{\pgfqpoint{4.742937in}{2.061837in}}%
\pgfpathlineto{\pgfqpoint{4.751213in}{2.061837in}}%
\pgfpathlineto{\pgfqpoint{4.759485in}{2.092923in}}%
\pgfpathlineto{\pgfqpoint{4.776036in}{2.061837in}}%
\pgfpathlineto{\pgfqpoint{4.784308in}{2.061837in}}%
\pgfpathlineto{\pgfqpoint{4.792584in}{2.077380in}}%
\pgfpathlineto{\pgfqpoint{4.800861in}{2.077380in}}%
\pgfpathlineto{\pgfqpoint{4.809137in}{2.092923in}}%
\pgfpathlineto{\pgfqpoint{4.817414in}{2.111574in}}%
\pgfpathlineto{\pgfqpoint{4.825692in}{2.061837in}}%
\pgfpathlineto{\pgfqpoint{4.833968in}{2.077380in}}%
\pgfpathlineto{\pgfqpoint{4.842243in}{2.077380in}}%
\pgfpathlineto{\pgfqpoint{4.850520in}{2.092923in}}%
\pgfpathlineto{\pgfqpoint{4.858795in}{2.092923in}}%
\pgfpathlineto{\pgfqpoint{4.867068in}{2.061837in}}%
\pgfpathlineto{\pgfqpoint{4.908449in}{2.061837in}}%
\pgfpathlineto{\pgfqpoint{4.916724in}{2.077380in}}%
\pgfpathlineto{\pgfqpoint{4.924996in}{2.077380in}}%
\pgfpathlineto{\pgfqpoint{4.933270in}{2.061837in}}%
\pgfpathlineto{\pgfqpoint{4.941548in}{2.077380in}}%
\pgfpathlineto{\pgfqpoint{4.958098in}{2.077380in}}%
\pgfpathlineto{\pgfqpoint{4.966376in}{2.092923in}}%
\pgfpathlineto{\pgfqpoint{4.982927in}{2.061837in}}%
\pgfpathlineto{\pgfqpoint{4.991206in}{2.096031in}}%
\pgfpathlineto{\pgfqpoint{4.999486in}{2.092923in}}%
\pgfpathlineto{\pgfqpoint{5.016036in}{2.061837in}}%
\pgfpathlineto{\pgfqpoint{5.024314in}{2.061837in}}%
\pgfpathlineto{\pgfqpoint{5.032594in}{2.127117in}}%
\pgfpathlineto{\pgfqpoint{5.040864in}{2.096031in}}%
\pgfpathlineto{\pgfqpoint{5.057421in}{2.096031in}}%
\pgfpathlineto{\pgfqpoint{5.065699in}{2.077380in}}%
\pgfpathlineto{\pgfqpoint{5.082254in}{2.077380in}}%
\pgfpathlineto{\pgfqpoint{5.090526in}{2.096031in}}%
\pgfpathlineto{\pgfqpoint{5.098804in}{2.092923in}}%
\pgfpathlineto{\pgfqpoint{5.115354in}{2.061837in}}%
\pgfpathlineto{\pgfqpoint{5.131910in}{2.092923in}}%
\pgfpathlineto{\pgfqpoint{5.140185in}{2.092923in}}%
\pgfpathlineto{\pgfqpoint{5.148459in}{2.077380in}}%
\pgfpathlineto{\pgfqpoint{5.156736in}{2.077380in}}%
\pgfpathlineto{\pgfqpoint{5.165012in}{2.092923in}}%
\pgfpathlineto{\pgfqpoint{5.173283in}{2.096031in}}%
\pgfpathlineto{\pgfqpoint{5.181559in}{2.092923in}}%
\pgfpathlineto{\pgfqpoint{5.189836in}{2.096031in}}%
\pgfpathlineto{\pgfqpoint{5.198110in}{2.077380in}}%
\pgfpathlineto{\pgfqpoint{5.214659in}{2.077380in}}%
\pgfpathlineto{\pgfqpoint{5.222936in}{2.096031in}}%
\pgfpathlineto{\pgfqpoint{5.231214in}{2.061837in}}%
\pgfpathlineto{\pgfqpoint{5.239492in}{2.061837in}}%
\pgfpathlineto{\pgfqpoint{5.247764in}{1.629746in}}%
\pgfpathlineto{\pgfqpoint{5.264314in}{1.458774in}}%
\pgfpathlineto{\pgfqpoint{5.272592in}{1.455666in}}%
\pgfpathlineto{\pgfqpoint{5.280867in}{1.440123in}}%
\pgfpathlineto{\pgfqpoint{5.289144in}{1.409037in}}%
\pgfpathlineto{\pgfqpoint{5.297422in}{1.405929in}}%
\pgfpathlineto{\pgfqpoint{5.305696in}{1.390386in}}%
\pgfpathlineto{\pgfqpoint{5.313972in}{1.409037in}}%
\pgfpathlineto{\pgfqpoint{5.322249in}{1.390386in}}%
\pgfpathlineto{\pgfqpoint{5.330524in}{1.393494in}}%
\pgfpathlineto{\pgfqpoint{5.338803in}{1.390386in}}%
\pgfpathlineto{\pgfqpoint{5.347076in}{1.359300in}}%
\pgfpathlineto{\pgfqpoint{5.355353in}{1.405929in}}%
\pgfpathlineto{\pgfqpoint{5.371906in}{1.374843in}}%
\pgfpathlineto{\pgfqpoint{5.380185in}{1.374843in}}%
\pgfpathlineto{\pgfqpoint{5.388456in}{1.359300in}}%
\pgfpathlineto{\pgfqpoint{5.396731in}{1.377951in}}%
\pgfpathlineto{\pgfqpoint{5.405008in}{1.359300in}}%
\pgfpathlineto{\pgfqpoint{5.413287in}{1.374843in}}%
\pgfpathlineto{\pgfqpoint{5.421567in}{1.359300in}}%
\pgfpathlineto{\pgfqpoint{5.438125in}{1.359300in}}%
\pgfpathlineto{\pgfqpoint{5.446404in}{1.374843in}}%
\pgfpathlineto{\pgfqpoint{5.454684in}{1.359300in}}%
\pgfpathlineto{\pgfqpoint{5.462963in}{1.393494in}}%
\pgfpathlineto{\pgfqpoint{5.471242in}{1.359300in}}%
\pgfpathlineto{\pgfqpoint{5.479521in}{1.390386in}}%
\pgfpathlineto{\pgfqpoint{5.487801in}{1.359300in}}%
\pgfpathlineto{\pgfqpoint{5.487801in}{1.359300in}}%
\pgfusepath{stroke}%
\end{pgfscope}%
\begin{pgfscope}%
\pgfpathrectangle{\pgfqpoint{0.530556in}{0.656763in}}{\pgfqpoint{4.960000in}{3.264000in}}%
\pgfusepath{clip}%
\pgfsetrectcap%
\pgfsetroundjoin%
\pgfsetlinewidth{1.505625pt}%
\definecolor{currentstroke}{rgb}{1.000000,0.498039,0.054902}%
\pgfsetstrokecolor{currentstroke}%
\pgfsetdash{}{0pt}%
\pgfpathmoveto{\pgfqpoint{0.520556in}{0.720329in}}%
\pgfpathlineto{\pgfqpoint{0.523030in}{0.720877in}}%
\pgfpathlineto{\pgfqpoint{0.531803in}{0.716991in}}%
\pgfpathlineto{\pgfqpoint{0.549350in}{0.716991in}}%
\pgfpathlineto{\pgfqpoint{0.566896in}{0.720877in}}%
\pgfpathlineto{\pgfqpoint{0.575670in}{0.720877in}}%
\pgfpathlineto{\pgfqpoint{0.584437in}{0.724763in}}%
\pgfpathlineto{\pgfqpoint{0.593194in}{0.724763in}}%
\pgfpathlineto{\pgfqpoint{0.601968in}{0.726706in}}%
\pgfpathlineto{\pgfqpoint{0.610741in}{0.724763in}}%
\pgfpathlineto{\pgfqpoint{0.619536in}{0.726706in}}%
\pgfpathlineto{\pgfqpoint{0.628310in}{0.726706in}}%
\pgfpathlineto{\pgfqpoint{0.645857in}{0.730591in}}%
\pgfpathlineto{\pgfqpoint{0.654630in}{0.728649in}}%
\pgfpathlineto{\pgfqpoint{0.663403in}{0.730591in}}%
\pgfpathlineto{\pgfqpoint{0.698497in}{0.730591in}}%
\pgfpathlineto{\pgfqpoint{0.716043in}{0.734477in}}%
\pgfpathlineto{\pgfqpoint{0.724817in}{0.732534in}}%
\pgfpathlineto{\pgfqpoint{0.733590in}{0.734477in}}%
\pgfpathlineto{\pgfqpoint{0.751137in}{0.730591in}}%
\pgfpathlineto{\pgfqpoint{0.777456in}{0.730591in}}%
\pgfpathlineto{\pgfqpoint{0.786252in}{0.728649in}}%
\pgfpathlineto{\pgfqpoint{0.803799in}{0.728649in}}%
\pgfpathlineto{\pgfqpoint{0.812572in}{0.730591in}}%
\pgfpathlineto{\pgfqpoint{0.830119in}{0.730591in}}%
\pgfpathlineto{\pgfqpoint{0.838893in}{0.728649in}}%
\pgfpathlineto{\pgfqpoint{0.847666in}{0.730591in}}%
\pgfpathlineto{\pgfqpoint{0.856439in}{0.728649in}}%
\pgfpathlineto{\pgfqpoint{0.865213in}{0.728649in}}%
\pgfpathlineto{\pgfqpoint{0.873986in}{0.730591in}}%
\pgfpathlineto{\pgfqpoint{0.882759in}{0.728649in}}%
\pgfpathlineto{\pgfqpoint{0.909079in}{0.728649in}}%
\pgfpathlineto{\pgfqpoint{0.917847in}{0.730591in}}%
\pgfpathlineto{\pgfqpoint{0.926604in}{0.730591in}}%
\pgfpathlineto{\pgfqpoint{0.935377in}{0.732534in}}%
\pgfpathlineto{\pgfqpoint{0.961697in}{0.732534in}}%
\pgfpathlineto{\pgfqpoint{0.970471in}{0.734477in}}%
\pgfpathlineto{\pgfqpoint{0.979244in}{0.734477in}}%
\pgfpathlineto{\pgfqpoint{0.988018in}{0.738363in}}%
\pgfpathlineto{\pgfqpoint{1.023105in}{0.738363in}}%
\pgfpathlineto{\pgfqpoint{1.031862in}{0.740306in}}%
\pgfpathlineto{\pgfqpoint{1.049409in}{0.740306in}}%
\pgfpathlineto{\pgfqpoint{1.058182in}{0.738363in}}%
\pgfpathlineto{\pgfqpoint{1.066956in}{0.742249in}}%
\pgfpathlineto{\pgfqpoint{1.075729in}{0.740306in}}%
\pgfpathlineto{\pgfqpoint{1.084502in}{0.740306in}}%
\pgfpathlineto{\pgfqpoint{1.093276in}{0.742249in}}%
\pgfpathlineto{\pgfqpoint{1.145894in}{0.742249in}}%
\pgfpathlineto{\pgfqpoint{1.154667in}{0.740306in}}%
\pgfpathlineto{\pgfqpoint{1.172214in}{0.740306in}}%
\pgfpathlineto{\pgfqpoint{1.180987in}{0.742249in}}%
\pgfpathlineto{\pgfqpoint{1.189761in}{0.740306in}}%
\pgfpathlineto{\pgfqpoint{1.207307in}{0.740306in}}%
\pgfpathlineto{\pgfqpoint{1.216081in}{0.738363in}}%
\pgfpathlineto{\pgfqpoint{1.224854in}{0.740306in}}%
\pgfpathlineto{\pgfqpoint{1.233621in}{0.738363in}}%
\pgfpathlineto{\pgfqpoint{1.251152in}{0.738363in}}%
\pgfpathlineto{\pgfqpoint{1.286246in}{0.730591in}}%
\pgfpathlineto{\pgfqpoint{1.295019in}{0.730591in}}%
\pgfpathlineto{\pgfqpoint{1.303792in}{0.728649in}}%
\pgfpathlineto{\pgfqpoint{1.312565in}{0.730591in}}%
\pgfpathlineto{\pgfqpoint{1.330112in}{0.726706in}}%
\pgfpathlineto{\pgfqpoint{1.338880in}{0.726706in}}%
\pgfpathlineto{\pgfqpoint{1.347637in}{0.722820in}}%
\pgfpathlineto{\pgfqpoint{1.373957in}{0.722820in}}%
\pgfpathlineto{\pgfqpoint{1.391504in}{0.718934in}}%
\pgfpathlineto{\pgfqpoint{1.409293in}{0.718934in}}%
\pgfpathlineto{\pgfqpoint{1.418066in}{0.716991in}}%
\pgfpathlineto{\pgfqpoint{1.426840in}{0.716991in}}%
\pgfpathlineto{\pgfqpoint{1.435613in}{0.715049in}}%
\pgfpathlineto{\pgfqpoint{1.523347in}{0.715049in}}%
\pgfpathlineto{\pgfqpoint{1.532120in}{0.713106in}}%
\pgfpathlineto{\pgfqpoint{1.540893in}{0.715049in}}%
\pgfpathlineto{\pgfqpoint{1.549667in}{0.711163in}}%
\pgfpathlineto{\pgfqpoint{1.558440in}{0.711163in}}%
\pgfpathlineto{\pgfqpoint{1.567214in}{0.713106in}}%
\pgfpathlineto{\pgfqpoint{1.584760in}{0.713106in}}%
\pgfpathlineto{\pgfqpoint{1.593534in}{0.715049in}}%
\pgfpathlineto{\pgfqpoint{1.619832in}{0.715049in}}%
\pgfpathlineto{\pgfqpoint{1.628605in}{0.716991in}}%
\pgfpathlineto{\pgfqpoint{1.654925in}{0.716991in}}%
\pgfpathlineto{\pgfqpoint{1.663698in}{0.718934in}}%
\pgfpathlineto{\pgfqpoint{1.672472in}{0.718934in}}%
\pgfpathlineto{\pgfqpoint{1.681245in}{0.722820in}}%
\pgfpathlineto{\pgfqpoint{1.690019in}{0.722820in}}%
\pgfpathlineto{\pgfqpoint{1.698792in}{0.724763in}}%
\pgfpathlineto{\pgfqpoint{1.707565in}{0.724763in}}%
\pgfpathlineto{\pgfqpoint{1.716339in}{0.726706in}}%
\pgfpathlineto{\pgfqpoint{1.760205in}{0.726706in}}%
\pgfpathlineto{\pgfqpoint{1.768979in}{0.728649in}}%
\pgfpathlineto{\pgfqpoint{1.777752in}{0.728649in}}%
\pgfpathlineto{\pgfqpoint{1.795299in}{0.732534in}}%
\pgfpathlineto{\pgfqpoint{1.804072in}{0.732534in}}%
\pgfpathlineto{\pgfqpoint{1.812846in}{0.730591in}}%
\pgfpathlineto{\pgfqpoint{1.821619in}{0.730591in}}%
\pgfpathlineto{\pgfqpoint{1.839166in}{0.734477in}}%
\pgfpathlineto{\pgfqpoint{1.874259in}{0.734477in}}%
\pgfpathlineto{\pgfqpoint{1.891806in}{0.730591in}}%
\pgfpathlineto{\pgfqpoint{1.900579in}{0.732534in}}%
\pgfpathlineto{\pgfqpoint{1.909352in}{0.730591in}}%
\pgfpathlineto{\pgfqpoint{1.961993in}{0.730591in}}%
\pgfpathlineto{\pgfqpoint{1.970766in}{0.728649in}}%
\pgfpathlineto{\pgfqpoint{1.988313in}{0.728649in}}%
\pgfpathlineto{\pgfqpoint{1.997086in}{0.726706in}}%
\pgfpathlineto{\pgfqpoint{2.005859in}{0.728649in}}%
\pgfpathlineto{\pgfqpoint{2.014633in}{0.728649in}}%
\pgfpathlineto{\pgfqpoint{2.032179in}{0.732534in}}%
\pgfpathlineto{\pgfqpoint{2.067273in}{0.732534in}}%
\pgfpathlineto{\pgfqpoint{2.076046in}{0.734477in}}%
\pgfpathlineto{\pgfqpoint{2.084820in}{0.732534in}}%
\pgfpathlineto{\pgfqpoint{2.119913in}{0.732534in}}%
\pgfpathlineto{\pgfqpoint{2.128686in}{0.734477in}}%
\pgfpathlineto{\pgfqpoint{2.137460in}{0.732534in}}%
\pgfpathlineto{\pgfqpoint{2.172575in}{0.732534in}}%
\pgfpathlineto{\pgfqpoint{2.181349in}{0.734477in}}%
\pgfpathlineto{\pgfqpoint{2.190122in}{0.732534in}}%
\pgfpathlineto{\pgfqpoint{2.198895in}{0.734477in}}%
\pgfpathlineto{\pgfqpoint{2.233989in}{0.734477in}}%
\pgfpathlineto{\pgfqpoint{2.277856in}{0.744191in}}%
\pgfpathlineto{\pgfqpoint{2.295402in}{0.744191in}}%
\pgfpathlineto{\pgfqpoint{2.304176in}{0.746134in}}%
\pgfpathlineto{\pgfqpoint{2.321722in}{0.746134in}}%
\pgfpathlineto{\pgfqpoint{2.330496in}{0.748077in}}%
\pgfpathlineto{\pgfqpoint{2.339269in}{0.746134in}}%
\pgfpathlineto{\pgfqpoint{2.348042in}{0.746134in}}%
\pgfpathlineto{\pgfqpoint{2.365589in}{0.750020in}}%
\pgfpathlineto{\pgfqpoint{2.374363in}{0.748077in}}%
\pgfpathlineto{\pgfqpoint{2.383136in}{0.750020in}}%
\pgfpathlineto{\pgfqpoint{2.470869in}{0.750020in}}%
\pgfpathlineto{\pgfqpoint{2.479643in}{0.751963in}}%
\pgfpathlineto{\pgfqpoint{2.523510in}{0.751963in}}%
\pgfpathlineto{\pgfqpoint{2.532283in}{0.750020in}}%
\pgfpathlineto{\pgfqpoint{2.541056in}{0.751963in}}%
\pgfpathlineto{\pgfqpoint{2.549830in}{0.750020in}}%
\pgfpathlineto{\pgfqpoint{2.646337in}{0.750020in}}%
\pgfpathlineto{\pgfqpoint{2.655110in}{0.753906in}}%
\pgfpathlineto{\pgfqpoint{2.664126in}{0.750020in}}%
\pgfpathlineto{\pgfqpoint{2.672899in}{0.751963in}}%
\pgfpathlineto{\pgfqpoint{2.681673in}{0.750020in}}%
\pgfpathlineto{\pgfqpoint{2.690446in}{0.750020in}}%
\pgfpathlineto{\pgfqpoint{2.699219in}{0.753906in}}%
\pgfpathlineto{\pgfqpoint{2.707993in}{0.751963in}}%
\pgfpathlineto{\pgfqpoint{2.725539in}{0.755849in}}%
\pgfpathlineto{\pgfqpoint{2.751831in}{0.755849in}}%
\pgfpathlineto{\pgfqpoint{2.760589in}{0.753906in}}%
\pgfpathlineto{\pgfqpoint{2.769362in}{0.753906in}}%
\pgfpathlineto{\pgfqpoint{2.778135in}{0.755849in}}%
\pgfpathlineto{\pgfqpoint{2.795682in}{0.755849in}}%
\pgfpathlineto{\pgfqpoint{2.804449in}{0.753906in}}%
\pgfpathlineto{\pgfqpoint{2.813207in}{0.755849in}}%
\pgfpathlineto{\pgfqpoint{2.839527in}{0.755849in}}%
\pgfpathlineto{\pgfqpoint{2.848300in}{0.757791in}}%
\pgfpathlineto{\pgfqpoint{2.857074in}{0.755849in}}%
\pgfpathlineto{\pgfqpoint{2.865847in}{0.759734in}}%
\pgfpathlineto{\pgfqpoint{2.874620in}{0.761677in}}%
\pgfpathlineto{\pgfqpoint{2.892167in}{0.761677in}}%
\pgfpathlineto{\pgfqpoint{2.900941in}{0.763620in}}%
\pgfpathlineto{\pgfqpoint{2.909714in}{0.761677in}}%
\pgfpathlineto{\pgfqpoint{2.918480in}{0.763620in}}%
\pgfpathlineto{\pgfqpoint{2.927239in}{0.763620in}}%
\pgfpathlineto{\pgfqpoint{2.936008in}{0.765563in}}%
\pgfpathlineto{\pgfqpoint{2.988630in}{0.765563in}}%
\pgfpathlineto{\pgfqpoint{2.997403in}{0.769449in}}%
\pgfpathlineto{\pgfqpoint{3.014950in}{0.769449in}}%
\pgfpathlineto{\pgfqpoint{3.023724in}{0.767506in}}%
\pgfpathlineto{\pgfqpoint{3.076364in}{0.767506in}}%
\pgfpathlineto{\pgfqpoint{3.085137in}{0.765563in}}%
\pgfpathlineto{\pgfqpoint{3.093910in}{0.767506in}}%
\pgfpathlineto{\pgfqpoint{3.111457in}{0.767506in}}%
\pgfpathlineto{\pgfqpoint{3.120230in}{0.769449in}}%
\pgfpathlineto{\pgfqpoint{3.129004in}{0.769449in}}%
\pgfpathlineto{\pgfqpoint{3.137776in}{0.771391in}}%
\pgfpathlineto{\pgfqpoint{3.146551in}{0.771391in}}%
\pgfpathlineto{\pgfqpoint{3.155324in}{0.773334in}}%
\pgfpathlineto{\pgfqpoint{3.164091in}{0.771391in}}%
\pgfpathlineto{\pgfqpoint{3.172849in}{0.771391in}}%
\pgfpathlineto{\pgfqpoint{3.181622in}{0.773334in}}%
\pgfpathlineto{\pgfqpoint{3.190395in}{0.771391in}}%
\pgfpathlineto{\pgfqpoint{3.199169in}{0.771391in}}%
\pgfpathlineto{\pgfqpoint{3.207942in}{0.769449in}}%
\pgfpathlineto{\pgfqpoint{3.216715in}{0.769449in}}%
\pgfpathlineto{\pgfqpoint{3.225489in}{0.771391in}}%
\pgfpathlineto{\pgfqpoint{3.269349in}{0.771391in}}%
\pgfpathlineto{\pgfqpoint{3.278107in}{0.773334in}}%
\pgfpathlineto{\pgfqpoint{3.286880in}{0.771391in}}%
\pgfpathlineto{\pgfqpoint{3.313200in}{0.771391in}}%
\pgfpathlineto{\pgfqpoint{3.321974in}{0.773334in}}%
\pgfpathlineto{\pgfqpoint{3.330747in}{0.771391in}}%
\pgfpathlineto{\pgfqpoint{3.348316in}{0.771391in}}%
\pgfpathlineto{\pgfqpoint{3.365862in}{0.767506in}}%
\pgfpathlineto{\pgfqpoint{3.392182in}{0.767506in}}%
\pgfpathlineto{\pgfqpoint{3.400956in}{0.765563in}}%
\pgfpathlineto{\pgfqpoint{3.409729in}{0.767506in}}%
\pgfpathlineto{\pgfqpoint{3.418503in}{0.767506in}}%
\pgfpathlineto{\pgfqpoint{3.427276in}{0.763620in}}%
\pgfpathlineto{\pgfqpoint{3.436049in}{0.763620in}}%
\pgfpathlineto{\pgfqpoint{3.453574in}{0.767506in}}%
\pgfpathlineto{\pgfqpoint{3.462347in}{0.765563in}}%
\pgfpathlineto{\pgfqpoint{3.471121in}{0.769449in}}%
\pgfpathlineto{\pgfqpoint{3.479894in}{0.771391in}}%
\pgfpathlineto{\pgfqpoint{3.488667in}{0.767506in}}%
\pgfpathlineto{\pgfqpoint{3.558854in}{0.767506in}}%
\pgfpathlineto{\pgfqpoint{3.567628in}{0.769449in}}%
\pgfpathlineto{\pgfqpoint{3.585174in}{0.769449in}}%
\pgfpathlineto{\pgfqpoint{3.593948in}{0.767506in}}%
\pgfpathlineto{\pgfqpoint{3.628997in}{0.767506in}}%
\pgfpathlineto{\pgfqpoint{3.637770in}{0.769449in}}%
\pgfpathlineto{\pgfqpoint{3.646544in}{0.767506in}}%
\pgfpathlineto{\pgfqpoint{3.655317in}{0.769449in}}%
\pgfpathlineto{\pgfqpoint{3.707957in}{0.769449in}}%
\pgfpathlineto{\pgfqpoint{3.716731in}{0.767506in}}%
\pgfpathlineto{\pgfqpoint{3.725504in}{0.767506in}}%
\pgfpathlineto{\pgfqpoint{3.734277in}{0.769449in}}%
\pgfpathlineto{\pgfqpoint{3.743051in}{0.769449in}}%
\pgfpathlineto{\pgfqpoint{3.751824in}{0.767506in}}%
\pgfpathlineto{\pgfqpoint{3.760597in}{0.769449in}}%
\pgfpathlineto{\pgfqpoint{3.769371in}{0.767506in}}%
\pgfpathlineto{\pgfqpoint{3.778144in}{0.767506in}}%
\pgfpathlineto{\pgfqpoint{3.786917in}{0.769449in}}%
\pgfpathlineto{\pgfqpoint{3.795691in}{0.767506in}}%
\pgfpathlineto{\pgfqpoint{3.822011in}{0.767506in}}%
\pgfpathlineto{\pgfqpoint{3.839558in}{0.771391in}}%
\pgfpathlineto{\pgfqpoint{3.848331in}{0.771391in}}%
\pgfpathlineto{\pgfqpoint{3.865878in}{0.775277in}}%
\pgfpathlineto{\pgfqpoint{3.874651in}{0.773334in}}%
\pgfpathlineto{\pgfqpoint{3.883424in}{0.773334in}}%
\pgfpathlineto{\pgfqpoint{3.892198in}{0.775277in}}%
\pgfpathlineto{\pgfqpoint{3.900971in}{0.775277in}}%
\pgfpathlineto{\pgfqpoint{3.909744in}{0.773334in}}%
\pgfpathlineto{\pgfqpoint{3.927291in}{0.773334in}}%
\pgfpathlineto{\pgfqpoint{3.936064in}{0.771391in}}%
\pgfpathlineto{\pgfqpoint{3.944838in}{0.771391in}}%
\pgfpathlineto{\pgfqpoint{3.953611in}{0.769449in}}%
\pgfpathlineto{\pgfqpoint{3.971158in}{0.769449in}}%
\pgfpathlineto{\pgfqpoint{3.979931in}{0.767506in}}%
\pgfpathlineto{\pgfqpoint{3.988705in}{0.769449in}}%
\pgfpathlineto{\pgfqpoint{3.997478in}{0.767506in}}%
\pgfpathlineto{\pgfqpoint{4.041345in}{0.767506in}}%
\pgfpathlineto{\pgfqpoint{4.050118in}{0.769449in}}%
\pgfpathlineto{\pgfqpoint{4.085212in}{0.769449in}}%
\pgfpathlineto{\pgfqpoint{4.093985in}{0.771391in}}%
\pgfpathlineto{\pgfqpoint{4.137852in}{0.771391in}}%
\pgfpathlineto{\pgfqpoint{4.146625in}{0.773334in}}%
\pgfpathlineto{\pgfqpoint{4.172945in}{0.773334in}}%
\pgfpathlineto{\pgfqpoint{4.181719in}{0.771391in}}%
\pgfpathlineto{\pgfqpoint{4.251883in}{0.771391in}}%
\pgfpathlineto{\pgfqpoint{4.260657in}{0.773334in}}%
\pgfpathlineto{\pgfqpoint{4.269430in}{0.771391in}}%
\pgfpathlineto{\pgfqpoint{4.295750in}{0.771391in}}%
\pgfpathlineto{\pgfqpoint{4.304524in}{0.773334in}}%
\pgfpathlineto{\pgfqpoint{4.313297in}{0.771391in}}%
\pgfpathlineto{\pgfqpoint{4.322070in}{0.773334in}}%
\pgfpathlineto{\pgfqpoint{4.330844in}{0.773334in}}%
\pgfpathlineto{\pgfqpoint{4.339617in}{0.777220in}}%
\pgfpathlineto{\pgfqpoint{4.348390in}{0.777220in}}%
\pgfpathlineto{\pgfqpoint{4.357164in}{0.779163in}}%
\pgfpathlineto{\pgfqpoint{4.374710in}{0.779163in}}%
\pgfpathlineto{\pgfqpoint{4.383484in}{0.777220in}}%
\pgfpathlineto{\pgfqpoint{4.506333in}{0.777220in}}%
\pgfpathlineto{\pgfqpoint{4.515106in}{0.779163in}}%
\pgfpathlineto{\pgfqpoint{4.523879in}{0.777220in}}%
\pgfpathlineto{\pgfqpoint{4.532653in}{0.777220in}}%
\pgfpathlineto{\pgfqpoint{4.541426in}{0.779163in}}%
\pgfpathlineto{\pgfqpoint{4.550200in}{0.779163in}}%
\pgfpathlineto{\pgfqpoint{4.558973in}{0.777220in}}%
\pgfpathlineto{\pgfqpoint{4.567746in}{0.777220in}}%
\pgfpathlineto{\pgfqpoint{4.576520in}{0.779163in}}%
\pgfpathlineto{\pgfqpoint{4.585293in}{0.777220in}}%
\pgfpathlineto{\pgfqpoint{4.637933in}{0.777220in}}%
\pgfpathlineto{\pgfqpoint{4.646706in}{0.779163in}}%
\pgfpathlineto{\pgfqpoint{4.673027in}{0.779163in}}%
\pgfpathlineto{\pgfqpoint{4.681800in}{0.781106in}}%
\pgfpathlineto{\pgfqpoint{4.690573in}{0.781106in}}%
\pgfpathlineto{\pgfqpoint{4.699347in}{0.783049in}}%
\pgfpathlineto{\pgfqpoint{4.716893in}{0.783049in}}%
\pgfpathlineto{\pgfqpoint{4.725667in}{0.781106in}}%
\pgfpathlineto{\pgfqpoint{4.804627in}{0.781106in}}%
\pgfpathlineto{\pgfqpoint{4.813400in}{0.779163in}}%
\pgfpathlineto{\pgfqpoint{4.822174in}{0.779163in}}%
\pgfpathlineto{\pgfqpoint{4.830947in}{0.781106in}}%
\pgfpathlineto{\pgfqpoint{4.892361in}{0.767506in}}%
\pgfpathlineto{\pgfqpoint{4.901134in}{0.763620in}}%
\pgfpathlineto{\pgfqpoint{4.918681in}{0.767506in}}%
\pgfpathlineto{\pgfqpoint{4.927454in}{0.771391in}}%
\pgfpathlineto{\pgfqpoint{4.936227in}{0.773334in}}%
\pgfpathlineto{\pgfqpoint{4.945023in}{0.773334in}}%
\pgfpathlineto{\pgfqpoint{4.953796in}{0.775277in}}%
\pgfpathlineto{\pgfqpoint{4.980116in}{0.775277in}}%
\pgfpathlineto{\pgfqpoint{4.988889in}{0.773334in}}%
\pgfpathlineto{\pgfqpoint{4.997663in}{0.773334in}}%
\pgfpathlineto{\pgfqpoint{5.015204in}{0.777220in}}%
\pgfpathlineto{\pgfqpoint{5.023960in}{0.777220in}}%
\pgfpathlineto{\pgfqpoint{5.032734in}{0.779163in}}%
\pgfpathlineto{\pgfqpoint{5.041508in}{0.777220in}}%
\pgfpathlineto{\pgfqpoint{5.050281in}{0.779163in}}%
\pgfpathlineto{\pgfqpoint{5.076601in}{0.779163in}}%
\pgfpathlineto{\pgfqpoint{5.094147in}{0.775277in}}%
\pgfpathlineto{\pgfqpoint{5.102921in}{0.777220in}}%
\pgfpathlineto{\pgfqpoint{5.120462in}{0.773334in}}%
\pgfpathlineto{\pgfqpoint{5.146766in}{0.773334in}}%
\pgfpathlineto{\pgfqpoint{5.155539in}{0.771391in}}%
\pgfpathlineto{\pgfqpoint{5.164313in}{0.773334in}}%
\pgfpathlineto{\pgfqpoint{5.181859in}{0.773334in}}%
\pgfpathlineto{\pgfqpoint{5.190633in}{0.771391in}}%
\pgfpathlineto{\pgfqpoint{5.208179in}{0.771391in}}%
\pgfpathlineto{\pgfqpoint{5.216953in}{0.769449in}}%
\pgfpathlineto{\pgfqpoint{5.225720in}{0.771391in}}%
\pgfpathlineto{\pgfqpoint{5.252024in}{0.771391in}}%
\pgfpathlineto{\pgfqpoint{5.260798in}{0.769449in}}%
\pgfpathlineto{\pgfqpoint{5.269571in}{0.769449in}}%
\pgfpathlineto{\pgfqpoint{5.278344in}{0.767506in}}%
\pgfpathlineto{\pgfqpoint{5.287118in}{0.763620in}}%
\pgfpathlineto{\pgfqpoint{5.295891in}{0.763620in}}%
\pgfpathlineto{\pgfqpoint{5.313438in}{0.759734in}}%
\pgfpathlineto{\pgfqpoint{5.322211in}{0.755849in}}%
\pgfpathlineto{\pgfqpoint{5.330984in}{0.755849in}}%
\pgfpathlineto{\pgfqpoint{5.339752in}{0.753906in}}%
\pgfpathlineto{\pgfqpoint{5.348509in}{0.750020in}}%
\pgfpathlineto{\pgfqpoint{5.366056in}{0.746134in}}%
\pgfpathlineto{\pgfqpoint{5.374829in}{0.746134in}}%
\pgfpathlineto{\pgfqpoint{5.383603in}{0.744191in}}%
\pgfpathlineto{\pgfqpoint{5.409901in}{0.744191in}}%
\pgfpathlineto{\pgfqpoint{5.427447in}{0.740306in}}%
\pgfpathlineto{\pgfqpoint{5.436221in}{0.740306in}}%
\pgfpathlineto{\pgfqpoint{5.444993in}{0.738363in}}%
\pgfpathlineto{\pgfqpoint{5.488861in}{0.738363in}}%
\pgfpathlineto{\pgfqpoint{5.500556in}{0.735773in}}%
\pgfpathlineto{\pgfqpoint{5.500556in}{0.735773in}}%
\pgfusepath{stroke}%
\end{pgfscope}%
\begin{pgfscope}%
\pgfpathrectangle{\pgfqpoint{0.530556in}{0.656763in}}{\pgfqpoint{4.960000in}{3.264000in}}%
\pgfusepath{clip}%
\pgfsetrectcap%
\pgfsetroundjoin%
\pgfsetlinewidth{1.505625pt}%
\definecolor{currentstroke}{rgb}{0.172549,0.627451,0.172549}%
\pgfsetstrokecolor{currentstroke}%
\pgfsetdash{}{0pt}%
\pgfpathmoveto{\pgfqpoint{0.520556in}{0.730591in}}%
\pgfpathlineto{\pgfqpoint{0.531803in}{0.730591in}}%
\pgfpathlineto{\pgfqpoint{0.540576in}{0.728649in}}%
\pgfpathlineto{\pgfqpoint{0.558122in}{0.728649in}}%
\pgfpathlineto{\pgfqpoint{0.566896in}{0.730591in}}%
\pgfpathlineto{\pgfqpoint{0.593194in}{0.730591in}}%
\pgfpathlineto{\pgfqpoint{0.601968in}{0.728649in}}%
\pgfpathlineto{\pgfqpoint{0.610741in}{0.728649in}}%
\pgfpathlineto{\pgfqpoint{0.619536in}{0.730591in}}%
\pgfpathlineto{\pgfqpoint{0.628310in}{0.728649in}}%
\pgfpathlineto{\pgfqpoint{0.637083in}{0.728649in}}%
\pgfpathlineto{\pgfqpoint{0.645857in}{0.730591in}}%
\pgfpathlineto{\pgfqpoint{0.663403in}{0.730591in}}%
\pgfpathlineto{\pgfqpoint{0.672177in}{0.728649in}}%
\pgfpathlineto{\pgfqpoint{0.689723in}{0.728649in}}%
\pgfpathlineto{\pgfqpoint{0.698497in}{0.730591in}}%
\pgfpathlineto{\pgfqpoint{0.707270in}{0.728649in}}%
\pgfpathlineto{\pgfqpoint{0.716043in}{0.730591in}}%
\pgfpathlineto{\pgfqpoint{0.768684in}{0.730591in}}%
\pgfpathlineto{\pgfqpoint{0.786252in}{0.726706in}}%
\pgfpathlineto{\pgfqpoint{0.795022in}{0.728649in}}%
\pgfpathlineto{\pgfqpoint{0.821346in}{0.728649in}}%
\pgfpathlineto{\pgfqpoint{0.830119in}{0.726706in}}%
\pgfpathlineto{\pgfqpoint{0.838893in}{0.726706in}}%
\pgfpathlineto{\pgfqpoint{0.847666in}{0.728649in}}%
\pgfpathlineto{\pgfqpoint{0.856439in}{0.726706in}}%
\pgfpathlineto{\pgfqpoint{0.926604in}{0.726706in}}%
\pgfpathlineto{\pgfqpoint{0.935377in}{0.728649in}}%
\pgfpathlineto{\pgfqpoint{0.961697in}{0.728649in}}%
\pgfpathlineto{\pgfqpoint{0.979244in}{0.732534in}}%
\pgfpathlineto{\pgfqpoint{1.023105in}{0.732534in}}%
\pgfpathlineto{\pgfqpoint{1.031862in}{0.734477in}}%
\pgfpathlineto{\pgfqpoint{1.066956in}{0.734477in}}%
\pgfpathlineto{\pgfqpoint{1.084502in}{0.738363in}}%
\pgfpathlineto{\pgfqpoint{1.093276in}{0.738363in}}%
\pgfpathlineto{\pgfqpoint{1.102049in}{0.740306in}}%
\pgfpathlineto{\pgfqpoint{1.110823in}{0.738363in}}%
\pgfpathlineto{\pgfqpoint{1.137121in}{0.738363in}}%
\pgfpathlineto{\pgfqpoint{1.145894in}{0.740306in}}%
\pgfpathlineto{\pgfqpoint{1.163441in}{0.736420in}}%
\pgfpathlineto{\pgfqpoint{1.233621in}{0.736420in}}%
\pgfpathlineto{\pgfqpoint{1.242379in}{0.734477in}}%
\pgfpathlineto{\pgfqpoint{1.251152in}{0.734477in}}%
\pgfpathlineto{\pgfqpoint{1.259926in}{0.732534in}}%
\pgfpathlineto{\pgfqpoint{1.268699in}{0.732534in}}%
\pgfpathlineto{\pgfqpoint{1.286246in}{0.728649in}}%
\pgfpathlineto{\pgfqpoint{1.295019in}{0.728649in}}%
\pgfpathlineto{\pgfqpoint{1.303792in}{0.726706in}}%
\pgfpathlineto{\pgfqpoint{1.312565in}{0.726706in}}%
\pgfpathlineto{\pgfqpoint{1.330112in}{0.722820in}}%
\pgfpathlineto{\pgfqpoint{1.338880in}{0.722820in}}%
\pgfpathlineto{\pgfqpoint{1.347637in}{0.720877in}}%
\pgfpathlineto{\pgfqpoint{1.365184in}{0.720877in}}%
\pgfpathlineto{\pgfqpoint{1.373957in}{0.718934in}}%
\pgfpathlineto{\pgfqpoint{1.391504in}{0.718934in}}%
\pgfpathlineto{\pgfqpoint{1.400277in}{0.716991in}}%
\pgfpathlineto{\pgfqpoint{1.409293in}{0.718934in}}%
\pgfpathlineto{\pgfqpoint{1.418066in}{0.718934in}}%
\pgfpathlineto{\pgfqpoint{1.426840in}{0.715049in}}%
\pgfpathlineto{\pgfqpoint{1.444387in}{0.715049in}}%
\pgfpathlineto{\pgfqpoint{1.453160in}{0.713106in}}%
\pgfpathlineto{\pgfqpoint{1.461933in}{0.713106in}}%
\pgfpathlineto{\pgfqpoint{1.470707in}{0.715049in}}%
\pgfpathlineto{\pgfqpoint{1.479477in}{0.713106in}}%
\pgfpathlineto{\pgfqpoint{1.488253in}{0.715049in}}%
\pgfpathlineto{\pgfqpoint{1.497027in}{0.713106in}}%
\pgfpathlineto{\pgfqpoint{1.505800in}{0.715049in}}%
\pgfpathlineto{\pgfqpoint{1.514570in}{0.715049in}}%
\pgfpathlineto{\pgfqpoint{1.523347in}{0.713106in}}%
\pgfpathlineto{\pgfqpoint{1.540893in}{0.713106in}}%
\pgfpathlineto{\pgfqpoint{1.549667in}{0.709220in}}%
\pgfpathlineto{\pgfqpoint{1.567214in}{0.713106in}}%
\pgfpathlineto{\pgfqpoint{1.575986in}{0.711163in}}%
\pgfpathlineto{\pgfqpoint{1.584760in}{0.713106in}}%
\pgfpathlineto{\pgfqpoint{1.619832in}{0.713106in}}%
\pgfpathlineto{\pgfqpoint{1.628605in}{0.715049in}}%
\pgfpathlineto{\pgfqpoint{1.663698in}{0.715049in}}%
\pgfpathlineto{\pgfqpoint{1.681245in}{0.718934in}}%
\pgfpathlineto{\pgfqpoint{1.690019in}{0.718934in}}%
\pgfpathlineto{\pgfqpoint{1.707565in}{0.722820in}}%
\pgfpathlineto{\pgfqpoint{1.733885in}{0.722820in}}%
\pgfpathlineto{\pgfqpoint{1.751432in}{0.726706in}}%
\pgfpathlineto{\pgfqpoint{1.760205in}{0.724763in}}%
\pgfpathlineto{\pgfqpoint{1.768979in}{0.726706in}}%
\pgfpathlineto{\pgfqpoint{1.786525in}{0.726706in}}%
\pgfpathlineto{\pgfqpoint{1.795299in}{0.728649in}}%
\pgfpathlineto{\pgfqpoint{1.830392in}{0.728649in}}%
\pgfpathlineto{\pgfqpoint{1.847939in}{0.732534in}}%
\pgfpathlineto{\pgfqpoint{1.883032in}{0.732534in}}%
\pgfpathlineto{\pgfqpoint{1.891806in}{0.730591in}}%
\pgfpathlineto{\pgfqpoint{1.935673in}{0.730591in}}%
\pgfpathlineto{\pgfqpoint{1.944446in}{0.728649in}}%
\pgfpathlineto{\pgfqpoint{2.005859in}{0.728649in}}%
\pgfpathlineto{\pgfqpoint{2.014633in}{0.730591in}}%
\pgfpathlineto{\pgfqpoint{2.023406in}{0.730591in}}%
\pgfpathlineto{\pgfqpoint{2.032179in}{0.732534in}}%
\pgfpathlineto{\pgfqpoint{2.040953in}{0.730591in}}%
\pgfpathlineto{\pgfqpoint{2.049723in}{0.730591in}}%
\pgfpathlineto{\pgfqpoint{2.058500in}{0.732534in}}%
\pgfpathlineto{\pgfqpoint{2.067273in}{0.732534in}}%
\pgfpathlineto{\pgfqpoint{2.076046in}{0.734477in}}%
\pgfpathlineto{\pgfqpoint{2.084820in}{0.734477in}}%
\pgfpathlineto{\pgfqpoint{2.093593in}{0.732534in}}%
\pgfpathlineto{\pgfqpoint{2.172575in}{0.732534in}}%
\pgfpathlineto{\pgfqpoint{2.181349in}{0.734477in}}%
\pgfpathlineto{\pgfqpoint{2.233989in}{0.734477in}}%
\pgfpathlineto{\pgfqpoint{2.242762in}{0.736420in}}%
\pgfpathlineto{\pgfqpoint{2.251535in}{0.736420in}}%
\pgfpathlineto{\pgfqpoint{2.269082in}{0.740306in}}%
\pgfpathlineto{\pgfqpoint{2.277856in}{0.740306in}}%
\pgfpathlineto{\pgfqpoint{2.286629in}{0.742249in}}%
\pgfpathlineto{\pgfqpoint{2.295402in}{0.742249in}}%
\pgfpathlineto{\pgfqpoint{2.304176in}{0.744191in}}%
\pgfpathlineto{\pgfqpoint{2.321722in}{0.744191in}}%
\pgfpathlineto{\pgfqpoint{2.330496in}{0.746134in}}%
\pgfpathlineto{\pgfqpoint{2.339269in}{0.744191in}}%
\pgfpathlineto{\pgfqpoint{2.348042in}{0.746134in}}%
\pgfpathlineto{\pgfqpoint{2.356816in}{0.746134in}}%
\pgfpathlineto{\pgfqpoint{2.365589in}{0.750020in}}%
\pgfpathlineto{\pgfqpoint{2.400683in}{0.750020in}}%
\pgfpathlineto{\pgfqpoint{2.409456in}{0.751963in}}%
\pgfpathlineto{\pgfqpoint{2.453323in}{0.751963in}}%
\pgfpathlineto{\pgfqpoint{2.462096in}{0.750020in}}%
\pgfpathlineto{\pgfqpoint{2.470869in}{0.751963in}}%
\pgfpathlineto{\pgfqpoint{2.479643in}{0.751963in}}%
\pgfpathlineto{\pgfqpoint{2.488416in}{0.750020in}}%
\pgfpathlineto{\pgfqpoint{2.497190in}{0.751963in}}%
\pgfpathlineto{\pgfqpoint{2.523510in}{0.751963in}}%
\pgfpathlineto{\pgfqpoint{2.532283in}{0.750020in}}%
\pgfpathlineto{\pgfqpoint{2.541056in}{0.750020in}}%
\pgfpathlineto{\pgfqpoint{2.549830in}{0.748077in}}%
\pgfpathlineto{\pgfqpoint{2.558603in}{0.750020in}}%
\pgfpathlineto{\pgfqpoint{2.567376in}{0.748077in}}%
\pgfpathlineto{\pgfqpoint{2.628790in}{0.748077in}}%
\pgfpathlineto{\pgfqpoint{2.637563in}{0.751963in}}%
\pgfpathlineto{\pgfqpoint{2.646337in}{0.750020in}}%
\pgfpathlineto{\pgfqpoint{2.655110in}{0.750020in}}%
\pgfpathlineto{\pgfqpoint{2.664126in}{0.751963in}}%
\pgfpathlineto{\pgfqpoint{2.707993in}{0.751963in}}%
\pgfpathlineto{\pgfqpoint{2.716766in}{0.753906in}}%
\pgfpathlineto{\pgfqpoint{2.734313in}{0.753906in}}%
\pgfpathlineto{\pgfqpoint{2.743080in}{0.755849in}}%
\pgfpathlineto{\pgfqpoint{2.795682in}{0.755849in}}%
\pgfpathlineto{\pgfqpoint{2.804449in}{0.753906in}}%
\pgfpathlineto{\pgfqpoint{2.857074in}{0.753906in}}%
\pgfpathlineto{\pgfqpoint{2.865847in}{0.755849in}}%
\pgfpathlineto{\pgfqpoint{2.874620in}{0.759734in}}%
\pgfpathlineto{\pgfqpoint{2.883394in}{0.757791in}}%
\pgfpathlineto{\pgfqpoint{2.892167in}{0.763620in}}%
\pgfpathlineto{\pgfqpoint{2.900941in}{0.761677in}}%
\pgfpathlineto{\pgfqpoint{2.918480in}{0.761677in}}%
\pgfpathlineto{\pgfqpoint{2.927239in}{0.763620in}}%
\pgfpathlineto{\pgfqpoint{2.944779in}{0.763620in}}%
\pgfpathlineto{\pgfqpoint{2.953537in}{0.765563in}}%
\pgfpathlineto{\pgfqpoint{2.988630in}{0.765563in}}%
\pgfpathlineto{\pgfqpoint{2.997403in}{0.767506in}}%
\pgfpathlineto{\pgfqpoint{3.023724in}{0.767506in}}%
\pgfpathlineto{\pgfqpoint{3.032497in}{0.769449in}}%
\pgfpathlineto{\pgfqpoint{3.050044in}{0.765563in}}%
\pgfpathlineto{\pgfqpoint{3.058817in}{0.767506in}}%
\pgfpathlineto{\pgfqpoint{3.067590in}{0.765563in}}%
\pgfpathlineto{\pgfqpoint{3.076364in}{0.765563in}}%
\pgfpathlineto{\pgfqpoint{3.093910in}{0.769449in}}%
\pgfpathlineto{\pgfqpoint{3.129004in}{0.769449in}}%
\pgfpathlineto{\pgfqpoint{3.137776in}{0.771391in}}%
\pgfpathlineto{\pgfqpoint{3.164091in}{0.771391in}}%
\pgfpathlineto{\pgfqpoint{3.172849in}{0.773334in}}%
\pgfpathlineto{\pgfqpoint{3.181622in}{0.773334in}}%
\pgfpathlineto{\pgfqpoint{3.190395in}{0.771391in}}%
\pgfpathlineto{\pgfqpoint{3.243035in}{0.771391in}}%
\pgfpathlineto{\pgfqpoint{3.251809in}{0.773334in}}%
\pgfpathlineto{\pgfqpoint{3.260582in}{0.771391in}}%
\pgfpathlineto{\pgfqpoint{3.269349in}{0.771391in}}%
\pgfpathlineto{\pgfqpoint{3.278107in}{0.773334in}}%
\pgfpathlineto{\pgfqpoint{3.295653in}{0.773334in}}%
\pgfpathlineto{\pgfqpoint{3.304427in}{0.771391in}}%
\pgfpathlineto{\pgfqpoint{3.313200in}{0.773334in}}%
\pgfpathlineto{\pgfqpoint{3.321974in}{0.773334in}}%
\pgfpathlineto{\pgfqpoint{3.330747in}{0.775277in}}%
\pgfpathlineto{\pgfqpoint{3.348316in}{0.775277in}}%
\pgfpathlineto{\pgfqpoint{3.357089in}{0.771391in}}%
\pgfpathlineto{\pgfqpoint{3.365862in}{0.769449in}}%
\pgfpathlineto{\pgfqpoint{3.374636in}{0.771391in}}%
\pgfpathlineto{\pgfqpoint{3.383409in}{0.769449in}}%
\pgfpathlineto{\pgfqpoint{3.400956in}{0.769449in}}%
\pgfpathlineto{\pgfqpoint{3.409729in}{0.767506in}}%
\pgfpathlineto{\pgfqpoint{3.427276in}{0.767506in}}%
\pgfpathlineto{\pgfqpoint{3.436049in}{0.769449in}}%
\pgfpathlineto{\pgfqpoint{3.453574in}{0.769449in}}%
\pgfpathlineto{\pgfqpoint{3.462347in}{0.771391in}}%
\pgfpathlineto{\pgfqpoint{3.479894in}{0.771391in}}%
\pgfpathlineto{\pgfqpoint{3.488667in}{0.773334in}}%
\pgfpathlineto{\pgfqpoint{3.558854in}{0.773334in}}%
\pgfpathlineto{\pgfqpoint{3.567628in}{0.771391in}}%
\pgfpathlineto{\pgfqpoint{3.576401in}{0.771391in}}%
\pgfpathlineto{\pgfqpoint{3.585174in}{0.769449in}}%
\pgfpathlineto{\pgfqpoint{3.593948in}{0.771391in}}%
\pgfpathlineto{\pgfqpoint{3.602716in}{0.769449in}}%
\pgfpathlineto{\pgfqpoint{3.611472in}{0.771391in}}%
\pgfpathlineto{\pgfqpoint{3.620241in}{0.769449in}}%
\pgfpathlineto{\pgfqpoint{3.628997in}{0.771391in}}%
\pgfpathlineto{\pgfqpoint{3.655317in}{0.771391in}}%
\pgfpathlineto{\pgfqpoint{3.664090in}{0.773334in}}%
\pgfpathlineto{\pgfqpoint{3.734277in}{0.773334in}}%
\pgfpathlineto{\pgfqpoint{3.743051in}{0.771391in}}%
\pgfpathlineto{\pgfqpoint{3.778144in}{0.771391in}}%
\pgfpathlineto{\pgfqpoint{3.786917in}{0.773334in}}%
\pgfpathlineto{\pgfqpoint{3.795691in}{0.771391in}}%
\pgfpathlineto{\pgfqpoint{3.804464in}{0.773334in}}%
\pgfpathlineto{\pgfqpoint{3.813237in}{0.771391in}}%
\pgfpathlineto{\pgfqpoint{3.822011in}{0.773334in}}%
\pgfpathlineto{\pgfqpoint{3.830784in}{0.771391in}}%
\pgfpathlineto{\pgfqpoint{3.839558in}{0.773334in}}%
\pgfpathlineto{\pgfqpoint{3.848331in}{0.773334in}}%
\pgfpathlineto{\pgfqpoint{3.857104in}{0.775277in}}%
\pgfpathlineto{\pgfqpoint{3.892198in}{0.775277in}}%
\pgfpathlineto{\pgfqpoint{3.909744in}{0.779163in}}%
\pgfpathlineto{\pgfqpoint{3.918518in}{0.775277in}}%
\pgfpathlineto{\pgfqpoint{3.927291in}{0.777220in}}%
\pgfpathlineto{\pgfqpoint{3.953611in}{0.777220in}}%
\pgfpathlineto{\pgfqpoint{3.962385in}{0.775277in}}%
\pgfpathlineto{\pgfqpoint{3.979931in}{0.775277in}}%
\pgfpathlineto{\pgfqpoint{3.988705in}{0.773334in}}%
\pgfpathlineto{\pgfqpoint{4.050118in}{0.773334in}}%
\pgfpathlineto{\pgfqpoint{4.058891in}{0.775277in}}%
\pgfpathlineto{\pgfqpoint{4.067665in}{0.775277in}}%
\pgfpathlineto{\pgfqpoint{4.076438in}{0.773334in}}%
\pgfpathlineto{\pgfqpoint{4.093985in}{0.773334in}}%
\pgfpathlineto{\pgfqpoint{4.102758in}{0.777220in}}%
\pgfpathlineto{\pgfqpoint{4.164172in}{0.777220in}}%
\pgfpathlineto{\pgfqpoint{4.172945in}{0.775277in}}%
\pgfpathlineto{\pgfqpoint{4.181719in}{0.775277in}}%
\pgfpathlineto{\pgfqpoint{4.190492in}{0.777220in}}%
\pgfpathlineto{\pgfqpoint{4.234337in}{0.777220in}}%
\pgfpathlineto{\pgfqpoint{4.243110in}{0.775277in}}%
\pgfpathlineto{\pgfqpoint{4.251883in}{0.775277in}}%
\pgfpathlineto{\pgfqpoint{4.260657in}{0.777220in}}%
\pgfpathlineto{\pgfqpoint{4.286977in}{0.777220in}}%
\pgfpathlineto{\pgfqpoint{4.295750in}{0.775277in}}%
\pgfpathlineto{\pgfqpoint{4.304524in}{0.777220in}}%
\pgfpathlineto{\pgfqpoint{4.330844in}{0.777220in}}%
\pgfpathlineto{\pgfqpoint{4.339617in}{0.779163in}}%
\pgfpathlineto{\pgfqpoint{4.348390in}{0.779163in}}%
\pgfpathlineto{\pgfqpoint{4.357164in}{0.781106in}}%
\pgfpathlineto{\pgfqpoint{4.374710in}{0.781106in}}%
\pgfpathlineto{\pgfqpoint{4.383484in}{0.783049in}}%
\pgfpathlineto{\pgfqpoint{4.392257in}{0.781106in}}%
\pgfpathlineto{\pgfqpoint{4.436124in}{0.781106in}}%
\pgfpathlineto{\pgfqpoint{4.444897in}{0.779163in}}%
\pgfpathlineto{\pgfqpoint{4.453671in}{0.779163in}}%
\pgfpathlineto{\pgfqpoint{4.462444in}{0.781106in}}%
\pgfpathlineto{\pgfqpoint{4.471239in}{0.779163in}}%
\pgfpathlineto{\pgfqpoint{4.480013in}{0.781106in}}%
\pgfpathlineto{\pgfqpoint{4.515106in}{0.781106in}}%
\pgfpathlineto{\pgfqpoint{4.523879in}{0.779163in}}%
\pgfpathlineto{\pgfqpoint{4.532653in}{0.781106in}}%
\pgfpathlineto{\pgfqpoint{4.576520in}{0.781106in}}%
\pgfpathlineto{\pgfqpoint{4.585293in}{0.783049in}}%
\pgfpathlineto{\pgfqpoint{4.594066in}{0.781106in}}%
\pgfpathlineto{\pgfqpoint{4.655480in}{0.781106in}}%
\pgfpathlineto{\pgfqpoint{4.664253in}{0.783049in}}%
\pgfpathlineto{\pgfqpoint{4.708120in}{0.783049in}}%
\pgfpathlineto{\pgfqpoint{4.716893in}{0.784991in}}%
\pgfpathlineto{\pgfqpoint{4.743213in}{0.784991in}}%
\pgfpathlineto{\pgfqpoint{4.751987in}{0.783049in}}%
\pgfpathlineto{\pgfqpoint{4.760760in}{0.783049in}}%
\pgfpathlineto{\pgfqpoint{4.769534in}{0.784991in}}%
\pgfpathlineto{\pgfqpoint{4.778307in}{0.784991in}}%
\pgfpathlineto{\pgfqpoint{4.787080in}{0.783049in}}%
\pgfpathlineto{\pgfqpoint{4.795854in}{0.784991in}}%
\pgfpathlineto{\pgfqpoint{4.804627in}{0.784991in}}%
\pgfpathlineto{\pgfqpoint{4.813400in}{0.783049in}}%
\pgfpathlineto{\pgfqpoint{4.822174in}{0.784991in}}%
\pgfpathlineto{\pgfqpoint{4.830947in}{0.783049in}}%
\pgfpathlineto{\pgfqpoint{4.839720in}{0.784991in}}%
\pgfpathlineto{\pgfqpoint{4.848494in}{0.781106in}}%
\pgfpathlineto{\pgfqpoint{4.857267in}{0.781106in}}%
\pgfpathlineto{\pgfqpoint{4.874814in}{0.777220in}}%
\pgfpathlineto{\pgfqpoint{4.883587in}{0.773334in}}%
\pgfpathlineto{\pgfqpoint{4.892361in}{0.771391in}}%
\pgfpathlineto{\pgfqpoint{4.927454in}{0.771391in}}%
\pgfpathlineto{\pgfqpoint{4.953796in}{0.777220in}}%
\pgfpathlineto{\pgfqpoint{4.962569in}{0.777220in}}%
\pgfpathlineto{\pgfqpoint{4.971343in}{0.775277in}}%
\pgfpathlineto{\pgfqpoint{4.980116in}{0.777220in}}%
\pgfpathlineto{\pgfqpoint{5.023960in}{0.777220in}}%
\pgfpathlineto{\pgfqpoint{5.032734in}{0.781106in}}%
\pgfpathlineto{\pgfqpoint{5.067828in}{0.781106in}}%
\pgfpathlineto{\pgfqpoint{5.076601in}{0.779163in}}%
\pgfpathlineto{\pgfqpoint{5.085374in}{0.781106in}}%
\pgfpathlineto{\pgfqpoint{5.094147in}{0.779163in}}%
\pgfpathlineto{\pgfqpoint{5.111695in}{0.779163in}}%
\pgfpathlineto{\pgfqpoint{5.120462in}{0.777220in}}%
\pgfpathlineto{\pgfqpoint{5.181859in}{0.777220in}}%
\pgfpathlineto{\pgfqpoint{5.190633in}{0.773334in}}%
\pgfpathlineto{\pgfqpoint{5.199405in}{0.775277in}}%
\pgfpathlineto{\pgfqpoint{5.208179in}{0.773334in}}%
\pgfpathlineto{\pgfqpoint{5.216953in}{0.775277in}}%
\pgfpathlineto{\pgfqpoint{5.225720in}{0.773334in}}%
\pgfpathlineto{\pgfqpoint{5.234477in}{0.773334in}}%
\pgfpathlineto{\pgfqpoint{5.243251in}{0.775277in}}%
\pgfpathlineto{\pgfqpoint{5.260798in}{0.775277in}}%
\pgfpathlineto{\pgfqpoint{5.269571in}{0.773334in}}%
\pgfpathlineto{\pgfqpoint{5.295891in}{0.773334in}}%
\pgfpathlineto{\pgfqpoint{5.304664in}{0.771391in}}%
\pgfpathlineto{\pgfqpoint{5.330984in}{0.771391in}}%
\pgfpathlineto{\pgfqpoint{5.339752in}{0.767506in}}%
\pgfpathlineto{\pgfqpoint{5.348509in}{0.765563in}}%
\pgfpathlineto{\pgfqpoint{5.366056in}{0.765563in}}%
\pgfpathlineto{\pgfqpoint{5.374829in}{0.763620in}}%
\pgfpathlineto{\pgfqpoint{5.409901in}{0.763620in}}%
\pgfpathlineto{\pgfqpoint{5.418674in}{0.759734in}}%
\pgfpathlineto{\pgfqpoint{5.436221in}{0.759734in}}%
\pgfpathlineto{\pgfqpoint{5.444993in}{0.757791in}}%
\pgfpathlineto{\pgfqpoint{5.453767in}{0.757791in}}%
\pgfpathlineto{\pgfqpoint{5.462541in}{0.755849in}}%
\pgfpathlineto{\pgfqpoint{5.471314in}{0.755849in}}%
\pgfpathlineto{\pgfqpoint{5.480087in}{0.757791in}}%
\pgfpathlineto{\pgfqpoint{5.500556in}{0.757791in}}%
\pgfpathlineto{\pgfqpoint{5.500556in}{0.757791in}}%
\pgfusepath{stroke}%
\end{pgfscope}%
\begin{pgfscope}%
\pgfsetrectcap%
\pgfsetmiterjoin%
\pgfsetlinewidth{0.803000pt}%
\definecolor{currentstroke}{rgb}{0.000000,0.000000,0.000000}%
\pgfsetstrokecolor{currentstroke}%
\pgfsetdash{}{0pt}%
\pgfpathmoveto{\pgfqpoint{0.530556in}{0.656763in}}%
\pgfpathlineto{\pgfqpoint{0.530556in}{3.920763in}}%
\pgfusepath{stroke}%
\end{pgfscope}%
\begin{pgfscope}%
\pgfsetrectcap%
\pgfsetmiterjoin%
\pgfsetlinewidth{0.803000pt}%
\definecolor{currentstroke}{rgb}{0.000000,0.000000,0.000000}%
\pgfsetstrokecolor{currentstroke}%
\pgfsetdash{}{0pt}%
\pgfpathmoveto{\pgfqpoint{5.490556in}{0.656763in}}%
\pgfpathlineto{\pgfqpoint{5.490556in}{3.920763in}}%
\pgfusepath{stroke}%
\end{pgfscope}%
\begin{pgfscope}%
\pgfsetrectcap%
\pgfsetmiterjoin%
\pgfsetlinewidth{0.803000pt}%
\definecolor{currentstroke}{rgb}{0.000000,0.000000,0.000000}%
\pgfsetstrokecolor{currentstroke}%
\pgfsetdash{}{0pt}%
\pgfpathmoveto{\pgfqpoint{0.530556in}{0.656763in}}%
\pgfpathlineto{\pgfqpoint{5.490556in}{0.656763in}}%
\pgfusepath{stroke}%
\end{pgfscope}%
\begin{pgfscope}%
\pgfsetrectcap%
\pgfsetmiterjoin%
\pgfsetlinewidth{0.803000pt}%
\definecolor{currentstroke}{rgb}{0.000000,0.000000,0.000000}%
\pgfsetstrokecolor{currentstroke}%
\pgfsetdash{}{0pt}%
\pgfpathmoveto{\pgfqpoint{0.530556in}{3.920763in}}%
\pgfpathlineto{\pgfqpoint{5.490556in}{3.920763in}}%
\pgfusepath{stroke}%
\end{pgfscope}%
\begin{pgfscope}%
\pgfsetbuttcap%
\pgfsetmiterjoin%
\definecolor{currentfill}{rgb}{1.000000,1.000000,1.000000}%
\pgfsetfillcolor{currentfill}%
\pgfsetfillopacity{0.800000}%
\pgfsetlinewidth{1.003750pt}%
\definecolor{currentstroke}{rgb}{0.800000,0.800000,0.800000}%
\pgfsetstrokecolor{currentstroke}%
\pgfsetstrokeopacity{0.800000}%
\pgfsetdash{}{0pt}%
\pgfpathmoveto{\pgfqpoint{3.801511in}{3.228633in}}%
\pgfpathlineto{\pgfqpoint{5.393334in}{3.228633in}}%
\pgfpathquadraticcurveto{\pgfqpoint{5.421112in}{3.228633in}}{\pgfqpoint{5.421112in}{3.256411in}}%
\pgfpathlineto{\pgfqpoint{5.421112in}{3.823541in}}%
\pgfpathquadraticcurveto{\pgfqpoint{5.421112in}{3.851318in}}{\pgfqpoint{5.393334in}{3.851318in}}%
\pgfpathlineto{\pgfqpoint{3.801511in}{3.851318in}}%
\pgfpathquadraticcurveto{\pgfqpoint{3.773733in}{3.851318in}}{\pgfqpoint{3.773733in}{3.823541in}}%
\pgfpathlineto{\pgfqpoint{3.773733in}{3.256411in}}%
\pgfpathquadraticcurveto{\pgfqpoint{3.773733in}{3.228633in}}{\pgfqpoint{3.801511in}{3.228633in}}%
\pgfpathclose%
\pgfusepath{stroke,fill}%
\end{pgfscope}%
\begin{pgfscope}%
\pgfsetrectcap%
\pgfsetroundjoin%
\pgfsetlinewidth{1.505625pt}%
\definecolor{currentstroke}{rgb}{0.121569,0.466667,0.705882}%
\pgfsetstrokecolor{currentstroke}%
\pgfsetdash{}{0pt}%
\pgfpathmoveto{\pgfqpoint{3.829289in}{3.747152in}}%
\pgfpathlineto{\pgfqpoint{4.107066in}{3.747152in}}%
\pgfusepath{stroke}%
\end{pgfscope}%
\begin{pgfscope}%
\definecolor{textcolor}{rgb}{0.000000,0.000000,0.000000}%
\pgfsetstrokecolor{textcolor}%
\pgfsetfillcolor{textcolor}%
\pgftext[x=4.218177in,y=3.698541in,left,base]{\color{textcolor}\rmfamily\fontsize{10.000000}{12.000000}\selectfont CPU}%
\end{pgfscope}%
\begin{pgfscope}%
\pgfsetrectcap%
\pgfsetroundjoin%
\pgfsetlinewidth{1.505625pt}%
\definecolor{currentstroke}{rgb}{1.000000,0.498039,0.054902}%
\pgfsetstrokecolor{currentstroke}%
\pgfsetdash{}{0pt}%
\pgfpathmoveto{\pgfqpoint{3.829289in}{3.553479in}}%
\pgfpathlineto{\pgfqpoint{4.107066in}{3.553479in}}%
\pgfusepath{stroke}%
\end{pgfscope}%
\begin{pgfscope}%
\definecolor{textcolor}{rgb}{0.000000,0.000000,0.000000}%
\pgfsetstrokecolor{textcolor}%
\pgfsetfillcolor{textcolor}%
\pgftext[x=4.218177in,y=3.504868in,left,base]{\color{textcolor}\rmfamily\fontsize{10.000000}{12.000000}\selectfont Zone Raspberry Pi}%
\end{pgfscope}%
\begin{pgfscope}%
\pgfsetrectcap%
\pgfsetroundjoin%
\pgfsetlinewidth{1.505625pt}%
\definecolor{currentstroke}{rgb}{0.172549,0.627451,0.172549}%
\pgfsetstrokecolor{currentstroke}%
\pgfsetdash{}{0pt}%
\pgfpathmoveto{\pgfqpoint{3.829289in}{3.359806in}}%
\pgfpathlineto{\pgfqpoint{4.107066in}{3.359806in}}%
\pgfusepath{stroke}%
\end{pgfscope}%
\begin{pgfscope}%
\definecolor{textcolor}{rgb}{0.000000,0.000000,0.000000}%
\pgfsetstrokecolor{textcolor}%
\pgfsetfillcolor{textcolor}%
\pgftext[x=4.218177in,y=3.311195in,left,base]{\color{textcolor}\rmfamily\fontsize{10.000000}{12.000000}\selectfont Zone step-down}%
\end{pgfscope}%
\end{pgfpicture}%
\makeatother%
\endgroup%

  \label{fig:test_5}
  \vspace{-0.2cm}
  \caption{\textbf{Test 5 :} Évolution de la température dans les tests réalisés dans le nouveau boîtier}
\end{figure}


\begin{figure}[ht!]
  \centering
  %% Creator: Matplotlib, PGF backend
%%
%% To include the figure in your LaTeX document, write
%%   \input{<filename>.pgf}
%%
%% Make sure the required packages are loaded in your preamble
%%   \usepackage{pgf}
%%
%% Figures using additional raster images can only be included by \input if
%% they are in the same directory as the main LaTeX file. For loading figures
%% from other directories you can use the `import` package
%%   \usepackage{import}
%% and then include the figures with
%%   \import{<path to file>}{<filename>.pgf}
%%
%% Matplotlib used the following preamble
%%
\begingroup%
\makeatletter%
\begin{pgfpicture}%
\pgfpathrectangle{\pgfpointorigin}{\pgfqpoint{6.400000in}{4.800000in}}%
\pgfusepath{use as bounding box, clip}%
\begin{pgfscope}%
\pgfsetbuttcap%
\pgfsetmiterjoin%
\definecolor{currentfill}{rgb}{1.000000,1.000000,1.000000}%
\pgfsetfillcolor{currentfill}%
\pgfsetlinewidth{0.000000pt}%
\definecolor{currentstroke}{rgb}{1.000000,1.000000,1.000000}%
\pgfsetstrokecolor{currentstroke}%
\pgfsetdash{}{0pt}%
\pgfpathmoveto{\pgfqpoint{0.000000in}{0.000000in}}%
\pgfpathlineto{\pgfqpoint{6.400000in}{0.000000in}}%
\pgfpathlineto{\pgfqpoint{6.400000in}{4.800000in}}%
\pgfpathlineto{\pgfqpoint{0.000000in}{4.800000in}}%
\pgfpathclose%
\pgfusepath{fill}%
\end{pgfscope}%
\begin{pgfscope}%
\pgfsetbuttcap%
\pgfsetmiterjoin%
\definecolor{currentfill}{rgb}{1.000000,1.000000,1.000000}%
\pgfsetfillcolor{currentfill}%
\pgfsetlinewidth{0.000000pt}%
\definecolor{currentstroke}{rgb}{0.000000,0.000000,0.000000}%
\pgfsetstrokecolor{currentstroke}%
\pgfsetstrokeopacity{0.000000}%
\pgfsetdash{}{0pt}%
\pgfpathmoveto{\pgfqpoint{0.800000in}{0.960000in}}%
\pgfpathlineto{\pgfqpoint{5.760000in}{0.960000in}}%
\pgfpathlineto{\pgfqpoint{5.760000in}{4.224000in}}%
\pgfpathlineto{\pgfqpoint{0.800000in}{4.224000in}}%
\pgfpathclose%
\pgfusepath{fill}%
\end{pgfscope}%
\begin{pgfscope}%
\pgfsetbuttcap%
\pgfsetroundjoin%
\definecolor{currentfill}{rgb}{0.000000,0.000000,0.000000}%
\pgfsetfillcolor{currentfill}%
\pgfsetlinewidth{0.803000pt}%
\definecolor{currentstroke}{rgb}{0.000000,0.000000,0.000000}%
\pgfsetstrokecolor{currentstroke}%
\pgfsetdash{}{0pt}%
\pgfsys@defobject{currentmarker}{\pgfqpoint{0.000000in}{-0.048611in}}{\pgfqpoint{0.000000in}{0.000000in}}{%
\pgfpathmoveto{\pgfqpoint{0.000000in}{0.000000in}}%
\pgfpathlineto{\pgfqpoint{0.000000in}{-0.048611in}}%
\pgfusepath{stroke,fill}%
}%
\begin{pgfscope}%
\pgfsys@transformshift{0.800000in}{0.960000in}%
\pgfsys@useobject{currentmarker}{}%
\end{pgfscope}%
\end{pgfscope}%
\begin{pgfscope}%
\definecolor{textcolor}{rgb}{0.000000,0.000000,0.000000}%
\pgfsetstrokecolor{textcolor}%
\pgfsetfillcolor{textcolor}%
\pgftext[x=0.512522in,y=0.621070in,left,base,rotate=30.000000]{\color{textcolor}\rmfamily\fontsize{10.000000}{12.000000}\selectfont 00:00}%
\end{pgfscope}%
\begin{pgfscope}%
\pgfsetbuttcap%
\pgfsetroundjoin%
\definecolor{currentfill}{rgb}{0.000000,0.000000,0.000000}%
\pgfsetfillcolor{currentfill}%
\pgfsetlinewidth{0.803000pt}%
\definecolor{currentstroke}{rgb}{0.000000,0.000000,0.000000}%
\pgfsetstrokecolor{currentstroke}%
\pgfsetdash{}{0pt}%
\pgfsys@defobject{currentmarker}{\pgfqpoint{0.000000in}{-0.048611in}}{\pgfqpoint{0.000000in}{0.000000in}}{%
\pgfpathmoveto{\pgfqpoint{0.000000in}{0.000000in}}%
\pgfpathlineto{\pgfqpoint{0.000000in}{-0.048611in}}%
\pgfusepath{stroke,fill}%
}%
\begin{pgfscope}%
\pgfsys@transformshift{1.791871in}{0.960000in}%
\pgfsys@useobject{currentmarker}{}%
\end{pgfscope}%
\end{pgfscope}%
\begin{pgfscope}%
\definecolor{textcolor}{rgb}{0.000000,0.000000,0.000000}%
\pgfsetstrokecolor{textcolor}%
\pgfsetfillcolor{textcolor}%
\pgftext[x=1.504393in,y=0.621070in,left,base,rotate=30.000000]{\color{textcolor}\rmfamily\fontsize{10.000000}{12.000000}\selectfont 01:00}%
\end{pgfscope}%
\begin{pgfscope}%
\pgfsetbuttcap%
\pgfsetroundjoin%
\definecolor{currentfill}{rgb}{0.000000,0.000000,0.000000}%
\pgfsetfillcolor{currentfill}%
\pgfsetlinewidth{0.803000pt}%
\definecolor{currentstroke}{rgb}{0.000000,0.000000,0.000000}%
\pgfsetstrokecolor{currentstroke}%
\pgfsetdash{}{0pt}%
\pgfsys@defobject{currentmarker}{\pgfqpoint{0.000000in}{-0.048611in}}{\pgfqpoint{0.000000in}{0.000000in}}{%
\pgfpathmoveto{\pgfqpoint{0.000000in}{0.000000in}}%
\pgfpathlineto{\pgfqpoint{0.000000in}{-0.048611in}}%
\pgfusepath{stroke,fill}%
}%
\begin{pgfscope}%
\pgfsys@transformshift{2.783742in}{0.960000in}%
\pgfsys@useobject{currentmarker}{}%
\end{pgfscope}%
\end{pgfscope}%
\begin{pgfscope}%
\definecolor{textcolor}{rgb}{0.000000,0.000000,0.000000}%
\pgfsetstrokecolor{textcolor}%
\pgfsetfillcolor{textcolor}%
\pgftext[x=2.496264in,y=0.621070in,left,base,rotate=30.000000]{\color{textcolor}\rmfamily\fontsize{10.000000}{12.000000}\selectfont 02:00}%
\end{pgfscope}%
\begin{pgfscope}%
\pgfsetbuttcap%
\pgfsetroundjoin%
\definecolor{currentfill}{rgb}{0.000000,0.000000,0.000000}%
\pgfsetfillcolor{currentfill}%
\pgfsetlinewidth{0.803000pt}%
\definecolor{currentstroke}{rgb}{0.000000,0.000000,0.000000}%
\pgfsetstrokecolor{currentstroke}%
\pgfsetdash{}{0pt}%
\pgfsys@defobject{currentmarker}{\pgfqpoint{0.000000in}{-0.048611in}}{\pgfqpoint{0.000000in}{0.000000in}}{%
\pgfpathmoveto{\pgfqpoint{0.000000in}{0.000000in}}%
\pgfpathlineto{\pgfqpoint{0.000000in}{-0.048611in}}%
\pgfusepath{stroke,fill}%
}%
\begin{pgfscope}%
\pgfsys@transformshift{3.775613in}{0.960000in}%
\pgfsys@useobject{currentmarker}{}%
\end{pgfscope}%
\end{pgfscope}%
\begin{pgfscope}%
\definecolor{textcolor}{rgb}{0.000000,0.000000,0.000000}%
\pgfsetstrokecolor{textcolor}%
\pgfsetfillcolor{textcolor}%
\pgftext[x=3.488135in,y=0.621070in,left,base,rotate=30.000000]{\color{textcolor}\rmfamily\fontsize{10.000000}{12.000000}\selectfont 03:00}%
\end{pgfscope}%
\begin{pgfscope}%
\pgfsetbuttcap%
\pgfsetroundjoin%
\definecolor{currentfill}{rgb}{0.000000,0.000000,0.000000}%
\pgfsetfillcolor{currentfill}%
\pgfsetlinewidth{0.803000pt}%
\definecolor{currentstroke}{rgb}{0.000000,0.000000,0.000000}%
\pgfsetstrokecolor{currentstroke}%
\pgfsetdash{}{0pt}%
\pgfsys@defobject{currentmarker}{\pgfqpoint{0.000000in}{-0.048611in}}{\pgfqpoint{0.000000in}{0.000000in}}{%
\pgfpathmoveto{\pgfqpoint{0.000000in}{0.000000in}}%
\pgfpathlineto{\pgfqpoint{0.000000in}{-0.048611in}}%
\pgfusepath{stroke,fill}%
}%
\begin{pgfscope}%
\pgfsys@transformshift{4.767484in}{0.960000in}%
\pgfsys@useobject{currentmarker}{}%
\end{pgfscope}%
\end{pgfscope}%
\begin{pgfscope}%
\definecolor{textcolor}{rgb}{0.000000,0.000000,0.000000}%
\pgfsetstrokecolor{textcolor}%
\pgfsetfillcolor{textcolor}%
\pgftext[x=4.480006in,y=0.621070in,left,base,rotate=30.000000]{\color{textcolor}\rmfamily\fontsize{10.000000}{12.000000}\selectfont 04:00}%
\end{pgfscope}%
\begin{pgfscope}%
\pgfsetbuttcap%
\pgfsetroundjoin%
\definecolor{currentfill}{rgb}{0.000000,0.000000,0.000000}%
\pgfsetfillcolor{currentfill}%
\pgfsetlinewidth{0.803000pt}%
\definecolor{currentstroke}{rgb}{0.000000,0.000000,0.000000}%
\pgfsetstrokecolor{currentstroke}%
\pgfsetdash{}{0pt}%
\pgfsys@defobject{currentmarker}{\pgfqpoint{0.000000in}{-0.048611in}}{\pgfqpoint{0.000000in}{0.000000in}}{%
\pgfpathmoveto{\pgfqpoint{0.000000in}{0.000000in}}%
\pgfpathlineto{\pgfqpoint{0.000000in}{-0.048611in}}%
\pgfusepath{stroke,fill}%
}%
\begin{pgfscope}%
\pgfsys@transformshift{5.759355in}{0.960000in}%
\pgfsys@useobject{currentmarker}{}%
\end{pgfscope}%
\end{pgfscope}%
\begin{pgfscope}%
\definecolor{textcolor}{rgb}{0.000000,0.000000,0.000000}%
\pgfsetstrokecolor{textcolor}%
\pgfsetfillcolor{textcolor}%
\pgftext[x=5.471877in,y=0.621070in,left,base,rotate=30.000000]{\color{textcolor}\rmfamily\fontsize{10.000000}{12.000000}\selectfont 05:00}%
\end{pgfscope}%
\begin{pgfscope}%
\definecolor{textcolor}{rgb}{0.000000,0.000000,0.000000}%
\pgfsetstrokecolor{textcolor}%
\pgfsetfillcolor{textcolor}%
\pgftext[x=3.280000in,y=0.542126in,,top]{\color{textcolor}\rmfamily\fontsize{10.000000}{12.000000}\selectfont Temps (hh:mm)}%
\end{pgfscope}%
\begin{pgfscope}%
\pgfsetbuttcap%
\pgfsetroundjoin%
\definecolor{currentfill}{rgb}{0.000000,0.000000,0.000000}%
\pgfsetfillcolor{currentfill}%
\pgfsetlinewidth{0.803000pt}%
\definecolor{currentstroke}{rgb}{0.000000,0.000000,0.000000}%
\pgfsetstrokecolor{currentstroke}%
\pgfsetdash{}{0pt}%
\pgfsys@defobject{currentmarker}{\pgfqpoint{-0.048611in}{0.000000in}}{\pgfqpoint{0.000000in}{0.000000in}}{%
\pgfpathmoveto{\pgfqpoint{0.000000in}{0.000000in}}%
\pgfpathlineto{\pgfqpoint{-0.048611in}{0.000000in}}%
\pgfusepath{stroke,fill}%
}%
\begin{pgfscope}%
\pgfsys@transformshift{0.800000in}{0.960000in}%
\pgfsys@useobject{currentmarker}{}%
\end{pgfscope}%
\end{pgfscope}%
\begin{pgfscope}%
\definecolor{textcolor}{rgb}{0.000000,0.000000,0.000000}%
\pgfsetstrokecolor{textcolor}%
\pgfsetfillcolor{textcolor}%
\pgftext[x=0.563888in,y=0.911775in,left,base]{\color{textcolor}\rmfamily\fontsize{10.000000}{12.000000}\selectfont \(\displaystyle 20\)}%
\end{pgfscope}%
\begin{pgfscope}%
\pgfsetbuttcap%
\pgfsetroundjoin%
\definecolor{currentfill}{rgb}{0.000000,0.000000,0.000000}%
\pgfsetfillcolor{currentfill}%
\pgfsetlinewidth{0.803000pt}%
\definecolor{currentstroke}{rgb}{0.000000,0.000000,0.000000}%
\pgfsetstrokecolor{currentstroke}%
\pgfsetdash{}{0pt}%
\pgfsys@defobject{currentmarker}{\pgfqpoint{-0.048611in}{0.000000in}}{\pgfqpoint{0.000000in}{0.000000in}}{%
\pgfpathmoveto{\pgfqpoint{0.000000in}{0.000000in}}%
\pgfpathlineto{\pgfqpoint{-0.048611in}{0.000000in}}%
\pgfusepath{stroke,fill}%
}%
\begin{pgfscope}%
\pgfsys@transformshift{0.800000in}{1.426286in}%
\pgfsys@useobject{currentmarker}{}%
\end{pgfscope}%
\end{pgfscope}%
\begin{pgfscope}%
\definecolor{textcolor}{rgb}{0.000000,0.000000,0.000000}%
\pgfsetstrokecolor{textcolor}%
\pgfsetfillcolor{textcolor}%
\pgftext[x=0.563888in,y=1.378060in,left,base]{\color{textcolor}\rmfamily\fontsize{10.000000}{12.000000}\selectfont \(\displaystyle 25\)}%
\end{pgfscope}%
\begin{pgfscope}%
\pgfsetbuttcap%
\pgfsetroundjoin%
\definecolor{currentfill}{rgb}{0.000000,0.000000,0.000000}%
\pgfsetfillcolor{currentfill}%
\pgfsetlinewidth{0.803000pt}%
\definecolor{currentstroke}{rgb}{0.000000,0.000000,0.000000}%
\pgfsetstrokecolor{currentstroke}%
\pgfsetdash{}{0pt}%
\pgfsys@defobject{currentmarker}{\pgfqpoint{-0.048611in}{0.000000in}}{\pgfqpoint{0.000000in}{0.000000in}}{%
\pgfpathmoveto{\pgfqpoint{0.000000in}{0.000000in}}%
\pgfpathlineto{\pgfqpoint{-0.048611in}{0.000000in}}%
\pgfusepath{stroke,fill}%
}%
\begin{pgfscope}%
\pgfsys@transformshift{0.800000in}{1.892571in}%
\pgfsys@useobject{currentmarker}{}%
\end{pgfscope}%
\end{pgfscope}%
\begin{pgfscope}%
\definecolor{textcolor}{rgb}{0.000000,0.000000,0.000000}%
\pgfsetstrokecolor{textcolor}%
\pgfsetfillcolor{textcolor}%
\pgftext[x=0.563888in,y=1.844346in,left,base]{\color{textcolor}\rmfamily\fontsize{10.000000}{12.000000}\selectfont \(\displaystyle 30\)}%
\end{pgfscope}%
\begin{pgfscope}%
\pgfsetbuttcap%
\pgfsetroundjoin%
\definecolor{currentfill}{rgb}{0.000000,0.000000,0.000000}%
\pgfsetfillcolor{currentfill}%
\pgfsetlinewidth{0.803000pt}%
\definecolor{currentstroke}{rgb}{0.000000,0.000000,0.000000}%
\pgfsetstrokecolor{currentstroke}%
\pgfsetdash{}{0pt}%
\pgfsys@defobject{currentmarker}{\pgfqpoint{-0.048611in}{0.000000in}}{\pgfqpoint{0.000000in}{0.000000in}}{%
\pgfpathmoveto{\pgfqpoint{0.000000in}{0.000000in}}%
\pgfpathlineto{\pgfqpoint{-0.048611in}{0.000000in}}%
\pgfusepath{stroke,fill}%
}%
\begin{pgfscope}%
\pgfsys@transformshift{0.800000in}{2.358857in}%
\pgfsys@useobject{currentmarker}{}%
\end{pgfscope}%
\end{pgfscope}%
\begin{pgfscope}%
\definecolor{textcolor}{rgb}{0.000000,0.000000,0.000000}%
\pgfsetstrokecolor{textcolor}%
\pgfsetfillcolor{textcolor}%
\pgftext[x=0.563888in,y=2.310632in,left,base]{\color{textcolor}\rmfamily\fontsize{10.000000}{12.000000}\selectfont \(\displaystyle 35\)}%
\end{pgfscope}%
\begin{pgfscope}%
\pgfsetbuttcap%
\pgfsetroundjoin%
\definecolor{currentfill}{rgb}{0.000000,0.000000,0.000000}%
\pgfsetfillcolor{currentfill}%
\pgfsetlinewidth{0.803000pt}%
\definecolor{currentstroke}{rgb}{0.000000,0.000000,0.000000}%
\pgfsetstrokecolor{currentstroke}%
\pgfsetdash{}{0pt}%
\pgfsys@defobject{currentmarker}{\pgfqpoint{-0.048611in}{0.000000in}}{\pgfqpoint{0.000000in}{0.000000in}}{%
\pgfpathmoveto{\pgfqpoint{0.000000in}{0.000000in}}%
\pgfpathlineto{\pgfqpoint{-0.048611in}{0.000000in}}%
\pgfusepath{stroke,fill}%
}%
\begin{pgfscope}%
\pgfsys@transformshift{0.800000in}{2.825143in}%
\pgfsys@useobject{currentmarker}{}%
\end{pgfscope}%
\end{pgfscope}%
\begin{pgfscope}%
\definecolor{textcolor}{rgb}{0.000000,0.000000,0.000000}%
\pgfsetstrokecolor{textcolor}%
\pgfsetfillcolor{textcolor}%
\pgftext[x=0.563888in,y=2.776918in,left,base]{\color{textcolor}\rmfamily\fontsize{10.000000}{12.000000}\selectfont \(\displaystyle 40\)}%
\end{pgfscope}%
\begin{pgfscope}%
\pgfsetbuttcap%
\pgfsetroundjoin%
\definecolor{currentfill}{rgb}{0.000000,0.000000,0.000000}%
\pgfsetfillcolor{currentfill}%
\pgfsetlinewidth{0.803000pt}%
\definecolor{currentstroke}{rgb}{0.000000,0.000000,0.000000}%
\pgfsetstrokecolor{currentstroke}%
\pgfsetdash{}{0pt}%
\pgfsys@defobject{currentmarker}{\pgfqpoint{-0.048611in}{0.000000in}}{\pgfqpoint{0.000000in}{0.000000in}}{%
\pgfpathmoveto{\pgfqpoint{0.000000in}{0.000000in}}%
\pgfpathlineto{\pgfqpoint{-0.048611in}{0.000000in}}%
\pgfusepath{stroke,fill}%
}%
\begin{pgfscope}%
\pgfsys@transformshift{0.800000in}{3.291429in}%
\pgfsys@useobject{currentmarker}{}%
\end{pgfscope}%
\end{pgfscope}%
\begin{pgfscope}%
\definecolor{textcolor}{rgb}{0.000000,0.000000,0.000000}%
\pgfsetstrokecolor{textcolor}%
\pgfsetfillcolor{textcolor}%
\pgftext[x=0.563888in,y=3.243203in,left,base]{\color{textcolor}\rmfamily\fontsize{10.000000}{12.000000}\selectfont \(\displaystyle 45\)}%
\end{pgfscope}%
\begin{pgfscope}%
\pgfsetbuttcap%
\pgfsetroundjoin%
\definecolor{currentfill}{rgb}{0.000000,0.000000,0.000000}%
\pgfsetfillcolor{currentfill}%
\pgfsetlinewidth{0.803000pt}%
\definecolor{currentstroke}{rgb}{0.000000,0.000000,0.000000}%
\pgfsetstrokecolor{currentstroke}%
\pgfsetdash{}{0pt}%
\pgfsys@defobject{currentmarker}{\pgfqpoint{-0.048611in}{0.000000in}}{\pgfqpoint{0.000000in}{0.000000in}}{%
\pgfpathmoveto{\pgfqpoint{0.000000in}{0.000000in}}%
\pgfpathlineto{\pgfqpoint{-0.048611in}{0.000000in}}%
\pgfusepath{stroke,fill}%
}%
\begin{pgfscope}%
\pgfsys@transformshift{0.800000in}{3.757714in}%
\pgfsys@useobject{currentmarker}{}%
\end{pgfscope}%
\end{pgfscope}%
\begin{pgfscope}%
\definecolor{textcolor}{rgb}{0.000000,0.000000,0.000000}%
\pgfsetstrokecolor{textcolor}%
\pgfsetfillcolor{textcolor}%
\pgftext[x=0.563888in,y=3.709489in,left,base]{\color{textcolor}\rmfamily\fontsize{10.000000}{12.000000}\selectfont \(\displaystyle 50\)}%
\end{pgfscope}%
\begin{pgfscope}%
\pgfsetbuttcap%
\pgfsetroundjoin%
\definecolor{currentfill}{rgb}{0.000000,0.000000,0.000000}%
\pgfsetfillcolor{currentfill}%
\pgfsetlinewidth{0.803000pt}%
\definecolor{currentstroke}{rgb}{0.000000,0.000000,0.000000}%
\pgfsetstrokecolor{currentstroke}%
\pgfsetdash{}{0pt}%
\pgfsys@defobject{currentmarker}{\pgfqpoint{-0.048611in}{0.000000in}}{\pgfqpoint{0.000000in}{0.000000in}}{%
\pgfpathmoveto{\pgfqpoint{0.000000in}{0.000000in}}%
\pgfpathlineto{\pgfqpoint{-0.048611in}{0.000000in}}%
\pgfusepath{stroke,fill}%
}%
\begin{pgfscope}%
\pgfsys@transformshift{0.800000in}{4.224000in}%
\pgfsys@useobject{currentmarker}{}%
\end{pgfscope}%
\end{pgfscope}%
\begin{pgfscope}%
\definecolor{textcolor}{rgb}{0.000000,0.000000,0.000000}%
\pgfsetstrokecolor{textcolor}%
\pgfsetfillcolor{textcolor}%
\pgftext[x=0.563888in,y=4.175775in,left,base]{\color{textcolor}\rmfamily\fontsize{10.000000}{12.000000}\selectfont \(\displaystyle 55\)}%
\end{pgfscope}%
\begin{pgfscope}%
\definecolor{textcolor}{rgb}{0.000000,0.000000,0.000000}%
\pgfsetstrokecolor{textcolor}%
\pgfsetfillcolor{textcolor}%
\pgftext[x=0.508333in,y=2.592000in,,bottom,rotate=90.000000]{\color{textcolor}\rmfamily\fontsize{10.000000}{12.000000}\selectfont Température (\textdegree{}C)}%
\end{pgfscope}%
\begin{pgfscope}%
\pgfpathrectangle{\pgfqpoint{0.800000in}{0.960000in}}{\pgfqpoint{4.960000in}{3.264000in}}%
\pgfusepath{clip}%
\pgfsetrectcap%
\pgfsetroundjoin%
\pgfsetlinewidth{1.505625pt}%
\definecolor{currentstroke}{rgb}{0.121569,0.466667,0.705882}%
\pgfsetstrokecolor{currentstroke}%
\pgfsetdash{}{0pt}%
\pgfpathmoveto{\pgfqpoint{0.808303in}{2.125714in}}%
\pgfpathlineto{\pgfqpoint{0.816579in}{2.206537in}}%
\pgfpathlineto{\pgfqpoint{0.824856in}{2.222080in}}%
\pgfpathlineto{\pgfqpoint{0.833132in}{2.240731in}}%
\pgfpathlineto{\pgfqpoint{0.841409in}{2.256274in}}%
\pgfpathlineto{\pgfqpoint{0.849686in}{2.274926in}}%
\pgfpathlineto{\pgfqpoint{0.857963in}{2.259383in}}%
\pgfpathlineto{\pgfqpoint{0.866239in}{2.290469in}}%
\pgfpathlineto{\pgfqpoint{0.874510in}{2.274926in}}%
\pgfpathlineto{\pgfqpoint{0.882787in}{2.290469in}}%
\pgfpathlineto{\pgfqpoint{0.891063in}{2.293577in}}%
\pgfpathlineto{\pgfqpoint{0.899341in}{2.309120in}}%
\pgfpathlineto{\pgfqpoint{0.907616in}{2.309120in}}%
\pgfpathlineto{\pgfqpoint{0.915888in}{2.293577in}}%
\pgfpathlineto{\pgfqpoint{0.924166in}{2.309120in}}%
\pgfpathlineto{\pgfqpoint{0.932441in}{2.293577in}}%
\pgfpathlineto{\pgfqpoint{0.940712in}{2.290469in}}%
\pgfpathlineto{\pgfqpoint{0.948989in}{2.259383in}}%
\pgfpathlineto{\pgfqpoint{0.957267in}{2.293577in}}%
\pgfpathlineto{\pgfqpoint{0.965542in}{2.274926in}}%
\pgfpathlineto{\pgfqpoint{0.973818in}{2.293577in}}%
\pgfpathlineto{\pgfqpoint{0.982095in}{2.309120in}}%
\pgfpathlineto{\pgfqpoint{0.990372in}{2.293577in}}%
\pgfpathlineto{\pgfqpoint{0.998648in}{2.324663in}}%
\pgfpathlineto{\pgfqpoint{1.006925in}{2.293577in}}%
\pgfpathlineto{\pgfqpoint{1.015201in}{2.324663in}}%
\pgfpathlineto{\pgfqpoint{1.023478in}{2.309120in}}%
\pgfpathlineto{\pgfqpoint{1.031755in}{2.309120in}}%
\pgfpathlineto{\pgfqpoint{1.040031in}{2.293577in}}%
\pgfpathlineto{\pgfqpoint{1.048306in}{2.293577in}}%
\pgfpathlineto{\pgfqpoint{1.056582in}{2.309120in}}%
\pgfpathlineto{\pgfqpoint{1.064857in}{2.293577in}}%
\pgfpathlineto{\pgfqpoint{1.073134in}{2.309120in}}%
\pgfpathlineto{\pgfqpoint{1.081409in}{2.309120in}}%
\pgfpathlineto{\pgfqpoint{1.089679in}{2.274926in}}%
\pgfpathlineto{\pgfqpoint{1.097954in}{2.309120in}}%
\pgfpathlineto{\pgfqpoint{1.106223in}{2.293577in}}%
\pgfpathlineto{\pgfqpoint{1.122776in}{2.293577in}}%
\pgfpathlineto{\pgfqpoint{1.131046in}{2.324663in}}%
\pgfpathlineto{\pgfqpoint{1.139322in}{2.327771in}}%
\pgfpathlineto{\pgfqpoint{1.147600in}{2.293577in}}%
\pgfpathlineto{\pgfqpoint{1.155876in}{2.293577in}}%
\pgfpathlineto{\pgfqpoint{1.164153in}{2.324663in}}%
\pgfpathlineto{\pgfqpoint{1.172429in}{2.293577in}}%
\pgfpathlineto{\pgfqpoint{1.180706in}{2.290469in}}%
\pgfpathlineto{\pgfqpoint{1.188983in}{2.324663in}}%
\pgfpathlineto{\pgfqpoint{1.205539in}{2.293577in}}%
\pgfpathlineto{\pgfqpoint{1.213815in}{2.293577in}}%
\pgfpathlineto{\pgfqpoint{1.222085in}{2.324663in}}%
\pgfpathlineto{\pgfqpoint{1.230355in}{2.274926in}}%
\pgfpathlineto{\pgfqpoint{1.238629in}{2.324663in}}%
\pgfpathlineto{\pgfqpoint{1.246900in}{2.340206in}}%
\pgfpathlineto{\pgfqpoint{1.263456in}{2.309120in}}%
\pgfpathlineto{\pgfqpoint{1.280007in}{2.340206in}}%
\pgfpathlineto{\pgfqpoint{1.288283in}{2.324663in}}%
\pgfpathlineto{\pgfqpoint{1.296560in}{2.340206in}}%
\pgfpathlineto{\pgfqpoint{1.304836in}{2.309120in}}%
\pgfpathlineto{\pgfqpoint{1.313113in}{2.324663in}}%
\pgfpathlineto{\pgfqpoint{1.329666in}{2.324663in}}%
\pgfpathlineto{\pgfqpoint{1.337942in}{2.293577in}}%
\pgfpathlineto{\pgfqpoint{1.362757in}{2.340206in}}%
\pgfpathlineto{\pgfqpoint{1.371032in}{2.290469in}}%
\pgfpathlineto{\pgfqpoint{1.379310in}{2.309120in}}%
\pgfpathlineto{\pgfqpoint{1.387585in}{2.340206in}}%
\pgfpathlineto{\pgfqpoint{1.404140in}{2.340206in}}%
\pgfpathlineto{\pgfqpoint{1.412410in}{2.358857in}}%
\pgfpathlineto{\pgfqpoint{1.420686in}{2.324663in}}%
\pgfpathlineto{\pgfqpoint{1.428964in}{2.358857in}}%
\pgfpathlineto{\pgfqpoint{1.437236in}{2.324663in}}%
\pgfpathlineto{\pgfqpoint{1.445508in}{2.324663in}}%
\pgfpathlineto{\pgfqpoint{1.453786in}{2.293577in}}%
\pgfpathlineto{\pgfqpoint{1.470332in}{2.358857in}}%
\pgfpathlineto{\pgfqpoint{1.478603in}{2.324663in}}%
\pgfpathlineto{\pgfqpoint{1.486881in}{2.340206in}}%
\pgfpathlineto{\pgfqpoint{1.495158in}{2.290469in}}%
\pgfpathlineto{\pgfqpoint{1.503428in}{2.358857in}}%
\pgfpathlineto{\pgfqpoint{1.511705in}{2.324663in}}%
\pgfpathlineto{\pgfqpoint{1.519976in}{2.340206in}}%
\pgfpathlineto{\pgfqpoint{1.528254in}{2.343314in}}%
\pgfpathlineto{\pgfqpoint{1.536532in}{2.321554in}}%
\pgfpathlineto{\pgfqpoint{1.544809in}{2.324663in}}%
\pgfpathlineto{\pgfqpoint{1.553080in}{2.358857in}}%
\pgfpathlineto{\pgfqpoint{1.561357in}{2.358857in}}%
\pgfpathlineto{\pgfqpoint{1.569634in}{2.324663in}}%
\pgfpathlineto{\pgfqpoint{1.577904in}{2.321554in}}%
\pgfpathlineto{\pgfqpoint{1.586175in}{2.340206in}}%
\pgfpathlineto{\pgfqpoint{1.594451in}{2.306011in}}%
\pgfpathlineto{\pgfqpoint{1.602726in}{2.343314in}}%
\pgfpathlineto{\pgfqpoint{1.611001in}{2.358857in}}%
\pgfpathlineto{\pgfqpoint{1.619273in}{2.321554in}}%
\pgfpathlineto{\pgfqpoint{1.627544in}{2.327771in}}%
\pgfpathlineto{\pgfqpoint{1.635819in}{2.290469in}}%
\pgfpathlineto{\pgfqpoint{1.644098in}{2.324663in}}%
\pgfpathlineto{\pgfqpoint{1.652375in}{2.274926in}}%
\pgfpathlineto{\pgfqpoint{1.660648in}{2.293577in}}%
\pgfpathlineto{\pgfqpoint{1.668927in}{2.324663in}}%
\pgfpathlineto{\pgfqpoint{1.677204in}{2.340206in}}%
\pgfpathlineto{\pgfqpoint{1.685481in}{2.340206in}}%
\pgfpathlineto{\pgfqpoint{1.693753in}{2.324663in}}%
\pgfpathlineto{\pgfqpoint{1.702028in}{2.343314in}}%
\pgfpathlineto{\pgfqpoint{1.710302in}{2.340206in}}%
\pgfpathlineto{\pgfqpoint{1.726853in}{2.340206in}}%
\pgfpathlineto{\pgfqpoint{1.735130in}{2.321554in}}%
\pgfpathlineto{\pgfqpoint{1.743408in}{2.340206in}}%
\pgfpathlineto{\pgfqpoint{1.751686in}{2.340206in}}%
\pgfpathlineto{\pgfqpoint{1.759963in}{2.358857in}}%
\pgfpathlineto{\pgfqpoint{1.768241in}{2.340206in}}%
\pgfpathlineto{\pgfqpoint{1.776521in}{2.324663in}}%
\pgfpathlineto{\pgfqpoint{1.784797in}{2.343314in}}%
\pgfpathlineto{\pgfqpoint{1.793074in}{2.340206in}}%
\pgfpathlineto{\pgfqpoint{1.801352in}{2.324663in}}%
\pgfpathlineto{\pgfqpoint{1.809630in}{2.324663in}}%
\pgfpathlineto{\pgfqpoint{1.817907in}{2.321554in}}%
\pgfpathlineto{\pgfqpoint{1.826185in}{2.309120in}}%
\pgfpathlineto{\pgfqpoint{1.834456in}{2.309120in}}%
\pgfpathlineto{\pgfqpoint{1.851011in}{2.340206in}}%
\pgfpathlineto{\pgfqpoint{1.859286in}{2.324663in}}%
\pgfpathlineto{\pgfqpoint{1.867564in}{2.321554in}}%
\pgfpathlineto{\pgfqpoint{1.875839in}{2.327771in}}%
\pgfpathlineto{\pgfqpoint{1.884116in}{2.324663in}}%
\pgfpathlineto{\pgfqpoint{1.892392in}{2.358857in}}%
\pgfpathlineto{\pgfqpoint{1.900668in}{2.306011in}}%
\pgfpathlineto{\pgfqpoint{1.908943in}{2.340206in}}%
\pgfpathlineto{\pgfqpoint{1.917220in}{2.343314in}}%
\pgfpathlineto{\pgfqpoint{1.925496in}{2.327771in}}%
\pgfpathlineto{\pgfqpoint{1.933773in}{2.358857in}}%
\pgfpathlineto{\pgfqpoint{1.942050in}{2.324663in}}%
\pgfpathlineto{\pgfqpoint{1.950326in}{2.340206in}}%
\pgfpathlineto{\pgfqpoint{1.958603in}{2.358857in}}%
\pgfpathlineto{\pgfqpoint{1.966880in}{2.358857in}}%
\pgfpathlineto{\pgfqpoint{1.975156in}{2.324663in}}%
\pgfpathlineto{\pgfqpoint{1.983434in}{2.327771in}}%
\pgfpathlineto{\pgfqpoint{1.991709in}{2.343314in}}%
\pgfpathlineto{\pgfqpoint{1.999987in}{2.324663in}}%
\pgfpathlineto{\pgfqpoint{2.008263in}{2.309120in}}%
\pgfpathlineto{\pgfqpoint{2.024816in}{2.309120in}}%
\pgfpathlineto{\pgfqpoint{2.033092in}{2.327771in}}%
\pgfpathlineto{\pgfqpoint{2.041369in}{2.309120in}}%
\pgfpathlineto{\pgfqpoint{2.049646in}{2.340206in}}%
\pgfpathlineto{\pgfqpoint{2.057923in}{2.343314in}}%
\pgfpathlineto{\pgfqpoint{2.066200in}{2.324663in}}%
\pgfpathlineto{\pgfqpoint{2.074476in}{2.343314in}}%
\pgfpathlineto{\pgfqpoint{2.082752in}{2.309120in}}%
\pgfpathlineto{\pgfqpoint{2.091029in}{2.309120in}}%
\pgfpathlineto{\pgfqpoint{2.099306in}{2.340206in}}%
\pgfpathlineto{\pgfqpoint{2.107582in}{2.324663in}}%
\pgfpathlineto{\pgfqpoint{2.115855in}{2.343314in}}%
\pgfpathlineto{\pgfqpoint{2.124134in}{2.355749in}}%
\pgfpathlineto{\pgfqpoint{2.132412in}{2.309120in}}%
\pgfpathlineto{\pgfqpoint{2.140692in}{2.327771in}}%
\pgfpathlineto{\pgfqpoint{2.148966in}{2.358857in}}%
\pgfpathlineto{\pgfqpoint{2.157242in}{2.358857in}}%
\pgfpathlineto{\pgfqpoint{2.165516in}{2.343314in}}%
\pgfpathlineto{\pgfqpoint{2.173793in}{2.358857in}}%
\pgfpathlineto{\pgfqpoint{2.182069in}{2.343314in}}%
\pgfpathlineto{\pgfqpoint{2.190341in}{2.343314in}}%
\pgfpathlineto{\pgfqpoint{2.198616in}{2.358857in}}%
\pgfpathlineto{\pgfqpoint{2.206894in}{2.358857in}}%
\pgfpathlineto{\pgfqpoint{2.215167in}{2.327771in}}%
\pgfpathlineto{\pgfqpoint{2.223445in}{2.306011in}}%
\pgfpathlineto{\pgfqpoint{2.231716in}{2.343314in}}%
\pgfpathlineto{\pgfqpoint{2.239993in}{2.324663in}}%
\pgfpathlineto{\pgfqpoint{2.248271in}{2.358857in}}%
\pgfpathlineto{\pgfqpoint{2.256548in}{2.340206in}}%
\pgfpathlineto{\pgfqpoint{2.264818in}{2.358857in}}%
\pgfpathlineto{\pgfqpoint{2.273096in}{2.358857in}}%
\pgfpathlineto{\pgfqpoint{2.281372in}{2.340206in}}%
\pgfpathlineto{\pgfqpoint{2.289649in}{2.274926in}}%
\pgfpathlineto{\pgfqpoint{2.297926in}{2.340206in}}%
\pgfpathlineto{\pgfqpoint{2.306204in}{2.340206in}}%
\pgfpathlineto{\pgfqpoint{2.314475in}{2.306011in}}%
\pgfpathlineto{\pgfqpoint{2.322763in}{2.306011in}}%
\pgfpathlineto{\pgfqpoint{2.331039in}{2.358857in}}%
\pgfpathlineto{\pgfqpoint{2.339316in}{2.343314in}}%
\pgfpathlineto{\pgfqpoint{2.347594in}{2.340206in}}%
\pgfpathlineto{\pgfqpoint{2.355869in}{2.358857in}}%
\pgfpathlineto{\pgfqpoint{2.364147in}{2.274926in}}%
\pgfpathlineto{\pgfqpoint{2.372423in}{2.340206in}}%
\pgfpathlineto{\pgfqpoint{2.380700in}{2.358857in}}%
\pgfpathlineto{\pgfqpoint{2.388979in}{2.340206in}}%
\pgfpathlineto{\pgfqpoint{2.397256in}{2.358857in}}%
\pgfpathlineto{\pgfqpoint{2.405532in}{2.324663in}}%
\pgfpathlineto{\pgfqpoint{2.413808in}{2.358857in}}%
\pgfpathlineto{\pgfqpoint{2.422085in}{2.377509in}}%
\pgfpathlineto{\pgfqpoint{2.430362in}{2.358857in}}%
\pgfpathlineto{\pgfqpoint{2.438634in}{2.343314in}}%
\pgfpathlineto{\pgfqpoint{2.446910in}{2.358857in}}%
\pgfpathlineto{\pgfqpoint{2.463465in}{2.358857in}}%
\pgfpathlineto{\pgfqpoint{2.471743in}{2.324663in}}%
\pgfpathlineto{\pgfqpoint{2.480022in}{2.340206in}}%
\pgfpathlineto{\pgfqpoint{2.488299in}{2.358857in}}%
\pgfpathlineto{\pgfqpoint{2.496577in}{2.327771in}}%
\pgfpathlineto{\pgfqpoint{2.504854in}{2.374400in}}%
\pgfpathlineto{\pgfqpoint{2.513132in}{2.374400in}}%
\pgfpathlineto{\pgfqpoint{2.521404in}{2.358857in}}%
\pgfpathlineto{\pgfqpoint{2.537953in}{2.358857in}}%
\pgfpathlineto{\pgfqpoint{2.546230in}{2.355749in}}%
\pgfpathlineto{\pgfqpoint{2.554505in}{2.358857in}}%
\pgfpathlineto{\pgfqpoint{2.562782in}{2.358857in}}%
\pgfpathlineto{\pgfqpoint{2.571059in}{2.374400in}}%
\pgfpathlineto{\pgfqpoint{2.579335in}{2.343314in}}%
\pgfpathlineto{\pgfqpoint{2.587612in}{2.340206in}}%
\pgfpathlineto{\pgfqpoint{2.595888in}{2.324663in}}%
\pgfpathlineto{\pgfqpoint{2.604165in}{2.340206in}}%
\pgfpathlineto{\pgfqpoint{2.612442in}{2.358857in}}%
\pgfpathlineto{\pgfqpoint{2.620718in}{2.358857in}}%
\pgfpathlineto{\pgfqpoint{2.628995in}{2.340206in}}%
\pgfpathlineto{\pgfqpoint{2.637272in}{2.374400in}}%
\pgfpathlineto{\pgfqpoint{2.645548in}{2.374400in}}%
\pgfpathlineto{\pgfqpoint{2.653825in}{2.358857in}}%
\pgfpathlineto{\pgfqpoint{2.670378in}{2.358857in}}%
\pgfpathlineto{\pgfqpoint{2.678654in}{2.340206in}}%
\pgfpathlineto{\pgfqpoint{2.686931in}{2.393051in}}%
\pgfpathlineto{\pgfqpoint{2.703484in}{2.393051in}}%
\pgfpathlineto{\pgfqpoint{2.711761in}{2.374400in}}%
\pgfpathlineto{\pgfqpoint{2.720038in}{2.340206in}}%
\pgfpathlineto{\pgfqpoint{2.728314in}{2.377509in}}%
\pgfpathlineto{\pgfqpoint{2.736591in}{2.358857in}}%
\pgfpathlineto{\pgfqpoint{2.744867in}{2.374400in}}%
\pgfpathlineto{\pgfqpoint{2.753144in}{2.358857in}}%
\pgfpathlineto{\pgfqpoint{2.761421in}{2.358857in}}%
\pgfpathlineto{\pgfqpoint{2.769698in}{2.393051in}}%
\pgfpathlineto{\pgfqpoint{2.777973in}{2.358857in}}%
\pgfpathlineto{\pgfqpoint{2.794527in}{2.358857in}}%
\pgfpathlineto{\pgfqpoint{2.802804in}{2.377509in}}%
\pgfpathlineto{\pgfqpoint{2.811081in}{2.377509in}}%
\pgfpathlineto{\pgfqpoint{2.819359in}{2.389943in}}%
\pgfpathlineto{\pgfqpoint{2.835913in}{2.358857in}}%
\pgfpathlineto{\pgfqpoint{2.844190in}{2.340206in}}%
\pgfpathlineto{\pgfqpoint{2.852468in}{2.377509in}}%
\pgfpathlineto{\pgfqpoint{2.860744in}{2.374400in}}%
\pgfpathlineto{\pgfqpoint{2.869020in}{2.358857in}}%
\pgfpathlineto{\pgfqpoint{2.877298in}{2.358857in}}%
\pgfpathlineto{\pgfqpoint{2.885573in}{2.377509in}}%
\pgfpathlineto{\pgfqpoint{2.893845in}{2.377509in}}%
\pgfpathlineto{\pgfqpoint{2.902114in}{2.358857in}}%
\pgfpathlineto{\pgfqpoint{2.910386in}{2.358857in}}%
\pgfpathlineto{\pgfqpoint{2.918662in}{2.343314in}}%
\pgfpathlineto{\pgfqpoint{2.926938in}{2.389943in}}%
\pgfpathlineto{\pgfqpoint{2.935216in}{2.377509in}}%
\pgfpathlineto{\pgfqpoint{2.943492in}{2.358857in}}%
\pgfpathlineto{\pgfqpoint{2.960045in}{2.358857in}}%
\pgfpathlineto{\pgfqpoint{2.968322in}{2.340206in}}%
\pgfpathlineto{\pgfqpoint{2.976599in}{2.374400in}}%
\pgfpathlineto{\pgfqpoint{2.984871in}{2.377509in}}%
\pgfpathlineto{\pgfqpoint{2.993149in}{2.358857in}}%
\pgfpathlineto{\pgfqpoint{3.001429in}{2.374400in}}%
\pgfpathlineto{\pgfqpoint{3.009699in}{2.358857in}}%
\pgfpathlineto{\pgfqpoint{3.017976in}{2.374400in}}%
\pgfpathlineto{\pgfqpoint{3.026247in}{2.358857in}}%
\pgfpathlineto{\pgfqpoint{3.034524in}{2.358857in}}%
\pgfpathlineto{\pgfqpoint{3.042800in}{2.377509in}}%
\pgfpathlineto{\pgfqpoint{3.051070in}{2.358857in}}%
\pgfpathlineto{\pgfqpoint{3.059340in}{2.377509in}}%
\pgfpathlineto{\pgfqpoint{3.067618in}{2.377509in}}%
\pgfpathlineto{\pgfqpoint{3.075890in}{2.358857in}}%
\pgfpathlineto{\pgfqpoint{3.092445in}{2.358857in}}%
\pgfpathlineto{\pgfqpoint{3.100722in}{2.324663in}}%
\pgfpathlineto{\pgfqpoint{3.108996in}{2.374400in}}%
\pgfpathlineto{\pgfqpoint{3.117269in}{2.377509in}}%
\pgfpathlineto{\pgfqpoint{3.125541in}{2.358857in}}%
\pgfpathlineto{\pgfqpoint{3.133811in}{2.358857in}}%
\pgfpathlineto{\pgfqpoint{3.142085in}{2.377509in}}%
\pgfpathlineto{\pgfqpoint{3.150359in}{2.377509in}}%
\pgfpathlineto{\pgfqpoint{3.158636in}{2.358857in}}%
\pgfpathlineto{\pgfqpoint{3.166914in}{2.377509in}}%
\pgfpathlineto{\pgfqpoint{3.175191in}{2.377509in}}%
\pgfpathlineto{\pgfqpoint{3.183467in}{2.358857in}}%
\pgfpathlineto{\pgfqpoint{3.224842in}{2.358857in}}%
\pgfpathlineto{\pgfqpoint{3.233119in}{2.393051in}}%
\pgfpathlineto{\pgfqpoint{3.241395in}{2.358857in}}%
\pgfpathlineto{\pgfqpoint{3.274501in}{2.358857in}}%
\pgfpathlineto{\pgfqpoint{3.282779in}{2.343314in}}%
\pgfpathlineto{\pgfqpoint{3.291050in}{2.393051in}}%
\pgfpathlineto{\pgfqpoint{3.299324in}{2.377509in}}%
\pgfpathlineto{\pgfqpoint{3.307596in}{2.340206in}}%
\pgfpathlineto{\pgfqpoint{3.315871in}{2.374400in}}%
\pgfpathlineto{\pgfqpoint{3.324149in}{2.393051in}}%
\pgfpathlineto{\pgfqpoint{3.332425in}{2.393051in}}%
\pgfpathlineto{\pgfqpoint{3.340699in}{2.358857in}}%
\pgfpathlineto{\pgfqpoint{3.348977in}{2.340206in}}%
\pgfpathlineto{\pgfqpoint{3.357258in}{2.340206in}}%
\pgfpathlineto{\pgfqpoint{3.365535in}{2.358857in}}%
\pgfpathlineto{\pgfqpoint{3.373812in}{2.358857in}}%
\pgfpathlineto{\pgfqpoint{3.382088in}{2.340206in}}%
\pgfpathlineto{\pgfqpoint{3.390365in}{2.358857in}}%
\pgfpathlineto{\pgfqpoint{3.398641in}{2.358857in}}%
\pgfpathlineto{\pgfqpoint{3.406916in}{2.393051in}}%
\pgfpathlineto{\pgfqpoint{3.415194in}{2.377509in}}%
\pgfpathlineto{\pgfqpoint{3.423470in}{2.393051in}}%
\pgfpathlineto{\pgfqpoint{3.431743in}{2.358857in}}%
\pgfpathlineto{\pgfqpoint{3.473119in}{2.358857in}}%
\pgfpathlineto{\pgfqpoint{3.481395in}{2.374400in}}%
\pgfpathlineto{\pgfqpoint{3.489672in}{2.358857in}}%
\pgfpathlineto{\pgfqpoint{3.497948in}{2.374400in}}%
\pgfpathlineto{\pgfqpoint{3.506222in}{2.377509in}}%
\pgfpathlineto{\pgfqpoint{3.514500in}{2.393051in}}%
\pgfpathlineto{\pgfqpoint{3.522775in}{2.358857in}}%
\pgfpathlineto{\pgfqpoint{3.531053in}{2.377509in}}%
\pgfpathlineto{\pgfqpoint{3.539330in}{2.377509in}}%
\pgfpathlineto{\pgfqpoint{3.547601in}{2.358857in}}%
\pgfpathlineto{\pgfqpoint{3.572433in}{2.358857in}}%
\pgfpathlineto{\pgfqpoint{3.580709in}{2.374400in}}%
\pgfpathlineto{\pgfqpoint{3.588985in}{2.358857in}}%
\pgfpathlineto{\pgfqpoint{3.597258in}{2.393051in}}%
\pgfpathlineto{\pgfqpoint{3.605534in}{2.377509in}}%
\pgfpathlineto{\pgfqpoint{3.622089in}{2.377509in}}%
\pgfpathlineto{\pgfqpoint{3.630367in}{2.358857in}}%
\pgfpathlineto{\pgfqpoint{3.638643in}{2.358857in}}%
\pgfpathlineto{\pgfqpoint{3.646919in}{2.377509in}}%
\pgfpathlineto{\pgfqpoint{3.655192in}{2.358857in}}%
\pgfpathlineto{\pgfqpoint{3.680023in}{2.358857in}}%
\pgfpathlineto{\pgfqpoint{3.688301in}{2.377509in}}%
\pgfpathlineto{\pgfqpoint{3.696579in}{2.358857in}}%
\pgfpathlineto{\pgfqpoint{3.704854in}{2.377509in}}%
\pgfpathlineto{\pgfqpoint{3.713130in}{2.358857in}}%
\pgfpathlineto{\pgfqpoint{3.721404in}{2.393051in}}%
\pgfpathlineto{\pgfqpoint{3.729678in}{2.374400in}}%
\pgfpathlineto{\pgfqpoint{3.737957in}{2.358857in}}%
\pgfpathlineto{\pgfqpoint{3.746234in}{2.358857in}}%
\pgfpathlineto{\pgfqpoint{3.754512in}{2.377509in}}%
\pgfpathlineto{\pgfqpoint{3.762789in}{2.358857in}}%
\pgfpathlineto{\pgfqpoint{3.771061in}{2.377509in}}%
\pgfpathlineto{\pgfqpoint{3.779336in}{2.393051in}}%
\pgfpathlineto{\pgfqpoint{3.787613in}{2.377509in}}%
\pgfpathlineto{\pgfqpoint{3.804159in}{2.377509in}}%
\pgfpathlineto{\pgfqpoint{3.820708in}{2.340206in}}%
\pgfpathlineto{\pgfqpoint{3.828985in}{2.393051in}}%
\pgfpathlineto{\pgfqpoint{3.837261in}{2.393051in}}%
\pgfpathlineto{\pgfqpoint{3.845536in}{2.358857in}}%
\pgfpathlineto{\pgfqpoint{3.853812in}{2.377509in}}%
\pgfpathlineto{\pgfqpoint{3.862089in}{2.358857in}}%
\pgfpathlineto{\pgfqpoint{3.870366in}{2.358857in}}%
\pgfpathlineto{\pgfqpoint{3.878638in}{2.377509in}}%
\pgfpathlineto{\pgfqpoint{3.886908in}{2.358857in}}%
\pgfpathlineto{\pgfqpoint{3.895181in}{2.358857in}}%
\pgfpathlineto{\pgfqpoint{3.903457in}{2.377509in}}%
\pgfpathlineto{\pgfqpoint{3.911732in}{2.393051in}}%
\pgfpathlineto{\pgfqpoint{3.920009in}{2.374400in}}%
\pgfpathlineto{\pgfqpoint{3.928285in}{2.358857in}}%
\pgfpathlineto{\pgfqpoint{3.936558in}{2.393051in}}%
\pgfpathlineto{\pgfqpoint{3.944835in}{2.358857in}}%
\pgfpathlineto{\pgfqpoint{3.953109in}{2.393051in}}%
\pgfpathlineto{\pgfqpoint{3.961380in}{2.374400in}}%
\pgfpathlineto{\pgfqpoint{3.969656in}{2.377509in}}%
\pgfpathlineto{\pgfqpoint{3.977930in}{2.377509in}}%
\pgfpathlineto{\pgfqpoint{3.986205in}{2.358857in}}%
\pgfpathlineto{\pgfqpoint{3.994481in}{2.358857in}}%
\pgfpathlineto{\pgfqpoint{4.002751in}{2.393051in}}%
\pgfpathlineto{\pgfqpoint{4.011027in}{2.377509in}}%
\pgfpathlineto{\pgfqpoint{4.019303in}{2.340206in}}%
\pgfpathlineto{\pgfqpoint{4.027578in}{2.358857in}}%
\pgfpathlineto{\pgfqpoint{4.052407in}{2.358857in}}%
\pgfpathlineto{\pgfqpoint{4.060684in}{2.377509in}}%
\pgfpathlineto{\pgfqpoint{4.068962in}{2.358857in}}%
\pgfpathlineto{\pgfqpoint{4.077239in}{2.374400in}}%
\pgfpathlineto{\pgfqpoint{4.110346in}{2.374400in}}%
\pgfpathlineto{\pgfqpoint{4.118622in}{2.377509in}}%
\pgfpathlineto{\pgfqpoint{4.126899in}{2.389943in}}%
\pgfpathlineto{\pgfqpoint{4.135176in}{2.393051in}}%
\pgfpathlineto{\pgfqpoint{4.143453in}{2.358857in}}%
\pgfpathlineto{\pgfqpoint{4.151724in}{2.374400in}}%
\pgfpathlineto{\pgfqpoint{4.159997in}{2.358857in}}%
\pgfpathlineto{\pgfqpoint{4.168274in}{2.374400in}}%
\pgfpathlineto{\pgfqpoint{4.176551in}{2.393051in}}%
\pgfpathlineto{\pgfqpoint{4.184827in}{2.358857in}}%
\pgfpathlineto{\pgfqpoint{4.193104in}{2.358857in}}%
\pgfpathlineto{\pgfqpoint{4.201381in}{2.377509in}}%
\pgfpathlineto{\pgfqpoint{4.209659in}{2.358857in}}%
\pgfpathlineto{\pgfqpoint{4.217937in}{2.377509in}}%
\pgfpathlineto{\pgfqpoint{4.226212in}{2.374400in}}%
\pgfpathlineto{\pgfqpoint{4.234490in}{2.424137in}}%
\pgfpathlineto{\pgfqpoint{4.242766in}{2.358857in}}%
\pgfpathlineto{\pgfqpoint{4.251044in}{2.358857in}}%
\pgfpathlineto{\pgfqpoint{4.259319in}{2.340206in}}%
\pgfpathlineto{\pgfqpoint{4.267597in}{2.374400in}}%
\pgfpathlineto{\pgfqpoint{4.275869in}{2.374400in}}%
\pgfpathlineto{\pgfqpoint{4.284138in}{2.358857in}}%
\pgfpathlineto{\pgfqpoint{4.292416in}{2.374400in}}%
\pgfpathlineto{\pgfqpoint{4.317248in}{2.374400in}}%
\pgfpathlineto{\pgfqpoint{4.325519in}{2.358857in}}%
\pgfpathlineto{\pgfqpoint{4.333796in}{2.358857in}}%
\pgfpathlineto{\pgfqpoint{4.342073in}{2.374400in}}%
\pgfpathlineto{\pgfqpoint{4.350351in}{2.358857in}}%
\pgfpathlineto{\pgfqpoint{4.358628in}{2.393051in}}%
\pgfpathlineto{\pgfqpoint{4.366904in}{2.389943in}}%
\pgfpathlineto{\pgfqpoint{4.375181in}{2.377509in}}%
\pgfpathlineto{\pgfqpoint{4.383458in}{2.374400in}}%
\pgfpathlineto{\pgfqpoint{4.391735in}{2.358857in}}%
\pgfpathlineto{\pgfqpoint{4.400009in}{2.389943in}}%
\pgfpathlineto{\pgfqpoint{4.408282in}{2.393051in}}%
\pgfpathlineto{\pgfqpoint{4.416556in}{2.358857in}}%
\pgfpathlineto{\pgfqpoint{4.424832in}{2.358857in}}%
\pgfpathlineto{\pgfqpoint{4.433110in}{2.374400in}}%
\pgfpathlineto{\pgfqpoint{4.441381in}{2.358857in}}%
\pgfpathlineto{\pgfqpoint{4.449658in}{2.358857in}}%
\pgfpathlineto{\pgfqpoint{4.457935in}{2.374400in}}%
\pgfpathlineto{\pgfqpoint{4.466211in}{2.358857in}}%
\pgfpathlineto{\pgfqpoint{4.474487in}{2.389943in}}%
\pgfpathlineto{\pgfqpoint{4.491037in}{2.324663in}}%
\pgfpathlineto{\pgfqpoint{4.499314in}{2.389943in}}%
\pgfpathlineto{\pgfqpoint{4.507591in}{2.389943in}}%
\pgfpathlineto{\pgfqpoint{4.515867in}{2.393051in}}%
\pgfpathlineto{\pgfqpoint{4.524143in}{2.374400in}}%
\pgfpathlineto{\pgfqpoint{4.540699in}{2.374400in}}%
\pgfpathlineto{\pgfqpoint{4.548976in}{2.389943in}}%
\pgfpathlineto{\pgfqpoint{4.557246in}{2.393051in}}%
\pgfpathlineto{\pgfqpoint{4.565518in}{2.389943in}}%
\pgfpathlineto{\pgfqpoint{4.573796in}{2.374400in}}%
\pgfpathlineto{\pgfqpoint{4.590350in}{2.374400in}}%
\pgfpathlineto{\pgfqpoint{4.598628in}{2.358857in}}%
\pgfpathlineto{\pgfqpoint{4.606904in}{2.374400in}}%
\pgfpathlineto{\pgfqpoint{4.615180in}{2.424137in}}%
\pgfpathlineto{\pgfqpoint{4.623455in}{2.389943in}}%
\pgfpathlineto{\pgfqpoint{4.631729in}{2.424137in}}%
\pgfpathlineto{\pgfqpoint{4.640003in}{2.374400in}}%
\pgfpathlineto{\pgfqpoint{4.664830in}{2.374400in}}%
\pgfpathlineto{\pgfqpoint{4.673103in}{2.389943in}}%
\pgfpathlineto{\pgfqpoint{4.681376in}{2.374400in}}%
\pgfpathlineto{\pgfqpoint{4.689650in}{2.408594in}}%
\pgfpathlineto{\pgfqpoint{4.697926in}{2.358857in}}%
\pgfpathlineto{\pgfqpoint{4.706202in}{2.408594in}}%
\pgfpathlineto{\pgfqpoint{4.714477in}{2.389943in}}%
\pgfpathlineto{\pgfqpoint{4.722752in}{2.405486in}}%
\pgfpathlineto{\pgfqpoint{4.731029in}{2.374400in}}%
\pgfpathlineto{\pgfqpoint{4.739306in}{2.389943in}}%
\pgfpathlineto{\pgfqpoint{4.747576in}{2.358857in}}%
\pgfpathlineto{\pgfqpoint{4.755853in}{2.374400in}}%
\pgfpathlineto{\pgfqpoint{4.764129in}{2.393051in}}%
\pgfpathlineto{\pgfqpoint{4.772401in}{2.374400in}}%
\pgfpathlineto{\pgfqpoint{4.780677in}{2.389943in}}%
\pgfpathlineto{\pgfqpoint{4.788954in}{2.389943in}}%
\pgfpathlineto{\pgfqpoint{4.797231in}{2.374400in}}%
\pgfpathlineto{\pgfqpoint{4.805504in}{2.374400in}}%
\pgfpathlineto{\pgfqpoint{4.813782in}{2.358857in}}%
\pgfpathlineto{\pgfqpoint{4.830332in}{2.358857in}}%
\pgfpathlineto{\pgfqpoint{4.838603in}{2.408594in}}%
\pgfpathlineto{\pgfqpoint{4.846872in}{2.374400in}}%
\pgfpathlineto{\pgfqpoint{4.855149in}{2.393051in}}%
\pgfpathlineto{\pgfqpoint{4.863424in}{2.408594in}}%
\pgfpathlineto{\pgfqpoint{4.871701in}{2.408594in}}%
\pgfpathlineto{\pgfqpoint{4.879973in}{2.374400in}}%
\pgfpathlineto{\pgfqpoint{4.888248in}{2.389943in}}%
\pgfpathlineto{\pgfqpoint{4.904800in}{2.358857in}}%
\pgfpathlineto{\pgfqpoint{4.913072in}{2.358857in}}%
\pgfpathlineto{\pgfqpoint{4.929622in}{2.424137in}}%
\pgfpathlineto{\pgfqpoint{4.937899in}{2.424137in}}%
\pgfpathlineto{\pgfqpoint{4.946177in}{2.358857in}}%
\pgfpathlineto{\pgfqpoint{4.954455in}{2.358857in}}%
\pgfpathlineto{\pgfqpoint{4.962730in}{2.374400in}}%
\pgfpathlineto{\pgfqpoint{4.971005in}{2.358857in}}%
\pgfpathlineto{\pgfqpoint{4.987554in}{2.358857in}}%
\pgfpathlineto{\pgfqpoint{4.995834in}{2.377509in}}%
\pgfpathlineto{\pgfqpoint{5.004111in}{2.374400in}}%
\pgfpathlineto{\pgfqpoint{5.012389in}{2.374400in}}%
\pgfpathlineto{\pgfqpoint{5.020665in}{2.358857in}}%
\pgfpathlineto{\pgfqpoint{5.028940in}{2.424137in}}%
\pgfpathlineto{\pgfqpoint{5.045495in}{2.393051in}}%
\pgfpathlineto{\pgfqpoint{5.053772in}{2.358857in}}%
\pgfpathlineto{\pgfqpoint{5.062049in}{2.389943in}}%
\pgfpathlineto{\pgfqpoint{5.070324in}{2.405486in}}%
\pgfpathlineto{\pgfqpoint{5.078602in}{2.405486in}}%
\pgfpathlineto{\pgfqpoint{5.103424in}{2.358857in}}%
\pgfpathlineto{\pgfqpoint{5.111701in}{2.389943in}}%
\pgfpathlineto{\pgfqpoint{5.128257in}{2.358857in}}%
\pgfpathlineto{\pgfqpoint{5.136534in}{2.393051in}}%
\pgfpathlineto{\pgfqpoint{5.144812in}{2.389943in}}%
\pgfpathlineto{\pgfqpoint{5.161366in}{2.389943in}}%
\pgfpathlineto{\pgfqpoint{5.169638in}{2.374400in}}%
\pgfpathlineto{\pgfqpoint{5.177909in}{2.374400in}}%
\pgfpathlineto{\pgfqpoint{5.186186in}{2.389943in}}%
\pgfpathlineto{\pgfqpoint{5.194461in}{2.393051in}}%
\pgfpathlineto{\pgfqpoint{5.202737in}{2.377509in}}%
\pgfpathlineto{\pgfqpoint{5.211014in}{2.374400in}}%
\pgfpathlineto{\pgfqpoint{5.219291in}{2.389943in}}%
\pgfpathlineto{\pgfqpoint{5.227567in}{2.389943in}}%
\pgfpathlineto{\pgfqpoint{5.235844in}{2.405486in}}%
\pgfpathlineto{\pgfqpoint{5.244121in}{2.405486in}}%
\pgfpathlineto{\pgfqpoint{5.252398in}{2.358857in}}%
\pgfpathlineto{\pgfqpoint{5.260675in}{2.374400in}}%
\pgfpathlineto{\pgfqpoint{5.268953in}{2.405486in}}%
\pgfpathlineto{\pgfqpoint{5.277227in}{2.374400in}}%
\pgfpathlineto{\pgfqpoint{5.285501in}{2.377509in}}%
\pgfpathlineto{\pgfqpoint{5.293777in}{2.374400in}}%
\pgfpathlineto{\pgfqpoint{5.302054in}{2.393051in}}%
\pgfpathlineto{\pgfqpoint{5.310331in}{2.374400in}}%
\pgfpathlineto{\pgfqpoint{5.318605in}{2.358857in}}%
\pgfpathlineto{\pgfqpoint{5.326881in}{2.389943in}}%
\pgfpathlineto{\pgfqpoint{5.335159in}{2.358857in}}%
\pgfpathlineto{\pgfqpoint{5.343436in}{2.389943in}}%
\pgfpathlineto{\pgfqpoint{5.351714in}{2.358857in}}%
\pgfpathlineto{\pgfqpoint{5.359990in}{2.389943in}}%
\pgfpathlineto{\pgfqpoint{5.368270in}{2.374400in}}%
\pgfpathlineto{\pgfqpoint{5.376546in}{2.377509in}}%
\pgfpathlineto{\pgfqpoint{5.384823in}{2.374400in}}%
\pgfpathlineto{\pgfqpoint{5.393101in}{2.377509in}}%
\pgfpathlineto{\pgfqpoint{5.401375in}{2.374400in}}%
\pgfpathlineto{\pgfqpoint{5.409651in}{2.358857in}}%
\pgfpathlineto{\pgfqpoint{5.417927in}{2.389943in}}%
\pgfpathlineto{\pgfqpoint{5.426200in}{2.374400in}}%
\pgfpathlineto{\pgfqpoint{5.434471in}{2.374400in}}%
\pgfpathlineto{\pgfqpoint{5.442748in}{2.389943in}}%
\pgfpathlineto{\pgfqpoint{5.459301in}{2.389943in}}%
\pgfpathlineto{\pgfqpoint{5.467578in}{2.358857in}}%
\pgfpathlineto{\pgfqpoint{5.475856in}{2.408594in}}%
\pgfpathlineto{\pgfqpoint{5.484132in}{2.374400in}}%
\pgfpathlineto{\pgfqpoint{5.492405in}{2.358857in}}%
\pgfpathlineto{\pgfqpoint{5.500682in}{2.389943in}}%
\pgfpathlineto{\pgfqpoint{5.508958in}{2.389943in}}%
\pgfpathlineto{\pgfqpoint{5.517234in}{1.920549in}}%
\pgfpathlineto{\pgfqpoint{5.525505in}{1.839726in}}%
\pgfpathlineto{\pgfqpoint{5.533784in}{1.786880in}}%
\pgfpathlineto{\pgfqpoint{5.542063in}{1.755794in}}%
\pgfpathlineto{\pgfqpoint{5.550341in}{1.740251in}}%
\pgfpathlineto{\pgfqpoint{5.558614in}{1.740251in}}%
\pgfpathlineto{\pgfqpoint{5.566887in}{1.706057in}}%
\pgfpathlineto{\pgfqpoint{5.575158in}{1.702949in}}%
\pgfpathlineto{\pgfqpoint{5.583433in}{1.752686in}}%
\pgfpathlineto{\pgfqpoint{5.591708in}{1.687406in}}%
\pgfpathlineto{\pgfqpoint{5.599987in}{1.740251in}}%
\pgfpathlineto{\pgfqpoint{5.608265in}{1.687406in}}%
\pgfpathlineto{\pgfqpoint{5.616544in}{1.721600in}}%
\pgfpathlineto{\pgfqpoint{5.624822in}{1.671863in}}%
\pgfpathlineto{\pgfqpoint{5.633092in}{1.702949in}}%
\pgfpathlineto{\pgfqpoint{5.649645in}{1.671863in}}%
\pgfpathlineto{\pgfqpoint{5.657923in}{1.687406in}}%
\pgfpathlineto{\pgfqpoint{5.666196in}{1.690514in}}%
\pgfpathlineto{\pgfqpoint{5.674471in}{1.687406in}}%
\pgfpathlineto{\pgfqpoint{5.691029in}{1.687406in}}%
\pgfpathlineto{\pgfqpoint{5.699308in}{1.706057in}}%
\pgfpathlineto{\pgfqpoint{5.707579in}{1.687406in}}%
\pgfpathlineto{\pgfqpoint{5.715858in}{1.671863in}}%
\pgfpathlineto{\pgfqpoint{5.724134in}{1.671863in}}%
\pgfpathlineto{\pgfqpoint{5.732409in}{1.721600in}}%
\pgfpathlineto{\pgfqpoint{5.740687in}{1.656320in}}%
\pgfpathlineto{\pgfqpoint{5.748966in}{1.687406in}}%
\pgfpathlineto{\pgfqpoint{5.757245in}{1.656320in}}%
\pgfpathlineto{\pgfqpoint{5.757245in}{1.656320in}}%
\pgfusepath{stroke}%
\end{pgfscope}%
\begin{pgfscope}%
\pgfpathrectangle{\pgfqpoint{0.800000in}{0.960000in}}{\pgfqpoint{4.960000in}{3.264000in}}%
\pgfusepath{clip}%
\pgfsetrectcap%
\pgfsetroundjoin%
\pgfsetlinewidth{1.505625pt}%
\definecolor{currentstroke}{rgb}{1.000000,0.498039,0.054902}%
\pgfsetstrokecolor{currentstroke}%
\pgfsetdash{}{0pt}%
\pgfpathmoveto{\pgfqpoint{0.790000in}{1.009118in}}%
\pgfpathlineto{\pgfqpoint{0.801238in}{1.006629in}}%
\pgfpathlineto{\pgfqpoint{0.810011in}{1.006629in}}%
\pgfpathlineto{\pgfqpoint{0.836329in}{1.012457in}}%
\pgfpathlineto{\pgfqpoint{0.871419in}{1.012457in}}%
\pgfpathlineto{\pgfqpoint{0.880191in}{1.010514in}}%
\pgfpathlineto{\pgfqpoint{0.888964in}{1.012457in}}%
\pgfpathlineto{\pgfqpoint{0.897736in}{1.012457in}}%
\pgfpathlineto{\pgfqpoint{0.906509in}{1.010514in}}%
\pgfpathlineto{\pgfqpoint{0.915281in}{1.012457in}}%
\pgfpathlineto{\pgfqpoint{0.924054in}{1.010514in}}%
\pgfpathlineto{\pgfqpoint{0.967917in}{1.010514in}}%
\pgfpathlineto{\pgfqpoint{0.976689in}{1.014400in}}%
\pgfpathlineto{\pgfqpoint{0.985462in}{1.014400in}}%
\pgfpathlineto{\pgfqpoint{0.994234in}{1.012457in}}%
\pgfpathlineto{\pgfqpoint{1.003007in}{1.016343in}}%
\pgfpathlineto{\pgfqpoint{1.011780in}{1.012457in}}%
\pgfpathlineto{\pgfqpoint{1.020552in}{1.014400in}}%
\pgfpathlineto{\pgfqpoint{1.029325in}{1.012457in}}%
\pgfpathlineto{\pgfqpoint{1.038097in}{1.014400in}}%
\pgfpathlineto{\pgfqpoint{1.047112in}{1.014400in}}%
\pgfpathlineto{\pgfqpoint{1.055885in}{1.016343in}}%
\pgfpathlineto{\pgfqpoint{1.082202in}{1.016343in}}%
\pgfpathlineto{\pgfqpoint{1.090975in}{1.018286in}}%
\pgfpathlineto{\pgfqpoint{1.117292in}{1.018286in}}%
\pgfpathlineto{\pgfqpoint{1.126087in}{1.016343in}}%
\pgfpathlineto{\pgfqpoint{1.143632in}{1.020229in}}%
\pgfpathlineto{\pgfqpoint{1.152399in}{1.018286in}}%
\pgfpathlineto{\pgfqpoint{1.161155in}{1.018286in}}%
\pgfpathlineto{\pgfqpoint{1.169928in}{1.020229in}}%
\pgfpathlineto{\pgfqpoint{1.214033in}{1.020229in}}%
\pgfpathlineto{\pgfqpoint{1.222806in}{1.016343in}}%
\pgfpathlineto{\pgfqpoint{1.231578in}{1.016343in}}%
\pgfpathlineto{\pgfqpoint{1.240350in}{1.014400in}}%
\pgfpathlineto{\pgfqpoint{1.249123in}{1.014400in}}%
\pgfpathlineto{\pgfqpoint{1.257896in}{1.020229in}}%
\pgfpathlineto{\pgfqpoint{1.266668in}{1.020229in}}%
\pgfpathlineto{\pgfqpoint{1.275441in}{1.018286in}}%
\pgfpathlineto{\pgfqpoint{1.310773in}{1.018286in}}%
\pgfpathlineto{\pgfqpoint{1.319546in}{1.020229in}}%
\pgfpathlineto{\pgfqpoint{1.328346in}{1.016343in}}%
\pgfpathlineto{\pgfqpoint{1.337113in}{1.018286in}}%
\pgfpathlineto{\pgfqpoint{1.345882in}{1.018286in}}%
\pgfpathlineto{\pgfqpoint{1.354658in}{1.016343in}}%
\pgfpathlineto{\pgfqpoint{1.363431in}{1.016343in}}%
\pgfpathlineto{\pgfqpoint{1.372203in}{1.014400in}}%
\pgfpathlineto{\pgfqpoint{1.380976in}{1.018286in}}%
\pgfpathlineto{\pgfqpoint{1.389748in}{1.016343in}}%
\pgfpathlineto{\pgfqpoint{1.398520in}{1.018286in}}%
\pgfpathlineto{\pgfqpoint{1.407293in}{1.016343in}}%
\pgfpathlineto{\pgfqpoint{1.416066in}{1.018286in}}%
\pgfpathlineto{\pgfqpoint{1.424839in}{1.026057in}}%
\pgfpathlineto{\pgfqpoint{1.442383in}{1.026057in}}%
\pgfpathlineto{\pgfqpoint{1.451156in}{1.029943in}}%
\pgfpathlineto{\pgfqpoint{1.468695in}{1.029943in}}%
\pgfpathlineto{\pgfqpoint{1.477452in}{1.031886in}}%
\pgfpathlineto{\pgfqpoint{1.521315in}{1.031886in}}%
\pgfpathlineto{\pgfqpoint{1.530087in}{1.033829in}}%
\pgfpathlineto{\pgfqpoint{1.538860in}{1.031886in}}%
\pgfpathlineto{\pgfqpoint{1.573944in}{1.031886in}}%
\pgfpathlineto{\pgfqpoint{1.582700in}{1.033829in}}%
\pgfpathlineto{\pgfqpoint{1.591473in}{1.029943in}}%
\pgfpathlineto{\pgfqpoint{1.600246in}{1.031886in}}%
\pgfpathlineto{\pgfqpoint{1.617791in}{1.028000in}}%
\pgfpathlineto{\pgfqpoint{1.635336in}{1.028000in}}%
\pgfpathlineto{\pgfqpoint{1.644108in}{1.026057in}}%
\pgfpathlineto{\pgfqpoint{1.652881in}{1.026057in}}%
\pgfpathlineto{\pgfqpoint{1.661653in}{1.024114in}}%
\pgfpathlineto{\pgfqpoint{1.670426in}{1.026057in}}%
\pgfpathlineto{\pgfqpoint{1.679199in}{1.024114in}}%
\pgfpathlineto{\pgfqpoint{1.687971in}{1.024114in}}%
\pgfpathlineto{\pgfqpoint{1.696744in}{1.026057in}}%
\pgfpathlineto{\pgfqpoint{1.714289in}{1.026057in}}%
\pgfpathlineto{\pgfqpoint{1.740606in}{1.031886in}}%
\pgfpathlineto{\pgfqpoint{1.749379in}{1.031886in}}%
\pgfpathlineto{\pgfqpoint{1.758157in}{1.033829in}}%
\pgfpathlineto{\pgfqpoint{1.766924in}{1.031886in}}%
\pgfpathlineto{\pgfqpoint{1.775697in}{1.033829in}}%
\pgfpathlineto{\pgfqpoint{1.784469in}{1.031886in}}%
\pgfpathlineto{\pgfqpoint{1.793242in}{1.031886in}}%
\pgfpathlineto{\pgfqpoint{1.802014in}{1.033829in}}%
\pgfpathlineto{\pgfqpoint{1.819559in}{1.033829in}}%
\pgfpathlineto{\pgfqpoint{1.828332in}{1.035771in}}%
\pgfpathlineto{\pgfqpoint{1.837105in}{1.035771in}}%
\pgfpathlineto{\pgfqpoint{1.845877in}{1.033829in}}%
\pgfpathlineto{\pgfqpoint{1.854650in}{1.033829in}}%
\pgfpathlineto{\pgfqpoint{1.863422in}{1.035771in}}%
\pgfpathlineto{\pgfqpoint{1.880967in}{1.035771in}}%
\pgfpathlineto{\pgfqpoint{1.889740in}{1.033829in}}%
\pgfpathlineto{\pgfqpoint{1.898512in}{1.033829in}}%
\pgfpathlineto{\pgfqpoint{1.907285in}{1.035771in}}%
\pgfpathlineto{\pgfqpoint{1.942375in}{1.035771in}}%
\pgfpathlineto{\pgfqpoint{1.959920in}{1.031886in}}%
\pgfpathlineto{\pgfqpoint{1.968693in}{1.033829in}}%
\pgfpathlineto{\pgfqpoint{1.986238in}{1.033829in}}%
\pgfpathlineto{\pgfqpoint{2.012577in}{1.028000in}}%
\pgfpathlineto{\pgfqpoint{2.021350in}{1.029943in}}%
\pgfpathlineto{\pgfqpoint{2.047712in}{1.029943in}}%
\pgfpathlineto{\pgfqpoint{2.056484in}{1.031886in}}%
\pgfpathlineto{\pgfqpoint{2.065257in}{1.031886in}}%
\pgfpathlineto{\pgfqpoint{2.074028in}{1.033829in}}%
\pgfpathlineto{\pgfqpoint{2.091574in}{1.033829in}}%
\pgfpathlineto{\pgfqpoint{2.109120in}{1.037714in}}%
\pgfpathlineto{\pgfqpoint{2.117892in}{1.037714in}}%
\pgfpathlineto{\pgfqpoint{2.126665in}{1.039657in}}%
\pgfpathlineto{\pgfqpoint{2.144210in}{1.035771in}}%
\pgfpathlineto{\pgfqpoint{2.152982in}{1.039657in}}%
\pgfpathlineto{\pgfqpoint{2.161755in}{1.035771in}}%
\pgfpathlineto{\pgfqpoint{2.170527in}{1.037714in}}%
\pgfpathlineto{\pgfqpoint{2.179300in}{1.035771in}}%
\pgfpathlineto{\pgfqpoint{2.188072in}{1.037714in}}%
\pgfpathlineto{\pgfqpoint{2.231935in}{1.037714in}}%
\pgfpathlineto{\pgfqpoint{2.240708in}{1.035771in}}%
\pgfpathlineto{\pgfqpoint{2.249480in}{1.035771in}}%
\pgfpathlineto{\pgfqpoint{2.258253in}{1.039657in}}%
\pgfpathlineto{\pgfqpoint{2.319661in}{1.039657in}}%
\pgfpathlineto{\pgfqpoint{2.328433in}{1.041600in}}%
\pgfpathlineto{\pgfqpoint{2.337202in}{1.041600in}}%
\pgfpathlineto{\pgfqpoint{2.345978in}{1.043543in}}%
\pgfpathlineto{\pgfqpoint{2.363523in}{1.043543in}}%
\pgfpathlineto{\pgfqpoint{2.372296in}{1.045486in}}%
\pgfpathlineto{\pgfqpoint{2.381068in}{1.045486in}}%
\pgfpathlineto{\pgfqpoint{2.398614in}{1.049371in}}%
\pgfpathlineto{\pgfqpoint{2.407386in}{1.049371in}}%
\pgfpathlineto{\pgfqpoint{2.416159in}{1.047429in}}%
\pgfpathlineto{\pgfqpoint{2.424931in}{1.049371in}}%
\pgfpathlineto{\pgfqpoint{2.433703in}{1.049371in}}%
\pgfpathlineto{\pgfqpoint{2.442476in}{1.047429in}}%
\pgfpathlineto{\pgfqpoint{2.451249in}{1.051314in}}%
\pgfpathlineto{\pgfqpoint{2.460015in}{1.049371in}}%
\pgfpathlineto{\pgfqpoint{2.468772in}{1.051314in}}%
\pgfpathlineto{\pgfqpoint{2.477537in}{1.049371in}}%
\pgfpathlineto{\pgfqpoint{2.486317in}{1.051314in}}%
\pgfpathlineto{\pgfqpoint{2.495086in}{1.055200in}}%
\pgfpathlineto{\pgfqpoint{2.512613in}{1.055200in}}%
\pgfpathlineto{\pgfqpoint{2.521385in}{1.057143in}}%
\pgfpathlineto{\pgfqpoint{2.530158in}{1.057143in}}%
\pgfpathlineto{\pgfqpoint{2.538930in}{1.061029in}}%
\pgfpathlineto{\pgfqpoint{2.565248in}{1.066857in}}%
\pgfpathlineto{\pgfqpoint{2.617883in}{1.066857in}}%
\pgfpathlineto{\pgfqpoint{2.626656in}{1.070743in}}%
\pgfpathlineto{\pgfqpoint{2.635428in}{1.070743in}}%
\pgfpathlineto{\pgfqpoint{2.644200in}{1.068800in}}%
\pgfpathlineto{\pgfqpoint{2.652973in}{1.068800in}}%
\pgfpathlineto{\pgfqpoint{2.661746in}{1.070743in}}%
\pgfpathlineto{\pgfqpoint{2.679269in}{1.070743in}}%
\pgfpathlineto{\pgfqpoint{2.688036in}{1.072686in}}%
\pgfpathlineto{\pgfqpoint{2.696792in}{1.070743in}}%
\pgfpathlineto{\pgfqpoint{2.714337in}{1.070743in}}%
\pgfpathlineto{\pgfqpoint{2.723110in}{1.072686in}}%
\pgfpathlineto{\pgfqpoint{2.740655in}{1.072686in}}%
\pgfpathlineto{\pgfqpoint{2.749427in}{1.074629in}}%
\pgfpathlineto{\pgfqpoint{2.758197in}{1.074629in}}%
\pgfpathlineto{\pgfqpoint{2.784517in}{1.080457in}}%
\pgfpathlineto{\pgfqpoint{2.793290in}{1.084343in}}%
\pgfpathlineto{\pgfqpoint{2.802063in}{1.082400in}}%
\pgfpathlineto{\pgfqpoint{2.810835in}{1.086286in}}%
\pgfpathlineto{\pgfqpoint{2.819608in}{1.084343in}}%
\pgfpathlineto{\pgfqpoint{2.828380in}{1.084343in}}%
\pgfpathlineto{\pgfqpoint{2.837153in}{1.086286in}}%
\pgfpathlineto{\pgfqpoint{2.845925in}{1.084343in}}%
\pgfpathlineto{\pgfqpoint{2.854697in}{1.084343in}}%
\pgfpathlineto{\pgfqpoint{2.872243in}{1.080457in}}%
\pgfpathlineto{\pgfqpoint{2.881009in}{1.082400in}}%
\pgfpathlineto{\pgfqpoint{2.889766in}{1.080457in}}%
\pgfpathlineto{\pgfqpoint{2.898538in}{1.080457in}}%
\pgfpathlineto{\pgfqpoint{2.907327in}{1.082400in}}%
\pgfpathlineto{\pgfqpoint{2.916084in}{1.080457in}}%
\pgfpathlineto{\pgfqpoint{2.924856in}{1.082400in}}%
\pgfpathlineto{\pgfqpoint{2.942401in}{1.082400in}}%
\pgfpathlineto{\pgfqpoint{2.951174in}{1.080457in}}%
\pgfpathlineto{\pgfqpoint{3.021376in}{1.080457in}}%
\pgfpathlineto{\pgfqpoint{3.030149in}{1.076571in}}%
\pgfpathlineto{\pgfqpoint{3.038921in}{1.078514in}}%
\pgfpathlineto{\pgfqpoint{3.056466in}{1.078514in}}%
\pgfpathlineto{\pgfqpoint{3.065240in}{1.080457in}}%
\pgfpathlineto{\pgfqpoint{3.074011in}{1.078514in}}%
\pgfpathlineto{\pgfqpoint{3.082784in}{1.080457in}}%
\pgfpathlineto{\pgfqpoint{3.161715in}{1.080457in}}%
\pgfpathlineto{\pgfqpoint{3.170488in}{1.082400in}}%
\pgfpathlineto{\pgfqpoint{3.179260in}{1.082400in}}%
\pgfpathlineto{\pgfqpoint{3.188033in}{1.084343in}}%
\pgfpathlineto{\pgfqpoint{3.196805in}{1.082400in}}%
\pgfpathlineto{\pgfqpoint{3.205820in}{1.082400in}}%
\pgfpathlineto{\pgfqpoint{3.214593in}{1.084343in}}%
\pgfpathlineto{\pgfqpoint{3.258456in}{1.084343in}}%
\pgfpathlineto{\pgfqpoint{3.267228in}{1.086286in}}%
\pgfpathlineto{\pgfqpoint{3.276001in}{1.084343in}}%
\pgfpathlineto{\pgfqpoint{3.284773in}{1.084343in}}%
\pgfpathlineto{\pgfqpoint{3.293546in}{1.082400in}}%
\pgfpathlineto{\pgfqpoint{3.311091in}{1.082400in}}%
\pgfpathlineto{\pgfqpoint{3.319863in}{1.080457in}}%
\pgfpathlineto{\pgfqpoint{3.372477in}{1.080457in}}%
\pgfpathlineto{\pgfqpoint{3.381249in}{1.082400in}}%
\pgfpathlineto{\pgfqpoint{3.390022in}{1.082400in}}%
\pgfpathlineto{\pgfqpoint{3.398794in}{1.084343in}}%
\pgfpathlineto{\pgfqpoint{3.442656in}{1.084343in}}%
\pgfpathlineto{\pgfqpoint{3.451429in}{1.082400in}}%
\pgfpathlineto{\pgfqpoint{3.530360in}{1.082400in}}%
\pgfpathlineto{\pgfqpoint{3.539138in}{1.084343in}}%
\pgfpathlineto{\pgfqpoint{3.565469in}{1.084343in}}%
\pgfpathlineto{\pgfqpoint{3.574245in}{1.086286in}}%
\pgfpathlineto{\pgfqpoint{3.591792in}{1.086286in}}%
\pgfpathlineto{\pgfqpoint{3.600563in}{1.082400in}}%
\pgfpathlineto{\pgfqpoint{3.609335in}{1.084343in}}%
\pgfpathlineto{\pgfqpoint{3.688266in}{1.084343in}}%
\pgfpathlineto{\pgfqpoint{3.697039in}{1.082400in}}%
\pgfpathlineto{\pgfqpoint{3.705811in}{1.082400in}}%
\pgfpathlineto{\pgfqpoint{3.714584in}{1.084343in}}%
\pgfpathlineto{\pgfqpoint{3.732129in}{1.084343in}}%
\pgfpathlineto{\pgfqpoint{3.749674in}{1.088229in}}%
\pgfpathlineto{\pgfqpoint{3.758447in}{1.088229in}}%
\pgfpathlineto{\pgfqpoint{3.767219in}{1.086286in}}%
\pgfpathlineto{\pgfqpoint{3.775992in}{1.086286in}}%
\pgfpathlineto{\pgfqpoint{3.784764in}{1.088229in}}%
\pgfpathlineto{\pgfqpoint{3.793531in}{1.086286in}}%
\pgfpathlineto{\pgfqpoint{3.802287in}{1.088229in}}%
\pgfpathlineto{\pgfqpoint{3.837374in}{1.088229in}}%
\pgfpathlineto{\pgfqpoint{3.846150in}{1.086286in}}%
\pgfpathlineto{\pgfqpoint{3.854923in}{1.086286in}}%
\pgfpathlineto{\pgfqpoint{3.863695in}{1.084343in}}%
\pgfpathlineto{\pgfqpoint{3.881240in}{1.084343in}}%
\pgfpathlineto{\pgfqpoint{3.890013in}{1.082400in}}%
\pgfpathlineto{\pgfqpoint{3.898810in}{1.086286in}}%
\pgfpathlineto{\pgfqpoint{3.916353in}{1.082400in}}%
\pgfpathlineto{\pgfqpoint{3.925125in}{1.088229in}}%
\pgfpathlineto{\pgfqpoint{3.933893in}{1.086286in}}%
\pgfpathlineto{\pgfqpoint{3.977738in}{1.086286in}}%
\pgfpathlineto{\pgfqpoint{3.986511in}{1.084343in}}%
\pgfpathlineto{\pgfqpoint{4.004056in}{1.084343in}}%
\pgfpathlineto{\pgfqpoint{4.012828in}{1.086286in}}%
\pgfpathlineto{\pgfqpoint{4.030616in}{1.086286in}}%
\pgfpathlineto{\pgfqpoint{4.039389in}{1.080457in}}%
\pgfpathlineto{\pgfqpoint{4.048158in}{1.082400in}}%
\pgfpathlineto{\pgfqpoint{4.092024in}{1.082400in}}%
\pgfpathlineto{\pgfqpoint{4.109569in}{1.086286in}}%
\pgfpathlineto{\pgfqpoint{4.118342in}{1.086286in}}%
\pgfpathlineto{\pgfqpoint{4.127114in}{1.088229in}}%
\pgfpathlineto{\pgfqpoint{4.135887in}{1.086286in}}%
\pgfpathlineto{\pgfqpoint{4.144659in}{1.088229in}}%
\pgfpathlineto{\pgfqpoint{4.153432in}{1.088229in}}%
\pgfpathlineto{\pgfqpoint{4.162204in}{1.090171in}}%
\pgfpathlineto{\pgfqpoint{4.197294in}{1.090171in}}%
\pgfpathlineto{\pgfqpoint{4.206082in}{1.088229in}}%
\pgfpathlineto{\pgfqpoint{4.241157in}{1.088229in}}%
\pgfpathlineto{\pgfqpoint{4.249930in}{1.086286in}}%
\pgfpathlineto{\pgfqpoint{4.258703in}{1.088229in}}%
\pgfpathlineto{\pgfqpoint{4.267475in}{1.084343in}}%
\pgfpathlineto{\pgfqpoint{4.293793in}{1.084343in}}%
\pgfpathlineto{\pgfqpoint{4.302565in}{1.086286in}}%
\pgfpathlineto{\pgfqpoint{4.311338in}{1.084343in}}%
\pgfpathlineto{\pgfqpoint{4.320110in}{1.084343in}}%
\pgfpathlineto{\pgfqpoint{4.328883in}{1.086286in}}%
\pgfpathlineto{\pgfqpoint{4.337655in}{1.084343in}}%
\pgfpathlineto{\pgfqpoint{4.346428in}{1.084343in}}%
\pgfpathlineto{\pgfqpoint{4.355201in}{1.082400in}}%
\pgfpathlineto{\pgfqpoint{4.390291in}{1.082400in}}%
\pgfpathlineto{\pgfqpoint{4.399063in}{1.084343in}}%
\pgfpathlineto{\pgfqpoint{4.407828in}{1.082400in}}%
\pgfpathlineto{\pgfqpoint{4.416586in}{1.084343in}}%
\pgfpathlineto{\pgfqpoint{4.425359in}{1.084343in}}%
\pgfpathlineto{\pgfqpoint{4.434131in}{1.086286in}}%
\pgfpathlineto{\pgfqpoint{4.442904in}{1.084343in}}%
\pgfpathlineto{\pgfqpoint{4.451677in}{1.086286in}}%
\pgfpathlineto{\pgfqpoint{4.477994in}{1.086286in}}%
\pgfpathlineto{\pgfqpoint{4.486767in}{1.084343in}}%
\pgfpathlineto{\pgfqpoint{4.495539in}{1.086286in}}%
\pgfpathlineto{\pgfqpoint{4.513084in}{1.086286in}}%
\pgfpathlineto{\pgfqpoint{4.530607in}{1.090171in}}%
\pgfpathlineto{\pgfqpoint{4.539380in}{1.086286in}}%
\pgfpathlineto{\pgfqpoint{4.574470in}{1.086286in}}%
\pgfpathlineto{\pgfqpoint{4.583243in}{1.084343in}}%
\pgfpathlineto{\pgfqpoint{4.592015in}{1.086286in}}%
\pgfpathlineto{\pgfqpoint{4.600788in}{1.086286in}}%
\pgfpathlineto{\pgfqpoint{4.609560in}{1.088229in}}%
\pgfpathlineto{\pgfqpoint{4.618332in}{1.092114in}}%
\pgfpathlineto{\pgfqpoint{4.662196in}{1.092114in}}%
\pgfpathlineto{\pgfqpoint{4.670968in}{1.090171in}}%
\pgfpathlineto{\pgfqpoint{4.679740in}{1.092114in}}%
\pgfpathlineto{\pgfqpoint{4.714831in}{1.092114in}}%
\pgfpathlineto{\pgfqpoint{4.723603in}{1.090171in}}%
\pgfpathlineto{\pgfqpoint{4.749921in}{1.090171in}}%
\pgfpathlineto{\pgfqpoint{4.758694in}{1.088229in}}%
\pgfpathlineto{\pgfqpoint{4.776239in}{1.088229in}}%
\pgfpathlineto{\pgfqpoint{4.785011in}{1.090171in}}%
\pgfpathlineto{\pgfqpoint{4.793784in}{1.088229in}}%
\pgfpathlineto{\pgfqpoint{4.802556in}{1.088229in}}%
\pgfpathlineto{\pgfqpoint{4.811329in}{1.090171in}}%
\pgfpathlineto{\pgfqpoint{4.820102in}{1.090171in}}%
\pgfpathlineto{\pgfqpoint{4.837647in}{1.094057in}}%
\pgfpathlineto{\pgfqpoint{4.855192in}{1.094057in}}%
\pgfpathlineto{\pgfqpoint{4.863964in}{1.096000in}}%
\pgfpathlineto{\pgfqpoint{4.872737in}{1.096000in}}%
\pgfpathlineto{\pgfqpoint{4.881509in}{1.094057in}}%
\pgfpathlineto{\pgfqpoint{4.899082in}{1.094057in}}%
\pgfpathlineto{\pgfqpoint{4.934166in}{1.086286in}}%
\pgfpathlineto{\pgfqpoint{4.942939in}{1.086286in}}%
\pgfpathlineto{\pgfqpoint{4.951712in}{1.088229in}}%
\pgfpathlineto{\pgfqpoint{4.960484in}{1.084343in}}%
\pgfpathlineto{\pgfqpoint{4.969257in}{1.086286in}}%
\pgfpathlineto{\pgfqpoint{4.995574in}{1.086286in}}%
\pgfpathlineto{\pgfqpoint{5.004347in}{1.088229in}}%
\pgfpathlineto{\pgfqpoint{5.013119in}{1.088229in}}%
\pgfpathlineto{\pgfqpoint{5.021892in}{1.094057in}}%
\pgfpathlineto{\pgfqpoint{5.039437in}{1.094057in}}%
\pgfpathlineto{\pgfqpoint{5.048210in}{1.092114in}}%
\pgfpathlineto{\pgfqpoint{5.056982in}{1.092114in}}%
\pgfpathlineto{\pgfqpoint{5.065755in}{1.094057in}}%
\pgfpathlineto{\pgfqpoint{5.100845in}{1.094057in}}%
\pgfpathlineto{\pgfqpoint{5.109617in}{1.092114in}}%
\pgfpathlineto{\pgfqpoint{5.118390in}{1.094057in}}%
\pgfpathlineto{\pgfqpoint{5.127163in}{1.092114in}}%
\pgfpathlineto{\pgfqpoint{5.135935in}{1.092114in}}%
\pgfpathlineto{\pgfqpoint{5.144708in}{1.090171in}}%
\pgfpathlineto{\pgfqpoint{5.153480in}{1.090171in}}%
\pgfpathlineto{\pgfqpoint{5.162253in}{1.092114in}}%
\pgfpathlineto{\pgfqpoint{5.179798in}{1.092114in}}%
\pgfpathlineto{\pgfqpoint{5.188570in}{1.090171in}}%
\pgfpathlineto{\pgfqpoint{5.197343in}{1.092114in}}%
\pgfpathlineto{\pgfqpoint{5.206118in}{1.090171in}}%
\pgfpathlineto{\pgfqpoint{5.232433in}{1.090171in}}%
\pgfpathlineto{\pgfqpoint{5.241206in}{1.088229in}}%
\pgfpathlineto{\pgfqpoint{5.249978in}{1.088229in}}%
\pgfpathlineto{\pgfqpoint{5.258751in}{1.086286in}}%
\pgfpathlineto{\pgfqpoint{5.267523in}{1.086286in}}%
\pgfpathlineto{\pgfqpoint{5.276297in}{1.084343in}}%
\pgfpathlineto{\pgfqpoint{5.285068in}{1.088229in}}%
\pgfpathlineto{\pgfqpoint{5.346476in}{1.088229in}}%
\pgfpathlineto{\pgfqpoint{5.355249in}{1.090171in}}%
\pgfpathlineto{\pgfqpoint{5.364021in}{1.088229in}}%
\pgfpathlineto{\pgfqpoint{5.372794in}{1.090171in}}%
\pgfpathlineto{\pgfqpoint{5.381566in}{1.088229in}}%
\pgfpathlineto{\pgfqpoint{5.390339in}{1.088229in}}%
\pgfpathlineto{\pgfqpoint{5.399111in}{1.090171in}}%
\pgfpathlineto{\pgfqpoint{5.425451in}{1.090171in}}%
\pgfpathlineto{\pgfqpoint{5.434219in}{1.088229in}}%
\pgfpathlineto{\pgfqpoint{5.442974in}{1.090171in}}%
\pgfpathlineto{\pgfqpoint{5.451747in}{1.090171in}}%
\pgfpathlineto{\pgfqpoint{5.460519in}{1.088229in}}%
\pgfpathlineto{\pgfqpoint{5.486837in}{1.088229in}}%
\pgfpathlineto{\pgfqpoint{5.495609in}{1.086286in}}%
\pgfpathlineto{\pgfqpoint{5.513155in}{1.086286in}}%
\pgfpathlineto{\pgfqpoint{5.521927in}{1.088229in}}%
\pgfpathlineto{\pgfqpoint{5.530700in}{1.084343in}}%
\pgfpathlineto{\pgfqpoint{5.539472in}{1.084343in}}%
\pgfpathlineto{\pgfqpoint{5.548245in}{1.078514in}}%
\pgfpathlineto{\pgfqpoint{5.583291in}{1.070743in}}%
\pgfpathlineto{\pgfqpoint{5.600836in}{1.070743in}}%
\pgfpathlineto{\pgfqpoint{5.635926in}{1.062971in}}%
\pgfpathlineto{\pgfqpoint{5.653471in}{1.062971in}}%
\pgfpathlineto{\pgfqpoint{5.662244in}{1.059086in}}%
\pgfpathlineto{\pgfqpoint{5.679789in}{1.059086in}}%
\pgfpathlineto{\pgfqpoint{5.714879in}{1.051314in}}%
\pgfpathlineto{\pgfqpoint{5.723652in}{1.051314in}}%
\pgfpathlineto{\pgfqpoint{5.732424in}{1.053257in}}%
\pgfpathlineto{\pgfqpoint{5.741197in}{1.053257in}}%
\pgfpathlineto{\pgfqpoint{5.749969in}{1.051314in}}%
\pgfpathlineto{\pgfqpoint{5.770000in}{1.051314in}}%
\pgfpathlineto{\pgfqpoint{5.770000in}{1.051314in}}%
\pgfusepath{stroke}%
\end{pgfscope}%
\begin{pgfscope}%
\pgfpathrectangle{\pgfqpoint{0.800000in}{0.960000in}}{\pgfqpoint{4.960000in}{3.264000in}}%
\pgfusepath{clip}%
\pgfsetrectcap%
\pgfsetroundjoin%
\pgfsetlinewidth{1.505625pt}%
\definecolor{currentstroke}{rgb}{0.172549,0.627451,0.172549}%
\pgfsetstrokecolor{currentstroke}%
\pgfsetdash{}{0pt}%
\pgfpathmoveto{\pgfqpoint{0.790000in}{1.004297in}}%
\pgfpathlineto{\pgfqpoint{0.827571in}{1.004297in}}%
\pgfpathlineto{\pgfqpoint{0.836329in}{1.006240in}}%
\pgfpathlineto{\pgfqpoint{0.862646in}{1.006240in}}%
\pgfpathlineto{\pgfqpoint{0.871419in}{1.002354in}}%
\pgfpathlineto{\pgfqpoint{0.915281in}{1.002354in}}%
\pgfpathlineto{\pgfqpoint{0.924054in}{1.000411in}}%
\pgfpathlineto{\pgfqpoint{0.932827in}{1.002354in}}%
\pgfpathlineto{\pgfqpoint{0.950372in}{0.998469in}}%
\pgfpathlineto{\pgfqpoint{0.959144in}{1.000411in}}%
\pgfpathlineto{\pgfqpoint{1.029325in}{1.000411in}}%
\pgfpathlineto{\pgfqpoint{1.038097in}{1.002354in}}%
\pgfpathlineto{\pgfqpoint{1.055885in}{0.998469in}}%
\pgfpathlineto{\pgfqpoint{1.073430in}{1.002354in}}%
\pgfpathlineto{\pgfqpoint{1.082202in}{1.000411in}}%
\pgfpathlineto{\pgfqpoint{1.108520in}{1.000411in}}%
\pgfpathlineto{\pgfqpoint{1.126087in}{1.004297in}}%
\pgfpathlineto{\pgfqpoint{1.134860in}{1.004297in}}%
\pgfpathlineto{\pgfqpoint{1.143632in}{1.002354in}}%
\pgfpathlineto{\pgfqpoint{1.152399in}{1.002354in}}%
\pgfpathlineto{\pgfqpoint{1.161155in}{1.004297in}}%
\pgfpathlineto{\pgfqpoint{1.187473in}{1.004297in}}%
\pgfpathlineto{\pgfqpoint{1.196245in}{1.006240in}}%
\pgfpathlineto{\pgfqpoint{1.214033in}{1.006240in}}%
\pgfpathlineto{\pgfqpoint{1.249123in}{1.014011in}}%
\pgfpathlineto{\pgfqpoint{1.257896in}{1.014011in}}%
\pgfpathlineto{\pgfqpoint{1.266668in}{1.015954in}}%
\pgfpathlineto{\pgfqpoint{1.284213in}{1.015954in}}%
\pgfpathlineto{\pgfqpoint{1.292983in}{1.017897in}}%
\pgfpathlineto{\pgfqpoint{1.302001in}{1.014011in}}%
\pgfpathlineto{\pgfqpoint{1.310773in}{1.014011in}}%
\pgfpathlineto{\pgfqpoint{1.319546in}{1.015954in}}%
\pgfpathlineto{\pgfqpoint{1.337113in}{1.012069in}}%
\pgfpathlineto{\pgfqpoint{1.345882in}{1.012069in}}%
\pgfpathlineto{\pgfqpoint{1.354658in}{1.014011in}}%
\pgfpathlineto{\pgfqpoint{1.363431in}{1.014011in}}%
\pgfpathlineto{\pgfqpoint{1.372203in}{1.015954in}}%
\pgfpathlineto{\pgfqpoint{1.380976in}{1.012069in}}%
\pgfpathlineto{\pgfqpoint{1.389748in}{1.014011in}}%
\pgfpathlineto{\pgfqpoint{1.398520in}{1.012069in}}%
\pgfpathlineto{\pgfqpoint{1.407293in}{1.012069in}}%
\pgfpathlineto{\pgfqpoint{1.416066in}{1.015954in}}%
\pgfpathlineto{\pgfqpoint{1.451156in}{1.015954in}}%
\pgfpathlineto{\pgfqpoint{1.459929in}{1.017897in}}%
\pgfpathlineto{\pgfqpoint{1.468695in}{1.017897in}}%
\pgfpathlineto{\pgfqpoint{1.477452in}{1.019840in}}%
\pgfpathlineto{\pgfqpoint{1.512542in}{1.019840in}}%
\pgfpathlineto{\pgfqpoint{1.521315in}{1.021783in}}%
\pgfpathlineto{\pgfqpoint{1.538860in}{1.021783in}}%
\pgfpathlineto{\pgfqpoint{1.547632in}{1.019840in}}%
\pgfpathlineto{\pgfqpoint{1.556405in}{1.021783in}}%
\pgfpathlineto{\pgfqpoint{1.565177in}{1.021783in}}%
\pgfpathlineto{\pgfqpoint{1.573944in}{1.019840in}}%
\pgfpathlineto{\pgfqpoint{1.591473in}{1.019840in}}%
\pgfpathlineto{\pgfqpoint{1.600246in}{1.021783in}}%
\pgfpathlineto{\pgfqpoint{1.609018in}{1.019840in}}%
\pgfpathlineto{\pgfqpoint{1.714289in}{1.019840in}}%
\pgfpathlineto{\pgfqpoint{1.731834in}{1.023726in}}%
\pgfpathlineto{\pgfqpoint{1.740606in}{1.027611in}}%
\pgfpathlineto{\pgfqpoint{1.749379in}{1.025669in}}%
\pgfpathlineto{\pgfqpoint{1.758157in}{1.029554in}}%
\pgfpathlineto{\pgfqpoint{1.766924in}{1.025669in}}%
\pgfpathlineto{\pgfqpoint{1.775697in}{1.029554in}}%
\pgfpathlineto{\pgfqpoint{1.784469in}{1.029554in}}%
\pgfpathlineto{\pgfqpoint{1.793242in}{1.027611in}}%
\pgfpathlineto{\pgfqpoint{1.810787in}{1.031497in}}%
\pgfpathlineto{\pgfqpoint{1.819559in}{1.029554in}}%
\pgfpathlineto{\pgfqpoint{1.828332in}{1.031497in}}%
\pgfpathlineto{\pgfqpoint{1.863422in}{1.031497in}}%
\pgfpathlineto{\pgfqpoint{1.872195in}{1.033440in}}%
\pgfpathlineto{\pgfqpoint{1.880967in}{1.033440in}}%
\pgfpathlineto{\pgfqpoint{1.889740in}{1.031497in}}%
\pgfpathlineto{\pgfqpoint{1.968693in}{1.031497in}}%
\pgfpathlineto{\pgfqpoint{1.977465in}{1.033440in}}%
\pgfpathlineto{\pgfqpoint{1.986238in}{1.031497in}}%
\pgfpathlineto{\pgfqpoint{2.030123in}{1.031497in}}%
\pgfpathlineto{\pgfqpoint{2.038892in}{1.029554in}}%
\pgfpathlineto{\pgfqpoint{2.047712in}{1.031497in}}%
\pgfpathlineto{\pgfqpoint{2.056484in}{1.031497in}}%
\pgfpathlineto{\pgfqpoint{2.065257in}{1.033440in}}%
\pgfpathlineto{\pgfqpoint{2.100347in}{1.033440in}}%
\pgfpathlineto{\pgfqpoint{2.109120in}{1.037326in}}%
\pgfpathlineto{\pgfqpoint{2.126665in}{1.037326in}}%
\pgfpathlineto{\pgfqpoint{2.135437in}{1.041211in}}%
\pgfpathlineto{\pgfqpoint{2.144210in}{1.037326in}}%
\pgfpathlineto{\pgfqpoint{2.152982in}{1.041211in}}%
\pgfpathlineto{\pgfqpoint{2.161755in}{1.037326in}}%
\pgfpathlineto{\pgfqpoint{2.170527in}{1.041211in}}%
\pgfpathlineto{\pgfqpoint{2.231935in}{1.041211in}}%
\pgfpathlineto{\pgfqpoint{2.240708in}{1.043154in}}%
\pgfpathlineto{\pgfqpoint{2.249480in}{1.041211in}}%
\pgfpathlineto{\pgfqpoint{2.258253in}{1.043154in}}%
\pgfpathlineto{\pgfqpoint{2.302115in}{1.043154in}}%
\pgfpathlineto{\pgfqpoint{2.310888in}{1.045097in}}%
\pgfpathlineto{\pgfqpoint{2.337202in}{1.045097in}}%
\pgfpathlineto{\pgfqpoint{2.345978in}{1.047040in}}%
\pgfpathlineto{\pgfqpoint{2.354751in}{1.047040in}}%
\pgfpathlineto{\pgfqpoint{2.363523in}{1.048983in}}%
\pgfpathlineto{\pgfqpoint{2.381068in}{1.048983in}}%
\pgfpathlineto{\pgfqpoint{2.398614in}{1.052869in}}%
\pgfpathlineto{\pgfqpoint{2.442476in}{1.052869in}}%
\pgfpathlineto{\pgfqpoint{2.451249in}{1.054811in}}%
\pgfpathlineto{\pgfqpoint{2.486317in}{1.054811in}}%
\pgfpathlineto{\pgfqpoint{2.503858in}{1.058697in}}%
\pgfpathlineto{\pgfqpoint{2.521385in}{1.058697in}}%
\pgfpathlineto{\pgfqpoint{2.530158in}{1.062583in}}%
\pgfpathlineto{\pgfqpoint{2.538930in}{1.062583in}}%
\pgfpathlineto{\pgfqpoint{2.565248in}{1.068411in}}%
\pgfpathlineto{\pgfqpoint{2.574020in}{1.068411in}}%
\pgfpathlineto{\pgfqpoint{2.582793in}{1.070354in}}%
\pgfpathlineto{\pgfqpoint{2.609111in}{1.070354in}}%
\pgfpathlineto{\pgfqpoint{2.617883in}{1.074240in}}%
\pgfpathlineto{\pgfqpoint{2.626656in}{1.074240in}}%
\pgfpathlineto{\pgfqpoint{2.635428in}{1.076183in}}%
\pgfpathlineto{\pgfqpoint{2.644200in}{1.074240in}}%
\pgfpathlineto{\pgfqpoint{2.652973in}{1.074240in}}%
\pgfpathlineto{\pgfqpoint{2.661746in}{1.076183in}}%
\pgfpathlineto{\pgfqpoint{2.696792in}{1.076183in}}%
\pgfpathlineto{\pgfqpoint{2.705565in}{1.078126in}}%
\pgfpathlineto{\pgfqpoint{2.714337in}{1.078126in}}%
\pgfpathlineto{\pgfqpoint{2.723110in}{1.080069in}}%
\pgfpathlineto{\pgfqpoint{2.731882in}{1.074240in}}%
\pgfpathlineto{\pgfqpoint{2.740655in}{1.078126in}}%
\pgfpathlineto{\pgfqpoint{2.758197in}{1.078126in}}%
\pgfpathlineto{\pgfqpoint{2.766972in}{1.080069in}}%
\pgfpathlineto{\pgfqpoint{2.775745in}{1.080069in}}%
\pgfpathlineto{\pgfqpoint{2.784517in}{1.083954in}}%
\pgfpathlineto{\pgfqpoint{2.793290in}{1.083954in}}%
\pgfpathlineto{\pgfqpoint{2.802063in}{1.082011in}}%
\pgfpathlineto{\pgfqpoint{2.810835in}{1.085897in}}%
\pgfpathlineto{\pgfqpoint{2.854697in}{1.085897in}}%
\pgfpathlineto{\pgfqpoint{2.863470in}{1.082011in}}%
\pgfpathlineto{\pgfqpoint{2.872243in}{1.083954in}}%
\pgfpathlineto{\pgfqpoint{2.881009in}{1.082011in}}%
\pgfpathlineto{\pgfqpoint{2.898538in}{1.082011in}}%
\pgfpathlineto{\pgfqpoint{2.907327in}{1.083954in}}%
\pgfpathlineto{\pgfqpoint{2.916084in}{1.082011in}}%
\pgfpathlineto{\pgfqpoint{2.924856in}{1.082011in}}%
\pgfpathlineto{\pgfqpoint{2.933629in}{1.083954in}}%
\pgfpathlineto{\pgfqpoint{2.942401in}{1.082011in}}%
\pgfpathlineto{\pgfqpoint{2.951174in}{1.083954in}}%
\pgfpathlineto{\pgfqpoint{2.959946in}{1.083954in}}%
\pgfpathlineto{\pgfqpoint{2.968719in}{1.082011in}}%
\pgfpathlineto{\pgfqpoint{2.977492in}{1.083954in}}%
\pgfpathlineto{\pgfqpoint{2.986283in}{1.083954in}}%
\pgfpathlineto{\pgfqpoint{2.995059in}{1.080069in}}%
\pgfpathlineto{\pgfqpoint{3.003831in}{1.080069in}}%
\pgfpathlineto{\pgfqpoint{3.012604in}{1.082011in}}%
\pgfpathlineto{\pgfqpoint{3.021376in}{1.080069in}}%
\pgfpathlineto{\pgfqpoint{3.030149in}{1.080069in}}%
\pgfpathlineto{\pgfqpoint{3.038921in}{1.082011in}}%
\pgfpathlineto{\pgfqpoint{3.065240in}{1.082011in}}%
\pgfpathlineto{\pgfqpoint{3.074011in}{1.083954in}}%
\pgfpathlineto{\pgfqpoint{3.082784in}{1.082011in}}%
\pgfpathlineto{\pgfqpoint{3.109080in}{1.082011in}}%
\pgfpathlineto{\pgfqpoint{3.117852in}{1.083954in}}%
\pgfpathlineto{\pgfqpoint{3.152942in}{1.083954in}}%
\pgfpathlineto{\pgfqpoint{3.161715in}{1.085897in}}%
\pgfpathlineto{\pgfqpoint{3.188033in}{1.085897in}}%
\pgfpathlineto{\pgfqpoint{3.196805in}{1.087840in}}%
\pgfpathlineto{\pgfqpoint{3.214593in}{1.087840in}}%
\pgfpathlineto{\pgfqpoint{3.223365in}{1.085897in}}%
\pgfpathlineto{\pgfqpoint{3.232138in}{1.087840in}}%
\pgfpathlineto{\pgfqpoint{3.240907in}{1.085897in}}%
\pgfpathlineto{\pgfqpoint{3.249682in}{1.085897in}}%
\pgfpathlineto{\pgfqpoint{3.267228in}{1.089783in}}%
\pgfpathlineto{\pgfqpoint{3.311091in}{1.089783in}}%
\pgfpathlineto{\pgfqpoint{3.319863in}{1.085897in}}%
\pgfpathlineto{\pgfqpoint{3.328636in}{1.087840in}}%
\pgfpathlineto{\pgfqpoint{3.337408in}{1.085897in}}%
\pgfpathlineto{\pgfqpoint{3.363720in}{1.085897in}}%
\pgfpathlineto{\pgfqpoint{3.372477in}{1.087840in}}%
\pgfpathlineto{\pgfqpoint{3.390022in}{1.087840in}}%
\pgfpathlineto{\pgfqpoint{3.398794in}{1.089783in}}%
\pgfpathlineto{\pgfqpoint{3.407567in}{1.087840in}}%
\pgfpathlineto{\pgfqpoint{3.451429in}{1.087840in}}%
\pgfpathlineto{\pgfqpoint{3.460202in}{1.089783in}}%
\pgfpathlineto{\pgfqpoint{3.468969in}{1.085897in}}%
\pgfpathlineto{\pgfqpoint{3.504043in}{1.085897in}}%
\pgfpathlineto{\pgfqpoint{3.512815in}{1.087840in}}%
\pgfpathlineto{\pgfqpoint{3.521588in}{1.085897in}}%
\pgfpathlineto{\pgfqpoint{3.530360in}{1.085897in}}%
\pgfpathlineto{\pgfqpoint{3.539138in}{1.087840in}}%
\pgfpathlineto{\pgfqpoint{3.547928in}{1.091726in}}%
\pgfpathlineto{\pgfqpoint{3.556700in}{1.087840in}}%
\pgfpathlineto{\pgfqpoint{3.565469in}{1.089783in}}%
\pgfpathlineto{\pgfqpoint{3.574245in}{1.089783in}}%
\pgfpathlineto{\pgfqpoint{3.583018in}{1.091726in}}%
\pgfpathlineto{\pgfqpoint{3.600563in}{1.091726in}}%
\pgfpathlineto{\pgfqpoint{3.609335in}{1.089783in}}%
\pgfpathlineto{\pgfqpoint{3.618108in}{1.093669in}}%
\pgfpathlineto{\pgfqpoint{3.626874in}{1.093669in}}%
\pgfpathlineto{\pgfqpoint{3.635631in}{1.091726in}}%
\pgfpathlineto{\pgfqpoint{3.644404in}{1.091726in}}%
\pgfpathlineto{\pgfqpoint{3.653176in}{1.093669in}}%
\pgfpathlineto{\pgfqpoint{3.670718in}{1.093669in}}%
\pgfpathlineto{\pgfqpoint{3.679494in}{1.091726in}}%
\pgfpathlineto{\pgfqpoint{3.723356in}{1.091726in}}%
\pgfpathlineto{\pgfqpoint{3.732129in}{1.093669in}}%
\pgfpathlineto{\pgfqpoint{3.740902in}{1.093669in}}%
\pgfpathlineto{\pgfqpoint{3.749674in}{1.091726in}}%
\pgfpathlineto{\pgfqpoint{3.767219in}{1.095611in}}%
\pgfpathlineto{\pgfqpoint{3.828605in}{1.095611in}}%
\pgfpathlineto{\pgfqpoint{3.837374in}{1.093669in}}%
\pgfpathlineto{\pgfqpoint{3.872468in}{1.093669in}}%
\pgfpathlineto{\pgfqpoint{3.881240in}{1.091726in}}%
\pgfpathlineto{\pgfqpoint{3.890013in}{1.093669in}}%
\pgfpathlineto{\pgfqpoint{3.916353in}{1.093669in}}%
\pgfpathlineto{\pgfqpoint{3.925125in}{1.091726in}}%
\pgfpathlineto{\pgfqpoint{3.933893in}{1.095611in}}%
\pgfpathlineto{\pgfqpoint{3.968966in}{1.095611in}}%
\pgfpathlineto{\pgfqpoint{3.977738in}{1.093669in}}%
\pgfpathlineto{\pgfqpoint{3.995284in}{1.097554in}}%
\pgfpathlineto{\pgfqpoint{4.012828in}{1.093669in}}%
\pgfpathlineto{\pgfqpoint{4.021844in}{1.095611in}}%
\pgfpathlineto{\pgfqpoint{4.030616in}{1.093669in}}%
\pgfpathlineto{\pgfqpoint{4.056934in}{1.093669in}}%
\pgfpathlineto{\pgfqpoint{4.083251in}{1.087840in}}%
\pgfpathlineto{\pgfqpoint{4.092024in}{1.089783in}}%
\pgfpathlineto{\pgfqpoint{4.118342in}{1.089783in}}%
\pgfpathlineto{\pgfqpoint{4.127114in}{1.091726in}}%
\pgfpathlineto{\pgfqpoint{4.135887in}{1.089783in}}%
\pgfpathlineto{\pgfqpoint{4.144659in}{1.091726in}}%
\pgfpathlineto{\pgfqpoint{4.153432in}{1.091726in}}%
\pgfpathlineto{\pgfqpoint{4.162204in}{1.093669in}}%
\pgfpathlineto{\pgfqpoint{4.170977in}{1.097554in}}%
\pgfpathlineto{\pgfqpoint{4.179749in}{1.097554in}}%
\pgfpathlineto{\pgfqpoint{4.188522in}{1.095611in}}%
\pgfpathlineto{\pgfqpoint{4.197294in}{1.095611in}}%
\pgfpathlineto{\pgfqpoint{4.206082in}{1.093669in}}%
\pgfpathlineto{\pgfqpoint{4.214836in}{1.093669in}}%
\pgfpathlineto{\pgfqpoint{4.223612in}{1.095611in}}%
\pgfpathlineto{\pgfqpoint{4.232385in}{1.093669in}}%
\pgfpathlineto{\pgfqpoint{4.241157in}{1.093669in}}%
\pgfpathlineto{\pgfqpoint{4.258703in}{1.089783in}}%
\pgfpathlineto{\pgfqpoint{4.276248in}{1.089783in}}%
\pgfpathlineto{\pgfqpoint{4.285019in}{1.087840in}}%
\pgfpathlineto{\pgfqpoint{4.293793in}{1.087840in}}%
\pgfpathlineto{\pgfqpoint{4.302565in}{1.089783in}}%
\pgfpathlineto{\pgfqpoint{4.311338in}{1.087840in}}%
\pgfpathlineto{\pgfqpoint{4.346428in}{1.087840in}}%
\pgfpathlineto{\pgfqpoint{4.355201in}{1.083954in}}%
\pgfpathlineto{\pgfqpoint{4.363973in}{1.082011in}}%
\pgfpathlineto{\pgfqpoint{4.372746in}{1.083954in}}%
\pgfpathlineto{\pgfqpoint{4.442904in}{1.083954in}}%
\pgfpathlineto{\pgfqpoint{4.451677in}{1.085897in}}%
\pgfpathlineto{\pgfqpoint{4.460449in}{1.083954in}}%
\pgfpathlineto{\pgfqpoint{4.469222in}{1.085897in}}%
\pgfpathlineto{\pgfqpoint{4.504312in}{1.085897in}}%
\pgfpathlineto{\pgfqpoint{4.513084in}{1.083954in}}%
\pgfpathlineto{\pgfqpoint{4.530607in}{1.087840in}}%
\pgfpathlineto{\pgfqpoint{4.548152in}{1.087840in}}%
\pgfpathlineto{\pgfqpoint{4.556925in}{1.083954in}}%
\pgfpathlineto{\pgfqpoint{4.565698in}{1.083954in}}%
\pgfpathlineto{\pgfqpoint{4.574470in}{1.085897in}}%
\pgfpathlineto{\pgfqpoint{4.592015in}{1.085897in}}%
\pgfpathlineto{\pgfqpoint{4.600788in}{1.083954in}}%
\pgfpathlineto{\pgfqpoint{4.609560in}{1.085897in}}%
\pgfpathlineto{\pgfqpoint{4.618332in}{1.085897in}}%
\pgfpathlineto{\pgfqpoint{4.627105in}{1.089783in}}%
\pgfpathlineto{\pgfqpoint{4.653423in}{1.089783in}}%
\pgfpathlineto{\pgfqpoint{4.662196in}{1.091726in}}%
\pgfpathlineto{\pgfqpoint{4.688535in}{1.091726in}}%
\pgfpathlineto{\pgfqpoint{4.697301in}{1.089783in}}%
\pgfpathlineto{\pgfqpoint{4.706058in}{1.091726in}}%
\pgfpathlineto{\pgfqpoint{4.714831in}{1.087840in}}%
\pgfpathlineto{\pgfqpoint{4.723603in}{1.087840in}}%
\pgfpathlineto{\pgfqpoint{4.732376in}{1.085897in}}%
\pgfpathlineto{\pgfqpoint{4.741149in}{1.087840in}}%
\pgfpathlineto{\pgfqpoint{4.749921in}{1.085897in}}%
\pgfpathlineto{\pgfqpoint{4.758694in}{1.087840in}}%
\pgfpathlineto{\pgfqpoint{4.802556in}{1.087840in}}%
\pgfpathlineto{\pgfqpoint{4.811329in}{1.089783in}}%
\pgfpathlineto{\pgfqpoint{4.820102in}{1.087840in}}%
\pgfpathlineto{\pgfqpoint{4.828874in}{1.089783in}}%
\pgfpathlineto{\pgfqpoint{4.846419in}{1.089783in}}%
\pgfpathlineto{\pgfqpoint{4.855192in}{1.093669in}}%
\pgfpathlineto{\pgfqpoint{4.863964in}{1.091726in}}%
\pgfpathlineto{\pgfqpoint{4.899082in}{1.091726in}}%
\pgfpathlineto{\pgfqpoint{4.916621in}{1.087840in}}%
\pgfpathlineto{\pgfqpoint{4.925394in}{1.087840in}}%
\pgfpathlineto{\pgfqpoint{4.942939in}{1.083954in}}%
\pgfpathlineto{\pgfqpoint{4.951712in}{1.083954in}}%
\pgfpathlineto{\pgfqpoint{4.960484in}{1.082011in}}%
\pgfpathlineto{\pgfqpoint{4.978029in}{1.082011in}}%
\pgfpathlineto{\pgfqpoint{4.986802in}{1.083954in}}%
\pgfpathlineto{\pgfqpoint{5.021892in}{1.083954in}}%
\pgfpathlineto{\pgfqpoint{5.030665in}{1.085897in}}%
\pgfpathlineto{\pgfqpoint{5.056982in}{1.085897in}}%
\pgfpathlineto{\pgfqpoint{5.065755in}{1.089783in}}%
\pgfpathlineto{\pgfqpoint{5.074527in}{1.089783in}}%
\pgfpathlineto{\pgfqpoint{5.083300in}{1.087840in}}%
\pgfpathlineto{\pgfqpoint{5.092072in}{1.089783in}}%
\pgfpathlineto{\pgfqpoint{5.109617in}{1.089783in}}%
\pgfpathlineto{\pgfqpoint{5.118390in}{1.087840in}}%
\pgfpathlineto{\pgfqpoint{5.127163in}{1.089783in}}%
\pgfpathlineto{\pgfqpoint{5.135935in}{1.087840in}}%
\pgfpathlineto{\pgfqpoint{5.153480in}{1.087840in}}%
\pgfpathlineto{\pgfqpoint{5.162253in}{1.089783in}}%
\pgfpathlineto{\pgfqpoint{5.171025in}{1.087840in}}%
\pgfpathlineto{\pgfqpoint{5.188570in}{1.087840in}}%
\pgfpathlineto{\pgfqpoint{5.197343in}{1.085897in}}%
\pgfpathlineto{\pgfqpoint{5.232433in}{1.085897in}}%
\pgfpathlineto{\pgfqpoint{5.241206in}{1.083954in}}%
\pgfpathlineto{\pgfqpoint{5.267523in}{1.083954in}}%
\pgfpathlineto{\pgfqpoint{5.276297in}{1.082011in}}%
\pgfpathlineto{\pgfqpoint{5.293841in}{1.082011in}}%
\pgfpathlineto{\pgfqpoint{5.302614in}{1.085897in}}%
\pgfpathlineto{\pgfqpoint{5.311386in}{1.083954in}}%
\pgfpathlineto{\pgfqpoint{5.328931in}{1.083954in}}%
\pgfpathlineto{\pgfqpoint{5.337704in}{1.085897in}}%
\pgfpathlineto{\pgfqpoint{5.346476in}{1.083954in}}%
\pgfpathlineto{\pgfqpoint{5.355249in}{1.085897in}}%
\pgfpathlineto{\pgfqpoint{5.364021in}{1.083954in}}%
\pgfpathlineto{\pgfqpoint{5.372794in}{1.085897in}}%
\pgfpathlineto{\pgfqpoint{5.381566in}{1.085897in}}%
\pgfpathlineto{\pgfqpoint{5.390339in}{1.083954in}}%
\pgfpathlineto{\pgfqpoint{5.399111in}{1.085897in}}%
\pgfpathlineto{\pgfqpoint{5.407884in}{1.085897in}}%
\pgfpathlineto{\pgfqpoint{5.416679in}{1.083954in}}%
\pgfpathlineto{\pgfqpoint{5.434219in}{1.083954in}}%
\pgfpathlineto{\pgfqpoint{5.442974in}{1.085897in}}%
\pgfpathlineto{\pgfqpoint{5.460519in}{1.082011in}}%
\pgfpathlineto{\pgfqpoint{5.478064in}{1.082011in}}%
\pgfpathlineto{\pgfqpoint{5.486837in}{1.083954in}}%
\pgfpathlineto{\pgfqpoint{5.504382in}{1.083954in}}%
\pgfpathlineto{\pgfqpoint{5.513155in}{1.082011in}}%
\pgfpathlineto{\pgfqpoint{5.521927in}{1.082011in}}%
\pgfpathlineto{\pgfqpoint{5.530700in}{1.080069in}}%
\pgfpathlineto{\pgfqpoint{5.539472in}{1.080069in}}%
\pgfpathlineto{\pgfqpoint{5.548245in}{1.076183in}}%
\pgfpathlineto{\pgfqpoint{5.565782in}{1.072297in}}%
\pgfpathlineto{\pgfqpoint{5.583291in}{1.072297in}}%
\pgfpathlineto{\pgfqpoint{5.592063in}{1.068411in}}%
\pgfpathlineto{\pgfqpoint{5.600836in}{1.068411in}}%
\pgfpathlineto{\pgfqpoint{5.609609in}{1.062583in}}%
\pgfpathlineto{\pgfqpoint{5.618381in}{1.062583in}}%
\pgfpathlineto{\pgfqpoint{5.627154in}{1.060640in}}%
\pgfpathlineto{\pgfqpoint{5.635926in}{1.062583in}}%
\pgfpathlineto{\pgfqpoint{5.644699in}{1.060640in}}%
\pgfpathlineto{\pgfqpoint{5.653471in}{1.062583in}}%
\pgfpathlineto{\pgfqpoint{5.662244in}{1.060640in}}%
\pgfpathlineto{\pgfqpoint{5.671016in}{1.062583in}}%
\pgfpathlineto{\pgfqpoint{5.679789in}{1.058697in}}%
\pgfpathlineto{\pgfqpoint{5.697334in}{1.058697in}}%
\pgfpathlineto{\pgfqpoint{5.706107in}{1.056754in}}%
\pgfpathlineto{\pgfqpoint{5.723652in}{1.056754in}}%
\pgfpathlineto{\pgfqpoint{5.741197in}{1.052869in}}%
\pgfpathlineto{\pgfqpoint{5.749969in}{1.054811in}}%
\pgfpathlineto{\pgfqpoint{5.770000in}{1.054811in}}%
\pgfpathlineto{\pgfqpoint{5.770000in}{1.054811in}}%
\pgfusepath{stroke}%
\end{pgfscope}%
\begin{pgfscope}%
\pgfsetrectcap%
\pgfsetmiterjoin%
\pgfsetlinewidth{0.803000pt}%
\definecolor{currentstroke}{rgb}{0.000000,0.000000,0.000000}%
\pgfsetstrokecolor{currentstroke}%
\pgfsetdash{}{0pt}%
\pgfpathmoveto{\pgfqpoint{0.800000in}{0.960000in}}%
\pgfpathlineto{\pgfqpoint{0.800000in}{4.224000in}}%
\pgfusepath{stroke}%
\end{pgfscope}%
\begin{pgfscope}%
\pgfsetrectcap%
\pgfsetmiterjoin%
\pgfsetlinewidth{0.803000pt}%
\definecolor{currentstroke}{rgb}{0.000000,0.000000,0.000000}%
\pgfsetstrokecolor{currentstroke}%
\pgfsetdash{}{0pt}%
\pgfpathmoveto{\pgfqpoint{5.760000in}{0.960000in}}%
\pgfpathlineto{\pgfqpoint{5.760000in}{4.224000in}}%
\pgfusepath{stroke}%
\end{pgfscope}%
\begin{pgfscope}%
\pgfsetrectcap%
\pgfsetmiterjoin%
\pgfsetlinewidth{0.803000pt}%
\definecolor{currentstroke}{rgb}{0.000000,0.000000,0.000000}%
\pgfsetstrokecolor{currentstroke}%
\pgfsetdash{}{0pt}%
\pgfpathmoveto{\pgfqpoint{0.800000in}{0.960000in}}%
\pgfpathlineto{\pgfqpoint{5.760000in}{0.960000in}}%
\pgfusepath{stroke}%
\end{pgfscope}%
\begin{pgfscope}%
\pgfsetrectcap%
\pgfsetmiterjoin%
\pgfsetlinewidth{0.803000pt}%
\definecolor{currentstroke}{rgb}{0.000000,0.000000,0.000000}%
\pgfsetstrokecolor{currentstroke}%
\pgfsetdash{}{0pt}%
\pgfpathmoveto{\pgfqpoint{0.800000in}{4.224000in}}%
\pgfpathlineto{\pgfqpoint{5.760000in}{4.224000in}}%
\pgfusepath{stroke}%
\end{pgfscope}%
\begin{pgfscope}%
\definecolor{textcolor}{rgb}{0.000000,0.000000,0.000000}%
\pgfsetstrokecolor{textcolor}%
\pgfsetfillcolor{textcolor}%
%\pgftext[x=3.280000in,y=4.307333in,,base]{\color{textcolor}\rmfamily\fontsize{12.000000}{14.400000}\selectfont Évolution de la température dans le nouveau boîtier}%
\end{pgfscope}%
\begin{pgfscope}%
\pgfsetbuttcap%
\pgfsetmiterjoin%
\definecolor{currentfill}{rgb}{1.000000,1.000000,1.000000}%
\pgfsetfillcolor{currentfill}%
\pgfsetfillopacity{0.800000}%
\pgfsetlinewidth{1.003750pt}%
\definecolor{currentstroke}{rgb}{0.800000,0.800000,0.800000}%
\pgfsetstrokecolor{currentstroke}%
\pgfsetstrokeopacity{0.800000}%
\pgfsetdash{}{0pt}%
\pgfpathmoveto{\pgfqpoint{4.070955in}{3.531871in}}%
\pgfpathlineto{\pgfqpoint{5.662778in}{3.531871in}}%
\pgfpathquadraticcurveto{\pgfqpoint{5.690556in}{3.531871in}}{\pgfqpoint{5.690556in}{3.559648in}}%
\pgfpathlineto{\pgfqpoint{5.690556in}{4.126778in}}%
\pgfpathquadraticcurveto{\pgfqpoint{5.690556in}{4.154556in}}{\pgfqpoint{5.662778in}{4.154556in}}%
\pgfpathlineto{\pgfqpoint{4.070955in}{4.154556in}}%
\pgfpathquadraticcurveto{\pgfqpoint{4.043177in}{4.154556in}}{\pgfqpoint{4.043177in}{4.126778in}}%
\pgfpathlineto{\pgfqpoint{4.043177in}{3.559648in}}%
\pgfpathquadraticcurveto{\pgfqpoint{4.043177in}{3.531871in}}{\pgfqpoint{4.070955in}{3.531871in}}%
\pgfpathclose%
\pgfusepath{stroke,fill}%
\end{pgfscope}%
\begin{pgfscope}%
\pgfsetrectcap%
\pgfsetroundjoin%
\pgfsetlinewidth{1.505625pt}%
\definecolor{currentstroke}{rgb}{0.121569,0.466667,0.705882}%
\pgfsetstrokecolor{currentstroke}%
\pgfsetdash{}{0pt}%
\pgfpathmoveto{\pgfqpoint{4.098733in}{4.050389in}}%
\pgfpathlineto{\pgfqpoint{4.376510in}{4.050389in}}%
\pgfusepath{stroke}%
\end{pgfscope}%
\begin{pgfscope}%
\definecolor{textcolor}{rgb}{0.000000,0.000000,0.000000}%
\pgfsetstrokecolor{textcolor}%
\pgfsetfillcolor{textcolor}%
\pgftext[x=4.487622in,y=4.001778in,left,base]{\color{textcolor}\rmfamily\fontsize{10.000000}{12.000000}\selectfont CPU}%
\end{pgfscope}%
\begin{pgfscope}%
\pgfsetrectcap%
\pgfsetroundjoin%
\pgfsetlinewidth{1.505625pt}%
\definecolor{currentstroke}{rgb}{1.000000,0.498039,0.054902}%
\pgfsetstrokecolor{currentstroke}%
\pgfsetdash{}{0pt}%
\pgfpathmoveto{\pgfqpoint{4.098733in}{3.856716in}}%
\pgfpathlineto{\pgfqpoint{4.376510in}{3.856716in}}%
\pgfusepath{stroke}%
\end{pgfscope}%
\begin{pgfscope}%
\definecolor{textcolor}{rgb}{0.000000,0.000000,0.000000}%
\pgfsetstrokecolor{textcolor}%
\pgfsetfillcolor{textcolor}%
\pgftext[x=4.487622in,y=3.808105in,left,base]{\color{textcolor}\rmfamily\fontsize{10.000000}{12.000000}\selectfont Zone Raspberry Pi}%
\end{pgfscope}%
\begin{pgfscope}%
\pgfsetrectcap%
\pgfsetroundjoin%
\pgfsetlinewidth{1.505625pt}%
\definecolor{currentstroke}{rgb}{0.172549,0.627451,0.172549}%
\pgfsetstrokecolor{currentstroke}%
\pgfsetdash{}{0pt}%
\pgfpathmoveto{\pgfqpoint{4.098733in}{3.663043in}}%
\pgfpathlineto{\pgfqpoint{4.376510in}{3.663043in}}%
\pgfusepath{stroke}%
\end{pgfscope}%
\begin{pgfscope}%
\definecolor{textcolor}{rgb}{0.000000,0.000000,0.000000}%
\pgfsetstrokecolor{textcolor}%
\pgfsetfillcolor{textcolor}%
\pgftext[x=4.487622in,y=3.614432in,left,base]{\color{textcolor}\rmfamily\fontsize{10.000000}{12.000000}\selectfont Zone step-down}%
\end{pgfscope}%
\end{pgfpicture}%
\makeatother%
\endgroup%

  \label{fig:test_6}
  \vspace{-1cm}
  \caption{\textbf{Test 6 :} Évolution de la température dans les tests réalisés dans le nouveau boîtier avec blindage magnétique}
\end{figure}


\begin{table}[h!]
  \centering
%\resizebox{\textwidth}{!}{%
\begin{tabular}{@{}lllll@{}}
\toprule
                                      & \textbf{Test 5}                                                        & \textbf{Test 6}         \\ \midrule
\textbf{Protocol:SendPackets Error}    & \begin{tabular}[c]{@{}l@{}}min : 4\\ max : 5\\ moy : 4,33\end{tabular} & \begin{tabular}[c]{@{}l@{}}min : 2\\ max : 3 \\ moy : 2,67\end{tabular}  \\ \midrule
\textbf{Receive:parseMessage Warning} & \begin{tabular}[c]{@{}l@{}}min : 0\\ max : 0\\ moy : 0\end{tabular} & \begin{tabular}[c]{@{}l@{}}min : 0\\ max : 0\\ moy : 0\end{tabular}     \\ \bottomrule
\end{tabular}%
%}

\caption{Statistiques loggés par le SDK de Thingstream pour les tests 5 et 6}
\label{tab:res22}
\end{table}
