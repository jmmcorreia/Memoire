
\chapter{Conclusion}

\noindent
Au cours de ce mémoire de fin d'études, une solution de monitoring pour le système d'information hospitalier CERHIS a été développée. Pour ce faire, les éléments matériels à utiliser devaient être établis en tenant compte des particularités de l'environnement. En effet, le prototype présenté ici est prêt à fonctionner dans des milieux avec des températures élevées et sans accès à une connexion Internet filaire. Le premier problème a été abordé avec le développement d'un boîtier spécifiquement conçu pour abriter les différents composants électroniques. Parmi ces composants figure un Raspberry Pi, qui fournit toute la puissance de calcul au prototype. Pour solutionner le second, le choix s'est porté sur l'utilisation d'un module SIM800L et la plateforme de Thingstream.  Les tests réalisés sur ce prototype ont mis en évidence les bonnes capacités de refroidissement de celui-ci. En outre, les problèmes de communication rencontrés avec le SIM800L lors du projet précédent ont été résolus. Cela a permis d'exploiter le SIM800L de manière très fiable.

~

\noindent
Le logiciel de monitoring développé est divisé en deux grandes parties. Premièrement, il y a le client qui tournera sur le Raspberry Pi dans le centre hospitalier. Cette application est basée sur une architecture de microservices et emploie le logiciel de Prometheus pour surveiller et stocker toutes les informations relatives au monitoring des dispositifs. L'outil de visualisation de données Grafana est couplé avec Prometheus pour donner aux techniciens un aperçu de l'état de l'infrastructure de CERHIS. Mais tout cela n'a pas été réalisé sans faute. Une première approche utilisant le protocole SNMP a été tentée. Cette solution s'est révélée efficace, mais elle n'était pas évolutive, ce qui causerait d'énormes problèmes dans l'avenir. Ceci constitue la plus grosse erreur qui a été commise lors de ce mémoire, puisqu'une recherche plus approfondie aurait montré que d'autres solutions plus performantes étaient disponibles sur le marché. Heureusement, cette erreur a pu être corrigée avec l'utilisation de Prometheus. Toutes les alertes détectées par le logiciel Prometheus sont transmises par SMS aux techniciens et vers un serveur distant.

~

\noindent
La deuxième partie de ce logiciel de monitoring consiste dans l'application tournant sur un serveur distant. Elle centralise toutes les données envoyées par les clients et les rend accessibles aux techniciens, même lorsqu'ils ne se trouvent pas dans le centre hospitalier. Le système de communication utilisé ne permet pas que toutes les métriques collectées par le client puissent être transmises sur ce serveur. Cependant, toutes les alertes y sont envoyées, incluant des mises à jour sur celles-ci. L'outil Grafana a été employé pour offrir une visualisation des données du serveur.

~

\noindent
Les tests effectués sur l'intégrité de la solution développée ont été très prometteurs. Malgré le fait qu'ils n'ont pas été réalisés en conjonction avec l'infrastructure de CERHIS, ils démontrent les capacités du prototype, en particulier celles de l'application de monitoring. Cependant, tout n'a pas été sans failles. Le SDK de Thingstream a présenté un bug qui a un grand impact sur l'utilisation de cette plateforme. Le support technique de Thingstream a assuré qu'il sera corrigé à l'avenir, mais aucune date n'a été donnée quant au moment où cela pourrait se produire.

\section{Suggestions d'amélioration}

\noindent
La solution de communication avec Thingstream s'est montrée comme étant le goulet d'étranglement de ce prototype. Malgré sa fiabilité irréprochable, son architecture opère de manière différente que celle du logiciel de monitoring (sauf pour les alertes). En outre, la plateforme de Thingstream n'a pas été conçue pour transmettre le volume de données qui est acquis par Prometheus. Dans l'avenir, une étude locale, c'est-à-dire en République Démocratique du Congo, doit être réalisée dans le dessein d'établir quelle est la solution de communication la plus adaptée pour ce projet. En effet, Thingstream offre une grande fiabilité au détriment de la quantité d'informations qui peut être communiquée. Les deux autres solutions proposées (Routeur LTE et le protocole point à point) peuvent résoudre le problème lié au volume de données, mais pourraient avoir un impact sur la fiabilité. C'est afin de comprendre ce dernier point que l'étude doit s'orienter, c'est-à-dire trouver la solution proposant le meilleur compromis entre coût, fiabilité et quantité de données qu'elle peut transmettre. Un deuxième point qui a encore besoin d'un peu plus de développement est le circuit imprimé employé avec ce prototype. Le circuit imprimé réalisé au cours de l'année précédente a été réutilisé cette année, mais il n'est plus adapté au nouveau prototype. En effet, avec l'addition d'un ventilateur, il serait tout à fait intéressant de pouvoir connecter ce dernier au Raspberry Pi dans le but de contrôler sa vitesse. Cela fournira une manière très simple de diminuer la consommation énergétique du prototype sans avoir un impact sur les capacités de refroidissement du boîtier. En outre, cela permettra aussi de surveiller l'état du ventilateur. Dernièrement, cette solution doit impérativement être testée dans l'environnement de CERHIS. Sans cela, les différents paramètres ne pourront pas être optimisés et des erreurs potentielles pourraient passer inaperçues.
